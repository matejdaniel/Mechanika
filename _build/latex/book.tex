%% Generated by Sphinx.
\def\sphinxdocclass{jupyterBook}
\documentclass[letterpaper,10pt,english]{jupyterBook}
\ifdefined\pdfpxdimen
   \let\sphinxpxdimen\pdfpxdimen\else\newdimen\sphinxpxdimen
\fi \sphinxpxdimen=.75bp\relax
\ifdefined\pdfimageresolution
    \pdfimageresolution= \numexpr \dimexpr1in\relax/\sphinxpxdimen\relax
\fi
%% let collapsible pdf bookmarks panel have high depth per default
\PassOptionsToPackage{bookmarksdepth=5}{hyperref}
%% turn off hyperref patch of \index as sphinx.xdy xindy module takes care of
%% suitable \hyperpage mark-up, working around hyperref-xindy incompatibility
\PassOptionsToPackage{hyperindex=false}{hyperref}
%% memoir class requires extra handling
\makeatletter\@ifclassloaded{memoir}
{\ifdefined\memhyperindexfalse\memhyperindexfalse\fi}{}\makeatother

\PassOptionsToPackage{booktabs}{sphinx}
\PassOptionsToPackage{colorrows}{sphinx}

\PassOptionsToPackage{warn}{textcomp}

\catcode`^^^^00a0\active\protected\def^^^^00a0{\leavevmode\nobreak\ }
\usepackage{cmap}
\usepackage{fontspec}
\defaultfontfeatures[\rmfamily,\sffamily,\ttfamily]{}
\usepackage{amsmath,amssymb,amstext}
\usepackage{polyglossia}
\setmainlanguage{english}



\setmainfont{FreeSerif}[
  Extension      = .otf,
  UprightFont    = *,
  ItalicFont     = *Italic,
  BoldFont       = *Bold,
  BoldItalicFont = *BoldItalic
]
\setsansfont{FreeSans}[
  Extension      = .otf,
  UprightFont    = *,
  ItalicFont     = *Oblique,
  BoldFont       = *Bold,
  BoldItalicFont = *BoldOblique,
]
\setmonofont{FreeMono}[
  Extension      = .otf,
  UprightFont    = *,
  ItalicFont     = *Oblique,
  BoldFont       = *Bold,
  BoldItalicFont = *BoldOblique,
]



\usepackage[Bjarne]{fncychap}
\usepackage[,numfigreset=1,mathnumfig]{sphinx}

\fvset{fontsize=\small}
\usepackage{geometry}


% Include hyperref last.
\usepackage{hyperref}
% Fix anchor placement for figures with captions.
\usepackage{hypcap}% it must be loaded after hyperref.
% Set up styles of URL: it should be placed after hyperref.
\urlstyle{same}

\addto\captionsenglish{\renewcommand{\contentsname}{O předmětu}}

\usepackage{sphinxmessages}



        % Start of preamble defined in sphinx-jupyterbook-latex %
         \usepackage[Latin,Greek]{ucharclasses}
        \usepackage{unicode-math}
        % fixing title of the toc
        \addto\captionsenglish{\renewcommand{\contentsname}{Contents}}
        \hypersetup{
            pdfencoding=auto,
            psdextra
        }
        % End of preamble defined in sphinx-jupyterbook-latex %
        

\title{Mechanika}
\date{Mar 02, 2025}
\release{}
\author{Matej Daniel}
\newcommand{\sphinxlogo}{\vbox{}}
\renewcommand{\releasename}{}
\makeindex
\begin{document}

\pagestyle{empty}
\sphinxmaketitle
\pagestyle{plain}
\sphinxtableofcontents
\pagestyle{normal}
\phantomsection\label{\detokenize{intro::doc}}


\sphinxAtStartPar
V těchto učebních textech se nachází vybrané kapitoly přednášené v rámci jednosemestrálního kurzu na Fakultě biomedicínského inženýrství, ČVUT v Praze.
\begin{itemize}
\item {} 
\sphinxAtStartPar
O předmětu

\begin{itemize}
\item {} 
\sphinxAtStartPar
{\hyperref[\detokenize{Misc/Podm_xednky_z_xe1po_u010dtu_a_zkou_u0161ky::doc}]{\sphinxcrossref{Zápočet:}}}

\item {} 
\sphinxAtStartPar
{\hyperref[\detokenize{Misc/Sylabus_LS2025::doc}]{\sphinxcrossref{Mechanika FBMI sylabus LS 2024/2025}}}

\end{itemize}
\end{itemize}
\begin{itemize}
\item {} 
\sphinxAtStartPar
Přednášky

\begin{itemize}
\item {} 
\sphinxAtStartPar
{\hyperref[\detokenize{Prednasky/0_1_Definice_a_historie_mechaniky::doc}]{\sphinxcrossref{Úvod do mechaniky}}}

\item {} 
\sphinxAtStartPar
{\hyperref[\detokenize{Prednasky/0_2_Skal_xe1ry_a_vektory::doc}]{\sphinxcrossref{Skalár a vektor}}}

\item {} 
\sphinxAtStartPar
{\hyperref[\detokenize{Prednasky/1_2_Kinematika_v_1D::doc}]{\sphinxcrossref{Kinematika v 1DOF prostoru}}}

\item {} 
\sphinxAtStartPar
{\hyperref[\detokenize{Prednasky/2_1_Mechanick_xe9_vlastnosti_materi_xe1lu::doc}]{\sphinxcrossref{Mechanické vlastnosti materiálu}}}

\end{itemize}
\end{itemize}
\begin{itemize}
\item {} 
\sphinxAtStartPar
Cvičení

\begin{itemize}
\item {} 
\sphinxAtStartPar
{\hyperref[\detokenize{Cviceni/C1::doc}]{\sphinxcrossref{1. cvičení}}}

\item {} 
\sphinxAtStartPar
{\hyperref[\detokenize{Cviceni/C2::doc}]{\sphinxcrossref{2. cvičení}}}

\end{itemize}
\end{itemize}

\sphinxstepscope


\part{O předmětu}

\sphinxstepscope


\chapter{Zápočet:}
\label{\detokenize{Misc/Podm_xednky_z_xe1po_u010dtu_a_zkou_u0161ky:zapocet}}\label{\detokenize{Misc/Podm_xednky_z_xe1po_u010dtu_a_zkou_u0161ky::doc}}\begin{itemize}
\item {} 
\sphinxAtStartPar
Cvičení jsou povinnou součástí výuky.

\item {} 
\sphinxAtStartPar
Během semestru se uskuteční 3 testy, z každého je možné získat maximálně 20 bodů, celkem je tedy možné získat 60 bodů.

\item {} 
\sphinxAtStartPar
Podmínkou získání zápočtu je úspěšné absolvování každého zápočtového testu s minimálním celkovým ziskem 50\% bodů (maximum 20
bodů, minimum 10 bodů). Účast na testech je povinná.

\item {} 
\sphinxAtStartPar
Každý student má právo absolvovat test jedenkrát znovu (tj. jeden opravný test).

\item {} 
\sphinxAtStartPar
Pokud se student nemohl z vážných důvodů (nemoc) testu zúčastnit, písemně se omluvil před termínem testu vyučujícímu a doložil oprávněnost své absence (lékařským potvrzením), může psát test v náhradním termínu. O oprávněnosti omluvy rozhoduje přednášející.

\item {} 
\sphinxAtStartPar
Pokuď student nesplní miniální podmínku u maximálně jednoho zápočtového testu ani u opravného termínu, je možné nahradit chybějící body vypracováním domácího úkolu.

\item {} 
\sphinxAtStartPar
Pokuď student nesplní podmínku 10 bodů u dva a více testů, nebude mu udělen zápočet.

\end{itemize}


\chapter{Zkouška:}
\label{\detokenize{Misc/Podm_xednky_z_xe1po_u010dtu_a_zkou_u0161ky:zkouska}}
\sphinxAtStartPar
Zkoušky se lze účastnit pouze při zapsaném zápočtu v KOSu a po zaspání na termín. Zkouška je převážně písemná a obsahuje
jak teoretické, tak početní otázky z přednášek, cvičení a domácích úkolů. Zkouška se skládá z těchto
částí:
\begin{enumerate}
\sphinxsetlistlabels{\arabic}{enumi}{enumii}{}{.}%
\item {} 
\sphinxAtStartPar
Test o délce cca 60 minut obsahuje krátké otázky z teorie a jednoduché výpočetní úlohy
s maximálním ziskem 40 bodů. Ze 40 možných bodů je nutné získat alespoň 50\%. Test probíhá na počítačích a student volí správnou odpověď z možných odpovědí.

\item {} 
\sphinxAtStartPar
Příkladová část o délce cca 80 minut obvykle obsahuje dva až tři příklady s maximálním ziskem 40 bodů.
\begin{itemize}
\item {} 
\sphinxAtStartPar
Ze 40 možných bodů je nutné získat alespoň 50\%. Lze používat seznamy vzorců bez dalších vepsaných poznámek (oficiální tahák).

\item {} 
\sphinxAtStartPar
Ani jeden příklad nesmí být hodnocen 0 body. V případě hodnocení příkladu za 0 bodů student neuspěl.

\end{itemize}

\item {} 
\sphinxAtStartPar
Ústní část. Lze získat 20 bodů. Při špatně zodpovězené otázce a prokázání neznalosti ztráta max. 20 bodů.

\end{enumerate}

\sphinxAtStartPar
Součtem bodů ze zápočtového testu a ze všech částí zkoušky vznikne výsledná známka.

\sphinxAtStartPar
Součet bodů Výsledná známka
90\sphinxhyphen{}100 A – výborně
80\sphinxhyphen{}89 B – velmi dobře
70\sphinxhyphen{}79 C – dobře
60\sphinxhyphen{}69 D – uspokojivě
50\sphinxhyphen{}59 E – dostatečně
<50 F

\sphinxstepscope


\chapter{Mechanika FBMI sylabus LS 2024/2025}
\label{\detokenize{Misc/Sylabus_LS2025:mechanika-fbmi-sylabus-ls-2024-2025}}\label{\detokenize{Misc/Sylabus_LS2025::doc}}
\sphinxAtStartPar
LS 2024/2025

\sphinxAtStartPar
Biomedicínský technik

\sphinxAtStartPar
Fakulta biomedicínského inženýrství, ČVUT v Praze


\begin{savenotes}
\sphinxatlongtablestart
\sphinxthistablewithglobalstyle
\makeatletter
  \LTleft \@totalleftmargin plus1fill
  \LTright\dimexpr\columnwidth-\@totalleftmargin-\linewidth\relax plus1fill
\makeatother
\begin{longtable}{llll}
\sphinxtoprule
\sphinxstyletheadfamily 
\sphinxAtStartPar
\sphinxstylestrong{Týden}
&\sphinxstyletheadfamily 
\sphinxAtStartPar
\sphinxstylestrong{Blok}
&\sphinxstyletheadfamily 
\sphinxAtStartPar
\sphinxstylestrong{Přednáška}
&\sphinxstyletheadfamily 
\sphinxAtStartPar
\sphinxstylestrong{Cvičení}
\\
\sphinxmidrule
\endfirsthead

\multicolumn{4}{c}{\sphinxnorowcolor
    \makebox[0pt]{\sphinxtablecontinued{\tablename\ \thetable{} \textendash{} continued from previous page}}%
}\\
\sphinxtoprule
\sphinxstyletheadfamily 
\sphinxAtStartPar
\sphinxstylestrong{Týden}
&\sphinxstyletheadfamily 
\sphinxAtStartPar
\sphinxstylestrong{Blok}
&\sphinxstyletheadfamily 
\sphinxAtStartPar
\sphinxstylestrong{Přednáška}
&\sphinxstyletheadfamily 
\sphinxAtStartPar
\sphinxstylestrong{Cvičení}
\\
\sphinxmidrule
\endhead

\sphinxbottomrule
\multicolumn{4}{r}{\sphinxnorowcolor
    \makebox[0pt][r]{\sphinxtablecontinued{continues on next page}}%
}\\
\endfoot

\endlastfoot
\sphinxtableatstartofbodyhook

\sphinxAtStartPar
1.
&
\sphinxAtStartPar
**Mechanika
&
\sphinxAtStartPar
Kinematika
&
\sphinxAtStartPar
Souřadnicové
\\
\sphinxhline
\sphinxAtStartPar

&
\sphinxAtStartPar
systémů s 1
&
\sphinxAtStartPar
přímkového
&
\sphinxAtStartPar
soustavy a
\\
\sphinxhline
\sphinxAtStartPar
20.2.
&
\sphinxAtStartPar
stupněm
&
\sphinxAtStartPar
pohybu, stupeň
&
\sphinxAtStartPar
jejich
\\
\sphinxhline
\sphinxAtStartPar

&
\sphinxAtStartPar
vonosti**
&
\sphinxAtStartPar
volnosti, vztah
&
\sphinxAtStartPar
transformace,
\\
\sphinxhline
\sphinxAtStartPar

&
\sphinxAtStartPar

&
\sphinxAtStartPar
dráha,
&
\sphinxAtStartPar
vektory a
\\
\sphinxhline
\sphinxAtStartPar

&
\sphinxAtStartPar

&
\sphinxAtStartPar
rychlost,
&
\sphinxAtStartPar
vektorové
\\
\sphinxhline
\sphinxAtStartPar

&
\sphinxAtStartPar

&
\sphinxAtStartPar
zrychlení,
&
\sphinxAtStartPar
operace –
\\
\sphinxhline
\sphinxAtStartPar

&
\sphinxAtStartPar

&
\sphinxAtStartPar
fyzikální
&
\sphinxAtStartPar
součet,
\\
\sphinxhline
\sphinxAtStartPar

&
\sphinxAtStartPar

&
\sphinxAtStartPar
interpretace
&
\sphinxAtStartPar
odečítání,
\\
\sphinxhline
\sphinxAtStartPar

&
\sphinxAtStartPar

&
\sphinxAtStartPar
derivace a
&
\sphinxAtStartPar
skalární a
\\
\sphinxhline
\sphinxAtStartPar

&
\sphinxAtStartPar

&
\sphinxAtStartPar
integrálu,
&
\sphinxAtStartPar
vektorový
\\
\sphinxhline
\sphinxAtStartPar

&
\sphinxAtStartPar

&
\sphinxAtStartPar
volný pád,
&
\sphinxAtStartPar
součin a jejich
\\
\sphinxhline
\sphinxAtStartPar

&
\sphinxAtStartPar

&
\sphinxAtStartPar
kolmý vrh.
&
\sphinxAtStartPar
fyzikální
\\
\sphinxhline
\sphinxAtStartPar

&
\sphinxAtStartPar

&
\sphinxAtStartPar
Dynamika
&
\sphinxAtStartPar
interpretace,
\\
\sphinxhline
\sphinxAtStartPar

&
\sphinxAtStartPar

&
\sphinxAtStartPar
pohybu,
&
\sphinxAtStartPar
grafické řešení
\\
\sphinxhline
\sphinxAtStartPar

&
\sphinxAtStartPar

&
\sphinxAtStartPar
Newtonovy
&
\sphinxAtStartPar
součtu a
\\
\sphinxhline
\sphinxAtStartPar

&
\sphinxAtStartPar

&
\sphinxAtStartPar
zákony, zákon
&
\sphinxAtStartPar
odečítání
\\
\sphinxhline
\sphinxAtStartPar

&
\sphinxAtStartPar

&
\sphinxAtStartPar
zachování
&
\sphinxAtStartPar
vektorů
\\
\sphinxhline
\sphinxAtStartPar

&
\sphinxAtStartPar

&
\sphinxAtStartPar
hybnosti,
&
\sphinxAtStartPar

\\
\sphinxhline
\sphinxAtStartPar

&
\sphinxAtStartPar

&
\sphinxAtStartPar
energie a její
&
\sphinxAtStartPar

\\
\sphinxhline
\sphinxAtStartPar

&
\sphinxAtStartPar

&
\sphinxAtStartPar
zachování,
&
\sphinxAtStartPar

\\
\sphinxhline
\sphinxAtStartPar

&
\sphinxAtStartPar

&
\sphinxAtStartPar
Lagranegeův
&
\sphinxAtStartPar

\\
\sphinxhline
\sphinxAtStartPar

&
\sphinxAtStartPar

&
\sphinxAtStartPar
popis pohybu,
&
\sphinxAtStartPar

\\
\sphinxhline
\sphinxAtStartPar

&
\sphinxAtStartPar

&
\sphinxAtStartPar
pasivní síly \sphinxhyphen{}
&
\sphinxAtStartPar

\\
\sphinxhline
\sphinxAtStartPar

&
\sphinxAtStartPar

&
\sphinxAtStartPar
tření, šikmá
&
\sphinxAtStartPar

\\
\sphinxhline
\sphinxAtStartPar

&
\sphinxAtStartPar

&
\sphinxAtStartPar
rovina, měření
&
\sphinxAtStartPar

\\
\sphinxhline
\sphinxAtStartPar

&
\sphinxAtStartPar

&
\sphinxAtStartPar
síly, rychlosti
&
\sphinxAtStartPar

\\
\sphinxhline
\sphinxAtStartPar

&
\sphinxAtStartPar

&
\sphinxAtStartPar
a zrychlení
&
\sphinxAtStartPar

\\
\sphinxhline
\sphinxAtStartPar
———–
&
\sphinxAtStartPar
—————–
&
\sphinxAtStartPar
—————–
&
\sphinxAtStartPar
—————–
\\
\sphinxhline
\sphinxAtStartPar
2.
&
\sphinxAtStartPar

&
\sphinxAtStartPar
Přímková silová
&
\sphinxAtStartPar
Pohyb po přímce
\\
\sphinxhline
\sphinxAtStartPar

&
\sphinxAtStartPar

&
\sphinxAtStartPar
soustava,
&
\sphinxAtStartPar
\sphinxhyphen{} statika,
\\
\sphinxhline
\sphinxAtStartPar
27.2.
&
\sphinxAtStartPar

&
\sphinxAtStartPar
rovnováha,
&
\sphinxAtStartPar
kinematika a
\\
\sphinxhline
\sphinxAtStartPar

&
\sphinxAtStartPar

&
\sphinxAtStartPar
uvolnění,
&
\sphinxAtStartPar
dynamika
\\
\sphinxhline
\sphinxAtStartPar

&
\sphinxAtStartPar

&
\sphinxAtStartPar
vnější a
&
\sphinxAtStartPar

\\
\sphinxhline
\sphinxAtStartPar

&
\sphinxAtStartPar

&
\sphinxAtStartPar
vnitřní síly,
&
\sphinxAtStartPar

\\
\sphinxhline
\sphinxAtStartPar

&
\sphinxAtStartPar

&
\sphinxAtStartPar
metoda řezu,
&
\sphinxAtStartPar

\\
\sphinxhline
\sphinxAtStartPar

&
\sphinxAtStartPar

&
\sphinxAtStartPar
posunutí,
&
\sphinxAtStartPar

\\
\sphinxhline
\sphinxAtStartPar

&
\sphinxAtStartPar

&
\sphinxAtStartPar
deformace
&
\sphinxAtStartPar

\\
\sphinxhline
\sphinxAtStartPar

&
\sphinxAtStartPar

&
\sphinxAtStartPar
(relativní a
&
\sphinxAtStartPar

\\
\sphinxhline
\sphinxAtStartPar

&
\sphinxAtStartPar

&
\sphinxAtStartPar
přirozené
&
\sphinxAtStartPar

\\
\sphinxhline
\sphinxAtStartPar

&
\sphinxAtStartPar

&
\sphinxAtStartPar
prodloužení),
&
\sphinxAtStartPar

\\
\sphinxhline
\sphinxAtStartPar

&
\sphinxAtStartPar

&
\sphinxAtStartPar
napětí, tahový
&
\sphinxAtStartPar

\\
\sphinxhline
\sphinxAtStartPar

&
\sphinxAtStartPar

&
\sphinxAtStartPar
diagram,
&
\sphinxAtStartPar

\\
\sphinxhline
\sphinxAtStartPar

&
\sphinxAtStartPar

&
\sphinxAtStartPar
charakteristiky
&
\sphinxAtStartPar

\\
\sphinxhline
\sphinxAtStartPar

&
\sphinxAtStartPar

&
\sphinxAtStartPar
materiálu,
&
\sphinxAtStartPar

\\
\sphinxhline
\sphinxAtStartPar

&
\sphinxAtStartPar

&
\sphinxAtStartPar
dimenzování
&
\sphinxAtStartPar

\\
\sphinxhline
\sphinxAtStartPar

&
\sphinxAtStartPar

&
\sphinxAtStartPar
prutu, prut
&
\sphinxAtStartPar

\\
\sphinxhline
\sphinxAtStartPar

&
\sphinxAtStartPar

&
\sphinxAtStartPar
stálé pevnosti,
&
\sphinxAtStartPar

\\
\sphinxhline
\sphinxAtStartPar

&
\sphinxAtStartPar

&
\sphinxAtStartPar
deformační
&
\sphinxAtStartPar

\\
\sphinxhline
\sphinxAtStartPar

&
\sphinxAtStartPar

&
\sphinxAtStartPar
energie, rázové
&
\sphinxAtStartPar

\\
\sphinxhline
\sphinxAtStartPar

&
\sphinxAtStartPar

&
\sphinxAtStartPar
namáhání,
&
\sphinxAtStartPar

\\
\sphinxhline
\sphinxAtStartPar

&
\sphinxAtStartPar

&
\sphinxAtStartPar
měření
&
\sphinxAtStartPar

\\
\sphinxhline
\sphinxAtStartPar

&
\sphinxAtStartPar

&
\sphinxAtStartPar
napjatosti
&
\sphinxAtStartPar

\\
\sphinxhline
\sphinxAtStartPar
———–
&
\sphinxAtStartPar
—————–
&
\sphinxAtStartPar
—————–
&
\sphinxAtStartPar
—————–
\\
\sphinxhline
\sphinxAtStartPar
3.
&
\sphinxAtStartPar

&
\sphinxAtStartPar
Periodický
&
\sphinxAtStartPar
Tah a tlak
\\
\sphinxhline
\sphinxAtStartPar

&
\sphinxAtStartPar

&
\sphinxAtStartPar
pohyb, pohyb
&
\sphinxAtStartPar
prutů,
\\
\sphinxhline
\sphinxAtStartPar
6.3.
&
\sphinxAtStartPar

&
\sphinxAtStartPar
hmotného bodu
&
\sphinxAtStartPar
dimenzování
\\
\sphinxhline
\sphinxAtStartPar

&
\sphinxAtStartPar

&
\sphinxAtStartPar
po kružnici,
&
\sphinxAtStartPar

\\
\sphinxhline
\sphinxAtStartPar

&
\sphinxAtStartPar

&
\sphinxAtStartPar
kmitání
&
\sphinxAtStartPar

\\
\sphinxhline
\sphinxAtStartPar

&
\sphinxAtStartPar

&
\sphinxAtStartPar
oscilátoru,
&
\sphinxAtStartPar

\\
\sphinxhline
\sphinxAtStartPar

&
\sphinxAtStartPar

&
\sphinxAtStartPar
vlastní, tlmené
&
\sphinxAtStartPar

\\
\sphinxhline
\sphinxAtStartPar

&
\sphinxAtStartPar

&
\sphinxAtStartPar
a buzené kmity,
&
\sphinxAtStartPar

\\
\sphinxhline
\sphinxAtStartPar

&
\sphinxAtStartPar

&
\sphinxAtStartPar
kyvadlo –
&
\sphinxAtStartPar

\\
\sphinxhline
\sphinxAtStartPar

&
\sphinxAtStartPar

&
\sphinxAtStartPar
fyzikální a
&
\sphinxAtStartPar

\\
\sphinxhline
\sphinxAtStartPar

&
\sphinxAtStartPar

&
\sphinxAtStartPar
matematické
&
\sphinxAtStartPar

\\
\sphinxhline
\sphinxAtStartPar

&
\sphinxAtStartPar

&
\sphinxAtStartPar
kyvadlo,
&
\sphinxAtStartPar

\\
\sphinxhline
\sphinxAtStartPar

&
\sphinxAtStartPar

&
\sphinxAtStartPar
dvojité kyvadlo
&
\sphinxAtStartPar

\\
\sphinxhline
\sphinxAtStartPar
———–
&
\sphinxAtStartPar
—————–
&
\sphinxAtStartPar
—————–
&
\sphinxAtStartPar
—————–
\\
\sphinxhline
\sphinxAtStartPar
4.
&
\sphinxAtStartPar
**Pohyb a
&
\sphinxAtStartPar
Pohyb hmotného
&
\sphinxAtStartPar
Kmitání a pohyb
\\
\sphinxhline
\sphinxAtStartPar

&
\sphinxAtStartPar
deformace v
&
\sphinxAtStartPar
bodu v rovině,
&
\sphinxAtStartPar
po kružnici
\\
\sphinxhline
\sphinxAtStartPar
13.3.
&
\sphinxAtStartPar
rovině**
&
\sphinxAtStartPar
superpozice
&
\sphinxAtStartPar

\\
\sphinxhline
\sphinxAtStartPar

&
\sphinxAtStartPar

&
\sphinxAtStartPar
pohybů,
&
\sphinxAtStartPar

\\
\sphinxhline
\sphinxAtStartPar

&
\sphinxAtStartPar

&
\sphinxAtStartPar
dynamické
&
\sphinxAtStartPar

\\
\sphinxhline
\sphinxAtStartPar

&
\sphinxAtStartPar

&
\sphinxAtStartPar
rovnice pohybu,
&
\sphinxAtStartPar

\\
\sphinxhline
\sphinxAtStartPar

&
\sphinxAtStartPar

&
\sphinxAtStartPar
pohyb tělesa v
&
\sphinxAtStartPar

\\
\sphinxhline
\sphinxAtStartPar

&
\sphinxAtStartPar

&
\sphinxAtStartPar
rovině,
&
\sphinxAtStartPar

\\
\sphinxhline
\sphinxAtStartPar

&
\sphinxAtStartPar

&
\sphinxAtStartPar
translační a
&
\sphinxAtStartPar

\\
\sphinxhline
\sphinxAtStartPar

&
\sphinxAtStartPar

&
\sphinxAtStartPar
rotační účinky
&
\sphinxAtStartPar

\\
\sphinxhline
\sphinxAtStartPar

&
\sphinxAtStartPar

&
\sphinxAtStartPar
síly, moment
&
\sphinxAtStartPar

\\
\sphinxhline
\sphinxAtStartPar

&
\sphinxAtStartPar

&
\sphinxAtStartPar
síly, páka,
&
\sphinxAtStartPar

\\
\sphinxhline
\sphinxAtStartPar

&
\sphinxAtStartPar

&
\sphinxAtStartPar
Varigninova
&
\sphinxAtStartPar

\\
\sphinxhline
\sphinxAtStartPar

&
\sphinxAtStartPar

&
\sphinxAtStartPar
věta, poloha
&
\sphinxAtStartPar

\\
\sphinxhline
\sphinxAtStartPar

&
\sphinxAtStartPar

&
\sphinxAtStartPar
těžiště,
&
\sphinxAtStartPar

\\
\sphinxhline
\sphinxAtStartPar

&
\sphinxAtStartPar

&
\sphinxAtStartPar
těžiště a
&
\sphinxAtStartPar

\\
\sphinxhline
\sphinxAtStartPar

&
\sphinxAtStartPar

&
\sphinxAtStartPar
rovnováha,
&
\sphinxAtStartPar

\\
\sphinxhline
\sphinxAtStartPar

&
\sphinxAtStartPar

&
\sphinxAtStartPar
těžiště a
&
\sphinxAtStartPar

\\
\sphinxhline
\sphinxAtStartPar

&
\sphinxAtStartPar

&
\sphinxAtStartPar
energie
&
\sphinxAtStartPar

\\
\sphinxhline
\sphinxAtStartPar
———–
&
\sphinxAtStartPar
—————–
&
\sphinxAtStartPar
—————–
&
\sphinxAtStartPar
—————–
\\
\sphinxhline
\sphinxAtStartPar
5.
&
\sphinxAtStartPar

&
\sphinxAtStartPar
Dynamika pohybu
&
\sphinxAtStartPar
1. zápočtová
\\
\sphinxhline
\sphinxAtStartPar

&
\sphinxAtStartPar

&
\sphinxAtStartPar
v rovině,
&
\sphinxAtStartPar
písemka
\\
\sphinxhline
\sphinxAtStartPar
20.3.
&
\sphinxAtStartPar

&
\sphinxAtStartPar
moment
&
\sphinxAtStartPar

\\
\sphinxhline
\sphinxAtStartPar

&
\sphinxAtStartPar

&
\sphinxAtStartPar
setrvačnosti,
&
\sphinxAtStartPar
Pohyb hmotného
\\
\sphinxhline
\sphinxAtStartPar

&
\sphinxAtStartPar

&
\sphinxAtStartPar
moment
&
\sphinxAtStartPar
bodu a tělesa v
\\
\sphinxhline
\sphinxAtStartPar

&
\sphinxAtStartPar

&
\sphinxAtStartPar
hybnosti,
&
\sphinxAtStartPar
rovině –
\\
\sphinxhline
\sphinxAtStartPar

&
\sphinxAtStartPar

&
\sphinxAtStartPar
Steinerova
&
\sphinxAtStartPar
statika a
\\
\sphinxhline
\sphinxAtStartPar

&
\sphinxAtStartPar

&
\sphinxAtStartPar
věta, rovnice
&
\sphinxAtStartPar
kinematika
\\
\sphinxhline
\sphinxAtStartPar

&
\sphinxAtStartPar

&
\sphinxAtStartPar
pohybu tělesa v
&
\sphinxAtStartPar

\\
\sphinxhline
\sphinxAtStartPar

&
\sphinxAtStartPar

&
\sphinxAtStartPar
rovině,
&
\sphinxAtStartPar

\\
\sphinxhline
\sphinxAtStartPar

&
\sphinxAtStartPar

&
\sphinxAtStartPar
interciální
&
\sphinxAtStartPar

\\
\sphinxhline
\sphinxAtStartPar

&
\sphinxAtStartPar

&
\sphinxAtStartPar
síly,
&
\sphinxAtStartPar

\\
\sphinxhline
\sphinxAtStartPar

&
\sphinxAtStartPar

&
\sphinxAtStartPar
Coriolisova
&
\sphinxAtStartPar

\\
\sphinxhline
\sphinxAtStartPar

&
\sphinxAtStartPar

&
\sphinxAtStartPar
síla, zákon
&
\sphinxAtStartPar

\\
\sphinxhline
\sphinxAtStartPar

&
\sphinxAtStartPar

&
\sphinxAtStartPar
zachování
&
\sphinxAtStartPar

\\
\sphinxhline
\sphinxAtStartPar

&
\sphinxAtStartPar

&
\sphinxAtStartPar
momentu
&
\sphinxAtStartPar

\\
\sphinxhline
\sphinxAtStartPar

&
\sphinxAtStartPar

&
\sphinxAtStartPar
hybnosti
&
\sphinxAtStartPar

\\
\sphinxhline
\sphinxAtStartPar
———–
&
\sphinxAtStartPar
—————–
&
\sphinxAtStartPar
—————–
&
\sphinxAtStartPar
—————–
\\
\sphinxhline
\sphinxAtStartPar
6.
&
\sphinxAtStartPar

&
\sphinxAtStartPar
Rovinný stav
&
\sphinxAtStartPar
Pohyb hmotného
\\
\sphinxhline
\sphinxAtStartPar

&
\sphinxAtStartPar

&
\sphinxAtStartPar
napjatosti,
&
\sphinxAtStartPar
bodu a tělesa v
\\
\sphinxhline
\sphinxAtStartPar
27.3.
&
\sphinxAtStartPar

&
\sphinxAtStartPar
tensor napětí a
&
\sphinxAtStartPar
rovině –
\\
\sphinxhline
\sphinxAtStartPar

&
\sphinxAtStartPar

&
\sphinxAtStartPar
deformace,
&
\sphinxAtStartPar
dynamika
\\
\sphinxhline
\sphinxAtStartPar

&
\sphinxAtStartPar

&
\sphinxAtStartPar
hlavní napětí a
&
\sphinxAtStartPar

\\
\sphinxhline
\sphinxAtStartPar

&
\sphinxAtStartPar

&
\sphinxAtStartPar
hlavní
&
\sphinxAtStartPar

\\
\sphinxhline
\sphinxAtStartPar

&
\sphinxAtStartPar

&
\sphinxAtStartPar
deformace,
&
\sphinxAtStartPar

\\
\sphinxhline
\sphinxAtStartPar

&
\sphinxAtStartPar

&
\sphinxAtStartPar
Mohrova
&
\sphinxAtStartPar

\\
\sphinxhline
\sphinxAtStartPar

&
\sphinxAtStartPar

&
\sphinxAtStartPar
kružnice,
&
\sphinxAtStartPar

\\
\sphinxhline
\sphinxAtStartPar

&
\sphinxAtStartPar

&
\sphinxAtStartPar
zatížení krutem
&
\sphinxAtStartPar

\\
\sphinxhline
\sphinxAtStartPar

&
\sphinxAtStartPar

&
\sphinxAtStartPar
a smykem,
&
\sphinxAtStartPar

\\
\sphinxhline
\sphinxAtStartPar

&
\sphinxAtStartPar

&
\sphinxAtStartPar
membránový stav
&
\sphinxAtStartPar

\\
\sphinxhline
\sphinxAtStartPar

&
\sphinxAtStartPar

&
\sphinxAtStartPar
napjatosti,
&
\sphinxAtStartPar

\\
\sphinxhline
\sphinxAtStartPar

&
\sphinxAtStartPar

&
\sphinxAtStartPar
Laplacova
&
\sphinxAtStartPar

\\
\sphinxhline
\sphinxAtStartPar

&
\sphinxAtStartPar

&
\sphinxAtStartPar
rovnice,
&
\sphinxAtStartPar

\\
\sphinxhline
\sphinxAtStartPar

&
\sphinxAtStartPar

&
\sphinxAtStartPar
tenzometrická
&
\sphinxAtStartPar

\\
\sphinxhline
\sphinxAtStartPar

&
\sphinxAtStartPar

&
\sphinxAtStartPar
ružice
&
\sphinxAtStartPar

\\
\sphinxhline
\sphinxAtStartPar
———–
&
\sphinxAtStartPar
—————–
&
\sphinxAtStartPar
—————–
&
\sphinxAtStartPar
—————–
\\
\sphinxhline
\sphinxAtStartPar
7.
&
\sphinxAtStartPar

&
\sphinxAtStartPar
Nosník a jeho
&
\sphinxAtStartPar
Rovinný stav
\\
\sphinxhline
\sphinxAtStartPar

&
\sphinxAtStartPar

&
\sphinxAtStartPar
zatížení,
&
\sphinxAtStartPar
napjatosti,
\\
\sphinxhline
\sphinxAtStartPar
3.4.
&
\sphinxAtStartPar

&
\sphinxAtStartPar
Geometrické
&
\sphinxAtStartPar
hlavní napětí,
\\
\sphinxhline
\sphinxAtStartPar

&
\sphinxAtStartPar

&
\sphinxAtStartPar
charakteristiky
&
\sphinxAtStartPar
dimenzování
\\
\sphinxhline
\sphinxAtStartPar

&
\sphinxAtStartPar

&
\sphinxAtStartPar
průřezu,
&
\sphinxAtStartPar

\\
\sphinxhline
\sphinxAtStartPar

&
\sphinxAtStartPar

&
\sphinxAtStartPar
definice
&
\sphinxAtStartPar

\\
\sphinxhline
\sphinxAtStartPar

&
\sphinxAtStartPar

&
\sphinxAtStartPar
neutrální osy,
&
\sphinxAtStartPar

\\
\sphinxhline
\sphinxAtStartPar

&
\sphinxAtStartPar

&
\sphinxAtStartPar
dimenzování
&
\sphinxAtStartPar

\\
\sphinxhline
\sphinxAtStartPar

&
\sphinxAtStartPar

&
\sphinxAtStartPar
nosníků, vzpěr
&
\sphinxAtStartPar

\\
\sphinxhline
\sphinxAtStartPar
———–
&
\sphinxAtStartPar
—————–
&
\sphinxAtStartPar
—————–
&
\sphinxAtStartPar
—————–
\\
\sphinxhline
\sphinxAtStartPar
8.
&
\sphinxAtStartPar
**Pohyb a
&
\sphinxAtStartPar
Pohyb ve třech
&
\sphinxAtStartPar
Dimenzování
\\
\sphinxhline
\sphinxAtStartPar

&
\sphinxAtStartPar
deformace v
&
\sphinxAtStartPar
rozměrech,
&
\sphinxAtStartPar
nosníků
\\
\sphinxhline
\sphinxAtStartPar
10.4.
&
\sphinxAtStartPar
prostoru**
&
\sphinxAtStartPar
rovnováha ve
&
\sphinxAtStartPar

\\
\sphinxhline
\sphinxAtStartPar

&
\sphinxAtStartPar

&
\sphinxAtStartPar
třech
&
\sphinxAtStartPar

\\
\sphinxhline
\sphinxAtStartPar

&
\sphinxAtStartPar

&
\sphinxAtStartPar
rozměrech,
&
\sphinxAtStartPar

\\
\sphinxhline
\sphinxAtStartPar

&
\sphinxAtStartPar

&
\sphinxAtStartPar
vektorový a
&
\sphinxAtStartPar

\\
\sphinxhline
\sphinxAtStartPar

&
\sphinxAtStartPar

&
\sphinxAtStartPar
maticový zápis
&
\sphinxAtStartPar

\\
\sphinxhline
\sphinxAtStartPar

&
\sphinxAtStartPar

&
\sphinxAtStartPar
rovnic
&
\sphinxAtStartPar

\\
\sphinxhline
\sphinxAtStartPar

&
\sphinxAtStartPar

&
\sphinxAtStartPar
rovnováhy a
&
\sphinxAtStartPar

\\
\sphinxhline
\sphinxAtStartPar

&
\sphinxAtStartPar

&
\sphinxAtStartPar
rovnic pohybu
&
\sphinxAtStartPar

\\
\sphinxhline
\sphinxAtStartPar
———–
&
\sphinxAtStartPar
—————–
&
\sphinxAtStartPar
—————–
&
\sphinxAtStartPar
—————–
\\
\sphinxhline
\sphinxAtStartPar
9.
&
\sphinxAtStartPar

&
\sphinxAtStartPar
Transformační
&
\sphinxAtStartPar
2. zápočtová
\\
\sphinxhline
\sphinxAtStartPar

&
\sphinxAtStartPar

&
\sphinxAtStartPar
matice pro
&
\sphinxAtStartPar
písemka
\\
\sphinxhline
\sphinxAtStartPar
17.4.
&
\sphinxAtStartPar

&
\sphinxAtStartPar
popis polohy a
&
\sphinxAtStartPar

\\
\sphinxhline
\sphinxAtStartPar

&
\sphinxAtStartPar

&
\sphinxAtStartPar
pohybu,
&
\sphinxAtStartPar
Jenoduché úlohy
\\
\sphinxhline
\sphinxAtStartPar

&
\sphinxAtStartPar

&
\sphinxAtStartPar
jednoduchý
&
\sphinxAtStartPar
na rovnováhu ve
\\
\sphinxhline
\sphinxAtStartPar

&
\sphinxAtStartPar

&
\sphinxAtStartPar
mechanismus a
&
\sphinxAtStartPar
3D řešené
\\
\sphinxhline
\sphinxAtStartPar

&
\sphinxAtStartPar

&
\sphinxAtStartPar
jeho analýza
&
\sphinxAtStartPar
pomocí
\\
\sphinxhline
\sphinxAtStartPar

&
\sphinxAtStartPar

&
\sphinxAtStartPar

&
\sphinxAtStartPar
maticového
\\
\sphinxhline
\sphinxAtStartPar

&
\sphinxAtStartPar

&
\sphinxAtStartPar

&
\sphinxAtStartPar
počtu.
\\
\sphinxhline
\sphinxAtStartPar
———–
&
\sphinxAtStartPar
—————–
&
\sphinxAtStartPar
—————–
&
\sphinxAtStartPar
—————–
\\
\sphinxhline
\sphinxAtStartPar
10.
&
\sphinxAtStartPar

&
\sphinxAtStartPar
Prostorová
&
\sphinxAtStartPar
Mechanika
\\
\sphinxhline
\sphinxAtStartPar

&
\sphinxAtStartPar

&
\sphinxAtStartPar
napjatost,
&
\sphinxAtStartPar
mechanismů,
\\
\sphinxhline
\sphinxAtStartPar
24.4.
&
\sphinxAtStartPar

&
\sphinxAtStartPar
hlavní napětí a
&
\sphinxAtStartPar
příklady a
\\
\sphinxhline
\sphinxAtStartPar

&
\sphinxAtStartPar

&
\sphinxAtStartPar
deformace,
&
\sphinxAtStartPar
aplikace v
\\
\sphinxhline
\sphinxAtStartPar

&
\sphinxAtStartPar

&
\sphinxAtStartPar
redukované
&
\sphinxAtStartPar
biologii
\\
\sphinxhline
\sphinxAtStartPar

&
\sphinxAtStartPar

&
\sphinxAtStartPar
napětí ve 3D,
&
\sphinxAtStartPar

\\
\sphinxhline
\sphinxAtStartPar

&
\sphinxAtStartPar

&
\sphinxAtStartPar
kombinované
&
\sphinxAtStartPar

\\
\sphinxhline
\sphinxAtStartPar

&
\sphinxAtStartPar

&
\sphinxAtStartPar
namáhání –
&
\sphinxAtStartPar

\\
\sphinxhline
\sphinxAtStartPar

&
\sphinxAtStartPar

&
\sphinxAtStartPar
ohyb\sphinxhyphen{}ohyb,
&
\sphinxAtStartPar

\\
\sphinxhline
\sphinxAtStartPar

&
\sphinxAtStartPar

&
\sphinxAtStartPar
ohyb\sphinxhyphen{}tlak,
&
\sphinxAtStartPar

\\
\sphinxhline
\sphinxAtStartPar

&
\sphinxAtStartPar

&
\sphinxAtStartPar
ohyb\sphinxhyphen{}krut.
&
\sphinxAtStartPar

\\
\sphinxhline
\sphinxAtStartPar
———–
&
\sphinxAtStartPar
—————–
&
\sphinxAtStartPar
—————–
&
\sphinxAtStartPar
—————–
\\
\sphinxhline
\sphinxAtStartPar
11.
&
\sphinxAtStartPar

&
\sphinxAtStartPar
Státní svátek
&
\sphinxAtStartPar
Kombinované
\\
\sphinxhline
\sphinxAtStartPar

&
\sphinxAtStartPar

&
\sphinxAtStartPar

&
\sphinxAtStartPar
namáhání a
\\
\sphinxhline
\sphinxAtStartPar
1.5.
&
\sphinxAtStartPar

&
\sphinxAtStartPar

&
\sphinxAtStartPar
dimenzování,
\\
\sphinxhline
\sphinxAtStartPar

&
\sphinxAtStartPar

&
\sphinxAtStartPar

&
\sphinxAtStartPar
příklady z
\\
\sphinxhline
\sphinxAtStartPar

&
\sphinxAtStartPar

&
\sphinxAtStartPar

&
\sphinxAtStartPar
biologie
\\
\sphinxhline
\sphinxAtStartPar
———–
&
\sphinxAtStartPar
—————–
&
\sphinxAtStartPar
—————–
&
\sphinxAtStartPar
—————–
\\
\sphinxhline
\sphinxAtStartPar
12.
&
\sphinxAtStartPar

&
\sphinxAtStartPar
Státní svátek
&
\sphinxAtStartPar
3. zápočtová
\\
\sphinxhline
\sphinxAtStartPar

&
\sphinxAtStartPar

&
\sphinxAtStartPar

&
\sphinxAtStartPar
písemka
\\
\sphinxhline
\sphinxAtStartPar
8.5.
&
\sphinxAtStartPar

&
\sphinxAtStartPar

&
\sphinxAtStartPar

\\
\sphinxhline
\sphinxAtStartPar
———–
&
\sphinxAtStartPar
—————–
&
\sphinxAtStartPar
—————–
&
\sphinxAtStartPar
—————–
\\
\sphinxhline
\sphinxAtStartPar
13.
&
\sphinxAtStartPar
**Hydro a
&
\sphinxAtStartPar
Mechanika
&
\sphinxAtStartPar
Opakování
\\
\sphinxhline
\sphinxAtStartPar

&
\sphinxAtStartPar
t
&
\sphinxAtStartPar
kapalin a
&
\sphinxAtStartPar
problematických
\\
\sphinxhline
\sphinxAtStartPar

&
\sphinxAtStartPar
ermomechanika**
&
\sphinxAtStartPar
plynů,
&
\sphinxAtStartPar
příkladů ze
\\
\sphinxhline
\sphinxAtStartPar

&
\sphinxAtStartPar

&
\sphinxAtStartPar
hydrostatický
&
\sphinxAtStartPar
zápočtových
\\
\sphinxhline
\sphinxAtStartPar

&
\sphinxAtStartPar

&
\sphinxAtStartPar
paradox,
&
\sphinxAtStartPar
písemek
\\
\sphinxhline
\sphinxAtStartPar

&
\sphinxAtStartPar

&
\sphinxAtStartPar
laminární a
&
\sphinxAtStartPar

\\
\sphinxhline
\sphinxAtStartPar

&
\sphinxAtStartPar

&
\sphinxAtStartPar
turbulentní
&
\sphinxAtStartPar

\\
\sphinxhline
\sphinxAtStartPar

&
\sphinxAtStartPar

&
\sphinxAtStartPar
proudění,
&
\sphinxAtStartPar

\\
\sphinxhline
\sphinxAtStartPar

&
\sphinxAtStartPar

&
\sphinxAtStartPar
Reynoldsovo
&
\sphinxAtStartPar

\\
\sphinxhline
\sphinxAtStartPar

&
\sphinxAtStartPar

&
\sphinxAtStartPar
číslo, tok a
&
\sphinxAtStartPar

\\
\sphinxhline
\sphinxAtStartPar

&
\sphinxAtStartPar

&
\sphinxAtStartPar
rovnice
&
\sphinxAtStartPar

\\
\sphinxhline
\sphinxAtStartPar

&
\sphinxAtStartPar

&
\sphinxAtStartPar
kontinuity,
&
\sphinxAtStartPar

\\
\sphinxhline
\sphinxAtStartPar

&
\sphinxAtStartPar

&
\sphinxAtStartPar
ideální a
&
\sphinxAtStartPar

\\
\sphinxhline
\sphinxAtStartPar

&
\sphinxAtStartPar

&
\sphinxAtStartPar
reálná kapalina
&
\sphinxAtStartPar

\\
\sphinxhline
\sphinxAtStartPar

&
\sphinxAtStartPar

&
\sphinxAtStartPar
a její popis.
&
\sphinxAtStartPar

\\
\sphinxhline
\sphinxAtStartPar

&
\sphinxAtStartPar

&
\sphinxAtStartPar
Ideální a
&
\sphinxAtStartPar

\\
\sphinxhline
\sphinxAtStartPar

&
\sphinxAtStartPar

&
\sphinxAtStartPar
reální plyn,
&
\sphinxAtStartPar

\\
\sphinxhline
\sphinxAtStartPar

&
\sphinxAtStartPar

&
\sphinxAtStartPar
stavová rovnice
&
\sphinxAtStartPar

\\
\sphinxhline
\sphinxAtStartPar

&
\sphinxAtStartPar

&
\sphinxAtStartPar
plynu
&
\sphinxAtStartPar

\\
\sphinxhline
\sphinxAtStartPar
———–
&
\sphinxAtStartPar
—————–
&
\sphinxAtStartPar
—————–
&
\sphinxAtStartPar
—————–
\\
\sphinxhline
\sphinxAtStartPar
14
&
\sphinxAtStartPar

&
\sphinxAtStartPar
Zákony
&
\sphinxAtStartPar
Zápočet,
\\
\sphinxhline
\sphinxAtStartPar

&
\sphinxAtStartPar

&
\sphinxAtStartPar
termodynamiky,
&
\sphinxAtStartPar
opravné písemky
\\
\sphinxhline
\sphinxAtStartPar

&
\sphinxAtStartPar

&
\sphinxAtStartPar
Carnotův cyklus
&
\sphinxAtStartPar

\\
\sphinxhline
\sphinxAtStartPar

&
\sphinxAtStartPar

&
\sphinxAtStartPar
statistický a
&
\sphinxAtStartPar

\\
\sphinxhline
\sphinxAtStartPar

&
\sphinxAtStartPar

&
\sphinxAtStartPar
termodynamický
&
\sphinxAtStartPar

\\
\sphinxhline
\sphinxAtStartPar

&
\sphinxAtStartPar

&
\sphinxAtStartPar
popis entropie,
&
\sphinxAtStartPar

\\
\sphinxhline
\sphinxAtStartPar

&
\sphinxAtStartPar

&
\sphinxAtStartPar
tepelné stroje,
&
\sphinxAtStartPar

\\
\sphinxhline
\sphinxAtStartPar

&
\sphinxAtStartPar

&
\sphinxAtStartPar
účinnost
&
\sphinxAtStartPar

\\
\sphinxhline
\sphinxAtStartPar
———–
&
\sphinxAtStartPar
—————–
&
\sphinxAtStartPar
—————–
&
\sphinxAtStartPar
—————–
\\
\sphinxbottomrule
\end{longtable}
\sphinxtableafterendhook
\sphinxatlongtableend
\end{savenotes}

\begin{sphinxuseclass}{cell}\begin{sphinxVerbatimInput}

\begin{sphinxuseclass}{cell_input}
\begin{sphinxVerbatim}[commandchars=\\\{\}]
\PYG{k+kn}{from} \PYG{n+nn}{IPython}\PYG{n+nn}{.}\PYG{n+nn}{display} \PYG{k+kn}{import} \PYG{n}{YouTubeVideo}

\PYG{n}{YouTubeVideo}\PYG{p}{(}\PYG{l+s+s1}{\PYGZsq{}}\PYG{l+s+s1}{rVBAevK07ys}\PYG{l+s+s1}{\PYGZsq{}}\PYG{p}{,} \PYG{n}{width}\PYG{o}{=}\PYG{l+m+mi}{800}\PYG{p}{)}
\end{sphinxVerbatim}

\end{sphinxuseclass}\end{sphinxVerbatimInput}
\begin{sphinxVerbatimOutput}

\begin{sphinxuseclass}{cell_output}
\noindent\sphinxincludegraphics{{5d513599da96e95c7949d4e38e3b81ef1fb6d0c3aec10d02b88e1960acad28a5}.jpg}

\end{sphinxuseclass}\end{sphinxVerbatimOutput}

\end{sphinxuseclass}
\sphinxstepscope


\part{Přednášky}

\sphinxstepscope


\chapter{Úvod do mechaniky}
\label{\detokenize{Prednasky/0_1_Definice_a_historie_mechaniky:uvod-do-mechaniky}}\label{\detokenize{Prednasky/0_1_Definice_a_historie_mechaniky::doc}}

\section{Co je to mechanika?}
\label{\detokenize{Prednasky/0_1_Definice_a_historie_mechaniky:co-je-to-mechanika}}

\bigskip\hrule\bigskip

\begin{quote}


\end{quote}


\bigskip\hrule\bigskip



\section{Historie mechaniky}
\label{\detokenize{Prednasky/0_1_Definice_a_historie_mechaniky:historie-mechaniky}}
\sphinxAtStartPar
\sphinxstylestrong{Galileo Galilei} zahájil moderní éru mechaniky použitím matematiky k popisu pohybu těles. Jeho \sphinxstyleemphasis{Mechanika}, publikovaná v roce 1623, představila koncept síly a popsala konstantně zrychlený pohyb objektů v blízkosti povrchu Země.

\sphinxAtStartPar
O šedesát let později \sphinxstylestrong{Isaac Newton} formuloval své zákony pohybu, které publikoval v roce 1687 pod názvem \sphinxstyleemphasis{Philosophiae Naturalis Principia Mathematica}. Ve třetí knize, nazvané \sphinxstyleemphasis{De mundi systemate}, Newton vyřešil největší vědecký problém své doby použitím svého univerzálního gravitačního zákona k určení pohybu planet. Newton zavedl matematický přístup k analýze fyzikálních jevů a odmítl hypotézy bez experimentálního základu.

\sphinxAtStartPar
Tento přístup vedl k rozsáhlému rozvoji Newtonovské mechaniky a vrcholil pracemi \sphinxstylestrong{Lenarda Eulera}, který systematicky studoval trojrozměrný pohyb tuhých těles a formuloval rovnice pohybu tuhého tělesa známé jako Eulerovy rovnice. Významný přínos k rozvoji mechaniky měl také \sphinxstylestrong{Joseph\sphinxhyphen{}Louis Lagrange}, který zformuloval analytickou mechaniku založenou na principech variačního počtu. Lagrangeova mechanika poskytla elegantní a obecný přístup k řešení mechanických problémů prostřednictvím Lagrangeových rovnic druhého druhu, které umožňují popis dynamiky systémů bez nutnosti explicitního uvažování sil působících na jednotlivé částice.
Během tohoto vývoje se postupně formoval i koncept energie, což vyvrcholilo v polovině 19. století objevem principu zachování energie a jeho aplikací na termodynamiku. Zachovávací principy, včetně zachování hybnosti, energie a momentu hybnosti, se staly klíčovými v klasické mechanice.

\sphinxAtStartPar
Newtonovská mechanika byla dále aplikována na systémy složené z mnoha částic, což vedlo k rozvoji mechaniky kontinua a teorií mechaniky tekutin, vlnové mechaniky a elektromagnetismu. Vývoj vlnové teorie světla přinesl otázky ohledně existence éteru, které byly vyvráceny Michelson\sphinxhyphen{}Morleyho experimentem v roce 1887. Následně \sphinxstylestrong{Albert Einstein} ve své speciální teorii relativity (1905) přehodnotil koncepty prostoru a času, čímž vyřešil rozpory mezi optikou a Newtonovskou mechanikou.

\sphinxAtStartPar
Další omezení Newtonovské mechaniky se objevily na mikroskopické úrovni. Statistická mechanika byla vyvinuta k propojení mikroskopických vlastností atomů a molekul s makroskopickými termodynamickými vlastnostmi materiálů. Začátkem 20. století kvantová mechanika poskytla matematický popis mikroskopických jevů, který plně odpovídal experimentálním pozorováním.

\sphinxAtStartPar
Ve 20. století se ukázalo, že Newtonův gravitační zákon již přesně nepopisuje velkorozměrový vesmír a byl nahrazen obecnou relativitou. Pozorování jako zrychlená expanze vesmíru na konci 20. a počátku 21. století vedla k zavedení nových konceptů, jako je temná energie, což opět vyžaduje přehodnocení základních fyzikálních konceptů.

\sphinxAtStartPar
\sphinxincludegraphics{{f94cac4896b1044d92ac3c94b46fcb274d83a0f1}.png}

\sphinxAtStartPar
Dělí se na několik hlavních oblastí:


\section{Dělení mechaniky}
\label{\detokenize{Prednasky/0_1_Definice_a_historie_mechaniky:deleni-mechaniky}}\begin{itemize}
\item {} 
\sphinxAtStartPar
\sphinxstylestrong{Klasická mechanika} – zahrnuje Newtonovu mechaniku, analytickou mechaniku (Lagrangeovu a Hamiltonovu) a mechaniku kontinua.

\item {} 
\sphinxAtStartPar
\sphinxstylestrong{Relativistická mechanika} – zabývá se pohybem těles při rychlostech blízkých rychlosti světla (Einsteinova teorie relativity).

\item {} 
\sphinxAtStartPar
\sphinxstylestrong{Kvantová mechanika} – popisuje chování částic na mikroskopické úrovni, kde selhává klasická mechanika.

\item {} 
\sphinxAtStartPar
\sphinxstylestrong{Statistická mechanika} – propojuje mikroskopické vlastnosti částic s makroskopickými vlastnostmi systémů.

\end{itemize}

\sphinxAtStartPar
V rámci jednosemstrálního kuzru mechaniky se budeme primárně zabývat klasickou mechaniku s jemným nahldénutím do statistické mechaniky při definici entropie. V mechanice, postupuje od elementárního k pokročilému čtyřmi obecnými způsoby:
\begin{enumerate}
\sphinxsetlistlabels{\arabic}{enumi}{enumii}{}{.}%
\item {} 
\sphinxAtStartPar
\sphinxstylestrong{Počet prostorových dimenzí.}
\begin{itemize}
\item {} 
\sphinxAtStartPar
Jednorozměrná mechanika je nejjednodušší. Všechny síly působí v jednom směru, například ve směru osy x (nebo opačném), a nebere se v úvahu 2D nebo 3D geometrie. Někteří lidé tomu říkají „skalární“ mechanika, protože vektory mají v 1D minimální využití.

\item {} 
\sphinxAtStartPar
Rovinná neboli 2D mechanika je další v pořadí obtížnosti. 2D mechanika je nejvíce zdůrazňována v jednoduchých aplikacích. Geometrie je důležitá, ale ne příliš složitá.

\item {} 
\sphinxAtStartPar
3D mechanika je nejobtížnější. Geometrie tří rozměrů je překvapivě obtížnější než dvě dimenze.

\end{itemize}

\item {} 
\sphinxAtStartPar
\sphinxstylestrong{Složitost pohybu.}
\begin{itemize}
\item {} 
\sphinxAtStartPar
Statika, která předpokládá žádný pohyb (nebo přesněji, zanedbatelné zrychlení), je nejjednodušší.

\item {} 
\sphinxAtStartPar
Přímočarý pohyb je další v pořadí obtížnosti za předpokladu, že se všechny body pohybují rovnoběžně s jednou danou přímkou, například osou \sphinxstyleemphasis{x}.

\item {} 
\sphinxAtStartPar
Pohyb po kružnici se týká systémů, kde se všechny body pohybují po kruzích kolem daného bodu v rovině nebo, ve 3D, kolem dané pevné osy.

\item {} 
\sphinxAtStartPar
Obecný pohyb, kde se body a objekty mohou pohybovat jakýmkoli způsobem, je nejobecnější a nejobtížnější.

\end{itemize}

\item {} 
\sphinxAtStartPar
\sphinxstylestrong{Složitost systému.} V přibližném pořadí obtížnosti mohou být systémy, které lze studovat v mechanice částic a tuhých těles, následující:
\begin{itemize}
\item {} 
\sphinxAtStartPar
Jedna částice (hmotný bod).

\item {} 
\sphinxAtStartPar
Systém částic.

\item {} 
\sphinxAtStartPar
Tuhé těleso.

\item {} 
\sphinxAtStartPar
Soustava částic a tuhých těles.

\end{itemize}

\item {} 
\sphinxAtStartPar
\sphinxstylestrong{Typ interakce.} Části systému interagují mezi sebou silami. V dynamice je s některými interakcemi snazší se vypořádat než s jinými.
\begin{itemize}
\item {} 
\sphinxAtStartPar
Síly určené polohami a rychlostmi. Nejjednodušší interakce jsou, když síly pocházejí z pružin, tlumičů a gravitace. To znamená, že síly lze nalézt přímo, pokud znáte polohy a rychlosti všech objektů.

\item {} 
\sphinxAtStartPar
Síly a zrychlení jsou spojené. Jedná se o systémy, které mají části, které interagují s vazbami, jako jsou závěsy a kluzné spoje. Tyto „kinematické“ vazby se používají k popisu mechanismů.

\end{itemize}

\end{enumerate}

\sphinxAtStartPar
\sphinxincludegraphics{{applmech-01-00001-g001}.png}


\chapter{Předchozí znalosti}
\label{\detokenize{Prednasky/0_1_Definice_a_historie_mechaniky:predchozi-znalosti}}
\sphinxAtStartPar
\sphinxstylestrong{Matematika}: Předpokládá se, že studenti mají  znalosti ze základní geometrie, algebry, trigonometrie, derivace a integrace. Některá z těchto témat budou krátce vysvětlena, ale ne jako ab initio tutoriály. Ukážeme si jak využít řešení algebraických a diferenciálních rovnic k řešení problémů mechaniky.

\sphinxAtStartPar
\sphinxstylestrong{Programování}: Potřebujete znát, nebo se současně učit počítačový jazyk nebo software, který dokáže řešit prlbémy lineární algebraiky
rovnic, numericky integrovat jednoduché obyčejné diferenciální rovnice a dělat slušné grafy. V rámci našeho kurzu budeme primárně používat programovací jazyk Python. V případě, že tento jazyk neovládáte, můžete také využít Matlab nebo Matematiku.


\chapter{Co obsahuje tento kurz}
\label{\detokenize{Prednasky/0_1_Definice_a_historie_mechaniky:co-obsahuje-tento-kurz}}\begin{quote}

\sphinxAtStartPar
\sphinxstylestrong{Upozornění}: Obsah kurzu se může měnit  v průběhu semestru.
\end{quote}

\sphinxAtStartPar
Obsah kurzu je následující:
\begin{enumerate}
\sphinxsetlistlabels{\arabic}{enumi}{enumii}{}{.}%
\setcounter{enumi}{-1}
\item {} 
\sphinxAtStartPar
\sphinxstylestrong{Úvod}
\begin{itemize}
\item {} 
\sphinxAtStartPar
Definice rozdělení a historie mechaniky

\item {} 
\sphinxAtStartPar
Skalární a vektorový počet

\end{itemize}

\item {} 
\sphinxAtStartPar
\sphinxstylestrong{Mechanika systémů s 1 stupněm vonosti}
\begin{itemize}
\item {} 
\sphinxAtStartPar
Kinematika přímkového pohybu, stupeň volnosti, vztah dráha, rychlost, zrychlení, fyzikální interpretace derivace a integrálu, volný pád, kolmý vrh.

\item {} 
\sphinxAtStartPar
Dynamika pohybu, Newtonovy zákony, zákon zachování hybnosti,

\item {} 
\sphinxAtStartPar
Energie a její zachování, Lagranegeův popis pohybu

\item {} 
\sphinxAtStartPar
Pasivní síly \sphinxhyphen{} tření, šikmá rovina,

\item {} 
\sphinxAtStartPar
Měření síly, rychlosti a zrychlení

\end{itemize}

\item {} 
\sphinxAtStartPar
\sphinxstylestrong{Přímková silová soustava}
\begin{itemize}
\item {} 
\sphinxAtStartPar
Rovnováha, uvolnění, vnější a vnitřní síly

\item {} 
\sphinxAtStartPar
Metoda řezu, posunutí, deformace (relativní a přirozené prodloužení), napětí, t

\item {} 
\sphinxAtStartPar
Tahový diagram, charakteristiky materiálu,

\item {} 
\sphinxAtStartPar
Dimenzování prutu, prut stálé pevnosti,

\item {} 
\sphinxAtStartPar
Deformační energie, rázové namáhání,

\item {} 
\sphinxAtStartPar
Měření napjatosti

\end{itemize}

\item {} 
\sphinxAtStartPar
\sphinxstylestrong{Periodický pohyb}
\begin{itemize}
\item {} 
\sphinxAtStartPar
Pohyb hmotného bodu po kružnici, kmitání oscilátoru, vlastní, tlmené a buzené kmity,

\item {} 
\sphinxAtStartPar
Kyvadlo – fyzikální a matematické kyvadlo,

\item {} 
\sphinxAtStartPar
Dvojité kyvadlo, Lagrange\sphinxhyphen{}Eulerovy rovnice

\end{itemize}

\item {} 
\sphinxAtStartPar
\sphinxstylestrong{Pohyb a deformace v rovině}
\begin{itemize}
\item {} 
\sphinxAtStartPar
Pohyb hmotného bodu v rovině, superpozice pohybů,

\item {} 
\sphinxAtStartPar
Dynamické rovnice pohybu pro hmotný bod

\item {} 
\sphinxAtStartPar
Pohyb tělesa v rovině, translační a rotační účinky síly, moment síly, páka, Varigninova věta

\item {} 
\sphinxAtStartPar
Poloha těžiště, těžiště a rovnováha, těžiště a energie

\end{itemize}

\item {} 
\sphinxAtStartPar
\sphinxstylestrong{Dynamika pohybu v rovině}
\begin{itemize}
\item {} 
\sphinxAtStartPar
Moment setrvačnosti, moment hybnosti, Steinerova věta,

\item {} 
\sphinxAtStartPar
Rovnice pohybu tělesa v rovině, interciální síly, Coriolisova síla,

\item {} 
\sphinxAtStartPar
Zákon zachování momentu hybnosti

\end{itemize}

\item {} 
\sphinxAtStartPar
\sphinxstylestrong{Rovinný stav napjatosti}
\begin{itemize}
\item {} 
\sphinxAtStartPar
Zatížení krutem a smykem

\item {} 
\sphinxAtStartPar
Tensor napětí a deformace, hlavní napětí a hlavní deformace, zobecněný Hookeův zákon

\item {} 
\sphinxAtStartPar
Mohrova kružnice

\item {} 
\sphinxAtStartPar
Membránový stav napjatosti, Laplacova rovnice

\item {} 
\sphinxAtStartPar
Tenzometrická ružice

\end{itemize}

\item {} 
\sphinxAtStartPar
\sphinxstylestrong{Nosník a jeho zatížení}
\begin{itemize}
\item {} 
\sphinxAtStartPar
Geometrické charakteristiky průřezu, definice neutrální osy,

\item {} 
\sphinxAtStartPar
Dimenzování nosníků

\item {} 
\sphinxAtStartPar
Vzpěr

\end{itemize}

\item {} 
\sphinxAtStartPar
\sphinxstylestrong{Pohyb a deformace v prostoru}
\begin{itemize}
\item {} 
\sphinxAtStartPar
Pohyb ve třech rozměrech, rovnováha ve třech rozměrech

\item {} 
\sphinxAtStartPar
Vektorový a maticový zápis rovnic rovnováhy a rovnic pohybu

\end{itemize}

\item {} 
\sphinxAtStartPar
\sphinxstylestrong{Mechanika mechanismů}
\begin{itemize}
\item {} 
\sphinxAtStartPar
Transformační matice pro popis polohy a pohybu

\item {} 
\sphinxAtStartPar
Jednoduchý mechanismus a jeho analýza

\end{itemize}

\item {} 
\sphinxAtStartPar
\sphinxstylestrong{Prostorová napjatost}
\begin{itemize}
\item {} 
\sphinxAtStartPar
Hlavní napětí a deformace

\item {} 
\sphinxAtStartPar
Redukované napětí ve 3D

\item {} 
\sphinxAtStartPar
Kombinované namáhání – ohyb\sphinxhyphen{}ohyb, ohyb\sphinxhyphen{}tlak, ohyb\sphinxhyphen{}krut.

\end{itemize}

\item {} 
\sphinxAtStartPar
\sphinxstylestrong{Hydromechanika}
\begin{itemize}
\item {} 
\sphinxAtStartPar
Mechanika kapalin a plynů, hydrostatický paradox,

\item {} 
\sphinxAtStartPar
Laminární a turbulentní proudění, Reynoldsovo číslo, tok a rovnice kontinuity,

\item {} 
\sphinxAtStartPar
Ideální a reálná kapalina a její popis.

\item {} 
\sphinxAtStartPar
Ideální a reální plyn, stavová rovnice plynu

\end{itemize}

\item {} 
\sphinxAtStartPar
\sphinxstylestrong{Zákony termodynamiky}
\begin{itemize}
\item {} 
\sphinxAtStartPar
Definice entorpie na základě statistické mechaniky

\item {} 
\sphinxAtStartPar
Carnotův cyklus

\item {} 
\sphinxAtStartPar
Tepelné stroje, účinnost

\end{itemize}

\end{enumerate}


\chapter{Co neobsahuje tento kurz}
\label{\detokenize{Prednasky/0_1_Definice_a_historie_mechaniky:co-neobsahuje-tento-kurz}}
\sphinxAtStartPar
\sphinxstylestrong{Relativita:} Pro fyziky zahrnuje klasická mechanika také speciální a obecnou teorii relativity. Pro fyzika slovo „klasický“ znamená „deterministický“ nebo „nekvantový“, takže relativita, i když přišla stovky let po Newtonovi, se stále nazývá klasickou. Zde však klasická mechanika znamená mechaniku tak, jak ji chápali Newton a Euler. Relativita zde není diskutována.

\sphinxAtStartPar
\sphinxstylestrong{Teorie ohybu a kroucení:} V průběhu semestru se sice naučíme určovat průbeh ohybových momentů, ale nediskutujeme vztah mezi zakřivením a momentem v nosnících ani Mohrův integrál.

\sphinxAtStartPar
\sphinxstylestrong{Inženýrská termodynamika a hydrodynamika} je popsána jenom velice jednoduše bez uvedení přesného inženýrského popisů a aplikací.


\chapter{Doporučená literatura}
\label{\detokenize{Prednasky/0_1_Definice_a_historie_mechaniky:doporucena-literatura}}\begin{quote}

\sphinxAtStartPar
\sphinxstylestrong{Upozornění}: Seznam literatury bude doplňován v průběhu semestru
\end{quote}
\begin{itemize}
\item {} 
\sphinxAtStartPar
\sphinxhref{https://ocw.mit.edu/courses/8-01sc-classical-mechanics-fall-2016/pages/online-textbook/}{Classical Mechanics}, online učebnice mechaniky, MIT

\item {} 
\sphinxAtStartPar
\sphinxhref{http://ruina.tam.cornell.edu/Book/index.html}{Andy Ruina and Rudra Pratap: Introduction to Statics and Dynamics}, volně dostupná učebnice s propracovanými příklady

\item {} 
\sphinxAtStartPar
\sphinxhref{https://www.lightandmatter.com/mechanics/}{Benjamin Crowel: Mechanics} opakování středoškolské mechaniky s jemným rozšířením

\end{itemize}

\sphinxstepscope


\chapter{Skalár a vektor}
\label{\detokenize{Prednasky/0_2_Skal_xe1ry_a_vektory:skalar-a-vektor}}\label{\detokenize{Prednasky/0_2_Skal_xe1ry_a_vektory::doc}}
\sphinxAtStartPar
připraveno na základě podkladů \sphinxhref{https://nbviewer.org/github/BMClab/BMC/blob/master/notebooks/ScalarVector.ipynb}{Marcos Duarte, Renato Naville Watanabe, Laboratory of Biomechanics and Motor Control}

\sphinxAtStartPar
Python velmi dobře zpracovává všechny matematické operace s číselnými skaláry a vektory a pro podobné věci můžete použít \sphinxhref{http://sympy.org}{Sympy}, ale s abstraktními symboly. Podívejme se stručně skaláry a vektory a ukážeme, jak používat Python pro numerický výpočet.

\sphinxAtStartPar
Přehled o skalárech a vektorech viz kapitola 2 \sphinxhref{http://ruina.tam.cornell.edu/Book/index.html}{Andy Ruina and Rudra Pratap: Introduction to Statics and Dynamics}.


\section{Nastavení Pythonu}
\label{\detokenize{Prednasky/0_2_Skal_xe1ry_a_vektory:nastaveni-pythonu}}
\begin{sphinxuseclass}{cell}\begin{sphinxVerbatimInput}

\begin{sphinxuseclass}{cell_input}
\begin{sphinxVerbatim}[commandchars=\\\{\}]
\PYG{k+kn}{from} \PYG{n+nn}{IPython}\PYG{n+nn}{.}\PYG{n+nn}{display} \PYG{k+kn}{import} \PYG{n}{IFrame}
\PYG{k+kn}{import} \PYG{n+nn}{math}
\PYG{k+kn}{import} \PYG{n+nn}{numpy} \PYG{k}{as} \PYG{n+nn}{np}
\end{sphinxVerbatim}

\end{sphinxuseclass}\end{sphinxVerbatimInput}

\end{sphinxuseclass}

\section{Skalární veličina}
\label{\detokenize{Prednasky/0_2_Skal_xe1ry_a_vektory:skalarni-velicina}}\begin{quote}

\sphinxAtStartPar
\sphinxstylestrong{Skalární veličina} je fyzikální veličina, která je plně určena jediným číselným údajem a jednotkou. To znamená, že k jejímu popisu stačí uvést, jaká je její velikost. Na rozdíl od vektorových veličin, které mají kromě velikosti také směr, skaláry nemají žádné směrové vlastnosti.
\end{quote}

\sphinxAtStartPar
\sphinxstylestrong{Příklady skalárních veličin:}
\begin{itemize}
\item {} 
\sphinxAtStartPar
Délka: Například délka tužky je 15 cm.

\item {} 
\sphinxAtStartPar
Hmotnost: Hmotnost knihy je 500 gramů.

\item {} 
\sphinxAtStartPar
Čas: Doba trvání filmu je 2 hodiny.

\item {} 
\sphinxAtStartPar
Teplota: Teplota vody je 20 stupňů Celsia.

\item {} 
\sphinxAtStartPar
Energie: Kinetická energie pohybujícího se tělesa je 100 Joule.

\end{itemize}

\sphinxAtStartPar
\sphinxstylestrong{Grafické znázornění skalárů}
Skaláry se obvykle graficky znázorňují pouze jako čísla s příslušnou jednotkou. Na rozdíl od vektorů, které se znázorňují jako úsečky s orientací, skaláry nemají žádný směr, a proto je jejich grafické znázornění jednoduché.


\subsection{Skalární operace v Pythonu}
\label{\detokenize{Prednasky/0_2_Skal_xe1ry_a_vektory:skalarni-operace-v-pythonu}}
\sphinxAtStartPar
Jednoduché aritmetické operace se skaláry jsou skutečně jednoduché:

\begin{sphinxuseclass}{cell}\begin{sphinxVerbatimInput}

\begin{sphinxuseclass}{cell_input}
\begin{sphinxVerbatim}[commandchars=\\\{\}]
\PYG{n}{a} \PYG{o}{=} \PYG{l+m+mi}{2}
\PYG{n}{b} \PYG{o}{=} \PYG{l+m+mi}{5}
\PYG{n+nb}{print}\PYG{p}{(}\PYG{l+s+s1}{\PYGZsq{}}\PYG{l+s+s1}{a =}\PYG{l+s+s1}{\PYGZsq{}}\PYG{p}{,} \PYG{n}{a}\PYG{p}{,} \PYG{l+s+s1}{\PYGZsq{}}\PYG{l+s+s1}{, b =}\PYG{l+s+s1}{\PYGZsq{}}\PYG{p}{,} \PYG{n}{b}\PYG{p}{)}
\PYG{n+nb}{print}\PYG{p}{(}\PYG{l+s+s1}{\PYGZsq{}}\PYG{l+s+s1}{a + b =}\PYG{l+s+s1}{\PYGZsq{}}\PYG{p}{,} \PYG{n}{a} \PYG{o}{+} \PYG{n}{b}\PYG{p}{)}
\PYG{n+nb}{print}\PYG{p}{(}\PYG{l+s+s1}{\PYGZsq{}}\PYG{l+s+s1}{a \PYGZhy{} b =}\PYG{l+s+s1}{\PYGZsq{}}\PYG{p}{,} \PYG{n}{a} \PYG{o}{\PYGZhy{}} \PYG{n}{b}\PYG{p}{)}
\PYG{n+nb}{print}\PYG{p}{(}\PYG{l+s+s1}{\PYGZsq{}}\PYG{l+s+s1}{a * b =}\PYG{l+s+s1}{\PYGZsq{}}\PYG{p}{,} \PYG{n}{a} \PYG{o}{*} \PYG{n}{b}\PYG{p}{)}
\PYG{n+nb}{print}\PYG{p}{(}\PYG{l+s+s1}{\PYGZsq{}}\PYG{l+s+s1}{a / b =}\PYG{l+s+s1}{\PYGZsq{}}\PYG{p}{,} \PYG{n}{a} \PYG{o}{/} \PYG{n}{b}\PYG{p}{)}
\PYG{n+nb}{print}\PYG{p}{(}\PYG{l+s+s1}{\PYGZsq{}}\PYG{l+s+s1}{a ** b =}\PYG{l+s+s1}{\PYGZsq{}}\PYG{p}{,} \PYG{n}{a} \PYG{o}{*}\PYG{o}{*} \PYG{n}{b}\PYG{p}{)}
\PYG{n+nb}{print}\PYG{p}{(}\PYG{l+s+s1}{\PYGZsq{}}\PYG{l+s+s1}{sqrt(b) =}\PYG{l+s+s1}{\PYGZsq{}}\PYG{p}{,} \PYG{n}{math}\PYG{o}{.}\PYG{n}{sqrt}\PYG{p}{(}\PYG{n}{b}\PYG{p}{)}\PYG{p}{)}
\end{sphinxVerbatim}

\end{sphinxuseclass}\end{sphinxVerbatimInput}
\begin{sphinxVerbatimOutput}

\begin{sphinxuseclass}{cell_output}
\begin{sphinxVerbatim}[commandchars=\\\{\}]
a = 2 , b = 5
a + b = 7
a \PYGZhy{} b = \PYGZhy{}3
a * b = 10
a / b = 0.4
a ** b = 32
sqrt(b) = 2.23606797749979
\end{sphinxVerbatim}

\end{sphinxuseclass}\end{sphinxVerbatimOutput}

\end{sphinxuseclass}
\sphinxAtStartPar
Pokud máte sadu čísel nebo pole, je pravděpodobně lepší použít NUMPY; Bude to rychlejší pro velké soubory dat a v kombinaci se Scipy, má mnohem více matematických funkcí.

\begin{sphinxuseclass}{cell}\begin{sphinxVerbatimInput}

\begin{sphinxuseclass}{cell_input}
\begin{sphinxVerbatim}[commandchars=\\\{\}]
\PYG{n}{a} \PYG{o}{=} \PYG{l+m+mi}{2}
\PYG{n}{b} \PYG{o}{=} \PYG{p}{[}\PYG{l+m+mi}{3}\PYG{p}{,} \PYG{l+m+mi}{4}\PYG{p}{,} \PYG{l+m+mi}{5}\PYG{p}{,} \PYG{l+m+mi}{6}\PYG{p}{,} \PYG{l+m+mi}{7}\PYG{p}{,} \PYG{l+m+mi}{8}\PYG{p}{]}
\PYG{n}{b} \PYG{o}{=} \PYG{n}{np}\PYG{o}{.}\PYG{n}{array}\PYG{p}{(}\PYG{n}{b}\PYG{p}{)}
\PYG{n+nb}{print}\PYG{p}{(}\PYG{l+s+s1}{\PYGZsq{}}\PYG{l+s+s1}{a =}\PYG{l+s+s1}{\PYGZsq{}}\PYG{p}{,} \PYG{n}{a}\PYG{p}{,} \PYG{l+s+s1}{\PYGZsq{}}\PYG{l+s+s1}{, b =}\PYG{l+s+s1}{\PYGZsq{}}\PYG{p}{,} \PYG{n}{b}\PYG{p}{)}
\PYG{n+nb}{print}\PYG{p}{(}\PYG{l+s+s1}{\PYGZsq{}}\PYG{l+s+s1}{a + b =}\PYG{l+s+s1}{\PYGZsq{}}\PYG{p}{,} \PYG{n}{a} \PYG{o}{+} \PYG{n}{b}\PYG{p}{)}
\PYG{n+nb}{print}\PYG{p}{(}\PYG{l+s+s1}{\PYGZsq{}}\PYG{l+s+s1}{a \PYGZhy{} b =}\PYG{l+s+s1}{\PYGZsq{}}\PYG{p}{,} \PYG{n}{a} \PYG{o}{\PYGZhy{}} \PYG{n}{b}\PYG{p}{)}
\PYG{n+nb}{print}\PYG{p}{(}\PYG{l+s+s1}{\PYGZsq{}}\PYG{l+s+s1}{a * b =}\PYG{l+s+s1}{\PYGZsq{}}\PYG{p}{,} \PYG{n}{a} \PYG{o}{*} \PYG{n}{b}\PYG{p}{)}
\PYG{n+nb}{print}\PYG{p}{(}\PYG{l+s+s1}{\PYGZsq{}}\PYG{l+s+s1}{a / b =}\PYG{l+s+s1}{\PYGZsq{}}\PYG{p}{,} \PYG{n}{a} \PYG{o}{/} \PYG{n}{b}\PYG{p}{)}
\PYG{n+nb}{print}\PYG{p}{(}\PYG{l+s+s1}{\PYGZsq{}}\PYG{l+s+s1}{a ** b =}\PYG{l+s+s1}{\PYGZsq{}}\PYG{p}{,} \PYG{n}{a} \PYG{o}{*}\PYG{o}{*} \PYG{n}{b}\PYG{p}{)}
\PYG{n+nb}{print}\PYG{p}{(}\PYG{l+s+s1}{\PYGZsq{}}\PYG{l+s+s1}{np.sqrt(b) =}\PYG{l+s+s1}{\PYGZsq{}}\PYG{p}{,} \PYG{n}{np}\PYG{o}{.}\PYG{n}{sqrt}\PYG{p}{(}\PYG{n}{b}\PYG{p}{)}\PYG{p}{)}  \PYG{c+c1}{\PYGZsh{} Použijte funkce numpy pro numpy pole}
\end{sphinxVerbatim}

\end{sphinxuseclass}\end{sphinxVerbatimInput}
\begin{sphinxVerbatimOutput}

\begin{sphinxuseclass}{cell_output}
\begin{sphinxVerbatim}[commandchars=\\\{\}]
a = 2 , b = [3 4 5 6 7 8]
a + b = [ 5  6  7  8  9 10]
a \PYGZhy{} b = [\PYGZhy{}1 \PYGZhy{}2 \PYGZhy{}3 \PYGZhy{}4 \PYGZhy{}5 \PYGZhy{}6]
a * b = [ 6  8 10 12 14 16]
a / b = [0.66666667 0.5        0.4        0.33333333 0.28571429 0.25      ]
a ** b = [  8  16  32  64 128 256]
np.sqrt(b) = [1.73205081 2.         2.23606798 2.44948974 2.64575131 2.82842712]
\end{sphinxVerbatim}

\end{sphinxuseclass}\end{sphinxVerbatimOutput}

\end{sphinxuseclass}
\sphinxAtStartPar
Numpy provádí aritmetické operace jediného čísla v \sphinxcode{\sphinxupquote{a}} se všemi čísly pole\sphinxcode{\sphinxupquote{ b}}. Tomu se říká v informatice pojmem \sphinxstyleemphasis{broadcasting}.
I když máte dvě pole (ale musí mít stejnou velikost), Numpy pravidla platí pro vás:

\begin{sphinxuseclass}{cell}\begin{sphinxVerbatimInput}

\begin{sphinxuseclass}{cell_input}
\begin{sphinxVerbatim}[commandchars=\\\{\}]
\PYG{n}{a} \PYG{o}{=} \PYG{n}{np}\PYG{o}{.}\PYG{n}{array}\PYG{p}{(}\PYG{p}{[}\PYG{l+m+mi}{1}\PYG{p}{,} \PYG{l+m+mi}{2}\PYG{p}{,} \PYG{l+m+mi}{3}\PYG{p}{]}\PYG{p}{)}
\PYG{n}{b} \PYG{o}{=} \PYG{n}{np}\PYG{o}{.}\PYG{n}{array}\PYG{p}{(}\PYG{p}{[}\PYG{l+m+mi}{4}\PYG{p}{,} \PYG{l+m+mi}{5}\PYG{p}{,} \PYG{l+m+mi}{6}\PYG{p}{]}\PYG{p}{)}
\PYG{n+nb}{print}\PYG{p}{(}\PYG{l+s+s1}{\PYGZsq{}}\PYG{l+s+s1}{a =}\PYG{l+s+s1}{\PYGZsq{}}\PYG{p}{,} \PYG{n}{a}\PYG{p}{,} \PYG{l+s+s1}{\PYGZsq{}}\PYG{l+s+s1}{, b =}\PYG{l+s+s1}{\PYGZsq{}}\PYG{p}{,} \PYG{n}{b}\PYG{p}{)}
\PYG{n+nb}{print}\PYG{p}{(}\PYG{l+s+s1}{\PYGZsq{}}\PYG{l+s+s1}{a + b =}\PYG{l+s+s1}{\PYGZsq{}}\PYG{p}{,} \PYG{n}{a} \PYG{o}{+} \PYG{n}{b}\PYG{p}{)}
\PYG{n+nb}{print}\PYG{p}{(}\PYG{l+s+s1}{\PYGZsq{}}\PYG{l+s+s1}{a \PYGZhy{} b =}\PYG{l+s+s1}{\PYGZsq{}}\PYG{p}{,} \PYG{n}{a} \PYG{o}{\PYGZhy{}} \PYG{n}{b}\PYG{p}{)}
\PYG{n+nb}{print}\PYG{p}{(}\PYG{l+s+s1}{\PYGZsq{}}\PYG{l+s+s1}{a * b =}\PYG{l+s+s1}{\PYGZsq{}}\PYG{p}{,} \PYG{n}{a} \PYG{o}{*} \PYG{n}{b}\PYG{p}{)}
\PYG{n+nb}{print}\PYG{p}{(}\PYG{l+s+s1}{\PYGZsq{}}\PYG{l+s+s1}{a / b =}\PYG{l+s+s1}{\PYGZsq{}}\PYG{p}{,} \PYG{n}{a} \PYG{o}{/} \PYG{n}{b}\PYG{p}{)}
\PYG{n+nb}{print}\PYG{p}{(}\PYG{l+s+s1}{\PYGZsq{}}\PYG{l+s+s1}{a ** b =}\PYG{l+s+s1}{\PYGZsq{}}\PYG{p}{,} \PYG{n}{a} \PYG{o}{*}\PYG{o}{*} \PYG{n}{b}\PYG{p}{)}
\end{sphinxVerbatim}

\end{sphinxuseclass}\end{sphinxVerbatimInput}
\begin{sphinxVerbatimOutput}

\begin{sphinxuseclass}{cell_output}
\begin{sphinxVerbatim}[commandchars=\\\{\}]
a = [1 2 3] , b = [4 5 6]
a + b = [5 7 9]
a \PYGZhy{} b = [\PYGZhy{}3 \PYGZhy{}3 \PYGZhy{}3]
a * b = [ 4 10 18]
a / b = [0.25 0.4  0.5 ]
a ** b = [  1  32 729]
\end{sphinxVerbatim}

\end{sphinxuseclass}\end{sphinxVerbatimOutput}

\end{sphinxuseclass}

\section{Extenzivní a intenzivní veličiny}
\label{\detokenize{Prednasky/0_2_Skal_xe1ry_a_vektory:extenzivni-a-intenzivni-veliciny}}
\sphinxAtStartPar
V termodynamice a fyzice obecně rozlišujeme dva základní typy veličin: extenzivní a intenzivní. Toto rozdělení je důležité pro pochopení chování systémů a jejich vlastností.


\subsection{Extenzivní veličiny}
\label{\detokenize{Prednasky/0_2_Skal_xe1ry_a_vektory:extenzivni-veliciny}}\begin{itemize}
\item {} 
\sphinxAtStartPar
\sphinxstylestrong{Definice:} Extenzivní veličiny jsou takové, jejichž hodnota závisí na velikosti systému, tedy na množství látky nebo hmotnosti. Pokud zdvojnásobíme množství látky v systému, zdvojnásobí se i hodnota extenzivní veličiny.

\item {} 
\sphinxAtStartPar
\sphinxstylestrong{Příklady:}
\begin{itemize}
\item {} 
\sphinxAtStartPar
\sphinxstylestrong{Hmotnost (m):} Celková hmotnost tělesa.

\item {} 
\sphinxAtStartPar
\sphinxstylestrong{Objem (V):} Prostor, který těleso zabírá.

\item {} 
\sphinxAtStartPar
\sphinxstylestrong{Energie (E):} Celková energie systému (vnitřní energie, kinetická energie, potenciální energie).

\item {} 
\sphinxAtStartPar
\sphinxstylestrong{Entropie (S):} Míra neuspořádanosti systému.

\item {} 
\sphinxAtStartPar
\sphinxstylestrong{Náboj (Q):} Celkový elektrický náboj.

\item {} 
\sphinxAtStartPar
\sphinxstylestrong{Délka (l):} Rozměr tělesa.

\end{itemize}

\end{itemize}


\subsection{Intenzivní veličiny}
\label{\detokenize{Prednasky/0_2_Skal_xe1ry_a_vektory:intenzivni-veliciny}}\begin{itemize}
\item {} 
\sphinxAtStartPar
\sphinxstylestrong{Definice:} Intenzivní veličiny jsou takové, jejichž hodnota nezávisí na velikosti systému. Zůstávají stejné, i když změníme množství látky v systému.

\item {} 
\sphinxAtStartPar
\sphinxstylestrong{Příklady:}
\begin{itemize}
\item {} 
\sphinxAtStartPar
\sphinxstylestrong{Teplota (T):} Míra horkosti nebo chladu.

\item {} 
\sphinxAtStartPar
\sphinxstylestrong{Tlak (p):} Síla působící na jednotku plochy.

\item {} 
\sphinxAtStartPar
\sphinxstylestrong{Hustota (ρ):} Hmotnost na jednotku objemu.

\item {} 
\sphinxAtStartPar
\sphinxstylestrong{Měrná tepelná kapacita (c):} Množství tepla potřebné k ohřátí jednotky hmotnosti o 1 stupeň.

\item {} 
\sphinxAtStartPar
\sphinxstylestrong{Koncentrace (c):} Množství látky v daném objemu.

\end{itemize}

\end{itemize}


\subsection{Vztah mezi extenzivními a intenzivními veličinami}
\label{\detokenize{Prednasky/0_2_Skal_xe1ry_a_vektory:vztah-mezi-extenzivnimi-a-intenzivnimi-velicinami}}\begin{itemize}
\item {} 
\sphinxAtStartPar
Poměr dvou extenzivních veličin dává často intenzivní veličinu (např. hustota = hmotnost/objem).

\item {} 
\sphinxAtStartPar
Intenzivní veličiny jsou “kvalitativní” vlastnosti, zatímco extenzivní veličiny jsou “kvantitativní”.

\end{itemize}


\subsection{Praktický význam}
\label{\detokenize{Prednasky/0_2_Skal_xe1ry_a_vektory:prakticky-vyznam}}
\sphinxAtStartPar
Rozlišení mezi extenzivními a intenzivními veličinami je klíčové pro správné pochopení a popis termodynamických systémů a procesů. Umožňuje nám například:
\begin{itemize}
\item {} 
\sphinxAtStartPar
Vytvářet termodynamické modely a rovnice.

\item {} 
\sphinxAtStartPar
Porovnávat vlastnosti různých látek.

\item {} 
\sphinxAtStartPar
Předpovídat chování systémů při změnách podmínek.

\end{itemize}


\section{Vektor}
\label{\detokenize{Prednasky/0_2_Skal_xe1ry_a_vektory:vektor}}\begin{quote}

\sphinxAtStartPar
\sphinxstylestrong{Vektorová veličina} je fyzikální veličina, která je charakterizována nejen svou velikostí, ale také směrem.
\end{quote}

\sphinxAtStartPar
Co to znamená?

\sphinxAtStartPar
\sphinxstylestrong{Velikost}: Určuje “intenzitu” nebo “množství” této veličiny. Například rychlost 50 km/h udává velikost rychlosti.
\sphinxstylestrong{Směr}: Určuje, kam veličina “směřuje” nebo “působí”. Například rychlost 50 km/h na východ udává jak velikost, tak i směr pohybu.
Grafické znázornění:

\sphinxAtStartPar
Vektor se graficky znázorňuje jako \sphinxstylestrong{orientovaná úsečka}.

\sphinxAtStartPar
Délka úsečky představuje velikost vektoru.
Směr úsečky (a šipka na jejím konci) představuje směr vektoru.

\sphinxAtStartPar
Symboly pro vektory jsou obvykle v knihách tištěny tučně, jako \sphinxstylestrong{a}. Mezi další zvyklosti označování patří \(\vec {a}\). Aby jsme předešli nedorozumění budeme v našem kurzu označovat vektory \(\vec{\mathbf{a}}\).

\sphinxAtStartPar
Například zvažte polohu bodu v prostoru představovaném vektorem:

\sphinxAtStartPar


\sphinxAtStartPar
Pozice bodu představovaného vektorem v karteziánském souřadném systému.

\sphinxAtStartPar
Poloha bodu (vektor) výše může být reprezentována jako n \sphinxhyphen{}tice hodnot:
\begin{equation*}
\begin{split} (x,\: y,\: z) \; \Rightarrow \; (5, 3, -2) \end{split}
\end{equation*}
\sphinxAtStartPar
nebo ve formě matice:
\begin{equation*}
\begin{split} \begin{bmatrix} x \\y \\z \end{bmatrix} \;\; \Rightarrow  \;\; \begin{bmatrix} 1 \\3 \\2 \end{bmatrix}\end{split}
\end{equation*}
\sphinxAtStartPar
Můžeme použít pole Numpy k reprezentaci komponent vektorů.
Například pro výše uvedený vektor je vyjádřen v Pythonu jako:

\begin{sphinxuseclass}{cell}\begin{sphinxVerbatimInput}

\begin{sphinxuseclass}{cell_input}
\begin{sphinxVerbatim}[commandchars=\\\{\}]
\PYG{n}{a} \PYG{o}{=} \PYG{n}{np}\PYG{o}{.}\PYG{n}{array}\PYG{p}{(}\PYG{p}{[}\PYG{l+m+mi}{1}\PYG{p}{,} \PYG{l+m+mi}{3}\PYG{p}{,} \PYG{l+m+mi}{2}\PYG{p}{]}\PYG{p}{)}
\PYG{n+nb}{print}\PYG{p}{(}\PYG{l+s+s1}{\PYGZsq{}}\PYG{l+s+s1}{a =}\PYG{l+s+s1}{\PYGZsq{}}\PYG{p}{,} \PYG{n}{a}\PYG{p}{)}
\end{sphinxVerbatim}

\end{sphinxuseclass}\end{sphinxVerbatimInput}
\begin{sphinxVerbatimOutput}

\begin{sphinxuseclass}{cell_output}
\begin{sphinxVerbatim}[commandchars=\\\{\}]
a = [1 3 2]
\end{sphinxVerbatim}

\end{sphinxuseclass}\end{sphinxVerbatimOutput}

\end{sphinxuseclass}
\sphinxAtStartPar
Vektor může být také reprezentován jako:
\$\( \overrightarrow{\mathbf{a}} = a_x\hat{\mathbf{i}} + a_y\hat{\mathbf{j}} + a_z\hat{\mathbf{k}} \)\$

\sphinxAtStartPar
kde vektory  \( \vec{\mathbf{i}}, \, \vec{\mathbf{j}}, \, \vec{\mathbf{k}} \, \) jsou jednotkové vektory ve směru jednotlivých souřadnicových os a jsou obvykle reprezentovány v barevné sekvenci \sphinxstylestrong{rgb} (červená, zelená, modrá) pro snadnější vizualizaci. Násobek \(a_x \vec{\mathbf{i}}, \: a_y \vec{\mathbf{j}}, \: a_z \vec{\mathbf{k}} \) jsou vektorové komponenty vektoru  \(\vec{\mathbf{a}}\) .

\sphinxAtStartPar
Jednotkový vektor (můžete se setkat také s označením verzor) je vektor, jehož délka (nebo norma) je 1.

\begin{sphinxuseclass}{cell}\begin{sphinxVerbatimInput}

\begin{sphinxuseclass}{cell_input}
\begin{sphinxVerbatim}[commandchars=\\\{\}]
\PYG{n}{IFrame}\PYG{p}{(}\PYG{l+s+s1}{\PYGZsq{}}\PYG{l+s+s1}{https://www.geogebra.org/classic/ydu8a7t7?embed}\PYG{l+s+s1}{\PYGZsq{}}\PYG{p}{,} \PYG{n}{width}\PYG{o}{=}\PYG{l+s+s1}{\PYGZsq{}}\PYG{l+s+s1}{100}\PYG{l+s+s1}{\PYGZpc{}}\PYG{l+s+s1}{\PYGZsq{}}\PYG{p}{,} \PYG{n}{height}\PYG{o}{=}\PYG{l+m+mi}{500}\PYG{p}{)}
\end{sphinxVerbatim}

\end{sphinxuseclass}\end{sphinxVerbatimInput}
\begin{sphinxVerbatimOutput}

\begin{sphinxuseclass}{cell_output}
\begin{sphinxVerbatim}[commandchars=\\\{\}]
\PYGZlt{}IPython.lib.display.IFrame at 0x745388114d00\PYGZgt{}
\end{sphinxVerbatim}

\end{sphinxuseclass}\end{sphinxVerbatimOutput}

\end{sphinxuseclass}

\section{Násobení vektoru skalárem}
\label{\detokenize{Prednasky/0_2_Skal_xe1ry_a_vektory:nasobeni-vektoru-skalarem}}
\sphinxAtStartPar
U složek vektoru \({a_x}\vec{\mathbf{i}}\) jsme již využili násobení vektoru \(\vec{\mathbf{i}}\) číslem \(a_x\).
\begin{quote}

\sphinxAtStartPar
Násobení vektoru skalárem je operace, která mění velikost vektoru, ale ne jeho směr. Můžeme si to představit jako “natažení” nebo “zkrácení” vektoru.
\end{quote}

\sphinxAtStartPar
\sphinxstylestrong{Jak to funguje}:
\begin{itemize}
\item {} 
\sphinxAtStartPar
\sphinxstylestrong{Zvětšení:} Když násobíme vektor kladným skalárem (číslem větším než 1), zvětšíme jeho velikost. Je to jako bychom vektor “natahovali”.

\item {} 
\sphinxAtStartPar
\sphinxstylestrong{Zmenšení:} Když násobíme vektor kladným skalárem mezi 0 a 1, zmenšíme jeho velikost. Je to jako bychom vektor “zkracovali”.

\item {} 
\sphinxAtStartPar
\sphinxstylestrong{Změna směru:} Když násobíme vektor záporným skalárem, změníme jeho směr na opačný. Je to jako bychom vektor “otočili” o 180 stupňů. Velikost vektoru se přitom může také změnit v závislosti na absolutní hodnotě skaláru.

\end{itemize}

\begin{sphinxuseclass}{cell}\begin{sphinxVerbatimInput}

\begin{sphinxuseclass}{cell_input}
\begin{sphinxVerbatim}[commandchars=\\\{\}]
\PYG{n}{a} \PYG{o}{=} \PYG{n}{np}\PYG{o}{.}\PYG{n}{array}\PYG{p}{(}\PYG{p}{[}\PYG{l+m+mi}{2}\PYG{p}{,}\PYG{l+m+mi}{1}\PYG{p}{]}\PYG{p}{)}
\PYG{n}{b} \PYG{o}{=} \PYG{l+m+mi}{2}
\PYG{n}{c} \PYG{o}{=} \PYG{l+m+mf}{0.5}
\PYG{n}{d} \PYG{o}{=} \PYG{o}{\PYGZhy{}}\PYG{l+m+mi}{1}
\PYG{n+nb}{print}\PYG{p}{(}\PYG{l+s+s1}{\PYGZsq{}}\PYG{l+s+s1}{a * b =}\PYG{l+s+s1}{\PYGZsq{}}\PYG{p}{,} \PYG{n}{a}\PYG{o}{*} \PYG{n}{b}\PYG{p}{)}
\PYG{n+nb}{print}\PYG{p}{(}\PYG{l+s+s1}{\PYGZsq{}}\PYG{l+s+s1}{a * b =}\PYG{l+s+s1}{\PYGZsq{}}\PYG{p}{,} \PYG{n}{a} \PYG{o}{*} \PYG{n}{c}\PYG{p}{)}
\PYG{n+nb}{print}\PYG{p}{(}\PYG{l+s+s1}{\PYGZsq{}}\PYG{l+s+s1}{a * d =}\PYG{l+s+s1}{\PYGZsq{}}\PYG{p}{,} \PYG{n}{a} \PYG{o}{*} \PYG{n}{d}\PYG{p}{)}
\end{sphinxVerbatim}

\end{sphinxuseclass}\end{sphinxVerbatimInput}
\begin{sphinxVerbatimOutput}

\begin{sphinxuseclass}{cell_output}
\begin{sphinxVerbatim}[commandchars=\\\{\}]
a * b = [4 2]
a * b = [1.  0.5]
a * d = [\PYGZhy{}2 \PYGZhy{}1]
\end{sphinxVerbatim}

\end{sphinxuseclass}\end{sphinxVerbatimOutput}

\end{sphinxuseclass}
\begin{sphinxuseclass}{cell}\begin{sphinxVerbatimInput}

\begin{sphinxuseclass}{cell_input}
\begin{sphinxVerbatim}[commandchars=\\\{\}]
\PYG{n}{IFrame}\PYG{p}{(}\PYG{l+s+s1}{\PYGZsq{}}\PYG{l+s+s1}{https://www.geogebra.org/classic/HYZXHadK?embed}\PYG{l+s+s1}{\PYGZsq{}}\PYG{p}{,} \PYG{n}{width}\PYG{o}{=}\PYG{l+s+s1}{\PYGZsq{}}\PYG{l+s+s1}{100}\PYG{l+s+s1}{\PYGZpc{}}\PYG{l+s+s1}{\PYGZsq{}}\PYG{p}{,} \PYG{n}{height}\PYG{o}{=}\PYG{l+m+mi}{500}\PYG{p}{)}
\end{sphinxVerbatim}

\end{sphinxuseclass}\end{sphinxVerbatimInput}
\begin{sphinxVerbatimOutput}

\begin{sphinxuseclass}{cell_output}
\begin{sphinxVerbatim}[commandchars=\\\{\}]
\PYGZlt{}IPython.lib.display.IFrame at 0x745388116860\PYGZgt{}
\end{sphinxVerbatim}

\end{sphinxuseclass}\end{sphinxVerbatimOutput}

\end{sphinxuseclass}
\sphinxAtStartPar
Jestliže máme vektor \(\vec{{\mathbf{a}}}\), existuje k němu opačný vektor \(-\vec{{\mathbf{a}}}\). Pro tento vektor platí:
\$\(\vec{{\mathbf{a}}} + (-\vec{{\mathbf{a}}}) = \vec{{\mathbf{0}}}\)\(
kde \)\textbackslash{}vec\}\( je nulový vektor \)\textbackslash{}vec\} = (0,0,0)\$.

\sphinxAtStartPar
Násobení skalárem funguje v Pythonu jednoduše jako násobení pole. Protože však nyní jednáme s vektory, nyní některé operace nedávají smysl. Například pro dvojici vektorů neexistují žádné násobení, dělení, síla a druhou odmocninu ve způsobu, jakým jsme vypočítali.

\sphinxAtStartPar
Jednotkový vektor \(\vec{\mathbf{u}}\) nenulového vektoru \(\vec{\mathbf{a}}\) je jednotkový vektor se stejným směrem:
\begin{equation*}
\begin{split} \mathbf{\vec{u}} = \frac{\overrightarrow{\mathbf{a}}}{||\overrightarrow{\mathbf{a}}||} = \frac{a_x\,\hat{\mathbf{i}} + a_y\,\vec{\mathbf{j}} + a_z\, \vec{\mathbf{k}}}{\sqrt{a_x^2+a_y^2+a_z^2}} \end{split}
\end{equation*}

\subsection{Velikost (délka nebo norma) vektoru}
\label{\detokenize{Prednasky/0_2_Skal_xe1ry_a_vektory:velikost-delka-nebo-norma-vektoru}}
\sphinxAtStartPar
Velikost (délka) vektoru je často reprezentována symbolem \( || \; || \), také známou jako norma (nebo euklidovská norma) vektoru a je definována jako:
\$\( ||\overrightarrow{\mathbf{a}}|| = \sqrt{a_x^2+a_y^2+a_z^2} \)\$
Funkce \sphinxcode{\sphinxupquote{numpy.linalg.norm}} počítá normu:

\begin{sphinxuseclass}{cell}\begin{sphinxVerbatimInput}

\begin{sphinxuseclass}{cell_input}
\begin{sphinxVerbatim}[commandchars=\\\{\}]
\PYG{n}{a} \PYG{o}{=} \PYG{n}{np}\PYG{o}{.}\PYG{n}{array}\PYG{p}{(}\PYG{p}{[}\PYG{l+m+mi}{1}\PYG{p}{,} \PYG{l+m+mi}{2}\PYG{p}{,} \PYG{l+m+mi}{3}\PYG{p}{]}\PYG{p}{)}
\PYG{n}{np}\PYG{o}{.}\PYG{n}{linalg}\PYG{o}{.}\PYG{n}{norm}\PYG{p}{(}\PYG{n}{a}\PYG{p}{)}
\end{sphinxVerbatim}

\end{sphinxuseclass}\end{sphinxVerbatimInput}
\begin{sphinxVerbatimOutput}

\begin{sphinxuseclass}{cell_output}
\begin{sphinxVerbatim}[commandchars=\\\{\}]
3.7416573867739413
\end{sphinxVerbatim}

\end{sphinxuseclass}\end{sphinxVerbatimOutput}

\end{sphinxuseclass}
\sphinxAtStartPar
Nebo můžeme definici použít přímo a spočítat:

\begin{sphinxuseclass}{cell}\begin{sphinxVerbatimInput}

\begin{sphinxuseclass}{cell_input}
\begin{sphinxVerbatim}[commandchars=\\\{\}]
\PYG{n}{np}\PYG{o}{.}\PYG{n}{sqrt}\PYG{p}{(}\PYG{n}{np}\PYG{o}{.}\PYG{n}{sum}\PYG{p}{(}\PYG{n}{a}\PYG{o}{*}\PYG{n}{a}\PYG{p}{)}\PYG{p}{)}
\end{sphinxVerbatim}

\end{sphinxuseclass}\end{sphinxVerbatimInput}
\begin{sphinxVerbatimOutput}

\begin{sphinxuseclass}{cell_output}
\begin{sphinxVerbatim}[commandchars=\\\{\}]
3.7416573867739413
\end{sphinxVerbatim}

\end{sphinxuseclass}\end{sphinxVerbatimOutput}

\end{sphinxuseclass}
\sphinxAtStartPar
Poté, verzor (jedotkový vektor) pro vektor \( \overrightarrow{\mathbf{a}} = (1, 2, 3)\) je:

\begin{sphinxuseclass}{cell}\begin{sphinxVerbatimInput}

\begin{sphinxuseclass}{cell_input}
\begin{sphinxVerbatim}[commandchars=\\\{\}]
\PYG{n}{a} \PYG{o}{=} \PYG{n}{np}\PYG{o}{.}\PYG{n}{array}\PYG{p}{(}\PYG{p}{[}\PYG{l+m+mi}{1}\PYG{p}{,} \PYG{l+m+mi}{2}\PYG{p}{,} \PYG{l+m+mi}{3}\PYG{p}{]}\PYG{p}{)}
\PYG{n}{u} \PYG{o}{=} \PYG{n}{a}\PYG{o}{/}\PYG{n}{np}\PYG{o}{.}\PYG{n}{linalg}\PYG{o}{.}\PYG{n}{norm}\PYG{p}{(}\PYG{n}{a}\PYG{p}{)}
\PYG{n+nb}{print}\PYG{p}{(}\PYG{l+s+s1}{\PYGZsq{}}\PYG{l+s+s1}{u =}\PYG{l+s+s1}{\PYGZsq{}}\PYG{p}{,} \PYG{n}{u}\PYG{p}{)}
\end{sphinxVerbatim}

\end{sphinxuseclass}\end{sphinxVerbatimInput}
\begin{sphinxVerbatimOutput}

\begin{sphinxuseclass}{cell_output}
\begin{sphinxVerbatim}[commandchars=\\\{\}]
u = [0.26726124 0.53452248 0.80178373]
\end{sphinxVerbatim}

\end{sphinxuseclass}\end{sphinxVerbatimOutput}

\end{sphinxuseclass}
\sphinxAtStartPar
A můžeme ověřit, že jeho velikost je skutečně 1:

\begin{sphinxuseclass}{cell}\begin{sphinxVerbatimInput}

\begin{sphinxuseclass}{cell_input}
\begin{sphinxVerbatim}[commandchars=\\\{\}]
\PYG{n}{np}\PYG{o}{.}\PYG{n}{linalg}\PYG{o}{.}\PYG{n}{norm}\PYG{p}{(}\PYG{n}{u}\PYG{p}{)}
\end{sphinxVerbatim}

\end{sphinxuseclass}\end{sphinxVerbatimInput}
\begin{sphinxVerbatimOutput}

\begin{sphinxuseclass}{cell_output}
\begin{sphinxVerbatim}[commandchars=\\\{\}]
1.0
\end{sphinxVerbatim}

\end{sphinxuseclass}\end{sphinxVerbatimOutput}

\end{sphinxuseclass}
\sphinxAtStartPar
Ale reprezentace vektoru jako n\sphinxhyphen{}tice hodnot je platná pouze pro vektor s jeho působištěm, který se shoduje s počátkem souřadnicového systému \( (0, 0, 0) \).
Například zvažte následující vektor:
\sphinxincludegraphics{{6}.png}

\sphinxAtStartPar
Takový vektor nelze reprezentovat \((B_x, B_y, B_z)\), protože by to bylo pro vektor od původu do bodu B., aby přesně reprezentoval tento vektor, který potřebujeme dva vektory \(\vec{OB}\) a \(\vec{OA}\). Tato skutečnost je důležitá, když provádíme některé výpočty v mechanice.


\subsection{Sčítání a odečítání vektorů}
\label{\detokenize{Prednasky/0_2_Skal_xe1ry_a_vektory:scitani-a-odecitani-vektoru}}
\sphinxAtStartPar
Sčítáním dvou vektorů získáme další vektor:
\$\( \overrightarrow{\mathbf{a}} + \overrightarrow{\mathbf{b}} = (a_x\vec{\mathbf{i}} + a_y\vec{\mathbf{j}} + a_z\vec{\mathbf{k}}) + (b_x\vec{\mathbf{i}} + b_y\vec{\mathbf{j}} + b_z\vec{\mathbf{k}}) = 
(a_x+b_x)\vec{\mathbf{i}} + (a_y+b_y)\vec{\mathbf{j}} + (a_z+b_z)\vec{\mathbf{k}} \)\$

\sphinxAtStartPar
V maticovém zápisu to bude vypadat:
\$\(\begin{bmatrix} a_x \\ a_y \\ a_z \end{bmatrix}  +  \begin{bmatrix} b_x \\ b_y \\ b_z \end{bmatrix} = \begin{bmatrix} a_x + b_x \\ a_y + b_y \\ a_z + b_z \end{bmatrix}\)\$

\sphinxAtStartPar
Metoda rovnoběžníku je grafická metoda, která nám umožňuje názorně sčítat dva vektory.

\sphinxAtStartPar
\sphinxstylestrong{Postup:}
\begin{enumerate}
\sphinxsetlistlabels{\arabic}{enumi}{enumii}{}{.}%
\item {} 
\sphinxAtStartPar
Nakreslení vektorů: Oba vektory, které chceme sčítat, nakreslíme tak, aby jejich počáteční body splývaly.

\item {} 
\sphinxAtStartPar
Sestrojení rovnoběžníku: Z koncových bodů obou vektorů sestrojíme rovnoběžníky. Tj. z koncového bodu prvního vektoru sestrojíme přímku rovnoběžnou s druhým vektorem a naopak. Tím vznikne rovnoběžník.

\item {} 
\sphinxAtStartPar
Výsledný vektor: Úhlopříčka rovnoběžníku, která vychází z bodu, kde se oba vektory setkávají, představuje výsledný vektor (vektorový součet).

\end{enumerate}

\begin{sphinxuseclass}{cell}\begin{sphinxVerbatimInput}

\begin{sphinxuseclass}{cell_input}
\begin{sphinxVerbatim}[commandchars=\\\{\}]
\PYG{n}{IFrame}\PYG{p}{(}\PYG{l+s+s1}{\PYGZsq{}}\PYG{l+s+s1}{https://www.geogebra.org/classic/FCknj7c3?embed}\PYG{l+s+s1}{\PYGZsq{}}\PYG{p}{,} \PYG{n}{width}\PYG{o}{=}\PYG{l+s+s1}{\PYGZsq{}}\PYG{l+s+s1}{100}\PYG{l+s+s1}{\PYGZpc{}}\PYG{l+s+s1}{\PYGZsq{}}\PYG{p}{,} \PYG{n}{height}\PYG{o}{=}\PYG{l+m+mi}{500}\PYG{p}{)}
\end{sphinxVerbatim}

\end{sphinxuseclass}\end{sphinxVerbatimInput}
\begin{sphinxVerbatimOutput}

\begin{sphinxuseclass}{cell_output}
\begin{sphinxVerbatim}[commandchars=\\\{\}]
\PYGZlt{}IPython.lib.display.IFrame at 0x745388116920\PYGZgt{}
\end{sphinxVerbatim}

\end{sphinxuseclass}\end{sphinxVerbatimOutput}

\end{sphinxuseclass}
\sphinxAtStartPar
V prostoru metoda rovnoběžníku funguje stejně, ale musíme jeho konstrukci provést v rovině určené oběma vektory. Proto se v prostoru raděi spoléháme na výpočet

\begin{sphinxuseclass}{cell}\begin{sphinxVerbatimInput}

\begin{sphinxuseclass}{cell_input}
\begin{sphinxVerbatim}[commandchars=\\\{\}]
\PYG{n}{IFrame}\PYG{p}{(}\PYG{l+s+s1}{\PYGZsq{}}\PYG{l+s+s1}{https://www.geogebra.org/classic/gkhaxb2b?embed}\PYG{l+s+s1}{\PYGZsq{}}\PYG{p}{,} \PYG{n}{width}\PYG{o}{=}\PYG{l+s+s1}{\PYGZsq{}}\PYG{l+s+s1}{100}\PYG{l+s+s1}{\PYGZpc{}}\PYG{l+s+s1}{\PYGZsq{}}\PYG{p}{,} \PYG{n}{height}\PYG{o}{=}\PYG{l+m+mi}{500}\PYG{p}{)}
\end{sphinxVerbatim}

\end{sphinxuseclass}\end{sphinxVerbatimInput}
\begin{sphinxVerbatimOutput}

\begin{sphinxuseclass}{cell_output}
\begin{sphinxVerbatim}[commandchars=\\\{\}]
\PYGZlt{}IPython.lib.display.IFrame at 0x745388115fc0\PYGZgt{}
\end{sphinxVerbatim}

\end{sphinxuseclass}\end{sphinxVerbatimOutput}

\end{sphinxuseclass}
\sphinxAtStartPar
Odčítání dvou vektorů je také dalším vektorem:
\begin{equation*}
\begin{split} \overrightarrow{\mathbf{a}} - \overrightarrow{\mathbf{b}} = (a_x\vec{\mathbf{i}} + a_y\vec{\mathbf{j}} + a_z\vec{\mathbf{k}}) - (b_x\vec{\mathbf{i}} + b_y\vec{\mathbf{j}} + b_z\vec{\mathbf{k}}) = 
(a_x-b_x)\vec{\mathbf{i}} + (a_y-b_y)\vec{\mathbf{j}} + (a_z-b_z)\vec{\mathbf{k}} \end{split}
\end{equation*}
\sphinxAtStartPar
V maticovém zápisu to bude vypadat:
\$\(\begin{bmatrix} a_x \\ a_y \\ a_z \end{bmatrix}  -  \begin{bmatrix} b_x \\ b_y \\ b_z \end{bmatrix} = \begin{bmatrix} a_x - b_x \\ a_y - b_y \\ a_z - b_z \end{bmatrix}\)\$

\sphinxAtStartPar
Zvažte dvě 2D pole (řádky a sloupce) představující polohu dvou objektů pohybujících se v prostoru. Sloupce představují vektorové komponenty a řádky hodnot polohového vektoru v různých okamžicích. Je snadné provést sčítání a odčítání s těmito vektory:

\begin{sphinxuseclass}{cell}\begin{sphinxVerbatimInput}

\begin{sphinxuseclass}{cell_input}
\begin{sphinxVerbatim}[commandchars=\\\{\}]
\PYG{n}{a} \PYG{o}{=} \PYG{n}{np}\PYG{o}{.}\PYG{n}{array}\PYG{p}{(}\PYG{p}{[}\PYG{p}{[}\PYG{l+m+mi}{1}\PYG{p}{,} \PYG{l+m+mi}{2}\PYG{p}{,} \PYG{l+m+mi}{3}\PYG{p}{]}\PYG{p}{,} \PYG{p}{[}\PYG{l+m+mi}{1}\PYG{p}{,} \PYG{l+m+mi}{1}\PYG{p}{,} \PYG{l+m+mi}{1}\PYG{p}{]}\PYG{p}{]}\PYG{p}{)}
\PYG{n}{b} \PYG{o}{=} \PYG{n}{np}\PYG{o}{.}\PYG{n}{array}\PYG{p}{(}\PYG{p}{[}\PYG{p}{[}\PYG{l+m+mi}{4}\PYG{p}{,} \PYG{l+m+mi}{5}\PYG{p}{,} \PYG{l+m+mi}{6}\PYG{p}{]}\PYG{p}{,} \PYG{p}{[}\PYG{l+m+mi}{7}\PYG{p}{,} \PYG{l+m+mi}{8}\PYG{p}{,} \PYG{l+m+mi}{9}\PYG{p}{]}\PYG{p}{]}\PYG{p}{)}
\PYG{n+nb}{print}\PYG{p}{(}\PYG{l+s+s1}{\PYGZsq{}}\PYG{l+s+s1}{a =}\PYG{l+s+s1}{\PYGZsq{}}\PYG{p}{,} \PYG{n}{a}\PYG{p}{,} \PYG{l+s+s1}{\PYGZsq{}}\PYG{l+s+se}{\PYGZbs{}n}\PYG{l+s+s1}{b =}\PYG{l+s+s1}{\PYGZsq{}}\PYG{p}{,} \PYG{n}{b}\PYG{p}{)}
\PYG{n+nb}{print}\PYG{p}{(}\PYG{l+s+s1}{\PYGZsq{}}\PYG{l+s+s1}{a + b =}\PYG{l+s+s1}{\PYGZsq{}}\PYG{p}{,} \PYG{n}{a} \PYG{o}{+} \PYG{n}{b}\PYG{p}{)}
\PYG{n+nb}{print}\PYG{p}{(}\PYG{l+s+s1}{\PYGZsq{}}\PYG{l+s+s1}{a \PYGZhy{} b =}\PYG{l+s+s1}{\PYGZsq{}}\PYG{p}{,} \PYG{n}{a} \PYG{o}{\PYGZhy{}} \PYG{n}{b}\PYG{p}{)}
\end{sphinxVerbatim}

\end{sphinxuseclass}\end{sphinxVerbatimInput}
\begin{sphinxVerbatimOutput}

\begin{sphinxuseclass}{cell_output}
\begin{sphinxVerbatim}[commandchars=\\\{\}]
a = [[1 2 3]
 [1 1 1]] 
b = [[4 5 6]
 [7 8 9]]
a + b = [[ 5  7  9]
 [ 8  9 10]]
a \PYGZhy{} b = [[\PYGZhy{}3 \PYGZhy{}3 \PYGZhy{}3]
 [\PYGZhy{}6 \PYGZhy{}7 \PYGZhy{}8]]
\end{sphinxVerbatim}

\end{sphinxuseclass}\end{sphinxVerbatimOutput}

\end{sphinxuseclass}
\sphinxAtStartPar
Numpy zvládne n\sphinxhyphen{}dimenzionální pole s velikostí omezenou dostupnou pamětí v počítači.

\sphinxAtStartPar
A můžeme provádět operace na každém vektoru, například vypočítat normu každého z nich.
Nejprve zkontrolujeme tvar proměnné \sphinxcode{\sphinxupquote{a}} pomocí metody„ tvar “nebo funkce\sphinxcode{\sphinxupquote{ numpy.shape}}:

\begin{sphinxuseclass}{cell}\begin{sphinxVerbatimInput}

\begin{sphinxuseclass}{cell_input}
\begin{sphinxVerbatim}[commandchars=\\\{\}]
\PYG{n+nb}{print}\PYG{p}{(}\PYG{n}{a}\PYG{o}{.}\PYG{n}{shape}\PYG{p}{)}
\PYG{n+nb}{print}\PYG{p}{(}\PYG{n}{np}\PYG{o}{.}\PYG{n}{shape}\PYG{p}{(}\PYG{n}{a}\PYG{p}{)}\PYG{p}{)}
\end{sphinxVerbatim}

\end{sphinxuseclass}\end{sphinxVerbatimInput}
\begin{sphinxVerbatimOutput}

\begin{sphinxuseclass}{cell_output}
\begin{sphinxVerbatim}[commandchars=\\\{\}]
(2, 3)
(2, 3)
\end{sphinxVerbatim}

\end{sphinxuseclass}\end{sphinxVerbatimOutput}

\end{sphinxuseclass}
\sphinxAtStartPar
To znamená, že proměnná \sphinxcode{\sphinxupquote{A}} má 2 řádky a 3 sloupce.
Musíme sdělit funkci \sphinxcode{\sphinxupquote{numpy.norm}} pro výpočet normy pro každý vektor, tj. Abychom fungovali skrz sloupce proměnné\sphinxcode{\sphinxupquote{ a}} pomocí paraneter \sphinxcode{\sphinxupquote{osy}}:

\begin{sphinxuseclass}{cell}\begin{sphinxVerbatimInput}

\begin{sphinxuseclass}{cell_input}
\begin{sphinxVerbatim}[commandchars=\\\{\}]
\PYG{n}{np}\PYG{o}{.}\PYG{n}{linalg}\PYG{o}{.}\PYG{n}{norm}\PYG{p}{(}\PYG{n}{a}\PYG{p}{,} \PYG{n}{axis}\PYG{o}{=}\PYG{l+m+mi}{1}\PYG{p}{)}
\end{sphinxVerbatim}

\end{sphinxuseclass}\end{sphinxVerbatimInput}
\begin{sphinxVerbatimOutput}

\begin{sphinxuseclass}{cell_output}
\begin{sphinxVerbatim}[commandchars=\\\{\}]
array([3.74165739, 1.73205081])
\end{sphinxVerbatim}

\end{sphinxuseclass}\end{sphinxVerbatimOutput}

\end{sphinxuseclass}

\subsection{Vektor mezi dvěma danými body}
\label{\detokenize{Prednasky/0_2_Skal_xe1ry_a_vektory:vektor-mezi-dvema-danymi-body}}
\sphinxAtStartPar
podle \sphinxhref{https://www.nagwa.com/en/explainers/694162356987/}{Lesson Explainer: Vectors in Space}

\sphinxAtStartPar
Začněme uvažováním dvou odlišných bodů a v prostoru, jak je znázorněno na obrázku.
\sphinxincludegraphics{{5}.png}

\sphinxAtStartPar
Chceme sestrojit vektor z \(A\) do \(B\), který je označen \(\vec{AB}\)
\sphinxincludegraphics{{6}.png}

\sphinxAtStartPar
Abychom toho dosáhli, můžeme cestovat přes počátek, jak je znázorněno na obrázku níže. Můžeme jít z bodu \(O\) do bodu \(A\) a pak z bodu \(O\) do bodu
\(B\).
\sphinxincludegraphics{{7}.png}

\sphinxAtStartPar
To lze zapsat jako následující rovnici pomocí vektorů:
\$\(\vec{AB} = \vec{AO} + \vec{OB} \)\(
Pro jakýkoli bod \)A\( platí
\)\(\vec{AO} = - \vec{OA}\)\(
Tuto vlastnost můžeme použít k přepsání naší rovnice takto: 
\)\(\vec{AB} = -\vec{OA} + \vec{BO} \)\(
Vektor \)\textbackslash{}vec\{AB\}\$ je tak možné vyjádřit

\sphinxAtStartPar
\$\(\vec{AB} =  \vec{BO} - \vec{OA}  = \vec{\mathbf{B}} - \vec{\mathbf{A}}\)\$\\
Vektor mezi dvěma body můžeme získat tak, že od polohového vektoru konečného bodu odčítáme polohový vektor počátečního bodu.

\begin{sphinxuseclass}{cell}\begin{sphinxVerbatimInput}

\begin{sphinxuseclass}{cell_input}
\begin{sphinxVerbatim}[commandchars=\\\{\}]
\PYG{n}{IFrame}\PYG{p}{(}\PYG{l+s+s1}{\PYGZsq{}}\PYG{l+s+s1}{https://www.geogebra.org/classic/NrzgxrQa?embed}\PYG{l+s+s1}{\PYGZsq{}}\PYG{p}{,} \PYG{n}{width}\PYG{o}{=}\PYG{l+s+s1}{\PYGZsq{}}\PYG{l+s+s1}{100}\PYG{l+s+s1}{\PYGZpc{}}\PYG{l+s+s1}{\PYGZsq{}}\PYG{p}{,} \PYG{n}{height}\PYG{o}{=}\PYG{l+m+mi}{500}\PYG{p}{)}
\end{sphinxVerbatim}

\end{sphinxuseclass}\end{sphinxVerbatimInput}
\begin{sphinxVerbatimOutput}

\begin{sphinxuseclass}{cell_output}
\begin{sphinxVerbatim}[commandchars=\\\{\}]
\PYGZlt{}IPython.lib.display.IFrame at 0x745388116560\PYGZgt{}
\end{sphinxVerbatim}

\end{sphinxuseclass}\end{sphinxVerbatimOutput}

\end{sphinxuseclass}

\section{Skalární součin}
\label{\detokenize{Prednasky/0_2_Skal_xe1ry_a_vektory:skalarni-soucin}}\begin{quote}

\sphinxAtStartPar
Skalární součin je operace, která přiřazuje dvěma vektorům jediné číslo (skalár).
\end{quote}

\sphinxAtStartPar
Toto číslo nám poskytuje informaci o tom, jak moc jsou tyto dva vektory “souhlasné” neboli jak moc “směřují stejným směrem”. Skalární součin umožňuje vypočítat projekci jednoho vektoru na druhý. Tato projekce nám říká, jak velká část jednoho vektoru “směřuje” ve směru druhého vektoru.

\sphinxAtStartPar
\sphinxstylestrong{Matematický zápis}
Skalární součin vektorů \(\vec{\mathbf{u}}\) a \(\vec{\mathbf{v}}\) se značí \(\vec{\mathbf{u}} \cdot \vec{\mathbf{u}}\).

\sphinxAtStartPar
Definujme skalární součin jednotkových vektorů v pravoúhlém souřadnicovém systému.
\$\( \vec{\mathbf{i}} \cdot \vec{\mathbf{i}} = \vec{\mathbf{j}} \cdot \vec{\mathbf{j}} = \vec{\mathbf{k}} \cdot \vec{\mathbf{k}}= 1 \quad \text{a} \quad \vec{\mathbf{i}} \cdot \vec{\mathbf{j}} = \vec{\mathbf{i}} \cdot \vec{\mathbf{k}} = \vec{\mathbf{j}} \cdot \vec{\mathbf{k}} = 0 \)\$

\sphinxAtStartPar
PSkalární součin mezi dvěma vektory je matematická operace algebraicky definovaná jako součet násobků jednotlivých složek (v každém směru) obou vektorů.
\$\( \overrightarrow{\mathbf{a}} \cdot \overrightarrow{\mathbf{b}} = (a_x\,\vec{\mathbf{i}}+a_y\,\vec{\mathbf{j}}+a_z\,\vec{\mathbf{k}}) \cdot (b_x\,\vec{\mathbf{i}}+b_y\,\vec{\mathbf{j}}+b_z\,\vec{\mathbf{k}}) = a_x b_x + a_y b_y + a_z b_z \)\$

\sphinxAtStartPar
Geometrický ekvivalent skalární součinu je násobek velikostí dvou vektorů a kosinunu úhlu mezi nimi:
\$\( \overrightarrow{\mathbf{a}} \cdot \overrightarrow{\mathbf{b}} = ||\overrightarrow{\mathbf{a}}||\:||\overrightarrow{\mathbf{b}}||\:cos(\theta) \)\$

\sphinxAtStartPar
Což je také ekvivalentní tomu, že skalární součin vektoru  \(\overrightarrow{\mathbf{a}}\) and \(\overrightarrow{\mathbf{b}}\) je velikostí vektoru \(\overrightarrow{\mathbf{a}}\)  a velikostí složky \(\overrightarrow{\mathbf{b}}\)  paralelní s \(\overrightarrow{\mathbf{a}}\)

\sphinxAtStartPar
V maticovém zápisu můžeme zapsat jako násobení dvojice vektorů
\$\( \overrightarrow{\mathbf{a}} \cdot \overrightarrow{\mathbf{b}} =  \begin{bmatrix} a_x \\ a_y \\ a_z \end{bmatrix}  \begin{bmatrix} b_x \\ b_y \\ b_z \end{bmatrix}^T = \begin{bmatrix} a_x \\ a_y \\ a_z \end{bmatrix} [b_x, b_y, b_z]\)\$

\sphinxAtStartPar
V této interaktivní animaci lze vizualizovat skalární součin mezi dvěma vektory:

\begin{sphinxuseclass}{cell}\begin{sphinxVerbatimInput}

\begin{sphinxuseclass}{cell_input}
\begin{sphinxVerbatim}[commandchars=\\\{\}]
\PYG{n}{IFrame}\PYG{p}{(}\PYG{l+s+s1}{\PYGZsq{}}\PYG{l+s+s1}{https://www.geogebra.org/classic/ncdf2jsw?embed}\PYG{l+s+s1}{\PYGZsq{}}\PYG{p}{,}
       \PYG{n}{width}\PYG{o}{=}\PYG{l+s+s1}{\PYGZsq{}}\PYG{l+s+s1}{100}\PYG{l+s+s1}{\PYGZpc{}}\PYG{l+s+s1}{\PYGZsq{}}\PYG{p}{,} \PYG{n}{height}\PYG{o}{=}\PYG{l+m+mi}{500}\PYG{p}{)}
\end{sphinxVerbatim}

\end{sphinxuseclass}\end{sphinxVerbatimInput}
\begin{sphinxVerbatimOutput}

\begin{sphinxuseclass}{cell_output}
\begin{sphinxVerbatim}[commandchars=\\\{\}]
\PYGZlt{}IPython.lib.display.IFrame at 0x745388116320\PYGZgt{}
\end{sphinxVerbatim}

\end{sphinxuseclass}\end{sphinxVerbatimOutput}

\end{sphinxuseclass}
\sphinxAtStartPar
Numpy funkce pro skalární součin je \sphinxcode{\sphinxupquote{numpy.dot}}:

\begin{sphinxuseclass}{cell}\begin{sphinxVerbatimInput}

\begin{sphinxuseclass}{cell_input}
\begin{sphinxVerbatim}[commandchars=\\\{\}]
\PYG{n}{a} \PYG{o}{=} \PYG{n}{np}\PYG{o}{.}\PYG{n}{array}\PYG{p}{(}\PYG{p}{[}\PYG{l+m+mi}{1}\PYG{p}{,} \PYG{l+m+mi}{2}\PYG{p}{,} \PYG{l+m+mi}{3}\PYG{p}{]}\PYG{p}{)}
\PYG{n}{b} \PYG{o}{=} \PYG{n}{np}\PYG{o}{.}\PYG{n}{array}\PYG{p}{(}\PYG{p}{[}\PYG{l+m+mi}{4}\PYG{p}{,} \PYG{l+m+mi}{5}\PYG{p}{,} \PYG{l+m+mi}{6}\PYG{p}{]}\PYG{p}{)}
\PYG{n+nb}{print}\PYG{p}{(}\PYG{l+s+s1}{\PYGZsq{}}\PYG{l+s+s1}{a =}\PYG{l+s+s1}{\PYGZsq{}}\PYG{p}{,} \PYG{n}{a}\PYG{p}{,} \PYG{l+s+s1}{\PYGZsq{}}\PYG{l+s+se}{\PYGZbs{}n}\PYG{l+s+s1}{b =}\PYG{l+s+s1}{\PYGZsq{}}\PYG{p}{,} \PYG{n}{b}\PYG{p}{)}
\PYG{n+nb}{print}\PYG{p}{(}\PYG{l+s+s1}{\PYGZsq{}}\PYG{l+s+s1}{np.dot(a, b) =}\PYG{l+s+s1}{\PYGZsq{}}\PYG{p}{,} \PYG{n}{np}\PYG{o}{.}\PYG{n}{dot}\PYG{p}{(}\PYG{n}{a}\PYG{p}{,} \PYG{n}{b}\PYG{p}{)}\PYG{p}{)}
\end{sphinxVerbatim}

\end{sphinxuseclass}\end{sphinxVerbatimInput}
\begin{sphinxVerbatimOutput}

\begin{sphinxuseclass}{cell_output}
\begin{sphinxVerbatim}[commandchars=\\\{\}]
a = [1 2 3] 
b = [4 5 6]
np.dot(a, b) = 32
\end{sphinxVerbatim}

\end{sphinxuseclass}\end{sphinxVerbatimOutput}

\end{sphinxuseclass}
\sphinxAtStartPar
Nebo můžeme definici použít přímo a spočítat přímo:

\begin{sphinxuseclass}{cell}\begin{sphinxVerbatimInput}

\begin{sphinxuseclass}{cell_input}
\begin{sphinxVerbatim}[commandchars=\\\{\}]
\PYG{n}{np}\PYG{o}{.}\PYG{n}{sum}\PYG{p}{(}\PYG{n}{a}\PYG{o}{*}\PYG{n}{b}\PYG{p}{)}
\end{sphinxVerbatim}

\end{sphinxuseclass}\end{sphinxVerbatimInput}
\begin{sphinxVerbatimOutput}

\begin{sphinxuseclass}{cell_output}
\begin{sphinxVerbatim}[commandchars=\\\{\}]
32
\end{sphinxVerbatim}

\end{sphinxuseclass}\end{sphinxVerbatimOutput}

\end{sphinxuseclass}
\sphinxAtStartPar
Pro 2D pole vykonává funkce \sphinxcode{\sphinxupquote{numpy.dot}} spíše násobení než skalárního součinu; Pojďme tedy použít funkci \sphinxcode{\sphinxupquote{numpy.sum}}:

\begin{sphinxuseclass}{cell}\begin{sphinxVerbatimInput}

\begin{sphinxuseclass}{cell_input}
\begin{sphinxVerbatim}[commandchars=\\\{\}]
\PYG{n}{a} \PYG{o}{=} \PYG{n}{np}\PYG{o}{.}\PYG{n}{array}\PYG{p}{(}\PYG{p}{[}\PYG{p}{[}\PYG{l+m+mi}{1}\PYG{p}{,} \PYG{l+m+mi}{2}\PYG{p}{,} \PYG{l+m+mi}{3}\PYG{p}{]}\PYG{p}{,} \PYG{p}{[}\PYG{l+m+mi}{1}\PYG{p}{,} \PYG{l+m+mi}{1}\PYG{p}{,} \PYG{l+m+mi}{1}\PYG{p}{]}\PYG{p}{]}\PYG{p}{)}
\PYG{n}{b} \PYG{o}{=} \PYG{n}{np}\PYG{o}{.}\PYG{n}{array}\PYG{p}{(}\PYG{p}{[}\PYG{p}{[}\PYG{l+m+mi}{4}\PYG{p}{,} \PYG{l+m+mi}{5}\PYG{p}{,} \PYG{l+m+mi}{6}\PYG{p}{]}\PYG{p}{,} \PYG{p}{[}\PYG{l+m+mi}{7}\PYG{p}{,} \PYG{l+m+mi}{8}\PYG{p}{,} \PYG{l+m+mi}{9}\PYG{p}{]}\PYG{p}{]}\PYG{p}{)}
\PYG{n}{np}\PYG{o}{.}\PYG{n}{sum}\PYG{p}{(}\PYG{n}{a}\PYG{o}{*}\PYG{n}{b}\PYG{p}{,} \PYG{n}{axis}\PYG{o}{=}\PYG{l+m+mi}{1}\PYG{p}{)}
\end{sphinxVerbatim}

\end{sphinxuseclass}\end{sphinxVerbatimInput}
\begin{sphinxVerbatimOutput}

\begin{sphinxuseclass}{cell_output}
\begin{sphinxVerbatim}[commandchars=\\\{\}]
array([32, 24])
\end{sphinxVerbatim}

\end{sphinxuseclass}\end{sphinxVerbatimOutput}

\end{sphinxuseclass}

\section{Vektorový součin}
\label{\detokenize{Prednasky/0_2_Skal_xe1ry_a_vektory:vektorovy-soucin}}\begin{quote}

\sphinxAtStartPar
Vektorový součin je binární operace definovaná ve třírozměrném euklidovském prostoru, která přiřadí dvěma vektorům nový vektor, který je na ně oba kolmý. Na rozdíl od skalárního součinu, jehož výsledkem je číslo (skalár), výsledkem vektorového součinu je opět vektor.
\end{quote}

\sphinxAtStartPar
\sphinxstylestrong{Matematický zápis}
Vektorový součin vektorů \(\vec{\mathbf{a}}\) a \(\vec{\mathbf{b}}\) se značí \(\vec{\mathbf{a}} \times \vec{\mathbf{b}}\).

\sphinxAtStartPar
\sphinxstylestrong{Geometrická interpretace}
\begin{itemize}
\item {} 
\sphinxAtStartPar
\sphinxstylestrong{Směr}: Výsledný vektor je kolmý k rovině určené vektory \(\vec{\mathbf{a}}\) a \(\vec{\mathbf{b}}\). Jeho směr se určuje pomocí pravidla pravé ruky: Pokud prsty pravé ruky ohneme od vektoru u k vektoru v, ukazuje nám palec směr výsledného vektoru.

\end{itemize}

\sphinxAtStartPar
\sphinxincludegraphics{{28.43.2}.png}
\begin{itemize}
\item {} 
\sphinxAtStartPar
\sphinxstylestrong{Velikost}: Velikost výsledného vektoru je rovna ploše rovnoběžníku určeného vektory \(\vec{\mathbf{a}}\) a \(\vec{\mathbf{b}}\).
\begin{equation*}
\begin{split} \overrightarrow{\mathbf{a}} \times \overrightarrow{\mathbf{b}} = ||\overrightarrow{\mathbf{a}}||\:||\overrightarrow{\mathbf{b}}||\:sin(\theta) \end{split}
\end{equation*}
\end{itemize}



\sphinxAtStartPar
\sphinxstylestrong{Analytické vyjádření}

\sphinxAtStartPar
Definujme vektorový součin jednotkových vektorů:
\$\( \begin{array}{l l}
\vec{\mathbf{i}} \times \vec{\mathbf{i}} = \vec{\mathbf{j}} \times \vec{\mathbf{j}} = \vec{\mathbf{k}} \times \vec{\mathbf{k}} = 0 \\
\vec{\mathbf{i}} \times \vec{\mathbf{j}} = \vec{\mathbf{k}}, \quad \vec{\mathbf{k}} \times \vec{\mathbf{k}} = \vec{\mathbf{i}}, \quad \vec{\mathbf{k}} \times \vec{\mathbf{i}} = \vec{\mathbf{j}} \\
\vec{\mathbf{j}} \times \vec{\mathbf{i}} = -\vec{\mathbf{k}}, \quad \vec{\mathbf{k}} \times \vec{\mathbf{j}}= -\vec{\mathbf{i}}, \quad \vec{\mathbf{i}} \times \vec{\mathbf{k}} = -\vec{\mathbf{j}}
\end{array} \)\$

\sphinxAtStartPar
získáme pak
\begin{equation*}
\begin{split} \overrightarrow{\mathbf{a}} \times \overrightarrow{\mathbf{b}} = (a_x\,\vec{\mathbf{i}} + a_y\,\vec{\mathbf{j}} + a_z\,\vec{\mathbf{k}}) \times (b_x\,\vec{\mathbf{i}}+b_y\,\vec{\mathbf{j}}+b_z\,\vec{\mathbf{k}}) = (a_y b_z-a_z b_y)\vec{\mathbf{i}} + (a_z b_x-a_x b_z)\vec{\mathbf{j}}+(a_x b_y- a_y b_x)\vec{\mathbf{k}} \end{split}
\end{equation*}
\sphinxAtStartPar
Vektorový součin lze také vypočítat jako determinant matice:
\$\( \overrightarrow{\mathbf{a}} \times \overrightarrow{\mathbf{b}} = \left| \begin{array}{ccc}
\vec{\mathbf{i}} & \vec{\mathbf{j}} & \vec{\mathbf{k}} \\
a_x & a_y & a_z \\
b_x & b_y & b_z 
\end{array} \right|
= a_y b_z \vec{\mathbf{i}} + a_z b_x \vec{\mathbf{j}} +  a_x b_y \vec{\mathbf{k}} - a_y b_x \vec{\mathbf{k}}-a_z b_y \vec{\mathbf{i}} - a_x b_z \vec{\mathbf{j}} \)\(
\)\(\overrightarrow{\mathbf{a}} \times \overrightarrow{\mathbf{b}} = (a_y b_z-a_z b_y)\vec{\mathbf{i}} + (a_z b_x-a_x b_z)\vec{\mathbf{j}} + (a_x b_y-a_y b_x)\vec{\mathbf{k}} \)\$

\sphinxAtStartPar
V této interaktivní animaci lze vizualizovat vektorový součin mezi dvěma vektory:

\begin{sphinxuseclass}{cell}\begin{sphinxVerbatimInput}

\begin{sphinxuseclass}{cell_input}
\begin{sphinxVerbatim}[commandchars=\\\{\}]
\PYG{n}{IFrame}\PYG{p}{(}\PYG{l+s+s1}{\PYGZsq{}}\PYG{l+s+s1}{https://www.geogebra.org/classic/cz6v2U99?embed}\PYG{l+s+s1}{\PYGZsq{}}\PYG{p}{,}
       \PYG{n}{width}\PYG{o}{=}\PYG{l+s+s1}{\PYGZsq{}}\PYG{l+s+s1}{100}\PYG{l+s+s1}{\PYGZpc{}}\PYG{l+s+s1}{\PYGZsq{}}\PYG{p}{,} \PYG{n}{height}\PYG{o}{=}\PYG{l+m+mi}{500}\PYG{p}{)}
\end{sphinxVerbatim}

\end{sphinxuseclass}\end{sphinxVerbatimInput}
\begin{sphinxVerbatimOutput}

\begin{sphinxuseclass}{cell_output}
\begin{sphinxVerbatim}[commandchars=\\\{\}]
\PYGZlt{}IPython.lib.display.IFrame at 0x745388116080\PYGZgt{}
\end{sphinxVerbatim}

\end{sphinxuseclass}\end{sphinxVerbatimOutput}

\end{sphinxuseclass}
\sphinxAtStartPar
Numpy funkce pro křížový produkt je \sphinxcode{\sphinxupquote{numpy.cross}}:

\begin{sphinxuseclass}{cell}\begin{sphinxVerbatimInput}

\begin{sphinxuseclass}{cell_input}
\begin{sphinxVerbatim}[commandchars=\\\{\}]
\PYG{n}{a} \PYG{o}{=} \PYG{p}{[}\PYG{l+m+mi}{0}\PYG{p}{,} \PYG{l+m+mi}{0}\PYG{p}{,} \PYG{l+m+mi}{5}\PYG{p}{]}
\PYG{n}{b} \PYG{o}{=} \PYG{p}{[}\PYG{l+m+mi}{3}\PYG{p}{,} \PYG{l+m+mi}{0}\PYG{p}{,} \PYG{l+m+mi}{0}\PYG{p}{]}

\PYG{n+nb}{print}\PYG{p}{(}\PYG{l+s+s1}{\PYGZsq{}}\PYG{l+s+s1}{a =}\PYG{l+s+s1}{\PYGZsq{}}\PYG{p}{,} \PYG{n}{a}\PYG{p}{,} \PYG{l+s+s1}{\PYGZsq{}}\PYG{l+s+se}{\PYGZbs{}n}\PYG{l+s+s1}{ b =}\PYG{l+s+s1}{\PYGZsq{}}\PYG{p}{,} \PYG{n}{b}\PYG{p}{)}
\PYG{n+nb}{print}\PYG{p}{(}\PYG{l+s+s1}{\PYGZsq{}}\PYG{l+s+s1}{np.cross(a, b) =}\PYG{l+s+s1}{\PYGZsq{}}\PYG{p}{,} \PYG{n}{np}\PYG{o}{.}\PYG{n}{cross}\PYG{p}{(}\PYG{n}{a}\PYG{p}{,} \PYG{n}{b}\PYG{p}{)}\PYG{p}{)}
\end{sphinxVerbatim}

\end{sphinxuseclass}\end{sphinxVerbatimInput}
\begin{sphinxVerbatimOutput}

\begin{sphinxuseclass}{cell_output}
\begin{sphinxVerbatim}[commandchars=\\\{\}]
a = [0, 0, 5] 
 b = [3, 0, 0]
np.cross(a, b) = [ 0 15  0]
\end{sphinxVerbatim}

\end{sphinxuseclass}\end{sphinxVerbatimOutput}

\end{sphinxuseclass}
\sphinxAtStartPar
Pro 2D pole s vektory v různých řádcích:

\begin{sphinxuseclass}{cell}\begin{sphinxVerbatimInput}

\begin{sphinxuseclass}{cell_input}
\begin{sphinxVerbatim}[commandchars=\\\{\}]
\PYG{n}{a} \PYG{o}{=} \PYG{n}{np}\PYG{o}{.}\PYG{n}{array}\PYG{p}{(}\PYG{p}{[}\PYG{p}{[}\PYG{l+m+mi}{1}\PYG{p}{,} \PYG{l+m+mi}{2}\PYG{p}{,} \PYG{l+m+mi}{3}\PYG{p}{]}\PYG{p}{,} \PYG{p}{[}\PYG{l+m+mi}{1}\PYG{p}{,} \PYG{l+m+mi}{1}\PYG{p}{,} \PYG{l+m+mi}{1}\PYG{p}{]}\PYG{p}{]}\PYG{p}{)}
\PYG{n}{b} \PYG{o}{=} \PYG{n}{np}\PYG{o}{.}\PYG{n}{array}\PYG{p}{(}\PYG{p}{[}\PYG{p}{[}\PYG{l+m+mi}{4}\PYG{p}{,} \PYG{l+m+mi}{5}\PYG{p}{,} \PYG{l+m+mi}{6}\PYG{p}{]}\PYG{p}{,} \PYG{p}{[}\PYG{l+m+mi}{7}\PYG{p}{,} \PYG{l+m+mi}{8}\PYG{p}{,} \PYG{l+m+mi}{9}\PYG{p}{]}\PYG{p}{]}\PYG{p}{)}
\PYG{n}{np}\PYG{o}{.}\PYG{n}{cross}\PYG{p}{(}\PYG{n}{a}\PYG{p}{,} \PYG{n}{b}\PYG{p}{,} \PYG{n}{axis}\PYG{o}{=}\PYG{l+m+mi}{1}\PYG{p}{)}
\end{sphinxVerbatim}

\end{sphinxuseclass}\end{sphinxVerbatimInput}
\begin{sphinxVerbatimOutput}

\begin{sphinxuseclass}{cell_output}
\begin{sphinxVerbatim}[commandchars=\\\{\}]
array([[\PYGZhy{}3,  6, \PYGZhy{}3],
       [ 1, \PYGZhy{}2,  1]])
\end{sphinxVerbatim}

\end{sphinxuseclass}\end{sphinxVerbatimOutput}

\end{sphinxuseclass}
\begin{sphinxuseclass}{cell}\begin{sphinxVerbatimInput}

\begin{sphinxuseclass}{cell_input}
\begin{sphinxVerbatim}[commandchars=\\\{\}]
\PYG{n}{a} \PYG{o}{=} \PYG{n}{np}\PYG{o}{.}\PYG{n}{array}\PYG{p}{(}\PYG{p}{[}\PYG{l+m+mi}{1}\PYG{p}{,} \PYG{l+m+mi}{2}\PYG{p}{,} \PYG{l+m+mi}{0}\PYG{p}{]}\PYG{p}{)}
\PYG{n}{b} \PYG{o}{=} \PYG{n}{np}\PYG{o}{.}\PYG{n}{array}\PYG{p}{(}\PYG{p}{[}\PYG{l+m+mi}{0}\PYG{p}{,} \PYG{l+m+mi}{1}\PYG{p}{,} \PYG{l+m+mi}{3}\PYG{p}{]}\PYG{p}{)}
\PYG{n}{c} \PYG{o}{=} \PYG{n}{np}\PYG{o}{.}\PYG{n}{array}\PYG{p}{(}\PYG{p}{[}\PYG{l+m+mi}{1}\PYG{p}{,} \PYG{l+m+mi}{0}\PYG{p}{,} \PYG{l+m+mi}{1}\PYG{p}{]}\PYG{p}{)}
\end{sphinxVerbatim}

\end{sphinxuseclass}\end{sphinxVerbatimInput}

\end{sphinxuseclass}

\section{Další čtení}
\label{\detokenize{Prednasky/0_2_Skal_xe1ry_a_vektory:dalsi-cteni}}\begin{itemize}
\item {} 
\sphinxAtStartPar
Přečtěte si stránky 44\sphinxhyphen{}92 první kapitoly \sphinxhref{http://ruina.tam.cornell.edu/Book/index.html}{Ruina and Rudra’s book} o skalárech a vektorech v mechanice.

\end{itemize}


\section{Video přednášky na internetu}
\label{\detokenize{Prednasky/0_2_Skal_xe1ry_a_vektory:video-prednasky-na-internetu}}\begin{itemize}
\item {} 
\sphinxAtStartPar
Khan Academy: \sphinxhref{https://www.khanacademy.org/math/algebra-home/alg-vectors}{Vectors}

\item {} 
\sphinxAtStartPar
\sphinxhref{https://youtu.be/fNk\_zzaMoSs}{Vectors, what even are they?}

\item {} 
\sphinxAtStartPar
\sphinxhref{https://is.muni.cz/do/rect/el/estud/prif/js17/pocetni\_praktikum1/web/ch02.html}{Základy vektorové a tenzorové algebry}

\end{itemize}


\section{Problémy}
\label{\detokenize{Prednasky/0_2_Skal_xe1ry_a_vektory:problemy}}\begin{enumerate}
\sphinxsetlistlabels{\arabic}{enumi}{enumii}{}{.}%
\item {} 
\sphinxAtStartPar
Vzhledem k vektorům, \(\overrightarrow{\mathbf{a}}=[1, 0, 0]\) a \(\overrightarrow{\mathbf{b}}=[1, 1, 1]\), vypočítejte jejich skalární a vektorový součin.

\item {} 
\sphinxAtStartPar
Vypočítejte jednotkové vektory pro \( [2, −2, 3] \) a \( [3, −3, 2] \) a určete ortogonální vektor těchto dvou vektorů.

\item {} 
\sphinxAtStartPar
Vzhledem k vektorům \(a = [1, 0, 0]\) a \(b = [1, 1, 1]\), vypočítejte \(\overrightarrow{\mathbf{a}} \times \overrightarrow{\mathbf{b}}\)  a ověřte, že tento vektor je ortogonální na původní vektory. Vypočítejte také  \(\overrightarrow{\mathbf{b}} \times \overrightarrow{\mathbf{a}}\) a porovnejte s  \(\overrightarrow{\mathbf{a}} \times \overrightarrow{\mathbf{b}}\).

\item {} 
\sphinxAtStartPar
Vzhledem k vektorům \( [1, 1, 0]; [1, 0, 1]; [0, 1, 1] \), vypočítejte ortonormální bázi souřadnicového systému.

\item {} 
\sphinxAtStartPar
Vyřešte úlohy  \sphinxstylestrong{1.1} až \sphinxstylestrong{1.9}, \sphinxstylestrong{1.11} (pomocí Pythonu), \sphinxstylestrong{1.12}, \sphinxstylestrong{1.14}, \sphinxstylestrong{1.17}, \sphinxstylestrong{1.18} do \sphinxstylestrong{1.24} knihy Ruiny a Rudry

\item {} 
\sphinxAtStartPar
Z knihy Ruiny a Rudry vyřešte problémy \sphinxstylestrong{1.1.1} do \sphinxstylestrong{1.3.16}.

\end{enumerate}

\sphinxAtStartPar
Pokud si nejste jisti ve skalárech a vektorech, měli byste tyto problémy vyřešit nejprve ručně a poté pomocí Pythonu zkontrolovat odpovědi.Znalost vektorového počtu je nezbytná pro další zvládnutí mechaniky.


\section{Reference}
\label{\detokenize{Prednasky/0_2_Skal_xe1ry_a_vektory:reference}}\begin{itemize}
\item {} 
\sphinxAtStartPar
Ruina A, Rudra P (2019) \sphinxhref{http://ruina.tam.cornell.edu/Book/index.html}{Introduction to Statics and Dynamics}. Oxford University Press.

\end{itemize}

\sphinxstepscope


\section{Souřadnicové systémy}
\label{\detokenize{Prednasky/0_3_Sou_u0159adnicov_xe9_syst_xe9my:souradnicove-systemy}}\label{\detokenize{Prednasky/0_3_Sou_u0159adnicov_xe9_syst_xe9my::doc}}
\sphinxAtStartPar
Pohyb (změna polohy v prostoru v závislosti na čase) není absolutní pojem; je potřeba referenční bod, vůči kterému popisujeme pohyb objektu. Stejně tak stav této reference nemůže být absolutní v prostoru, a proto je pohyb relativní.

\sphinxAtStartPar
Referenční soustava je místo, vzhledem k němuž si volíme popis pohybu objektu. V této referenční soustavě definujeme soustavu souřadnic (soubor os), ve které měříme pohyb objektu (ale pojmy referenční soustava a soustava souřadnic se často používají zaměnitelně).

\sphinxAtStartPar
Volba referenční soustavy a souřadného systému je často věcí úsudku a praktičnosti. Existuje však důležitý rozdíl mezi referenčními soustavami, když se zabýváme dynamikou pohybu, kde nás zajímají síly související s pohybem objektu. V dynamice hovoříme o inerciální soustavě (také známé jako Galileova soustava), když Newtonovy pohybové zákony v jejich jednoduché formě v této soustavě platí, a o neinerciální soustavě (také známé jako neinerciální referenční soustava), když Newtonovy zákony v jejich jednoduché formě neplatí (v takové soustavě se objevují zdánlivé zrychlení/síly). Inerciální soustava je v klidu nebo se pohybuje konstantní rychlostí (protože neexistuje absolutní klid!), zatímco neinerciální soustava je zrychlena (vzhledem k inerciální soustavě).

\sphinxAtStartPar
Pojem referenční soustavy se od dob Aristotela, Galileiho, Newtona a Einsteina výrazně změnil. Více o tom a filozofických implikacích si můžete přečíst v článku \sphinxhref{http://plato.stanford.edu/entries/spacetime-iframes/}{Space and Time: Inertial Frames}.


\subsection{Souřadnicová soustava pro analýzu lidského pohybu}
\label{\detokenize{Prednasky/0_3_Sou_u0159adnicov_xe9_syst_xe9my:souradnicova-soustava-pro-analyzu-lidskeho-pohybu}}
\sphinxAtStartPar
V anatomii používáme zjednodušenou referenční soustavu složenou z kolmých rovin, abychom poskytli standardní referenci pro kvalitativní popis struktur a pohybů lidského těla, jak je znázorněno na dalším obrázku.




\subsection{Báze souřadnicového systému}
\label{\detokenize{Prednasky/0_3_Sou_u0159adnicov_xe9_syst_xe9my:baze-souradnicoveho-systemu}}
\sphinxAtStartPar
\sphinxstylestrong{Báze souřadnicového systému} je množina vektorů, které nám umožňují jednoznačně popsat polohu libovolného bodu v prostoru. Představte si to jako soustavu os, které se vzájemně protínají v jednom bodě (počátku) a určují směry, ve kterých měříme vzdálenosti.

\sphinxAtStartPar
\sphinxstylestrong{Proč potřebujeme bázi?}
\begin{itemize}
\item {} 
\sphinxAtStartPar
\sphinxstylestrong{Jednoznačný popis polohy:} Díky bázi můžeme každý bod v prostoru jednoznačně identifikovat pomocí souřadnic, které udávají, jak daleko je tento bod od počátku ve směru jednotlivých os.

\item {} 
\sphinxAtStartPar
\sphinxstylestrong{Zjednodušení výpočtů:} Báze nám umožňuje převést geometrické problémy na algebraické, což výrazně usnadňuje jejich řešení.

\item {} 
\sphinxAtStartPar
\sphinxstylestrong{Standardizace:} Báze nám poskytuje společný jazyk pro popis prostorových vztahů.

\end{itemize}

\sphinxAtStartPar
\sphinxstylestrong{Jak vypadá báze?}
\begin{itemize}
\item {} 
\sphinxAtStartPar
\sphinxstylestrong{V rovině:} Bázi tvoří dva lineárně nezávislé vektory, které určují směr os x a y. Nejčastěji se používá kartézská soustava souřadnic, kde jsou osy x a y kolmé na sebe a mají stejnou jednotku.

\item {} 
\sphinxAtStartPar
\sphinxstylestrong{V prostoru:} Bázi tvoří tři lineárně nezávislé vektory, které určují směr os x, y a z. Opět se nejčastěji používá kartézská soustava souřadnic, kde jsou osy vzájemně kolmé a mají stejnou jednotku.

\end{itemize}

\sphinxAtStartPar
\sphinxstylestrong{Vlastnosti báze:}
\begin{itemize}
\item {} 
\sphinxAtStartPar
\sphinxstylestrong{Lineární nezávislost:} Žádný vektor báze nesmí být lineární kombinací ostatních vektorů báze.

\item {} 
\sphinxAtStartPar
\sphinxstylestrong{Rozsah:} Počet vektorů v bázi se rovná dimenzi prostoru. V rovině jsou to dva vektory, v prostoru tři.

\end{itemize}


\subsection{Ortogonální a ortonormální báze}
\label{\detokenize{Prednasky/0_3_Sou_u0159adnicov_xe9_syst_xe9my:ortogonalni-a-ortonormalni-baze}}
\sphinxAtStartPar
\sphinxstylestrong{Ortogonální báze} je speciální typ báze vektorového prostoru, kde jsou všechny vektory vzájemně kolmé. To znamená, že skalární součin libovolných dvou různých vektorů z této báze je roven nule.

\sphinxAtStartPar
\sphinxstylestrong{Proč je ortogonální báze výhodná?}
\begin{itemize}
\item {} 
\sphinxAtStartPar
\sphinxstylestrong{Jednodušší výpočty:} Mnohé výpočty, jako například projekce vektorů nebo hledání souřadnic vektoru vzhledem k bázi, jsou pro ortogonální báze výrazně jednodušší.

\item {} 
\sphinxAtStartPar
\sphinxstylestrong{Geometrická interpretace:} Ortogonálnost vektorů má přímou geometrickou interpretaci \sphinxhyphen{} vektory svírají pravý úhel.

\end{itemize}

\sphinxAtStartPar
\sphinxstylestrong{Ortonormální báze} je ještě speciálnější případ ortogonální báze. Kromě toho, že jsou všechny vektory v této bázi vzájemně kolmé, mají také jednotkovou délku (tj. jejich velikost je rovna jedné).

\sphinxAtStartPar
\sphinxstylestrong{Proč je ortonormální báze výhodná?}
\begin{itemize}
\item {} 
\sphinxAtStartPar
\sphinxstylestrong{Normalizace:} Jednotková délka vektorů zjednodušuje mnohé výpočty, zejména při práci s normami a vzdálenostmi.

\item {} 
\sphinxAtStartPar
\sphinxstylestrong{Ortogonalita:} Zachovává všechny výhody ortogonální báze.

\end{itemize}

\sphinxAtStartPar
\sphinxstylestrong{Druhy bází}
\begin{itemize}
\item {} 
\sphinxAtStartPar
\sphinxstylestrong{Kartézská báze:} Nejběžnější typ báze, kde osy jsou vzájemně kolmé a mají stejnou jednotku.

\item {} 
\sphinxAtStartPar
\sphinxstylestrong{Sférická báze:} Používá se pro popis polohy bodů ve sférických souřadnicích (poloměr, úhel v horizontální rovině, úhel od vertikály).

\item {} 
\sphinxAtStartPar
\sphinxstylestrong{Cylindrická báze:} Používá se pro popis polohy bodů v cylindrických souřadnicích (poloměr, úhel, výška).

\end{itemize}


\subsection{Kartézský souřadnicový systém}
\label{\detokenize{Prednasky/0_3_Sou_u0159adnicov_xe9_syst_xe9my:kartezsky-souradnicovy-system}}
\sphinxAtStartPar
Protože okolní prostor vnímáme jako trojrozměrný, vhodný souřadnicový systém je \sphinxhref{http://en.wikipedia.org/wiki/Cartesian\_coordinate\_system}{Kartézsky souřadnicový systém} v \sphinxhref{http://en.wikipedia.org/wiki/Cartesian\_coordinate\_system}{Kartézkom souřadnicovém systému} se třemi ortogonálními osami, jak je znázorněno níže. Ortogonalita kartézského souřadnicového systému je vhodná pro jeho použití v klasické mechanice, většinou se předpokládá, že struktura prostoru má \sphinxhref{http://en.wikipedia.org/wiki/Cartesian\_coordinate\_system}{Cartesian coordinate system} a v důsledku toho jsou pohyby v různých směrech na sobě nezávislé.



\sphinxAtStartPar
Další iniciativa pro standardizaci referenčních rámců pochází z projektu \sphinxhref{https://github.com/BMClab/BMC/blob/master/courses/refs/VAKHUM.pdf}{Virtual Animation of the Kinematics of the Human for Industrial, Educational and Research Purposes (VAKHUM)}.


\subsection{Určení souřadnicového systému}
\label{\detokenize{Prednasky/0_3_Sou_u0159adnicov_xe9_syst_xe9my:urceni-souradnicoveho-systemu}}
\sphinxAtStartPar
Nejčastěji používáme kartézskou soustavu souřadnic, která využívá tři vzájemně kolmé osy (x, y, z) a počátek. Každý bod v prostoru můžeme jednoznačně určit pomocí tří čísel, které udávají vzdálenosti od těchto os.



\sphinxAtStartPar
Je vidět, že verzory báze zobrazené na obrázku výše mají v kartézském souřadnicovém systému následující souřadnice:


\textbackslash{}begin\{equation\}
\textbackslash{}hat\{\textbackslash{}mathbf\{i\}\} = \textbackslash{}begin\{bmatrix\}1\textbackslash{}0\textbackslash{}0 \textbackslash{}end\{bmatrix\}, \textbackslash{}quad \textbackslash{}hat\{\textbackslash{}mathbf\{j\}\} = \textbackslash{}begin\{bmatrix\}0\textbackslash{}1\textbackslash{}0 \textbackslash{}end\{bmatrix\}, \textbackslash{}quad \textbackslash{}hat\{\textbackslash{}mathbf\{k\}\} = \textbackslash{}begin\{bmatrix\} 0 \textbackslash{} 0 \textbackslash{} 1 \textbackslash{}end\{bmatrix\}
\textbackslash{}end\{equation\}


\sphinxAtStartPar
Pomocí zápisu popsaného na obrázku výše lze poziční vektor \(\overrightarrow{\mathbf{r}}\) (nebo bod \(\overrightarrow{\mathbf{P}}\)) vyjádřit jako:


\textbackslash{}begin\{equation\}
\textbackslash{}overrightarrow\{\textbackslash{}mathbf\{r\}\} = x \textbackslash{}vec\{\textbackslash{}mathbf\{i\}\} + y \textbackslash{}vec\{\textbackslash{}mathbf\{j\}\} + z\textbackslash{}vec\{\textbackslash{}mathbf\{k\}\}
\textbackslash{}end\{equation\}



\subsubsection{Definice báze}
\label{\detokenize{Prednasky/0_3_Sou_u0159adnicov_xe9_syst_xe9my:definice-baze}}
\sphinxAtStartPar
Matematickým problémem určení souřadnicového systému je najít pro něj základ a počátek (základ je pouze množina vektorů bez počátku). Existují různé metody pro výpočet základny dané množinou bodů (souřadnic), například pro tento problém lze použít skalární součin nebo  vektorový součin.


\paragraph{Použití vektorového součinu}
\label{\detokenize{Prednasky/0_3_Sou_u0159adnicov_xe9_syst_xe9my:pouziti-vektoroveho-soucinu}}
\sphinxAtStartPar
Pojďme nyní definovat základ pomocí běžné metody v analýze pohybu (za použití vektorového součinu):
Vzhledem k souřadnicím tří nekolineárních bodů ve 3D prostoru (body, které neleží všechny na stejné přímce), \(\overrightarrow{\mathbf{m}}_1, \overrightarrow{\mathbf{m}}_2, \overrightarrow{\mathbf{m}}_3\), které mohou představovat např. polohu značek zachycených z relace analýzy pohybu, lze na základě následujících kroků:
\begin{enumerate}
\sphinxsetlistlabels{\arabic}{enumi}{enumii}{}{.}%
\item {} 
\sphinxAtStartPar
První osa, \(\overrightarrow{\mathbf{v}}_1\), vektor \(\overrightarrow{\mathbf{m}}_2-\overrightarrow{\mathbf{m}}_1\) (nebo jakýkoli jiný vektorový rozdíl);

\item {} 
\sphinxAtStartPar
Druhá osa, \(\overrightarrow{\mathbf{v}}_2\), vektorový součin mezi vektory \(\overrightarrow{\mathbf{v}}_1\) a \(\overrightarrow{\mathbf{m}}_3-\overrightarrow{\mathbf{m}}_1\) (nebo \(\overrightarrow{\mathbf{m}}_3-\overrightarrow{\mathbf{m}}_2\));

\item {} 
\sphinxAtStartPar
Třetí osa, \(\overrightarrow{\mathbf{v}}_3\), vektorový součin mezi vektory \(\overrightarrow{\mathbf{v}}_1\) a \(\overrightarrow{\mathbf{v}}_2\); a

\item {} 
\sphinxAtStartPar
Normalizace vektorů tak, aby měly normu 1, dělení každého vektoru jeho normou.

\end{enumerate}

\sphinxAtStartPar
Polohy bodů použitých ke konstrukci souřadnicového systému musí být podle definice specifikovány ve vztahu k již existujícímu souřadnicovému systému. Při analýze pohybu je tento souřadnicový systém souřadnicovým systémem ze systému zachycení pohybu a je stanoven ve fázi kalibrace. V této fázi jsou pozice značek umístěných na objektu s kolmými osami a známými vzdálenostmi mezi značkami zachyceny a použity jako referenční (laboratorní) souřadnicový systém.

\sphinxAtStartPar
Například vzhledem k pozicím \(\overrightarrow{\mathbf{m}}_1 = [1,2,5], \overrightarrow{\mathbf{m}}_2 = [2,3,3], \overrightarrow{\mathbf{m}}_3 = [4,0,2]\) lze nalézt základ s:

\begin{sphinxuseclass}{cell}\begin{sphinxVerbatimInput}

\begin{sphinxuseclass}{cell_input}
\begin{sphinxVerbatim}[commandchars=\\\{\}]
\PYG{k+kn}{import} \PYG{n+nn}{numpy} \PYG{k}{as} \PYG{n+nn}{np}

\PYG{n}{m1} \PYG{o}{=} \PYG{n}{np}\PYG{o}{.}\PYG{n}{array}\PYG{p}{(}\PYG{p}{[}\PYG{l+m+mi}{1}\PYG{p}{,} \PYG{l+m+mi}{2}\PYG{p}{,} \PYG{l+m+mi}{5}\PYG{p}{]}\PYG{p}{)}
\PYG{n}{m2} \PYG{o}{=} \PYG{n}{np}\PYG{o}{.}\PYG{n}{array}\PYG{p}{(}\PYG{p}{[}\PYG{l+m+mi}{2}\PYG{p}{,} \PYG{l+m+mi}{3}\PYG{p}{,} \PYG{l+m+mi}{3}\PYG{p}{]}\PYG{p}{)}
\PYG{n}{m3} \PYG{o}{=} \PYG{n}{np}\PYG{o}{.}\PYG{n}{array}\PYG{p}{(}\PYG{p}{[}\PYG{l+m+mi}{4}\PYG{p}{,} \PYG{l+m+mi}{0}\PYG{p}{,} \PYG{l+m+mi}{2}\PYG{p}{]}\PYG{p}{)}

\PYG{n}{v1} \PYG{o}{=} \PYG{n}{m2} \PYG{o}{\PYGZhy{}} \PYG{n}{m1}                \PYG{c+c1}{\PYGZsh{} první osa}
\PYG{n}{v2} \PYG{o}{=} \PYG{n}{np}\PYG{o}{.}\PYG{n}{cross}\PYG{p}{(}\PYG{n}{v1}\PYG{p}{,} \PYG{n}{m3} \PYG{o}{\PYGZhy{}} \PYG{n}{m1}\PYG{p}{)}  \PYG{c+c1}{\PYGZsh{} druhá osa}
\PYG{n}{v3} \PYG{o}{=} \PYG{n}{np}\PYG{o}{.}\PYG{n}{cross}\PYG{p}{(}\PYG{n}{v1}\PYG{p}{,} \PYG{n}{v2}\PYG{p}{)}       \PYG{c+c1}{\PYGZsh{} třetí osa}

\PYG{c+c1}{\PYGZsh{} Vektorová normalizace}
\PYG{n}{e1} \PYG{o}{=} \PYG{n}{v1}\PYG{o}{/}\PYG{n}{np}\PYG{o}{.}\PYG{n}{linalg}\PYG{o}{.}\PYG{n}{norm}\PYG{p}{(}\PYG{n}{v1}\PYG{p}{)}
\PYG{n}{e2} \PYG{o}{=} \PYG{n}{v2}\PYG{o}{/}\PYG{n}{np}\PYG{o}{.}\PYG{n}{linalg}\PYG{o}{.}\PYG{n}{norm}\PYG{p}{(}\PYG{n}{v2}\PYG{p}{)}
\PYG{n}{e3} \PYG{o}{=} \PYG{n}{v3}\PYG{o}{/}\PYG{n}{np}\PYG{o}{.}\PYG{n}{linalg}\PYG{o}{.}\PYG{n}{norm}\PYG{p}{(}\PYG{n}{v3}\PYG{p}{)}

\PYG{n+nb}{print}\PYG{p}{(}\PYG{l+s+s1}{\PYGZsq{}}\PYG{l+s+s1}{Verzory:}\PYG{l+s+s1}{\PYGZsq{}}\PYG{p}{,} \PYG{l+s+s1}{\PYGZsq{}}\PYG{l+s+se}{\PYGZbs{}n}\PYG{l+s+s1}{e1 =}\PYG{l+s+s1}{\PYGZsq{}}\PYG{p}{,} \PYG{n}{e1}\PYG{p}{,} \PYG{l+s+s1}{\PYGZsq{}}\PYG{l+s+se}{\PYGZbs{}n}\PYG{l+s+s1}{e2 =}\PYG{l+s+s1}{\PYGZsq{}}\PYG{p}{,} \PYG{n}{e2}\PYG{p}{,} \PYG{l+s+s1}{\PYGZsq{}}\PYG{l+s+se}{\PYGZbs{}n}\PYG{l+s+s1}{e3 =}\PYG{l+s+s1}{\PYGZsq{}}\PYG{p}{,} \PYG{n}{e3}\PYG{p}{)}

\PYG{n+nb}{print}\PYG{p}{(}\PYG{l+s+s1}{\PYGZsq{}}\PYG{l+s+se}{\PYGZbs{}n}\PYG{l+s+s1}{Norm of každého z verzorů:}\PYG{l+s+s1}{\PYGZsq{}}\PYG{p}{,}
      \PYG{l+s+s1}{\PYGZsq{}}\PYG{l+s+se}{\PYGZbs{}n}\PYG{l+s+s1}{||e1|| =}\PYG{l+s+s1}{\PYGZsq{}}\PYG{p}{,} \PYG{n}{np}\PYG{o}{.}\PYG{n}{linalg}\PYG{o}{.}\PYG{n}{norm}\PYG{p}{(}\PYG{n}{e1}\PYG{p}{)}\PYG{p}{,}
      \PYG{l+s+s1}{\PYGZsq{}}\PYG{l+s+se}{\PYGZbs{}n}\PYG{l+s+s1}{||e2|| =}\PYG{l+s+s1}{\PYGZsq{}}\PYG{p}{,} \PYG{n}{np}\PYG{o}{.}\PYG{n}{linalg}\PYG{o}{.}\PYG{n}{norm}\PYG{p}{(}\PYG{n}{e2}\PYG{p}{)}\PYG{p}{,}
      \PYG{l+s+s1}{\PYGZsq{}}\PYG{l+s+se}{\PYGZbs{}n}\PYG{l+s+s1}{||e3|| =}\PYG{l+s+s1}{\PYGZsq{}}\PYG{p}{,} \PYG{n}{np}\PYG{o}{.}\PYG{n}{linalg}\PYG{o}{.}\PYG{n}{norm}\PYG{p}{(}\PYG{n}{e3}\PYG{p}{)}\PYG{p}{)}

\PYG{n+nb}{print}\PYG{p}{(}\PYG{l+s+s1}{\PYGZsq{}}\PYG{l+s+se}{\PYGZbs{}n}\PYG{l+s+s1}{Test of ortogonality (pomocí vektorového součinu):}\PYG{l+s+s1}{\PYGZsq{}}\PYG{p}{,}
      \PYG{l+s+s1}{\PYGZsq{}}\PYG{l+s+se}{\PYGZbs{}n}\PYG{l+s+s1}{e1 x e2:}\PYG{l+s+s1}{\PYGZsq{}}\PYG{p}{,} \PYG{n}{np}\PYG{o}{.}\PYG{n}{linalg}\PYG{o}{.}\PYG{n}{norm}\PYG{p}{(}\PYG{n}{np}\PYG{o}{.}\PYG{n}{cross}\PYG{p}{(}\PYG{n}{e1}\PYG{p}{,} \PYG{n}{e2}\PYG{p}{)}\PYG{p}{)}\PYG{p}{,}
      \PYG{l+s+s1}{\PYGZsq{}}\PYG{l+s+se}{\PYGZbs{}n}\PYG{l+s+s1}{e1 x e3:}\PYG{l+s+s1}{\PYGZsq{}}\PYG{p}{,} \PYG{n}{np}\PYG{o}{.}\PYG{n}{linalg}\PYG{o}{.}\PYG{n}{norm}\PYG{p}{(}\PYG{n}{np}\PYG{o}{.}\PYG{n}{cross}\PYG{p}{(}\PYG{n}{e1}\PYG{p}{,} \PYG{n}{e3}\PYG{p}{)}\PYG{p}{)}\PYG{p}{,}
      \PYG{l+s+s1}{\PYGZsq{}}\PYG{l+s+se}{\PYGZbs{}n}\PYG{l+s+s1}{e2 x e3:}\PYG{l+s+s1}{\PYGZsq{}}\PYG{p}{,} \PYG{n}{np}\PYG{o}{.}\PYG{n}{linalg}\PYG{o}{.}\PYG{n}{norm}\PYG{p}{(}\PYG{n}{np}\PYG{o}{.}\PYG{n}{cross}\PYG{p}{(}\PYG{n}{e2}\PYG{p}{,} \PYG{n}{e3}\PYG{p}{)}\PYG{p}{)}\PYG{p}{)}
\end{sphinxVerbatim}

\end{sphinxuseclass}\end{sphinxVerbatimInput}
\begin{sphinxVerbatimOutput}

\begin{sphinxuseclass}{cell_output}
\begin{sphinxVerbatim}[commandchars=\\\{\}]
Verzory: 
e1 = [ 0.40824829  0.40824829 \PYGZhy{}0.81649658] 
e2 = [\PYGZhy{}0.76834982 \PYGZhy{}0.32929278 \PYGZhy{}0.5488213 ] 
e3 = [\PYGZhy{}0.49292179  0.85141036  0.17924429]

Norm of každého z verzorů: 
||e1|| = 1.0 
||e2|| = 1.0 
||e3|| = 1.0

Test of ortogonality (pomocí vektorového součinu): 
e1 x e2: 1.0 
e1 x e3: 1.0000000000000002 
e2 x e3: 0.9999999999999999
\end{sphinxVerbatim}

\end{sphinxuseclass}\end{sphinxVerbatimOutput}

\end{sphinxuseclass}
\sphinxAtStartPar
Kvůli chybám zaokrouhlování (\sphinxhref{https://en.wikipedia.org/wiki/Round-off\_error}{blíže tady}) a nepřesnosti výpočtu s desetinnými čísly s plovoucí desetinnou čárkou (\sphinxhref{https://en.wikipedia.org/wiki/Round-off\_error}{blíže tady}) se normy přesně nerovnají 1.
V případě potřeby můžeme výsledek zaokrouhlit:

\begin{sphinxuseclass}{cell}\begin{sphinxVerbatimInput}

\begin{sphinxuseclass}{cell_input}
\begin{sphinxVerbatim}[commandchars=\\\{\}]
\PYG{n+nb}{print}\PYG{p}{(}\PYG{l+s+s1}{\PYGZsq{}}\PYG{l+s+se}{\PYGZbs{}n}\PYG{l+s+s1}{Test of orthogonality (pomocí vektorového součinu):}\PYG{l+s+s1}{\PYGZsq{}}\PYG{p}{,}
      \PYG{l+s+s1}{\PYGZsq{}}\PYG{l+s+se}{\PYGZbs{}n}\PYG{l+s+s1}{e1 x e2:}\PYG{l+s+s1}{\PYGZsq{}}\PYG{p}{,} \PYG{n}{np}\PYG{o}{.}\PYG{n}{linalg}\PYG{o}{.}\PYG{n}{norm}\PYG{p}{(}\PYG{n}{np}\PYG{o}{.}\PYG{n}{cross}\PYG{p}{(}\PYG{n}{e1}\PYG{p}{,} \PYG{n}{e2}\PYG{p}{)}\PYG{p}{)}\PYG{o}{.}\PYG{n}{round}\PYG{p}{(}\PYG{l+m+mi}{8}\PYG{p}{)}\PYG{p}{,}
      \PYG{l+s+s1}{\PYGZsq{}}\PYG{l+s+se}{\PYGZbs{}n}\PYG{l+s+s1}{e1 x e3:}\PYG{l+s+s1}{\PYGZsq{}}\PYG{p}{,} \PYG{n}{np}\PYG{o}{.}\PYG{n}{linalg}\PYG{o}{.}\PYG{n}{norm}\PYG{p}{(}\PYG{n}{np}\PYG{o}{.}\PYG{n}{cross}\PYG{p}{(}\PYG{n}{e1}\PYG{p}{,} \PYG{n}{e3}\PYG{p}{)}\PYG{p}{)}\PYG{o}{.}\PYG{n}{round}\PYG{p}{(}\PYG{l+m+mi}{8}\PYG{p}{)}\PYG{p}{,}
      \PYG{l+s+s1}{\PYGZsq{}}\PYG{l+s+se}{\PYGZbs{}n}\PYG{l+s+s1}{e2 x e3:}\PYG{l+s+s1}{\PYGZsq{}}\PYG{p}{,} \PYG{n}{np}\PYG{o}{.}\PYG{n}{linalg}\PYG{o}{.}\PYG{n}{norm}\PYG{p}{(}\PYG{n}{np}\PYG{o}{.}\PYG{n}{cross}\PYG{p}{(}\PYG{n}{e2}\PYG{p}{,} \PYG{n}{e3}\PYG{p}{)}\PYG{p}{)}\PYG{o}{.}\PYG{n}{round}\PYG{p}{(}\PYG{l+m+mi}{8}\PYG{p}{)}\PYG{p}{)}
\end{sphinxVerbatim}

\end{sphinxuseclass}\end{sphinxVerbatimInput}
\begin{sphinxVerbatimOutput}

\begin{sphinxuseclass}{cell_output}
\begin{sphinxVerbatim}[commandchars=\\\{\}]
Test of orthogonality (pomocí vektorového součinu): 
e1 x e2: 1.0 
e1 x e3: 1.0 
e2 x e3: 1.0
\end{sphinxVerbatim}

\end{sphinxuseclass}\end{sphinxVerbatimOutput}

\end{sphinxuseclass}
\sphinxAtStartPar
Nebo formátujte text představující výsledek:

\begin{sphinxuseclass}{cell}\begin{sphinxVerbatimInput}

\begin{sphinxuseclass}{cell_input}
\begin{sphinxVerbatim}[commandchars=\\\{\}]
\PYG{n+nb}{print}\PYG{p}{(}\PYG{l+s+s1}{\PYGZsq{}}\PYG{l+s+se}{\PYGZbs{}n}\PYG{l+s+s1}{Test of orthogonality (pomocí vektorového součinu):}\PYG{l+s+s1}{\PYGZsq{}}\PYG{p}{,}
      \PYG{l+s+sa}{f}\PYG{l+s+s1}{\PYGZsq{}}\PYG{l+s+se}{\PYGZbs{}n}\PYG{l+s+s1}{e1 x e2: }\PYG{l+s+si}{\PYGZob{}}\PYG{n}{np}\PYG{o}{.}\PYG{n}{linalg}\PYG{o}{.}\PYG{n}{norm}\PYG{p}{(}\PYG{n}{np}\PYG{o}{.}\PYG{n}{cross}\PYG{p}{(}\PYG{n}{e1}\PYG{p}{,}\PYG{+w}{ }\PYG{n}{e2}\PYG{p}{)}\PYG{p}{)}\PYG{l+s+si}{:}\PYG{l+s+s1}{g}\PYG{l+s+si}{\PYGZcb{}}\PYG{l+s+s1}{\PYGZsq{}}\PYG{p}{,}
      \PYG{l+s+sa}{f}\PYG{l+s+s1}{\PYGZsq{}}\PYG{l+s+se}{\PYGZbs{}n}\PYG{l+s+s1}{e1 x e3: }\PYG{l+s+si}{\PYGZob{}}\PYG{n}{np}\PYG{o}{.}\PYG{n}{linalg}\PYG{o}{.}\PYG{n}{norm}\PYG{p}{(}\PYG{n}{np}\PYG{o}{.}\PYG{n}{cross}\PYG{p}{(}\PYG{n}{e1}\PYG{p}{,}\PYG{+w}{ }\PYG{n}{e3}\PYG{p}{)}\PYG{p}{)}\PYG{l+s+si}{:}\PYG{l+s+s1}{g}\PYG{l+s+si}{\PYGZcb{}}\PYG{l+s+s1}{\PYGZsq{}}\PYG{p}{,}
      \PYG{l+s+sa}{f}\PYG{l+s+s1}{\PYGZsq{}}\PYG{l+s+se}{\PYGZbs{}n}\PYG{l+s+s1}{e2 x e3: }\PYG{l+s+si}{\PYGZob{}}\PYG{n}{np}\PYG{o}{.}\PYG{n}{linalg}\PYG{o}{.}\PYG{n}{norm}\PYG{p}{(}\PYG{n}{np}\PYG{o}{.}\PYG{n}{cross}\PYG{p}{(}\PYG{n}{e2}\PYG{p}{,}\PYG{+w}{ }\PYG{n}{e3}\PYG{p}{)}\PYG{p}{)}\PYG{l+s+si}{:}\PYG{l+s+s1}{g}\PYG{l+s+si}{\PYGZcb{}}\PYG{l+s+s1}{\PYGZsq{}}\PYG{p}{)}
\end{sphinxVerbatim}

\end{sphinxuseclass}\end{sphinxVerbatimInput}
\begin{sphinxVerbatimOutput}

\begin{sphinxuseclass}{cell_output}
\begin{sphinxVerbatim}[commandchars=\\\{\}]
Test of orthogonality (pomocí vektorového součinu): 
e1 x e2: 1 
e1 x e3: 1 
e2 x e3: 1
\end{sphinxVerbatim}

\end{sphinxuseclass}\end{sphinxVerbatimOutput}

\end{sphinxuseclass}
\sphinxAtStartPar
Kde jsme použili \sphinxcode{\sphinxupquote{f'\{:.g\}'}} k formátování čísla na řetězec (\sphinxhref{https://docs.python.org/dev/library/stdtypes.html\#printf-style-string-formatting}{blíže tady} \sphinxhref{https://docs.python.org/dev/library/stdtypes.html\#printf-style-string-formatting}{nebo tady}).

\sphinxAtStartPar
Pro trvalou změnu bychom mohli použít numpy nastavení pro reprezentaci přesnosti desetinné čárky s funkcí \sphinxhref{https://numpy.org/doc/stable/reference/generated/numpy.set\_printoptions.html}{\sphinxcode{\sphinxupquote{np.set\_printoptions}}}, ale ta funguje pouze pro numpy pole, ne s čísly. Nebo použijte kouzelný příkaz IPython \sphinxhref{https://numpy.org/doc/stable/reference/generated/numpy.set\_printoptions.html}{\sphinxcode{\sphinxupquote{np.set\_printoptions}}}, ale v tomto případě nemůžete použít funkci print pro zobrazení výsledku.


\subsubsection{Definice počátku}
\label{\detokenize{Prednasky/0_3_Sou_u0159adnicov_xe9_syst_xe9my:definice-pocatku}}
\sphinxAtStartPar
Abychom mohli definovat souřadnicový systém pomocí vypočítané báze, musíme také definovat počátek. V zásadě bychom mohli jako počátek použít jakýkoli bod, ale pokud by se měl vypočítaný souřadnicový systém řídit anatomickými konvencemi, např. počátek souřadnicového systému by měl být ve středu kloubu, budeme muset vypočítat bázi a počátek podle standardů používaných v analýze pohybu.

\sphinxAtStartPar
Pokud je souřadnicový systém technickým základem a nikoli anatomickým, běžným postupem při analýze pohybu je definovat počátek souřadnicového systému jako těžiště (průměrnou) polohu mezi značkami v referenčním rámci. Použití průměrné polohy napříč značkami potenciálně snižuje vliv šumu (například z artefaktu měkkých tkání) na výpočet.

\sphinxAtStartPar
Pro značky ve výše uvedeném příkladu bude počátek souřadného systému:

\begin{sphinxuseclass}{cell}\begin{sphinxVerbatimInput}

\begin{sphinxuseclass}{cell_input}
\begin{sphinxVerbatim}[commandchars=\\\{\}]
\PYG{n}{origin} \PYG{o}{=} \PYG{n}{np}\PYG{o}{.}\PYG{n}{mean}\PYG{p}{(}\PYG{p}{(}\PYG{n}{m1}\PYG{p}{,} \PYG{n}{m2}\PYG{p}{,} \PYG{n}{m3}\PYG{p}{)}\PYG{p}{,} \PYG{n}{axis}\PYG{o}{=}\PYG{l+m+mi}{0}\PYG{p}{)}
\PYG{n+nb}{print}\PYG{p}{(}\PYG{l+s+s1}{\PYGZsq{}}\PYG{l+s+s1}{Počátek: }\PYG{l+s+s1}{\PYGZsq{}}\PYG{p}{,} \PYG{n}{origin}\PYG{p}{)}
\end{sphinxVerbatim}

\end{sphinxuseclass}\end{sphinxVerbatimInput}
\begin{sphinxVerbatimOutput}

\begin{sphinxuseclass}{cell_output}
\begin{sphinxVerbatim}[commandchars=\\\{\}]
Počátek:  [2.33333333 1.66666667 3.33333333]
\end{sphinxVerbatim}

\end{sphinxuseclass}\end{sphinxVerbatimOutput}

\end{sphinxuseclass}
\sphinxAtStartPar
Nakreslete souřadnicový systém a základ pomocí vlastní funkce Pythonu \sphinxcode{\sphinxupquote{CCS.py}}.
Můžeme sem zkopírovat a vložit obsah souboru \sphinxcode{\sphinxupquote{CCS.py}} (a spustit buňku) nebo můžeme funkci načíst z adresáře, udělejme to. Nejprve vytvořte složku “functions” a umístěte do ní soubor \sphinxcode{\sphinxupquote{CCS.py}}.

\begin{sphinxuseclass}{cell}\begin{sphinxVerbatimInput}

\begin{sphinxuseclass}{cell_input}
\begin{sphinxVerbatim}[commandchars=\\\{\}]
\PYG{n}{path2} \PYG{o}{=} \PYG{l+s+sa}{r}\PYG{l+s+s1}{\PYGZsq{}}\PYG{l+s+s1}{./functions}\PYG{l+s+s1}{\PYGZsq{}}
\PYG{k+kn}{import} \PYG{n+nn}{sys}
\PYG{n}{sys}\PYG{o}{.}\PYG{n}{path}\PYG{o}{.}\PYG{n}{insert}\PYG{p}{(}\PYG{l+m+mi}{1}\PYG{p}{,} \PYG{n}{path2}\PYG{p}{)}  \PYG{c+c1}{\PYGZsh{} přidat do pythonpath}
\PYG{k+kn}{from} \PYG{n+nn}{CCS} \PYG{k+kn}{import} \PYG{n}{CCS}
\end{sphinxVerbatim}

\end{sphinxuseclass}\end{sphinxVerbatimInput}
\begin{sphinxVerbatimOutput}

\begin{sphinxuseclass}{cell_output}
\begin{sphinxVerbatim}[commandchars=\\\{\}]
\PYG{g+gt}{\PYGZhy{}\PYGZhy{}\PYGZhy{}\PYGZhy{}\PYGZhy{}\PYGZhy{}\PYGZhy{}\PYGZhy{}\PYGZhy{}\PYGZhy{}\PYGZhy{}\PYGZhy{}\PYGZhy{}\PYGZhy{}\PYGZhy{}\PYGZhy{}\PYGZhy{}\PYGZhy{}\PYGZhy{}\PYGZhy{}\PYGZhy{}\PYGZhy{}\PYGZhy{}\PYGZhy{}\PYGZhy{}\PYGZhy{}\PYGZhy{}\PYGZhy{}\PYGZhy{}\PYGZhy{}\PYGZhy{}\PYGZhy{}\PYGZhy{}\PYGZhy{}\PYGZhy{}\PYGZhy{}\PYGZhy{}\PYGZhy{}\PYGZhy{}\PYGZhy{}\PYGZhy{}\PYGZhy{}\PYGZhy{}\PYGZhy{}\PYGZhy{}\PYGZhy{}\PYGZhy{}\PYGZhy{}\PYGZhy{}\PYGZhy{}\PYGZhy{}\PYGZhy{}\PYGZhy{}\PYGZhy{}\PYGZhy{}\PYGZhy{}\PYGZhy{}\PYGZhy{}\PYGZhy{}\PYGZhy{}\PYGZhy{}\PYGZhy{}\PYGZhy{}\PYGZhy{}\PYGZhy{}\PYGZhy{}\PYGZhy{}\PYGZhy{}\PYGZhy{}\PYGZhy{}\PYGZhy{}\PYGZhy{}\PYGZhy{}\PYGZhy{}\PYGZhy{}}
\PYG{n+ne}{ModuleNotFoundError}\PYG{g+gWhitespace}{                       }Traceback (most recent call last)
\PYG{o}{/}\PYG{n}{tmp}\PYG{o}{/}\PYG{n}{ipykernel\PYGZus{}641092}\PYG{o}{/}\PYG{l+m+mf}{3498384649.}\PYG{n}{py} \PYG{o+ow}{in} \PYG{o}{\PYGZlt{}}\PYG{n}{module}\PYG{o}{\PYGZgt{}}
\PYG{g+gWhitespace}{      }\PYG{l+m+mi}{2} \PYG{k+kn}{import} \PYG{n+nn}{sys}
\PYG{g+gWhitespace}{      }\PYG{l+m+mi}{3} \PYG{n}{sys}\PYG{o}{.}\PYG{n}{path}\PYG{o}{.}\PYG{n}{insert}\PYG{p}{(}\PYG{l+m+mi}{1}\PYG{p}{,} \PYG{n}{path2}\PYG{p}{)}  \PYG{c+c1}{\PYGZsh{} přidat do pythonpath}
\PYG{n+ne}{\PYGZhy{}\PYGZhy{}\PYGZhy{}\PYGZhy{}\PYGZgt{} }\PYG{l+m+mi}{4} \PYG{k+kn}{from} \PYG{n+nn}{CCS} \PYG{k+kn}{import} \PYG{n}{CCS}

\PYG{n+ne}{ModuleNotFoundError}: No module named \PYGZsq{}CCS\PYGZsq{}
\end{sphinxVerbatim}

\end{sphinxuseclass}\end{sphinxVerbatimOutput}

\end{sphinxuseclass}
\begin{sphinxuseclass}{cell}\begin{sphinxVerbatimInput}

\begin{sphinxuseclass}{cell_input}
\begin{sphinxVerbatim}[commandchars=\\\{\}]
\PYG{n}{markers} \PYG{o}{=} \PYG{n}{np}\PYG{o}{.}\PYG{n}{vstack}\PYG{p}{(}\PYG{p}{(}\PYG{n}{m1}\PYG{p}{,} \PYG{n}{m2}\PYG{p}{,} \PYG{n}{m3}\PYG{p}{)}\PYG{p}{)}
\PYG{n}{basis} \PYG{o}{=} \PYG{n}{np}\PYG{o}{.}\PYG{n}{vstack}\PYG{p}{(}\PYG{p}{(}\PYG{n}{e1}\PYG{p}{,} \PYG{n}{e2}\PYG{p}{,} \PYG{n}{e3}\PYG{p}{)}\PYG{p}{)}
\end{sphinxVerbatim}

\end{sphinxuseclass}\end{sphinxVerbatimInput}

\end{sphinxuseclass}

\paragraph{Vizualizace souřadnicového systému}
\label{\detokenize{Prednasky/0_3_Sou_u0159adnicov_xe9_syst_xe9my:vizualizace-souradnicoveho-systemu}}
\sphinxAtStartPar
Abychom měli interaktivní graf matplotlib, musíme před prvním použitím nainstalovat knihovnu \sphinxcode{\sphinxupquote{ipympl}} a restartovat prostředí Google Colab.

\begin{sphinxuseclass}{cell}\begin{sphinxVerbatimInput}

\begin{sphinxuseclass}{cell_input}
\begin{sphinxVerbatim}[commandchars=\\\{\}]
\PYG{o}{!}pip3\PYG{+w}{ }install\PYG{+w}{ }\PYGZhy{}q\PYG{+w}{ }ipympl
\end{sphinxVerbatim}

\end{sphinxuseclass}\end{sphinxVerbatimInput}

\end{sphinxuseclass}
\begin{sphinxuseclass}{cell}\begin{sphinxVerbatimInput}

\begin{sphinxuseclass}{cell_input}
\begin{sphinxVerbatim}[commandchars=\\\{\}]
\PYG{o}{\PYGZpc{}}\PYG{k}{matplotlib} widget  

\PYG{n}{CCS}\PYG{p}{(}\PYG{n}{xyz}\PYG{o}{=}\PYG{p}{[}\PYG{p}{]}\PYG{p}{,} \PYG{n}{Oijk}\PYG{o}{=}\PYG{n}{origin}\PYG{p}{,} \PYG{n}{ijk}\PYG{o}{=}\PYG{n}{basis}\PYG{p}{,} \PYG{n}{point}\PYG{o}{=}\PYG{n}{markers}\PYG{p}{,} \PYG{n}{vector}\PYG{o}{=}\PYG{k+kc}{True}\PYG{p}{)}\PYG{p}{;}
\end{sphinxVerbatim}

\end{sphinxuseclass}\end{sphinxVerbatimInput}
\begin{sphinxVerbatimOutput}

\begin{sphinxuseclass}{cell_output}
\noindent\sphinxincludegraphics{{a2c51cd3807f60e6a0d22834d91cc41ec5cfa0a3f2af3d73a82ff793a92a555a}.png}

\end{sphinxuseclass}\end{sphinxVerbatimOutput}

\end{sphinxuseclass}

\subsubsection{Gramův–Schmidtův proces}
\label{\detokenize{Prednasky/0_3_Sou_u0159adnicov_xe9_syst_xe9my:gramuvschmidtuv-proces}}
\sphinxAtStartPar
Další klasický postup v matematice, využívající skalární součin, je známý jako \sphinxhref{http://en.wikipedia.org/wiki/Gram\%E2\%80\%93Schmidt\_process}{Gram–Schmidt process}. Podívejte se na wikipedii \sphinxhref{http://en.wikipedia.org/wiki/Gram\%E2\%80\%93Schmidt\_process}{Gram–Schmidt process}, kde je ukázka Gram–Schmidtova procesu a jak jej implementovat v Pythonu.

\sphinxAtStartPar
\sphinxhref{http://en.wikipedia.org/wiki/Gram\%E2\%80\%93Schmidt\_process}{Gram–Schmidt process} je metoda pro ortonormalizaci (ortogonální jednotkové versory) sady vektorů pomocí skalárního součinu. Gram\sphinxhyphen{}Schmidtův proces funguje pro libovolný počet vektorů.

\sphinxAtStartPar
Například za předpokladu tří vektorů 
\(\overrightarrow{\mathbf{a}}, \overrightarrow{\mathbf{b}}, \overrightarrow{\mathbf{c}}\), ve 3D prostoru lze základ \(\{\vec{e}_a, \vec{e}_b, \vec{e}_c\}\) nalézt pomocí procesu Gram–Schmidta

\sphinxAtStartPar
První versor je v 
\(\overrightarrow{\mathbf{a}}\) směr (nebo ve směru kteréhokoli z dalších vektorů):


\textbackslash{}begin\{equation\}
\textbackslash{}vec\{e\}\_a = \textbackslash{}frac\{\textbackslash{}overrightarrow\{\textbackslash{}mathbf\{a\}\}\}\{||\textbackslash{}overrightarrow\{\textbackslash{}mathbf\{a\}\}||\}
\textbackslash{}end\{equation\}


\sphinxAtStartPar
Druhý versor, ortogonální k \(\vec{e}_a\), lze nalézt za předpokladu, že můžeme vyjádřit vektor \(\overrightarrow{\mathbf{b}}\) ve směru \(\vec{e}_a\) jako:


\textbackslash{}begin\{equation\}
\textbackslash{}overrightarrow\{\textbackslash{}mathbf\{b\}\} = \textbackslash{}overrightarrow\{\textbackslash{}mathbf\{b\}\}\textasciicircum{}| + \textbackslash{}overrightarrow\{\textbackslash{}mathbf\{b\}\}\textasciicircum{}\textbackslash{}bot
\textbackslash{}end\{equation\}


\sphinxAtStartPar
Pak:



\sphinxAtStartPar
Konečně:



\sphinxAtStartPar
Třetí versor, ortogonální k \(\{\vec{e}_a, \vec{e}_b\}\), lze nalézt vyjadřující vektor \(\overrightarrow{\mathbf{C}}\) ve směrech \(\vec{e}_a\) a \(\vec{e}_b\) jako:


\textbackslash{}begin\{equation\}
\textbackslash{}overrightarrow\{\textbackslash{}mathbf\{c\}\} = \textbackslash{}overrightarrow\{\textbackslash{}mathbf\{c\}\}\textasciicircum{}| + \textbackslash{}overrightarrow\{\textbackslash{}mathbf\{c\}\}\textasciicircum{}\textbackslash{}bot
\textbackslash{}end\{equation\}


\sphinxAtStartPar
Pak:

\textbackslash{}begin\{equation\}
\textbackslash{}overrightarrow\{\textbackslash{}mathbf\{c\}\}\textasciicircum{}\textbackslash{}bot = \textbackslash{}overrightarrow\{\textbackslash{}mathbf\{c\}\} \sphinxhyphen{} \textbackslash{}overrightarrow\{\textbackslash{}mathbf\{c\}\}\textasciicircum{}|
\textbackslash{}end\{equation\}


\sphinxAtStartPar
Kde:

\textbackslash{}begin\{equation\}
\textbackslash{}overrightarrow\{\textbackslash{}mathbf\{c\}\}\textasciicircum{}| = (\textbackslash{}overrightarrow\{\textbackslash{}mathbf\{c\}\} \textbackslash{}cdot \textbackslash{}vec\{e\}\_a ) \textbackslash{}hat\{e\}\_a + (\textbackslash{}overrightarrow\{\textbackslash{}mathbf\{c\}\} \textbackslash{}cdot \textbackslash{}vec\{e\}\_b ) \textbackslash{}hat\{e\}\_b
\textbackslash{}end\{equation\}


\sphinxAtStartPar
Konečně:

\textbackslash{}begin\{equation\}
\textbackslash{}vec\{e\}\_c = \textbackslash{}frac\{\textbackslash{}overrightarrow\{\textbackslash{}mathbf\{c\}\}\textasciicircum{}\textbackslash{}bot\}\{||\textbackslash{}overrightarrow\{\textbackslash{}mathbf\{c\}\}\textasciicircum{}\textbackslash{}bot||\}
\textbackslash{}end\{equation\}


\sphinxAtStartPar
A animace probíhajícího Gram\sphinxhyphen{}Schmidtova procesu:



\sphinxAtStartPar
Pojďme implementovat Gram–Schmidtův proces v Pythonu.

\sphinxAtStartPar
Například vzhledem k pozicím, které jsme viděli dříve, \(\overrightarrow{\mathbf{m}}_1 = [1,2,5], \overrightarrow{\mathbf{m}}_2 = [2,3,3], \overrightarrow{\mathbf{m}}_3 = [4,0,2]\), základ lze nalézt s:

\sphinxAtStartPar
První verze je:

\begin{sphinxuseclass}{cell}\begin{sphinxVerbatimInput}

\begin{sphinxuseclass}{cell_input}
\begin{sphinxVerbatim}[commandchars=\\\{\}]
\PYG{n}{ea} \PYG{o}{=} \PYG{n}{m1}\PYG{o}{/}\PYG{n}{np}\PYG{o}{.}\PYG{n}{linalg}\PYG{o}{.}\PYG{n}{norm}\PYG{p}{(}\PYG{n}{m1}\PYG{p}{)}
\PYG{n+nb}{print}\PYG{p}{(}\PYG{n}{ea}\PYG{p}{)}
\end{sphinxVerbatim}

\end{sphinxuseclass}\end{sphinxVerbatimInput}
\begin{sphinxVerbatimOutput}

\begin{sphinxuseclass}{cell_output}
\begin{sphinxVerbatim}[commandchars=\\\{\}]
[0.18257419 0.36514837 0.91287093]
\end{sphinxVerbatim}

\end{sphinxuseclass}\end{sphinxVerbatimOutput}

\end{sphinxuseclass}
\sphinxAtStartPar
Druhý versor je:

\begin{sphinxuseclass}{cell}\begin{sphinxVerbatimInput}

\begin{sphinxuseclass}{cell_input}
\begin{sphinxVerbatim}[commandchars=\\\{\}]
\PYG{n}{eb} \PYG{o}{=} \PYG{n}{m2} \PYG{o}{\PYGZhy{}} \PYG{n}{np}\PYG{o}{.}\PYG{n}{dot}\PYG{p}{(}\PYG{n}{m2}\PYG{p}{,} \PYG{n}{ea}\PYG{p}{)}\PYG{o}{*}\PYG{n}{ea}
\PYG{n}{eb} \PYG{o}{=} \PYG{n}{eb}\PYG{o}{/}\PYG{n}{np}\PYG{o}{.}\PYG{n}{linalg}\PYG{o}{.}\PYG{n}{norm}\PYG{p}{(}\PYG{n}{eb}\PYG{p}{)}
\PYG{n+nb}{print}\PYG{p}{(}\PYG{n}{eb}\PYG{p}{)}
\end{sphinxVerbatim}

\end{sphinxuseclass}\end{sphinxVerbatimInput}
\begin{sphinxVerbatimOutput}

\begin{sphinxuseclass}{cell_output}
\begin{sphinxVerbatim}[commandchars=\\\{\}]
[ 0.59020849  0.70186955 \PYGZhy{}0.39878952]
\end{sphinxVerbatim}

\end{sphinxuseclass}\end{sphinxVerbatimOutput}

\end{sphinxuseclass}
\sphinxAtStartPar
A třetí verzor je:

\begin{sphinxuseclass}{cell}\begin{sphinxVerbatimInput}

\begin{sphinxuseclass}{cell_input}
\begin{sphinxVerbatim}[commandchars=\\\{\}]
\PYG{n}{ec} \PYG{o}{=} \PYG{n}{m3} \PYG{o}{\PYGZhy{}} \PYG{n}{np}\PYG{o}{.}\PYG{n}{dot}\PYG{p}{(}\PYG{n}{m3}\PYG{p}{,} \PYG{n}{ea}\PYG{p}{)}\PYG{o}{*}\PYG{n}{ea} \PYG{o}{\PYGZhy{}} \PYG{n}{np}\PYG{o}{.}\PYG{n}{dot}\PYG{p}{(}\PYG{n}{m3}\PYG{p}{,} \PYG{n}{eb}\PYG{p}{)}\PYG{o}{*}\PYG{n}{eb}
\PYG{n}{ec} \PYG{o}{=} \PYG{n}{ec}\PYG{o}{/}\PYG{n}{np}\PYG{o}{.}\PYG{n}{linalg}\PYG{o}{.}\PYG{n}{norm}\PYG{p}{(}\PYG{n}{ec}\PYG{p}{)}
\PYG{n+nb}{print}\PYG{p}{(}\PYG{n}{ec}\PYG{p}{)}
\end{sphinxVerbatim}

\end{sphinxuseclass}\end{sphinxVerbatimInput}
\begin{sphinxVerbatimOutput}

\begin{sphinxuseclass}{cell_output}
\begin{sphinxVerbatim}[commandchars=\\\{\}]
[ 0.78633365 \PYGZhy{}0.61159284  0.08737041]
\end{sphinxVerbatim}

\end{sphinxuseclass}\end{sphinxVerbatimOutput}

\end{sphinxuseclass}
\sphinxAtStartPar
Pojďme zkontrolovat ortonormalitu mezi těmito versory:

\begin{sphinxuseclass}{cell}\begin{sphinxVerbatimInput}

\begin{sphinxuseclass}{cell_input}
\begin{sphinxVerbatim}[commandchars=\\\{\}]
\PYG{n+nb}{print}\PYG{p}{(}\PYG{l+s+s2}{\PYGZdq{}}\PYG{l+s+s2}{ Versors:}\PYG{l+s+s2}{\PYGZdq{}}\PYG{p}{,} \PYG{l+s+s2}{\PYGZdq{}}\PYG{l+s+se}{\PYGZbs{}n}\PYG{l+s+s2}{ea =}\PYG{l+s+s2}{\PYGZdq{}}\PYG{p}{,} \PYG{n}{ea}\PYG{p}{,} \PYG{l+s+s2}{\PYGZdq{}}\PYG{l+s+se}{\PYGZbs{}n}\PYG{l+s+s2}{eb =}\PYG{l+s+s2}{\PYGZdq{}}\PYG{p}{,} \PYG{n}{eb}\PYG{p}{,} \PYG{l+s+s2}{\PYGZdq{}}\PYG{l+s+se}{\PYGZbs{}n}\PYG{l+s+s2}{ec =}\PYG{l+s+s2}{\PYGZdq{}}\PYG{p}{,} \PYG{n}{ec}\PYG{p}{)}

\PYG{n+nb}{print}\PYG{p}{(}
    \PYG{l+s+s2}{\PYGZdq{}}\PYG{l+s+se}{\PYGZbs{}n}\PYG{l+s+s2}{ Norma každého verzoru:}\PYG{l+s+s2}{\PYGZdq{}}\PYG{p}{,}
    \PYG{l+s+s2}{\PYGZdq{}}\PYG{l+s+se}{\PYGZbs{}n}\PYG{l+s+s2}{ ||ea|| =}\PYG{l+s+s2}{\PYGZdq{}}\PYG{p}{,}
    \PYG{n}{np}\PYG{o}{.}\PYG{n}{linalg}\PYG{o}{.}\PYG{n}{norm}\PYG{p}{(}\PYG{n}{ea}\PYG{p}{)}\PYG{p}{,}
    \PYG{l+s+s2}{\PYGZdq{}}\PYG{l+s+se}{\PYGZbs{}n}\PYG{l+s+s2}{ ||eb|| =}\PYG{l+s+s2}{\PYGZdq{}}\PYG{p}{,}
    \PYG{n}{np}\PYG{o}{.}\PYG{n}{linalg}\PYG{o}{.}\PYG{n}{norm}\PYG{p}{(}\PYG{n}{eb}\PYG{p}{)}\PYG{p}{,}
    \PYG{l+s+s2}{\PYGZdq{}}\PYG{l+s+se}{\PYGZbs{}n}\PYG{l+s+s2}{ ||ec|| =}\PYG{l+s+s2}{\PYGZdq{}}\PYG{p}{,}
    \PYG{n}{np}\PYG{o}{.}\PYG{n}{linalg}\PYG{o}{.}\PYG{n}{norm}\PYG{p}{(}\PYG{n}{ec}\PYG{p}{)}\PYG{p}{,}
\PYG{p}{)}

\PYG{n+nb}{print}\PYG{p}{(}
    \PYG{l+s+s2}{\PYGZdq{}}\PYG{l+s+se}{\PYGZbs{}n}\PYG{l+s+s2}{ Test ortogonality (skalární součin):}\PYG{l+s+s2}{\PYGZdq{}}\PYG{p}{,}
    \PYG{l+s+s2}{\PYGZdq{}}\PYG{l+s+se}{\PYGZbs{}n}\PYG{l+s+s2}{ ea . eb:}\PYG{l+s+s2}{\PYGZdq{}}\PYG{p}{,}
    \PYG{n}{np}\PYG{o}{.}\PYG{n}{dot}\PYG{p}{(}\PYG{n}{ea}\PYG{p}{,} \PYG{n}{eb}\PYG{p}{)}\PYG{p}{,}
    \PYG{l+s+s2}{\PYGZdq{}}\PYG{l+s+se}{\PYGZbs{}n}\PYG{l+s+s2}{ eb . ec:}\PYG{l+s+s2}{\PYGZdq{}}\PYG{p}{,}
    \PYG{n}{np}\PYG{o}{.}\PYG{n}{dot}\PYG{p}{(}\PYG{n}{eb}\PYG{p}{,} \PYG{n}{ec}\PYG{p}{)}\PYG{p}{,}
    \PYG{l+s+s2}{\PYGZdq{}}\PYG{l+s+se}{\PYGZbs{}n}\PYG{l+s+s2}{ ec . ea:}\PYG{l+s+s2}{\PYGZdq{}}\PYG{p}{,}
    \PYG{n}{np}\PYG{o}{.}\PYG{n}{dot}\PYG{p}{(}\PYG{n}{ec}\PYG{p}{,} \PYG{n}{ea}\PYG{p}{)}\PYG{p}{,}
\PYG{p}{)}
\end{sphinxVerbatim}

\end{sphinxuseclass}\end{sphinxVerbatimInput}
\begin{sphinxVerbatimOutput}

\begin{sphinxuseclass}{cell_output}
\begin{sphinxVerbatim}[commandchars=\\\{\}]
 Versors: 
ea = [0.18257419 0.36514837 0.91287093] 
eb = [ 0.59020849  0.70186955 \PYGZhy{}0.39878952] 
ec = [ 0.78633365 \PYGZhy{}0.61159284  0.08737041]

 Norma každého verzoru: 
 ||ea|| = 1.0 
 ||eb|| = 1.0 
 ||ec|| = 1.0

 Test ortogonality (skalární součin): 
 ea . eb: \PYGZhy{}3.3306690738754696e\PYGZhy{}16 
 eb . ec: 6.245004513516506e\PYGZhy{}17 
 ec . ea: 8.326672684688674e\PYGZhy{}17
\end{sphinxVerbatim}

\end{sphinxuseclass}\end{sphinxVerbatimOutput}

\end{sphinxuseclass}

\paragraph{Vizualizace souřadnicového systému}
\label{\detokenize{Prednasky/0_3_Sou_u0159adnicov_xe9_syst_xe9my:id1}}
\begin{sphinxuseclass}{cell}\begin{sphinxVerbatimInput}

\begin{sphinxuseclass}{cell_input}
\begin{sphinxVerbatim}[commandchars=\\\{\}]
\PYG{c+c1}{\PYGZsh{} zkus to sám}
\end{sphinxVerbatim}

\end{sphinxuseclass}\end{sphinxVerbatimInput}

\end{sphinxuseclass}

\subsection{Sférický souřadnicový systém}
\label{\detokenize{Prednasky/0_3_Sou_u0159adnicov_xe9_syst_xe9my:sfericky-souradnicovy-system}}
\sphinxAtStartPar
\sphinxstylestrong{Souřadnice sférické} jsou trojrozměrný souřadnicový systém, který využívá k určení polohy bodu v prostoru:
\begin{itemize}
\item {} 
\sphinxAtStartPar
\sphinxstylestrong{Radiální vzdálenost (\(\rho\)):} Vzdálenost bodu od počátku souřadnic.

\item {} 
\sphinxAtStartPar
\sphinxstylestrong{Azimutální úhel (\(\theta\)):} Úhel v rovině xy mezi kladnou osou x a projekcí bodu do této roviny.

\item {} 
\sphinxAtStartPar
\sphinxstylestrong{Zenitový úhel (\(\phi\)):} Úhel mezi kladnou osou z a spojnicí bodu s počátkem.

\end{itemize}

\sphinxAtStartPar
\sphinxincludegraphics{{Spherical-coordinate-system}.png}

\sphinxAtStartPar
\sphinxstylestrong{Báze sférických souřadnic}

\sphinxAtStartPar
Bázi sférických souřadnic tvoří tři jednotkové vektory:
\begin{itemize}
\item {} 
\sphinxAtStartPar
\sphinxstylestrong{\(\vec{\mathbf{e}}_\rho\):} Směřuje radiálně od počátku k bodu.

\item {} 
\sphinxAtStartPar
\sphinxstylestrong{\(\vec{\mathbf{e}}_\theta\):} Směřuje ve směru rostoucího azimutálního úhlu \(\theta\).

\item {} 
\sphinxAtStartPar
\sphinxstylestrong{\(\vec{\mathbf{e}}_\varphi\):} Směřuje ve směru rostoucího zenitového úhlu \(\varphi\).

\end{itemize}

\sphinxAtStartPar
Tato báze je ortonormální, což znamená, že vektory jsou vzájemně kolmé a mají jednotkovou velikost.

\sphinxAtStartPar
\sphinxstylestrong{Transformace z kartézských souřadnic}

\sphinxAtStartPar
Pro převod kartézských souřadnic (\(x, y, z\)) na sférické souřadnice (\(\rho, \theta, \varphi\)) platí následující vztahy:
\begin{itemize}
\item {} 
\sphinxAtStartPar
\(\rho = \sqrt{x^2 + y^2 + z^2}\)

\item {} 
\sphinxAtStartPar
\(\theta = \arctan{\frac{y}{x}}\)

\item {} 
\sphinxAtStartPar
\(\varphi = \arccos{\frac{z}{\rho}}\)

\end{itemize}

\sphinxAtStartPar
\sphinxstylestrong{Transformace do kartézských souřadnic}

\sphinxAtStartPar
Pro převod sférických souřadnic na kartézské souřadnice platí následující vztahy:
\begin{itemize}
\item {} 
\sphinxAtStartPar
\(x = \rho \sin{\varphi} \cos{\theta}\)

\item {} 
\sphinxAtStartPar
\(y = \rho \sin{\varphi} \sin{\theta}\)

\item {} 
\sphinxAtStartPar
\(z = \rho \cos{\varphi}\)

\end{itemize}

\sphinxAtStartPar
\sphinxstylestrong{Výhody sférických souřadnic:}
\begin{itemize}
\item {} 
\sphinxAtStartPar
\sphinxstylestrong{Symetrie:} Využívají se výhodně pro problémy s kulovou symetrií (např. gravitační pole, elektrostatické pole).

\item {} 
\sphinxAtStartPar
\sphinxstylestrong{Zjednodušení výpočtů:} V některých případech mohou zjednodušit řešení fyzikálních problémů.

\end{itemize}

\sphinxAtStartPar
\sphinxstylestrong{Nevýhody sférických souřadnic:}
\begin{itemize}
\item {} 
\sphinxAtStartPar
\sphinxstylestrong{Složitější derivace:} Výpočet gradientu, divergence a rotace v sférických souřadnicích je složitější než v kartézských.

\item {} 
\sphinxAtStartPar
\sphinxstylestrong{Singularity:} V počátku souřadnic (ρ = 0) a na pólech (φ = 0, π) mohou nastat singularity.

\end{itemize}

\begin{sphinxuseclass}{cell}\begin{sphinxVerbatimInput}

\begin{sphinxuseclass}{cell_input}
\begin{sphinxVerbatim}[commandchars=\\\{\}]
\PYG{n}{IFrame}\PYG{p}{(}\PYG{l+s+s1}{\PYGZsq{}}\PYG{l+s+s1}{https://www.geogebra.org/classic/c5nsjx3t?embed}\PYG{l+s+s1}{\PYGZsq{}}\PYG{p}{)}
\end{sphinxVerbatim}

\end{sphinxuseclass}\end{sphinxVerbatimInput}
\begin{sphinxVerbatimOutput}

\begin{sphinxuseclass}{cell_output}
\begin{sphinxVerbatim}[commandchars=\\\{\}]
\PYG{g+gt}{\PYGZhy{}\PYGZhy{}\PYGZhy{}\PYGZhy{}\PYGZhy{}\PYGZhy{}\PYGZhy{}\PYGZhy{}\PYGZhy{}\PYGZhy{}\PYGZhy{}\PYGZhy{}\PYGZhy{}\PYGZhy{}\PYGZhy{}\PYGZhy{}\PYGZhy{}\PYGZhy{}\PYGZhy{}\PYGZhy{}\PYGZhy{}\PYGZhy{}\PYGZhy{}\PYGZhy{}\PYGZhy{}\PYGZhy{}\PYGZhy{}\PYGZhy{}\PYGZhy{}\PYGZhy{}\PYGZhy{}\PYGZhy{}\PYGZhy{}\PYGZhy{}\PYGZhy{}\PYGZhy{}\PYGZhy{}\PYGZhy{}\PYGZhy{}\PYGZhy{}\PYGZhy{}\PYGZhy{}\PYGZhy{}\PYGZhy{}\PYGZhy{}\PYGZhy{}\PYGZhy{}\PYGZhy{}\PYGZhy{}\PYGZhy{}\PYGZhy{}\PYGZhy{}\PYGZhy{}\PYGZhy{}\PYGZhy{}\PYGZhy{}\PYGZhy{}\PYGZhy{}\PYGZhy{}\PYGZhy{}\PYGZhy{}\PYGZhy{}\PYGZhy{}\PYGZhy{}\PYGZhy{}\PYGZhy{}\PYGZhy{}\PYGZhy{}\PYGZhy{}\PYGZhy{}\PYGZhy{}\PYGZhy{}\PYGZhy{}\PYGZhy{}\PYGZhy{}}
\PYG{n+ne}{NameError}\PYG{g+gWhitespace}{                                 }Traceback (most recent call last)
\PYG{o}{/}\PYG{n}{tmp}\PYG{o}{/}\PYG{n}{ipykernel\PYGZus{}28856}\PYG{o}{/}\PYG{l+m+mf}{37843828.}\PYG{n}{py} \PYG{o+ow}{in} \PYG{o}{\PYGZlt{}}\PYG{n}{module}\PYG{o}{\PYGZgt{}}
\PYG{n+ne}{\PYGZhy{}\PYGZhy{}\PYGZhy{}\PYGZhy{}\PYGZgt{} }\PYG{l+m+mi}{1} \PYG{n}{IFrame}\PYG{p}{(}\PYG{l+s+s1}{\PYGZsq{}}\PYG{l+s+s1}{https://www.geogebra.org/classic/c5nsjx3t?embed}\PYG{l+s+s1}{\PYGZsq{}}\PYG{p}{)}

\PYG{n+ne}{NameError}: name \PYGZsq{}IFrame\PYGZsq{} is not defined
\end{sphinxVerbatim}

\end{sphinxuseclass}\end{sphinxVerbatimOutput}

\end{sphinxuseclass}

\subsection{Cylindrické souřadnice}
\label{\detokenize{Prednasky/0_3_Sou_u0159adnicov_xe9_syst_xe9my:cylindricke-souradnice}}
\sphinxAtStartPar
Cylindrické souřadnice jsou vhodné pro systémy s válcovou symetrií. Používají se tři souřadnice:
\begin{itemize}
\item {} 
\sphinxAtStartPar
\sphinxstylestrong{Radiální vzdálenost (ρ):} Vzdálenost bodu od osy z.

\item {} 
\sphinxAtStartPar
\sphinxstylestrong{Azimutální úhel (θ):} Úhel v rovině xy měřený od kladné osy x.

\item {} 
\sphinxAtStartPar
\sphinxstylestrong{Výška (z):} Vzdálenost bodu od roviny xy (stejná jako v kartézských souřadnicích).

\end{itemize}

\sphinxAtStartPar
\sphinxincludegraphics{{Coord_system_CY_1}.png}

\sphinxAtStartPar
\sphinxstylestrong{Transformační vztahy:}
\begin{itemize}
\item {} 
\sphinxAtStartPar
\sphinxstylestrong{Z cylindrických do kartézských:}
\begin{itemize}
\item {} 
\sphinxAtStartPar
\(x = \rho \cos{\theta}\)

\item {} 
\sphinxAtStartPar
\(y = \rho \sin{\theta}\)

\item {} 
\sphinxAtStartPar
\(z = z\)

\end{itemize}

\item {} 
\sphinxAtStartPar
\sphinxstylestrong{Z kartézských do cylindrických:}
\begin{itemize}
\item {} 
\sphinxAtStartPar
\(\rho = \sqrt{x^2 + y^2}\)

\item {} 
\sphinxAtStartPar
\(\theta = \arctan{\frac{y}{x}}\)

\item {} 
\sphinxAtStartPar
\(z = z\)

\end{itemize}

\end{itemize}

\begin{sphinxuseclass}{cell}\begin{sphinxVerbatimInput}

\begin{sphinxuseclass}{cell_input}
\begin{sphinxVerbatim}[commandchars=\\\{\}]
\PYG{n}{IFrame}\PYG{p}{(}\PYG{l+s+s1}{\PYGZsq{}}\PYG{l+s+s1}{https://www.geogebra.org/classic/X3j28ZkC?embed}\PYG{l+s+s1}{\PYGZsq{}}\PYG{p}{,} \PYG{n}{width}\PYG{o}{=}\PYG{l+s+s1}{\PYGZsq{}}\PYG{l+s+s1}{100}\PYG{l+s+s1}{\PYGZpc{}}\PYG{l+s+s1}{\PYGZsq{}}\PYG{p}{,} \PYG{n}{height}\PYG{o}{=}\PYG{l+m+mi}{500}\PYG{p}{)}
\end{sphinxVerbatim}

\end{sphinxuseclass}\end{sphinxVerbatimInput}

\end{sphinxuseclass}
\sphinxAtStartPar
\sphinxstylestrong{Srovnání:}


\begin{savenotes}\sphinxattablestart
\sphinxthistablewithglobalstyle
\centering
\begin{tabulary}{\linewidth}[t]{TTTT}
\sphinxtoprule
\sphinxstyletheadfamily 
\sphinxAtStartPar
Vlastnost
&\sphinxstyletheadfamily 
\sphinxAtStartPar
Kartézské
&\sphinxstyletheadfamily 
\sphinxAtStartPar
Sférické
&\sphinxstyletheadfamily 
\sphinxAtStartPar
Cylindrické
\\
\sphinxmidrule
\sphinxtableatstartofbodyhook
\sphinxAtStartPar
Symetrie
&
\sphinxAtStartPar
Obdélníková
&
\sphinxAtStartPar
Kulová
&
\sphinxAtStartPar
Válcová
\\
\sphinxhline
\sphinxAtStartPar
Souřadnice
&
\sphinxAtStartPar
x, y, z
&
\sphinxAtStartPar
ρ, θ, φ
&
\sphinxAtStartPar
ρ, θ, z
\\
\sphinxhline
\sphinxAtStartPar
Použití
&
\sphinxAtStartPar
Obecné
&
\sphinxAtStartPar
Kulová symetrie
&
\sphinxAtStartPar
Válcová symetrie
\\
\sphinxbottomrule
\end{tabulary}
\sphinxtableafterendhook\par
\sphinxattableend\end{savenotes}

\sphinxAtStartPar
\sphinxstylestrong{Kdy použít který systém?}
\begin{itemize}
\item {} 
\sphinxAtStartPar
\sphinxstylestrong{Kartézské:} Většina každodenních problémů, lineární pohyb.

\item {} 
\sphinxAtStartPar
\sphinxstylestrong{Sférické:} Problémy s kulovou symetrií (např. gravitace, elektrostatika).

\item {} 
\sphinxAtStartPar
\sphinxstylestrong{Cylindrické:} Problémy s válcovou symetrií (např. proudění tekutin v potrubí).

\end{itemize}

\sphinxAtStartPar
\sphinxstylestrong{Proč používat jiné souřadnicové systémy než kartézské?}
\begin{itemize}
\item {} 
\sphinxAtStartPar
\sphinxstylestrong{Zjednodušení výpočtů:} Problémy s určitou symetrií mohou být v jiných souřadnicových systémech jednodušší.

\item {} 
\sphinxAtStartPar
\sphinxstylestrong{Přirozenější popis:} Některé fyzikální jevy jsou přirozeněji popsány v jiných souřadnicových systémech.

\end{itemize}


\subsection{Další čtení}
\label{\detokenize{Prednasky/0_3_Sou_u0159adnicov_xe9_syst_xe9my:dalsi-cteni}}\begin{itemize}
\item {} 
\sphinxAtStartPar
\sphinxhref{https://www.researchgate.net/publication/267761615\_The\_right\_frame\_of\_reference\_makes\_it\_simple\_An\_example\_of\_introductory\_mechanics\_supported\_by\_video\_analysis\_of\_motion}{The right frame of reference makes it simple: An example of introductory mechanics supported by video analysis of motion}

\end{itemize}


\subsection{Video přednášky na internetu}
\label{\detokenize{Prednasky/0_3_Sou_u0159adnicov_xe9_syst_xe9my:video-prednasky-na-internetu}}\begin{itemize}
\item {} 
\sphinxAtStartPar
\sphinxhref{https://www.khanacademy.org/science/physics/one-dimensional-motion/displacement-velocity-time/v/introduction-to-reference-frames}{Introduction to reference \sphinxhyphen{} Khan Academy}

\item {} 
\sphinxAtStartPar
\sphinxhref{https://www.khanacademy.org/math/linear-algebra/alternate-bases/orthonormal-basis/v/linear-algebra-introduction-to-orthonormal-bases}{Introduction to orthonormal bases \sphinxhyphen{} Khan Academy}

\item {} 
\sphinxAtStartPar
\sphinxhref{https://www.khanacademy.org/math/linear-algebra/alternate-bases/orthonormal-basis/v/linear-algebra-the-gram-schmidt-process}{The Gram\sphinxhyphen{}Schmidt process \sphinxhyphen{} Khan Academy}

\item {} 
\sphinxAtStartPar
\sphinxhref{https://youtu.be/ctwoH59Obew}{Biomechanics of Movement | Demo: Motion Capture Placement and Reference Frames}

\end{itemize}


\subsection{Problémy}
\label{\detokenize{Prednasky/0_3_Sou_u0159adnicov_xe9_syst_xe9my:problemy}}\begin{enumerate}
\sphinxsetlistlabels{\arabic}{enumi}{enumii}{}{.}%
\item {} 
\sphinxAtStartPar
Jak rychle se právě teď pohybujete? Ve své odpovědi zvažte svůj pohyb ve vztahu k Zemi a ve vztahu ke Slunci.

\item {} 
\sphinxAtStartPar
Přejděte na webovou stránku \sphinxurl{http://www.wisc-online.com/Objects/ViewObject.aspx?ID=AP15305} a dokončete interaktivní lekci, kde se dozvíte o anatomické terminologii pro popis relativní polohy v lidském těle.

\item {} 
\sphinxAtStartPar
Chcete\sphinxhyphen{}li se dozvědět více o kartézských souřadnicových systémech, přejděte na webovou stránku \sphinxurl{http://www.mathsisfun.com/data/cartesian-coordinates.html}, prostudujte si materiál a odpovězte na 10 otázek na konci.

\item {} 
\sphinxAtStartPar
Zadané body v 3D prostoru, m1 = {[}2, 2, 0{]}, m2 = {[}0, 1, 1{]}, m3 = {[}1, 2, 0{]}, najděte ortonormální bázi.

\item {} 
\sphinxAtStartPar
Určete, zda následující body tvoří základ v 3D prostoru, m1 = {[}2, 2, 0{]}, m2 = {[}1, 1, 1{]}, m3 = {[}1, 1, 0{]}.

\item {} 
\sphinxAtStartPar
Odvoďte výrazy pro tři osy páneve s ohledem na konvenci projektu \sphinxhref{https://github.com/BMClab/BMC/blob/master/courses/refs/VAKHUM.pdf}{Virtual Animation of the Kinematics of the Human for Industrial, Educational and Research Purposes (VAKHUM)} (každý použijte RASIS, LASIS, RPSIS a LPSIS jako názvy pro axi orientační bod pánve a výraz).

\item {} 
\sphinxAtStartPar
Určete vektorovou bázi pro pánev podle konvence projektu \sphinxhref{https://github.com/BMClab/BMC/blob/master/courses/refs/VAKHUM.pdf}{Virtual Animation of the Kinematics of the Human for Industrial, Educational and Research Purposes (VAKHUM)} pro následující polohy anatomických orientačních bodů (jednotky v metrech): RASIS={[}0.5, 4{]} 0.8, 70 . 0,1{]}, RPSIS = {[}0,3, 0,85, 0,2{]}, LPSIS = {[}0,29, 0,78, 0,3{]}.

\end{enumerate}


\subsection{Reference}
\label{\detokenize{Prednasky/0_3_Sou_u0159adnicov_xe9_syst_xe9my:reference}}\begin{itemize}
\item {} 
\sphinxAtStartPar
Corke P (2017) \sphinxhref{https://petercorke.com/RVC/}{Robotics, Vision and Control: Fundamental Algorithms in MATLAB}. 2. vyd. Springer\sphinxhyphen{}Verlag Berlín.

\item {} 
\sphinxAtStartPar
\sphinxhref{https://isbweb.org/activities/standards}{Standards \sphinxhyphen{} International Society of Biomechanics}.

\item {} 
\sphinxAtStartPar
Stanfordská encyklopedie filozofie. \sphinxhref{http://plato.stanford.edu/entries/spacetime-iframes/}{Space and Time: Inertial Frames}.

\item {} 
\sphinxAtStartPar
\sphinxhref{https://github.com/BMClab/BMC/blob/master/courses/refs/VAKHUM.pdf}{Virtual Animation of the Kinematics of the Human for Industrial, Educational and Research Purposes (VAKHUM)}.

\end{itemize}

\sphinxstepscope


\section{Stupně volnosti}
\label{\detokenize{Prednasky/1_1_Stupe_u0148_volnosti:stupne-volnosti}}\label{\detokenize{Prednasky/1_1_Stupe_u0148_volnosti::doc}}\begin{quote}

\sphinxAtStartPar
\sphinxstylestrong{Stupeň volnosti} (DOF, degrees of freedom) je v mechanice počet nezávislých parametrů, které definují konfiguraci mechanického systému. Jinými slovy, je to počet nezávislých pohybů, které může těleso vykonávat.
\end{quote}
\begin{itemize}
\item {} 
\sphinxAtStartPar
V ploše má bod dva stupně volnosti
\begin{itemize}
\item {} 
\sphinxAtStartPar
posun podél osy x a podél osy y

\end{itemize}

\item {} 
\sphinxAtStartPar
V ploše má těleso pak tři stupně volnosti
\begin{itemize}
\item {} 
\sphinxAtStartPar
posun podél osy x, osy y

\item {} 
\sphinxAtStartPar
otočení kolem bodu.

\end{itemize}

\end{itemize}

\noindent{\hspace*{\fill}\sphinxincludegraphics[width=200\sphinxpxdimen]{{DOF-2d}.png}\hspace*{\fill}}
\begin{itemize}
\item {} 
\sphinxAtStartPar
V prostoru má bod tři stupně volnosti
\begin{itemize}
\item {} 
\sphinxAtStartPar
posuny podél os x, y, z

\end{itemize}

\item {} 
\sphinxAtStartPar
V trojrozměrném prostoru má tuhé těleso 6 stupňů volnosti:
\begin{itemize}
\item {} 
\sphinxAtStartPar
3 posuvné (pohyb podél os x, y, z)

\item {} 
\sphinxAtStartPar
3 rotační (rotace kolem os x, y, z)

\end{itemize}

\end{itemize}

\noindent{\hspace*{\fill}\sphinxincludegraphics[width=200\sphinxpxdimen]{{DOF-3d}.png}\hspace*{\fill}}


\begin{savenotes}\sphinxattablestart
\sphinxthistablewithglobalstyle
\centering
\begin{tabulary}{\linewidth}[t]{TTTT}
\sphinxtoprule
\sphinxstyletheadfamily 
\sphinxAtStartPar
Dimenzionalita
&\sphinxstyletheadfamily 
\sphinxAtStartPar
Objekt
&\sphinxstyletheadfamily 
\sphinxAtStartPar
Počet stupňů volnosti
&\sphinxstyletheadfamily 
\sphinxAtStartPar
Popis
\\
\sphinxmidrule
\sphinxtableatstartofbodyhook
\sphinxAtStartPar
2D
&
\sphinxAtStartPar
Bod
&
\sphinxAtStartPar
2
&
\sphinxAtStartPar
Posunutí v ose x a y
\\
\sphinxhline
\sphinxAtStartPar
2D
&
\sphinxAtStartPar
Těleso
&
\sphinxAtStartPar
3
&
\sphinxAtStartPar
Posunutí v ose x a y, rotace kolem osy z
\\
\sphinxhline
\sphinxAtStartPar
3D
&
\sphinxAtStartPar
Bod
&
\sphinxAtStartPar
3
&
\sphinxAtStartPar
Posunutí v ose x, y a z
\\
\sphinxhline
\sphinxAtStartPar
3D
&
\sphinxAtStartPar
Těleso
&
\sphinxAtStartPar
6
&
\sphinxAtStartPar
Posunutí v ose x, y a z, rotace kolem os x, y a z
\\
\sphinxbottomrule
\end{tabulary}
\sphinxtableafterendhook\par
\sphinxattableend\end{savenotes}


\section{Vazba}
\label{\detokenize{Prednasky/1_1_Stupe_u0148_volnosti:vazba}}\begin{quote}

\sphinxAtStartPar
\sphinxstylestrong{Vazba} je omezení, které omezuje nebo zakazuje některé stupně volnosti tělesa.
\end{quote}
\begin{itemize}
\item {} 
\sphinxAtStartPar
\sphinxstylestrong{Geometrická omezení}: Pevné spojení mezi tělesy, kontaktní plochy, klouby.

\item {} 
\sphinxAtStartPar
\sphinxstylestrong{Silová omezení}: Působení vnějších sil, které omezují pohyb (např. tíhová síla).

\item {} 
\sphinxAtStartPar
\sphinxstylestrong{Kinematická omezení}: Omezení daná předepsaným pohybem částí systému.

\end{itemize}


\subsection{Příklady vazeb}
\label{\detokenize{Prednasky/1_1_Stupe_u0148_volnosti:priklady-vazeb}}\begin{itemize}
\item {} 
\sphinxAtStartPar
\sphinxstylestrong{Kloub}: Kloub umožňuje rotaci kolem jedné osy, ale omezuje posunutí v ostatních směrech. Příkladem je kloub dveří nebo kloub v koleni.

\item {} 
\sphinxAtStartPar
\sphinxstylestrong{Posuvné uložení}: Posuvné uložení umožňuje posunutí v jednom směru, ale omezuje posunutí v ostatních směrech a rotaci. Příkladem je píst v motoru.

\item {} 
\sphinxAtStartPar
\sphinxstylestrong{Vetknutí}: Vetknutí omezuje všechny posuvy a rotace v daném bodě. Příkladem je trám vetknutý do zdi.

\item {} 
\sphinxAtStartPar
\sphinxstylestrong{Lano}: Lano omezuje posunutí ve směru tahu lana, ale neomezuje posunutí v kolmém směru.

\item {} 
\sphinxAtStartPar
\sphinxstylestrong{Podpora}: Podpora omezuje posunutí v jednom nebo více směrech, v závislosti na typu podpory. Příkladem je podpora mostu.

\item {} 
\sphinxAtStartPar
\sphinxstylestrong{Kontaktní plocha}: Kontaktní plocha mezi dvěma tělesy omezuje posunutí ve směru kolmém k ploše.

\end{itemize}


\subsection{Dělení vazeb}
\label{\detokenize{Prednasky/1_1_Stupe_u0148_volnosti:deleni-vazeb}}
\sphinxAtStartPar
Vazby můžeme dělit podle různých kritérií:
\begin{itemize}
\item {} 
\sphinxAtStartPar
\sphinxstylestrong{Podle způsobu omezení}:
\begin{itemize}
\item {} 
\sphinxAtStartPar
\sphinxstylestrong{Holonomní vazby}: Omezují polohu tělesa.

\item {} 
\sphinxAtStartPar
\sphinxstylestrong{Neholonomní vazby}: Omezují rychlost nebo zrychlení tělesa.

\item {} 
\sphinxAtStartPar
\sphinxstylestrong{Skleronomní vazby}: Nezávisejí na čase.

\item {} 
\sphinxAtStartPar
\sphinxstylestrong{Reonomní vazby}: Závisejí na čase.

\end{itemize}

\item {} 
\sphinxAtStartPar
\sphinxstylestrong{Podle disipace energie}:
\begin{itemize}
\item {} 
\sphinxAtStartPar
\sphinxstylestrong{Ideální vazby}: Nevykonávají práci.

\item {} 
\sphinxAtStartPar
\sphinxstylestrong{Reálné vazby}: Vykonávají práci (např. tření).

\end{itemize}

\item {} 
\sphinxAtStartPar
\sphinxstylestrong{Podle počtu omezení}:
\begin{itemize}
\item {} 
\sphinxAtStartPar
\sphinxstylestrong{Jednostranné vazby}: Omezují pohyb pouze v jednom směru.

\item {} 
\sphinxAtStartPar
\sphinxstylestrong{Oboustranné vazby}: Omezují pohyb v obou směrech.

\end{itemize}

\end{itemize}


\subsubsection{Staticky určitý a neurčitý systém}
\label{\detokenize{Prednasky/1_1_Stupe_u0148_volnosti:staticky-urcity-a-neurcity-system}}
\sphinxAtStartPar
Pro to, aby těleso bylo ve statické rovnováze, musí být všechny možné pohyby adekvátně omezeny. Pokud stupeň volnosti není omezen, těleso je v nestabilním stavu a může se volně pohybovat jedním nebo více způsoby. Stabilita je velmi žádoucí z důvodů bezpečnosti a tělesa jsou často omezena redundantními omezeními, takže i kdyby jedno selhalo, těleso by stále zůstalo stabilní.  Pokud jsou omezení správně interpretována, pak rovné počty omezení a stupňů volnosti vytvářejí stabilní systém a hodnoty reakčních sil a momentů lze určit pomocí rovnic rovnováhy. Pokud počet omezení překračuje počet stupňů volnosti, těleso je v rovnováze, ale k určení reakcí budete potřebovat techniky, které jdou za rámec rovnic rovnováhy ve statice.

\sphinxAtStartPar
V mechanice rozlišujeme dva základní typy statických systémů podle vztahu mezi počtem neznámých reakcí a počtem rovnic statické rovnováhy:
\begin{enumerate}
\sphinxsetlistlabels{\arabic}{enumi}{enumii}{}{.}%
\item {} 
\sphinxAtStartPar
\sphinxstylestrong{Staticky určitý systém}
\begin{itemize}
\item {} 
\sphinxAtStartPar
\sphinxstylestrong{Definice:} Systém, kde počet neznámých reakčních sil a momentů je \sphinxstylestrong{roven} počtu rovnic statické rovnováhy.

\item {} 
\sphinxAtStartPar
\sphinxstylestrong{Řešení:} Všechny neznámé reakce lze stanovit pouze pomocí rovnic statické rovnováhy.

\item {} 
\sphinxAtStartPar
\sphinxstylestrong{Příklad:} Prostý nosník s jednou kloubovou a jednou posuvnou podporou.

\end{itemize}

\item {} 
\sphinxAtStartPar
\sphinxstylestrong{Staticky neurčitý systém}
\begin{itemize}
\item {} 
\sphinxAtStartPar
\sphinxstylestrong{Definice:} Systém, kde počet neznámých reakčních sil a momentů je \sphinxstylestrong{větší} než počet rovnic statické rovnováhy.

\item {} 
\sphinxAtStartPar
\sphinxstylestrong{Řešení:} K stanovení všech neznámých reakcí nestačí pouze rovnice statické rovnováhy. Je nutné zohlednit deformační chování materiálu a geometrii systému.

\item {} 
\sphinxAtStartPar
\sphinxstylestrong{Metody řešení:} rovnice komtabilitiy, metoda konečných prvků.

\item {} 
\sphinxAtStartPar
\sphinxstylestrong{Příklad:} Spojitý nosník s více podporami.

\item {} 
\sphinxAtStartPar
\sphinxstylestrong{Stupeň statické neurčitosti:} Rozdíl mezi počtem neznámých a počtem rovnic.

\end{itemize}

\end{enumerate}


\begin{savenotes}\sphinxattablestart
\sphinxthistablewithglobalstyle
\centering
\begin{tabulary}{\linewidth}[t]{TTTT}
\sphinxtoprule
\sphinxstyletheadfamily 
\sphinxAtStartPar
Typ systému
&\sphinxstyletheadfamily 
\sphinxAtStartPar
Charakteristika
&\sphinxstyletheadfamily 
\sphinxAtStartPar
Řešení
&\sphinxstyletheadfamily 
\sphinxAtStartPar
Příklady
\\
\sphinxmidrule
\sphinxtableatstartofbodyhook
\sphinxAtStartPar
Staticky určitý
&
\sphinxAtStartPar
Počet neznámých = počet rovnic
&
\sphinxAtStartPar
Rovnice rovnováhy stačí
&
\sphinxAtStartPar
Prostý nosník s 2 podporami
\\
\sphinxhline
\sphinxAtStartPar
Staticky neurčitý
&
\sphinxAtStartPar
Počet neznámých > počet rovnic
&
\sphinxAtStartPar
Nutné deformační podmínky
&
\sphinxAtStartPar
Vetknutý nosník s podpěrou
\\
\sphinxbottomrule
\end{tabulary}
\sphinxtableafterendhook\par
\sphinxattableend\end{savenotes}

\sphinxstepscope


\chapter{Kinematika v 1DOF prostoru}
\label{\detokenize{Prednasky/1_2_Kinematika_v_1D:kinematika-v-1dof-prostoru}}\label{\detokenize{Prednasky/1_2_Kinematika_v_1D::doc}}
\begin{sphinxuseclass}{cell}\begin{sphinxVerbatimInput}

\begin{sphinxuseclass}{cell_input}
\begin{sphinxVerbatim}[commandchars=\\\{\}]
\PYG{k+kn}{import} \PYG{n+nn}{numpy} \PYG{k}{as} \PYG{n+nn}{np}
\PYG{k+kn}{import} \PYG{n+nn}{matplotlib}\PYG{n+nn}{.}\PYG{n+nn}{pyplot} \PYG{k}{as} \PYG{n+nn}{plt}
\PYG{k+kn}{from} \PYG{n+nn}{IPython}\PYG{n+nn}{.}\PYG{n+nn}{display} \PYG{k+kn}{import} \PYG{n}{IFrame}
\end{sphinxVerbatim}

\end{sphinxuseclass}\end{sphinxVerbatimInput}

\end{sphinxuseclass}

\section{Poloha v 1D}
\label{\detokenize{Prednasky/1_2_Kinematika_v_1D:poloha-v-1d}}
\sphinxAtStartPar
1D prostor, neboli jednorozměrný prostor, je koncept, který se používá k popisu objektů nebo jevů, které mají pouze jednu dimenzi. Představte si přímku: ta je nekonečně dlouhá, ale nemá žádnou šířku ani hloubku. Vše, co se v 1D prostoru nachází, se dá popsat pouze pomocí jedné souřadnice.

\sphinxAtStartPar
Za 1D prostor můžeme považovat také prostor, kdy pro popsání polohý nám stačí jedna souřadnice. I když se auto pohybuje v 3D prostoru, pro popsání jeho polohy na dálnici nám stačí pouze jedna souřadnice – vzdálenost od počátečního bodu dálnice. Poloha je omezena na jednu dimenzi, i když se odehrává v prostoru s vyšší dimenzí.

\begin{sphinxadmonition}{warning}{Warning:}
\sphinxAtStartPar
V případě obecného křivočarého pohybu vznikají neinerciální zrychlení a síly, např dostředivé zrychlení při pohybu po kružnici.
\end{sphinxadmonition}

\sphinxAtStartPar
\sphinxstylestrong{Jak si představit 1D prostor}
\begin{itemize}
\item {} 
\sphinxAtStartPar
\sphinxstyleemphasis{Přímka:} Nejjednodušší způsob, jak si představit 1D prostor, je jako přímku. Každý bod na této přímce je určen pouze jednou souřadnicí, která udává jeho vzdálenost od počátku.

\item {} 
\sphinxAtStartPar
\sphinxstyleemphasis{Časová osa:} Časová osa je také příkladem 1D prostoru. Události se odehrávají v čase a dají se popsat pouze jednou souřadnicí – časem.

\item {} 
\sphinxAtStartPar
\sphinxstyleemphasis{Struna:} Vibrující struna je fyzikální příklad 1D prostoru. Pohyb bodů na struně se dá popsat pouze jednou souřadnicí – vzdáleností od jednoho konce struny.

\end{itemize}

\sphinxAtStartPar
\sphinxstylestrong{Matematický popis 1D prostoru}

\sphinxAtStartPar
1D prostor se matematicky popisuje pomocí:
\begin{itemize}
\item {} 
\sphinxAtStartPar
\sphinxstylestrong{Reálných čísel:} Každý bod v 1D prostoru je reprezentován reálným číslem.

\item {} 
\sphinxAtStartPar
\sphinxstylestrong{Souřadnicové osy:} 1D prostor má pouze jednu souřadnicovou osu, obvykle označenou jako osa x.

\item {} 
\sphinxAtStartPar
\sphinxstylestrong{Vzdálenosti:} Vzdálenost mezi dvěma body v 1D prostoru se vypočítá jako absolutní hodnota rozdílu jejich souřadnic.

\end{itemize}


\section{Pohyb v 1D}
\label{\detokenize{Prednasky/1_2_Kinematika_v_1D:pohyb-v-1d}}
\sphinxAtStartPar
Pro další popis budeme vycházet z definice pohybu. Samotnou změnu fyzikální veličiny označujeme v mechanice symbolem \(\Delta\). Například pro změnu času označíme jako \(\Delta t\). Znamená to:
\begin{equation*}
\begin{split}\Delta t = t_2 - t_1\end{split}
\end{equation*}
\sphinxAtStartPar
Začneme\sphinxhyphen{}li pohyb v čase 12 hodin 20 minut a ukočníme ho v čase 12 hodin 21 minut, trval daný pohyb 1 minutu. S popisem změny polohy je to o něco komplikovanější vzhledem k rozdělení pohybu. Kůli jednoduchosti budeme uvažovat nejjednodušší způsob měření pohybu u pohybu posuvného. Pohyb po kružnici si probereme samostatně v části periodického pohybu.


\subsection{Translační pohyb}
\label{\detokenize{Prednasky/1_2_Kinematika_v_1D:translacni-pohyb}}
\sphinxAtStartPar
Za nejjednodušší posuvný pohyb budeme považovat pohyb běžce po přímé dráze. Jakým způsobem by jsme mohli určit jeho polohu? Příkladem může být původní měření běhu.

\sphinxAtStartPar
\sphinxincludegraphics{{analyze-run-form}.jpg}


\subsection{Souřadnicová osa}
\label{\detokenize{Prednasky/1_2_Kinematika_v_1D:souradnicova-osa}}\begin{itemize}
\item {} 
\sphinxAtStartPar
\sphinxstylestrong{Počátek souřadnic:} Zvolený bod na přímce, od kterého měříme vzdálenosti.

\item {} 
\sphinxAtStartPar
\sphinxstylestrong{Kladný směr:} Směr, ve kterém rostou kladné hodnoty souřadnice.

\item {} 
\sphinxAtStartPar
\sphinxstylestrong{Záporný směr:} Směr opačný ke kladnému směru.

\end{itemize}

\sphinxAtStartPar
\sphinxincludegraphics{{original_13}.png}


\subsection{Poloha bodu}
\label{\detokenize{Prednasky/1_2_Kinematika_v_1D:poloha-bodu}}\begin{itemize}
\item {} 
\sphinxAtStartPar
\sphinxstylestrong{Souřadnice:} Číselná hodnota udávající vzdálenost bodu od počátku souřadnic ve zvoleném směru. V případě pohybu po přímce nám stačí jedna hodnota \(x\).

\end{itemize}


\subsection{Změna polohy}
\label{\detokenize{Prednasky/1_2_Kinematika_v_1D:zmena-polohy}}\begin{itemize}
\item {} 
\sphinxAtStartPar
\sphinxstylestrong{Posunutí:} Změna polohy bodu v čase. Je to vektorová veličina, která má velikost (dráhu) a směr. Označujeme jí \(\vec{\mathbf{d}}\).

\end{itemize}


\subsection{Zápis pohybu}
\label{\detokenize{Prednasky/1_2_Kinematika_v_1D:zapis-pohybu}}
\sphinxAtStartPar
Možný zápis pohybu je tabulkou, kde vyjádříme jednotlivé body


\begin{savenotes}\sphinxattablestart
\sphinxthistablewithglobalstyle
\centering
\begin{tabulary}{\linewidth}[t]{TT}
\sphinxtoprule
\sphinxstyletheadfamily 
\sphinxAtStartPar
Čas \(t\) {[}s{]}
&\sphinxstyletheadfamily 
\sphinxAtStartPar
Poloha \(x\) {[}m{]}
\\
\sphinxmidrule
\sphinxtableatstartofbodyhook
\sphinxAtStartPar
0
&
\sphinxAtStartPar
1
\\
\sphinxhline
\sphinxAtStartPar
1
&
\sphinxAtStartPar
3
\\
\sphinxhline
\sphinxAtStartPar
2
&
\sphinxAtStartPar
5
\\
\sphinxhline
\sphinxAtStartPar
3
&
\sphinxAtStartPar
5
\\
\sphinxhline
\sphinxAtStartPar
4
&
\sphinxAtStartPar
2
\\
\sphinxbottomrule
\end{tabulary}
\sphinxtableafterendhook\par
\sphinxattableend\end{savenotes}

\sphinxAtStartPar
Zobrazení pomocí tabulky může být nepřehledné když máme velké množství bodů. Proto je výhodnější po popis polohy použít graf. Při spojení jednotlivých naměřených bodů vždy pohyb aproximujeme.

\begin{sphinxuseclass}{cell}\begin{sphinxVerbatimInput}

\begin{sphinxuseclass}{cell_input}
\begin{sphinxVerbatim}[commandchars=\\\{\}]
\PYG{n}{t} \PYG{o}{=} \PYG{p}{[}\PYG{l+m+mi}{0}\PYG{p}{,}\PYG{l+m+mi}{1}\PYG{p}{,}\PYG{l+m+mi}{2}\PYG{p}{,}\PYG{l+m+mi}{3}\PYG{p}{,}\PYG{l+m+mi}{4}\PYG{p}{]}
\PYG{n}{x} \PYG{o}{=} \PYG{p}{[}\PYG{l+m+mi}{1}\PYG{p}{,}\PYG{l+m+mi}{3}\PYG{p}{,}\PYG{l+m+mi}{5}\PYG{p}{,}\PYG{l+m+mi}{5}\PYG{p}{,}\PYG{l+m+mi}{2}\PYG{p}{]}

\PYG{n}{plt}\PYG{o}{.}\PYG{n}{plot}\PYG{p}{(}\PYG{n}{t}\PYG{p}{,}\PYG{n}{x}\PYG{p}{,} \PYG{l+s+s2}{\PYGZdq{}}\PYG{l+s+s2}{o}\PYG{l+s+s2}{\PYGZdq{}}\PYG{p}{,} \PYG{n}{label}\PYG{o}{=}\PYG{l+s+s2}{\PYGZdq{}}\PYG{l+s+s2}{Naměřené hodnoty}\PYG{l+s+s2}{\PYGZdq{}}\PYG{p}{)}
\PYG{n}{plt}\PYG{o}{.}\PYG{n}{plot}\PYG{p}{(}\PYG{n}{t}\PYG{p}{,}\PYG{n}{x}\PYG{p}{,} \PYG{l+s+s2}{\PYGZdq{}}\PYG{l+s+s2}{\PYGZhy{}\PYGZhy{}}\PYG{l+s+s2}{\PYGZdq{}}\PYG{p}{,} \PYG{n}{label}\PYG{o}{=}\PYG{l+s+s2}{\PYGZdq{}}\PYG{l+s+s2}{Aproximace pohybu}\PYG{l+s+s2}{\PYGZdq{}}\PYG{p}{)}
\PYG{n}{plt}\PYG{o}{.}\PYG{n}{xlabel}\PYG{p}{(}\PYG{l+s+s2}{\PYGZdq{}}\PYG{l+s+s2}{Čas [s]}\PYG{l+s+s2}{\PYGZdq{}}\PYG{p}{)}
\PYG{n}{plt}\PYG{o}{.}\PYG{n}{ylabel}\PYG{p}{(}\PYG{l+s+s2}{\PYGZdq{}}\PYG{l+s+s2}{Poloha [m]}\PYG{l+s+s2}{\PYGZdq{}}\PYG{p}{)}
\PYG{n}{plt}\PYG{o}{.}\PYG{n}{legend}\PYG{p}{(}\PYG{p}{)}
\PYG{n}{plt}\PYG{o}{.}\PYG{n}{show}\PYG{p}{(}\PYG{p}{)}
\end{sphinxVerbatim}

\end{sphinxuseclass}\end{sphinxVerbatimInput}
\begin{sphinxVerbatimOutput}

\begin{sphinxuseclass}{cell_output}
\noindent\sphinxincludegraphics{{ada86577407f7ab14e062334eaa8f47b7422139ff5751b10ef8d8e9566923e85}.png}

\end{sphinxuseclass}\end{sphinxVerbatimOutput}

\end{sphinxuseclass}
\begin{sphinxadmonition}{warning}{Warning:}
\sphinxAtStartPar
V případě malého počtu naměřených bodů nebo jejich malé frekvence měření vůči pohybu je aproximace pohybu nepřesná.
\end{sphinxadmonition}


\section{Kinematika}
\label{\detokenize{Prednasky/1_2_Kinematika_v_1D:kinematika}}
\sphinxAtStartPar
\sphinxstylestrong{Kinematika} je odvětví klasické mechaniky, které popisuje pohyb objektů bez uvážení příčin pohybu (\sphinxhref{http://en.wikipedia.org/wiki/Kinematics}{Wikipedia}).

\sphinxAtStartPar
Kinematika částice je popis pohybu, když je objekt považován za částici.

\sphinxAtStartPar
Částice jako fyzický objekt v přírodě neexistuje; je to zjednodušení pro pochopení pohybu tělesa nebo je to pojmová definice, jako je těžiště soustavy objektů.

\sphinxAtStartPar
\sphinxstylestrong{Dráha, rychlost a zrychlení} jsou tři základní veličiny popisující pohyb tělesa. Mezi nimi existují úzké vztahy, které nám umožňují popsat a analyzovat pohyb tělesa.


\subsection{Dráha}
\label{\detokenize{Prednasky/1_2_Kinematika_v_1D:draha}}\begin{itemize}
\item {} 
\sphinxAtStartPar
\sphinxstylestrong{Dráha} je délka trajektorie, kterou urazí těleso při svém pohybu. Je to skalární veličina a značí se obvykle písmenem \sphinxstyleemphasis{s}.

\end{itemize}


\subsection{Rychlost}
\label{\detokenize{Prednasky/1_2_Kinematika_v_1D:rychlost}}
\sphinxAtStartPar
Průměrná rychlost mezi dvěma okamžiky je:
\begin{equation*}
\begin{split}\overrightarrow{\mathbf{v_x}}(t) = \frac{\overrightarrow{\mathbf{x}}(t_2)-\overrightarrow{\mathbf{x}}(t_1)}{t_2-t_1} = \frac{\Delta \overrightarrow{\mathbf{x}}}{\Delta t}\end{split}
\end{equation*}
\sphinxAtStartPar
Okamžitá rychlost částice se získá, když se \(\Delta t\) přiblíží k nule.
\begin{equation*}
\begin{split}\overrightarrow{\mathbf{v_x}}(t) = \lim_{\Delta t \to 0} \frac{\Delta \overrightarrow{\mathbf{x}}}{\Delta t} = \lim_{\Delta t \to 0} \frac{\overrightarrow{\mathbf{x}}(t+\Delta t)-\overrightarrow{\mathbf{r}}(t)}{\Delta t} = \frac{\mathrm{d}\overrightarrow{\mathbf{x}}}{\mathrm{d}t} = \dot{\vec{\mathbf{x}}}\end{split}
\end{equation*}
\begin{sphinxadmonition}{note}{Note:}
\sphinxAtStartPar
V mechanice se často setkáváme s časovými derivacemi, tedy derivacemi funkcí, které závisí na čase. Pro zjednodušení zápisu a usnadnění práce s těmito derivacemi se používá speciální značení pomocí teček nad symbolem veličiny. Toto značení zavedl \sphinxstylestrong{Isaac Newton}.
\begin{itemize}
\item {} 
\sphinxAtStartPar
\sphinxstylestrong{První časová derivace:} Značí se jednou tečkou nad symbolem veličiny. Například, pokud \(x\) značí polohu, pak \(\dot{x}\) (x s tečkou) značí její první časovou derivaci, tedy rychlost:
\begin{equation*}
\begin{split}\dot{x} = \frac{dx}{dt}\end{split}
\end{equation*}
\item {} 
\sphinxAtStartPar
\sphinxstylestrong{Druhá časová derivace:} Značí se dvěma tečkami nad symbolem veličiny. Například, \(\ddot{x}\) (x se dvěma tečkami) značí druhou časovou derivaci polohy, tedy zrychlení:

\end{itemize}
\begin{equation*}
\begin{split}\ddot{x} = \frac{d^2x}{dt^2} = \frac{d}{dt}\left(\frac{dx}{dt}\right) = \frac{d\dot{x}}{dt}\end{split}
\end{equation*}\begin{itemize}
\item {} 
\sphinxAtStartPar
\sphinxstylestrong{Třetí a vyšší časové derivace:} Značí se třemi a více tečkami nad symbolem veličiny. Vyšší časové derivace se používají méně často, ale mohou se objevit například při popisu dynamiky pohybu.

\end{itemize}
\end{sphinxadmonition}


\subsubsection{Rychlost je relativní}
\label{\detokenize{Prednasky/1_2_Kinematika_v_1D:rychlost-je-relativni}}
\sphinxAtStartPar
Rychlost je vždy relativní. To znamená, že se udává vzhledem k nějakému referenčnímu bodu nebo soustavě. Neexistuje absolutní rychlost. Pojďme si to ilustrovat na několika příkladech s ohledem na Zemi, Slunce a centrum naší Galaxie, Mléčné dráhy.

\sphinxAtStartPar
\sphinxstylestrong{1. Vzhledem k Zemi:}
\begin{itemize}
\item {} 
\sphinxAtStartPar
\sphinxstylestrong{Chůze:} Člověk jde rychlostí například 5 km/h. Tato rychlost je \sphinxstyleemphasis{vzhledem k povrchu Země}.

\item {} 
\sphinxAtStartPar
\sphinxstylestrong{Auto:} Auto jede rychlostí 100 km/h. Opět, tato rychlost je \sphinxstyleemphasis{vzhledem k Zemi}.

\item {} 
\sphinxAtStartPar
\sphinxstylestrong{Letadlo:} Letadlo letí rychlostí 900 km/h. Tato rychlost je \sphinxstyleemphasis{vzhledem k Zemi}.

\end{itemize}

\sphinxAtStartPar
V těchto příkladech je Země naše nejpřirozenější referenční soustava. Všichni se pohybujeme po Zemi a proto nám tyto rychlosti dávají největší smysl.

\sphinxAtStartPar
\sphinxstylestrong{2. Vzhledem ke Slunci:}

\sphinxAtStartPar
Země obíhá kolem Slunce. To znamená, že i když stojíme na Zemi, pohybujeme se obrovskou rychlostí \sphinxstyleemphasis{vzhledem ke Slunci}.
\begin{itemize}
\item {} 
\sphinxAtStartPar
\sphinxstylestrong{Rychlost Země kolem Slunce:} Přibližně 30 km/s (108 000 km/h). To je mnohem rychleji, než cokoliv, co běžně zažíváme na Zemi. I když sedíte v klidu doma, pohybujete se touto obrovskou rychlostí \sphinxstyleemphasis{vzhledem ke Slunci}.

\item {} 
\sphinxAtStartPar
\sphinxstylestrong{Rychlost sondy Voyager 1 (vzhledem ke Slunci):} Voyager 1, jedna z nejvzdálenějších sond od Země, se pohybuje rychlostí přibližně 17 km/s (61 200 km/h) \sphinxstyleemphasis{vzhledem ke Slunci}.

\end{itemize}

\sphinxAtStartPar
Zde vidíme, že i relativně “pomalu” se pohybující tělesa (jako Země) mají obrovské rychlosti, pokud změníme referenční soustavu na Slunce.

\sphinxAtStartPar
\sphinxstylestrong{3. Vzhledem k centru Galaxie:}

\sphinxAtStartPar
Slunce a celá Sluneční soustava obíhají kolem centra Mléčné dráhy. I zde se pohybujeme obrovskou rychlostí.
\begin{itemize}
\item {} 
\sphinxAtStartPar
\sphinxstylestrong{Rychlost Slunce kolem centra Galaxie:} Přibližně 220 km/s (828 000 km/h). To je ještě mnohem rychlejší než pohyb Země kolem Slunce! I Slunce, obrovská a zdánlivě nehybná hvězda, se řítí vesmírem obrovskou rychlostí \sphinxstyleemphasis{vzhledem k centru Galaxie}.

\end{itemize}

\sphinxAtStartPar
Zde je Země pouze malou planetou obíhající kolem Slunce, které je jednou z miliard hvězd obíhajících centrum Galaxie. Proto se naše rychlost \sphinxstyleemphasis{vzhledem k centru Galaxie} skládá z rychlosti Země kolem Slunce a rychlosti Slunce kolem centra Galaxie.

\sphinxAtStartPar
\sphinxstylestrong{Shrnutí:}


\begin{savenotes}\sphinxattablestart
\sphinxthistablewithglobalstyle
\centering
\begin{tabulary}{\linewidth}[t]{TTT}
\sphinxtoprule
\sphinxstyletheadfamily 
\sphinxAtStartPar
Vzhledem k
&\sphinxstyletheadfamily 
\sphinxAtStartPar
Příklad
&\sphinxstyletheadfamily 
\sphinxAtStartPar
Rychlost (přibližně)
\\
\sphinxmidrule
\sphinxtableatstartofbodyhook
\sphinxAtStartPar
Zemi
&
\sphinxAtStartPar
Chůze
&
\sphinxAtStartPar
5 km/h
\\
\sphinxhline
\sphinxAtStartPar
Rovník
&
\sphinxAtStartPar
Osa otáčení Země
&
\sphinxAtStartPar
1674 km/h
\\
\sphinxhline
\sphinxAtStartPar
Slunci
&
\sphinxAtStartPar
Země
&
\sphinxAtStartPar
108 000 km/h)
\\
\sphinxhline
\sphinxAtStartPar
Centru Galaxie
&
\sphinxAtStartPar
Slunce
&
\sphinxAtStartPar
828 000 km/h)
\\
\sphinxhline
\sphinxAtStartPar
Mléčna dráha
&
\sphinxAtStartPar
ostatní galaxie
&
\sphinxAtStartPar
2.2 miliona km/h
\\
\sphinxhline
\sphinxAtStartPar
\sphinxstyleemphasis{Zdroj: \sphinxhref{https://www.britannica.com/story/what-is-earths-velocity}{Britannica.com}}
&
\sphinxAtStartPar

&
\sphinxAtStartPar

\\
\sphinxbottomrule
\end{tabulary}
\sphinxtableafterendhook\par
\sphinxattableend\end{savenotes}




\subsection{Zrychlení}
\label{\detokenize{Prednasky/1_2_Kinematika_v_1D:zrychleni}}
\sphinxAtStartPar
Zrychlení je změna rychlosti bodu, která může být dána i rychlostí změny polohy druhého řádu. Střední zrychlení mezi dvěma okamžiky je:
\begin{equation*}
\begin{split}\overrightarrow{\mathbf{a}}(t) = \frac{\overrightarrow{\mathbf{v}}(t_2)-\overrightarrow{\mathbf{v}}(t_1)}{t_2-t_1} = \frac{\Delta \overrightarrow{\mathbf{v}}}{\Delta t}\end{split}
\end{equation*}
\sphinxAtStartPar
Podobně okamžité zrychlení je derivace prvního řádu rychlosti nebo derivace druhého řádu polohového vektoru:
\begin{equation*}
\begin{split}\overrightarrow{\mathbf{a}}(t) = \frac{\mathrm{d}\overrightarrow{\mathbf{v}}(t)}{\mathrm{d}t} = \frac{\mathrm{d}^2\overrightarrow{\mathbf{x}}(t)}{\mathrm{d}t^2} = \dot{\vec{\mathbf{v}}} =\ddot{\vec{\mathbf{x}}}\end{split}
\end{equation*}

\subsection{Ryv (Jerk)}
\label{\detokenize{Prednasky/1_2_Kinematika_v_1D:ryv-jerk}}
\sphinxAtStartPar
\sphinxstylestrong{Ryv} (anglicky jerk) je fyzikální veličina, která popisuje \sphinxstylestrong{změnu zrychlení v čase}. Udává, jak rychle se zrychlení tělesa mění. Je to vektorová veličina, což znamená, že má velikost a směr.

\sphinxAtStartPar
Ryv se značí písmenem \(j\) a jeho definiční vztah je dán derivací zrychlení (\(a\)) podle času (\(t\)):
\begin{equation*}
\begin{split}
\vec{\mathbf{j}} = \frac{\mathrm{d}\vec{\mathbf{a}}}{\mathrm{d}t} = \frac{\mathrm{d}^3\mathbf{x}}{\mathrm{d}t^3}
\end{split}
\end{equation*}

\subsection{Příklady pohybu v 1D}
\label{\detokenize{Prednasky/1_2_Kinematika_v_1D:priklady-pohybu-v-1d}}

\subsubsection{Bod při konstantní rychlosti}
\label{\detokenize{Prednasky/1_2_Kinematika_v_1D:bod-pri-konstantni-rychlosti}}\begin{equation*}
\begin{split}
\begin{array}{l l}
\overrightarrow{\mathbf{a}}(t) = 0 \\
\overrightarrow{\mathbf{v}}(t) = \overrightarrow{\mathbf{v}}_0 \\
\overrightarrow{\mathbf{r}}(t) = \overrightarrow{\mathbf{r}}_0 + \overrightarrow{\mathbf{v}}_0t
\end{array}
\end{split}
\end{equation*}

\subsubsection{Bod při konstantním zrychlení}
\label{\detokenize{Prednasky/1_2_Kinematika_v_1D:bod-pri-konstantnim-zrychleni}}\begin{equation*}
\begin{split}
\begin{array}{l l}
\overrightarrow{\mathbf{a}}(t) = \overrightarrow{\mathbf{a}}_0 \\
\overrightarrow{\mathbf{v}}(t) = \overrightarrow{\mathbf{v}}_0 + \overrightarrow{\mathbf{a}}_0t \\
\overrightarrow{\mathbf{r}}(t) = \overrightarrow{\mathbf{r}}_0 + \overrightarrow{\mathbf{v}}_0t +
\frac{1}{2}\overrightarrow{\mathbf{a}}_0 t^2
\end{array}
\end{split}
\end{equation*}
\begin{sphinxuseclass}{cell}\begin{sphinxVerbatimInput}

\begin{sphinxuseclass}{cell_input}
\begin{sphinxVerbatim}[commandchars=\\\{\}]
\PYG{k+kn}{import} \PYG{n+nn}{numpy} \PYG{k}{as} \PYG{n+nn}{np}
\PYG{k+kn}{import} \PYG{n+nn}{matplotlib}\PYG{n+nn}{.}\PYG{n+nn}{pyplot} \PYG{k}{as} \PYG{n+nn}{plt}
\PYG{o}{\PYGZpc{}}\PYG{k}{matplotlib} inline

\PYG{n}{t} \PYG{o}{=} \PYG{n}{np}\PYG{o}{.}\PYG{n}{linspace}\PYG{p}{(}\PYG{l+m+mi}{0}\PYG{p}{,} \PYG{l+m+mi}{2}\PYG{p}{,} \PYG{l+m+mi}{101}\PYG{p}{)}
\PYG{n}{r0} \PYG{o}{=} \PYG{l+m+mi}{1}
\PYG{n}{v0} \PYG{o}{=} \PYG{l+m+mi}{2}
\PYG{n}{a0} \PYG{o}{=} \PYG{l+m+mi}{4}

\PYG{n}{plt}\PYG{o}{.}\PYG{n}{rc}\PYG{p}{(}\PYG{l+s+s1}{\PYGZsq{}}\PYG{l+s+s1}{axes}\PYG{l+s+s1}{\PYGZsq{}}\PYG{p}{,}  \PYG{n}{labelsize}\PYG{o}{=}\PYG{l+m+mi}{14}\PYG{p}{,}  \PYG{n}{titlesize}\PYG{o}{=}\PYG{l+m+mi}{14}\PYG{p}{)} 
\PYG{n}{plt}\PYG{o}{.}\PYG{n}{rc}\PYG{p}{(}\PYG{l+s+s1}{\PYGZsq{}}\PYG{l+s+s1}{xtick}\PYG{l+s+s1}{\PYGZsq{}}\PYG{p}{,} \PYG{n}{labelsize}\PYG{o}{=}\PYG{l+m+mi}{10}\PYG{p}{)}
\PYG{n}{plt}\PYG{o}{.}\PYG{n}{rc}\PYG{p}{(}\PYG{l+s+s1}{\PYGZsq{}}\PYG{l+s+s1}{ytick}\PYG{l+s+s1}{\PYGZsq{}}\PYG{p}{,} \PYG{n}{labelsize}\PYG{o}{=}\PYG{l+m+mi}{10}\PYG{p}{)} 
\PYG{n}{f}\PYG{p}{,} \PYG{n}{axarr} \PYG{o}{=} \PYG{n}{plt}\PYG{o}{.}\PYG{n}{subplots}\PYG{p}{(}\PYG{l+m+mi}{3}\PYG{p}{,} \PYG{l+m+mi}{3}\PYG{p}{,} \PYG{n}{sharex} \PYG{o}{=} \PYG{k+kc}{True}\PYG{p}{,} \PYG{n}{sharey} \PYG{o}{=} \PYG{k+kc}{True}\PYG{p}{,} \PYG{n}{figsize}\PYG{o}{=}\PYG{p}{(}\PYG{l+m+mi}{14}\PYG{p}{,}\PYG{l+m+mi}{7}\PYG{p}{)}\PYG{p}{)}
\PYG{n}{plt}\PYG{o}{.}\PYG{n}{suptitle}\PYG{p}{(}\PYG{l+s+s1}{\PYGZsq{}}\PYG{l+s+s1}{Pohyb po přímce}\PYG{l+s+s1}{\PYGZsq{}}\PYG{p}{,} \PYG{n}{fontsize}\PYG{o}{=}\PYG{l+m+mi}{20}\PYG{p}{)}\PYG{p}{;}

\PYG{n}{tones} \PYG{o}{=} \PYG{n}{np}\PYG{o}{.}\PYG{n}{ones}\PYG{p}{(}\PYG{n}{np}\PYG{o}{.}\PYG{n}{size}\PYG{p}{(}\PYG{n}{t}\PYG{p}{)}\PYG{p}{)}

\PYG{n}{axarr}\PYG{p}{[}\PYG{l+m+mi}{0}\PYG{p}{,} \PYG{l+m+mi}{0}\PYG{p}{]}\PYG{o}{.}\PYG{n}{set\PYGZus{}title}\PYG{p}{(}\PYG{l+s+s1}{\PYGZsq{}}\PYG{l+s+s1}{v klidu}\PYG{l+s+s1}{\PYGZsq{}}\PYG{p}{,} \PYG{n}{fontsize}\PYG{o}{=}\PYG{l+m+mi}{14}\PYG{p}{)}\PYG{p}{;}
\PYG{n}{axarr}\PYG{p}{[}\PYG{l+m+mi}{0}\PYG{p}{,} \PYG{l+m+mi}{0}\PYG{p}{]}\PYG{o}{.}\PYG{n}{plot}\PYG{p}{(}\PYG{n}{t}\PYG{p}{,} \PYG{n}{r0}\PYG{o}{*}\PYG{n}{tones}\PYG{p}{,} \PYG{l+s+s1}{\PYGZsq{}}\PYG{l+s+s1}{g}\PYG{l+s+s1}{\PYGZsq{}}\PYG{p}{,} \PYG{n}{linewidth}\PYG{o}{=}\PYG{l+m+mi}{4}\PYG{p}{,} \PYG{n}{label}\PYG{o}{=}\PYG{l+s+s1}{\PYGZsq{}}\PYG{l+s+s1}{\PYGZdl{}r(t)=1\PYGZdl{}}\PYG{l+s+s1}{\PYGZsq{}}\PYG{p}{)}
\PYG{n}{axarr}\PYG{p}{[}\PYG{l+m+mi}{1}\PYG{p}{,} \PYG{l+m+mi}{0}\PYG{p}{]}\PYG{o}{.}\PYG{n}{plot}\PYG{p}{(}\PYG{n}{t}\PYG{p}{,}  \PYG{l+m+mi}{0}\PYG{o}{*}\PYG{n}{tones}\PYG{p}{,} \PYG{l+s+s1}{\PYGZsq{}}\PYG{l+s+s1}{b}\PYG{l+s+s1}{\PYGZsq{}}\PYG{p}{,} \PYG{n}{linewidth}\PYG{o}{=}\PYG{l+m+mi}{4}\PYG{p}{,} \PYG{n}{label}\PYG{o}{=}\PYG{l+s+s1}{\PYGZsq{}}\PYG{l+s+s1}{\PYGZdl{}v(t)=0\PYGZdl{}}\PYG{l+s+s1}{\PYGZsq{}}\PYG{p}{)}
\PYG{n}{axarr}\PYG{p}{[}\PYG{l+m+mi}{2}\PYG{p}{,} \PYG{l+m+mi}{0}\PYG{p}{]}\PYG{o}{.}\PYG{n}{plot}\PYG{p}{(}\PYG{n}{t}\PYG{p}{,}  \PYG{l+m+mi}{0}\PYG{o}{*}\PYG{n}{tones}\PYG{p}{,} \PYG{l+s+s1}{\PYGZsq{}}\PYG{l+s+s1}{r}\PYG{l+s+s1}{\PYGZsq{}}\PYG{p}{,} \PYG{n}{linewidth}\PYG{o}{=}\PYG{l+m+mi}{4}\PYG{p}{,} \PYG{n}{label}\PYG{o}{=}\PYG{l+s+s1}{\PYGZsq{}}\PYG{l+s+s1}{\PYGZdl{}a(t)=0\PYGZdl{}}\PYG{l+s+s1}{\PYGZsq{}}\PYG{p}{)}
\PYG{n}{axarr}\PYG{p}{[}\PYG{l+m+mi}{0}\PYG{p}{,} \PYG{l+m+mi}{0}\PYG{p}{]}\PYG{o}{.}\PYG{n}{set\PYGZus{}ylabel}\PYG{p}{(}\PYG{l+s+s1}{\PYGZsq{}}\PYG{l+s+s1}{r(t) [m]}\PYG{l+s+s1}{\PYGZsq{}}\PYG{p}{)}
\PYG{n}{axarr}\PYG{p}{[}\PYG{l+m+mi}{1}\PYG{p}{,} \PYG{l+m+mi}{0}\PYG{p}{]}\PYG{o}{.}\PYG{n}{set\PYGZus{}ylabel}\PYG{p}{(}\PYG{l+s+s1}{\PYGZsq{}}\PYG{l+s+s1}{v(t) [m/s]}\PYG{l+s+s1}{\PYGZsq{}}\PYG{p}{)}
\PYG{n}{axarr}\PYG{p}{[}\PYG{l+m+mi}{2}\PYG{p}{,} \PYG{l+m+mi}{0}\PYG{p}{]}\PYG{o}{.}\PYG{n}{set\PYGZus{}ylabel}\PYG{p}{(}\PYG{l+s+s1}{\PYGZsq{}}\PYG{l+s+s1}{a(t) [m/s\PYGZdl{}\PYGZca{}2\PYGZdl{}]}\PYG{l+s+s1}{\PYGZsq{}}\PYG{p}{)}

\PYG{n}{axarr}\PYG{p}{[}\PYG{l+m+mi}{0}\PYG{p}{,} \PYG{l+m+mi}{1}\PYG{p}{]}\PYG{o}{.}\PYG{n}{set\PYGZus{}title}\PYG{p}{(}\PYG{l+s+s1}{\PYGZsq{}}\PYG{l+s+s1}{při konstatní rychlosti}\PYG{l+s+s1}{\PYGZsq{}}\PYG{p}{)}\PYG{p}{;}
\PYG{n}{axarr}\PYG{p}{[}\PYG{l+m+mi}{0}\PYG{p}{,} \PYG{l+m+mi}{1}\PYG{p}{]}\PYG{o}{.}\PYG{n}{plot}\PYG{p}{(}\PYG{n}{t}\PYG{p}{,} \PYG{n}{r0}\PYG{o}{*}\PYG{n}{tones}\PYG{o}{+}\PYG{n}{v0}\PYG{o}{*}\PYG{n}{t}\PYG{p}{,} \PYG{l+s+s1}{\PYGZsq{}}\PYG{l+s+s1}{g}\PYG{l+s+s1}{\PYGZsq{}}\PYG{p}{,} \PYG{n}{linewidth}\PYG{o}{=}\PYG{l+m+mi}{4}\PYG{p}{,} \PYG{n}{label}\PYG{o}{=}\PYG{l+s+s1}{\PYGZsq{}}\PYG{l+s+s1}{\PYGZdl{}r(t)=1+2t\PYGZdl{}}\PYG{l+s+s1}{\PYGZsq{}}\PYG{p}{)}
\PYG{n}{axarr}\PYG{p}{[}\PYG{l+m+mi}{1}\PYG{p}{,} \PYG{l+m+mi}{1}\PYG{p}{]}\PYG{o}{.}\PYG{n}{plot}\PYG{p}{(}\PYG{n}{t}\PYG{p}{,} \PYG{n}{v0}\PYG{o}{*}\PYG{n}{tones}\PYG{p}{,}      \PYG{l+s+s1}{\PYGZsq{}}\PYG{l+s+s1}{b}\PYG{l+s+s1}{\PYGZsq{}}\PYG{p}{,} \PYG{n}{linewidth}\PYG{o}{=}\PYG{l+m+mi}{4}\PYG{p}{,} \PYG{n}{label}\PYG{o}{=}\PYG{l+s+s1}{\PYGZsq{}}\PYG{l+s+s1}{\PYGZdl{}v(t)=2\PYGZdl{}}\PYG{l+s+s1}{\PYGZsq{}}\PYG{p}{)}
\PYG{n}{axarr}\PYG{p}{[}\PYG{l+m+mi}{2}\PYG{p}{,} \PYG{l+m+mi}{1}\PYG{p}{]}\PYG{o}{.}\PYG{n}{plot}\PYG{p}{(}\PYG{n}{t}\PYG{p}{,}  \PYG{l+m+mi}{0}\PYG{o}{*}\PYG{n}{tones}\PYG{p}{,}      \PYG{l+s+s1}{\PYGZsq{}}\PYG{l+s+s1}{r}\PYG{l+s+s1}{\PYGZsq{}}\PYG{p}{,} \PYG{n}{linewidth}\PYG{o}{=}\PYG{l+m+mi}{4}\PYG{p}{,} \PYG{n}{label}\PYG{o}{=}\PYG{l+s+s1}{\PYGZsq{}}\PYG{l+s+s1}{\PYGZdl{}a(t)=0\PYGZdl{}}\PYG{l+s+s1}{\PYGZsq{}}\PYG{p}{)}

\PYG{n}{axarr}\PYG{p}{[}\PYG{l+m+mi}{0}\PYG{p}{,} \PYG{l+m+mi}{2}\PYG{p}{]}\PYG{o}{.}\PYG{n}{set\PYGZus{}title}\PYG{p}{(}\PYG{l+s+s1}{\PYGZsq{}}\PYG{l+s+s1}{při konstantním zrychlení}\PYG{l+s+s1}{\PYGZsq{}}\PYG{p}{)}\PYG{p}{;}
\PYG{n}{axarr}\PYG{p}{[}\PYG{l+m+mi}{0}\PYG{p}{,} \PYG{l+m+mi}{2}\PYG{p}{]}\PYG{o}{.}\PYG{n}{plot}\PYG{p}{(}\PYG{n}{t}\PYG{p}{,} \PYG{n}{r0}\PYG{o}{*}\PYG{n}{tones}\PYG{o}{+}\PYG{n}{v0}\PYG{o}{*}\PYG{n}{t}\PYG{o}{+}\PYG{l+m+mi}{1}\PYG{o}{/}\PYG{l+m+mf}{2.}\PYG{o}{*}\PYG{n}{a0}\PYG{o}{*}\PYG{n}{t}\PYG{o}{*}\PYG{o}{*}\PYG{l+m+mi}{2}\PYG{p}{,}\PYG{l+s+s1}{\PYGZsq{}}\PYG{l+s+s1}{g}\PYG{l+s+s1}{\PYGZsq{}}\PYG{p}{,} \PYG{n}{linewidth}\PYG{o}{=}\PYG{l+m+mi}{4}\PYG{p}{,}
                 \PYG{n}{label}\PYG{o}{=}\PYG{l+s+s1}{\PYGZsq{}}\PYG{l+s+s1}{\PYGZdl{}r(t)=1+2t+}\PYG{l+s+se}{\PYGZbs{}\PYGZbs{}}\PYG{l+s+s1}{frac}\PYG{l+s+si}{\PYGZob{}1\PYGZcb{}}\PYG{l+s+si}{\PYGZob{}2\PYGZcb{}}\PYG{l+s+s1}{4t\PYGZca{}2\PYGZdl{}}\PYG{l+s+s1}{\PYGZsq{}}\PYG{p}{)}
\PYG{n}{axarr}\PYG{p}{[}\PYG{l+m+mi}{1}\PYG{p}{,} \PYG{l+m+mi}{2}\PYG{p}{]}\PYG{o}{.}\PYG{n}{plot}\PYG{p}{(}\PYG{n}{t}\PYG{p}{,} \PYG{n}{v0}\PYG{o}{*}\PYG{n}{tones}\PYG{o}{+}\PYG{n}{a0}\PYG{o}{*}\PYG{n}{t}\PYG{p}{,}             \PYG{l+s+s1}{\PYGZsq{}}\PYG{l+s+s1}{b}\PYG{l+s+s1}{\PYGZsq{}}\PYG{p}{,} \PYG{n}{linewidth}\PYG{o}{=}\PYG{l+m+mi}{4}\PYG{p}{,}
                 \PYG{n}{label}\PYG{o}{=}\PYG{l+s+s1}{\PYGZsq{}}\PYG{l+s+s1}{\PYGZdl{}v(t)=2+4t\PYGZdl{}}\PYG{l+s+s1}{\PYGZsq{}}\PYG{p}{)}
\PYG{n}{axarr}\PYG{p}{[}\PYG{l+m+mi}{2}\PYG{p}{,} \PYG{l+m+mi}{2}\PYG{p}{]}\PYG{o}{.}\PYG{n}{plot}\PYG{p}{(}\PYG{n}{t}\PYG{p}{,} \PYG{n}{a0}\PYG{o}{*}\PYG{n}{tones}\PYG{p}{,}                  \PYG{l+s+s1}{\PYGZsq{}}\PYG{l+s+s1}{r}\PYG{l+s+s1}{\PYGZsq{}}\PYG{p}{,} \PYG{n}{linewidth}\PYG{o}{=}\PYG{l+m+mi}{4}\PYG{p}{,}
                 \PYG{n}{label}\PYG{o}{=}\PYG{l+s+s1}{\PYGZsq{}}\PYG{l+s+s1}{\PYGZdl{}a(t)=4\PYGZdl{}}\PYG{l+s+s1}{\PYGZsq{}}\PYG{p}{)}

\PYG{k}{for} \PYG{n}{i} \PYG{o+ow}{in} \PYG{n+nb}{range}\PYG{p}{(}\PYG{l+m+mi}{3}\PYG{p}{)}\PYG{p}{:} 
    \PYG{n}{axarr}\PYG{p}{[}\PYG{l+m+mi}{2}\PYG{p}{,} \PYG{n}{i}\PYG{p}{]}\PYG{o}{.}\PYG{n}{set\PYGZus{}xlabel}\PYG{p}{(}\PYG{l+s+s1}{\PYGZsq{}}\PYG{l+s+s1}{Čas [s]}\PYG{l+s+s1}{\PYGZsq{}}\PYG{p}{)}\PYG{p}{;}
    \PYG{k}{for} \PYG{n}{j} \PYG{o+ow}{in} \PYG{n+nb}{range}\PYG{p}{(}\PYG{l+m+mi}{3}\PYG{p}{)}\PYG{p}{:}
        \PYG{n}{axarr}\PYG{p}{[}\PYG{n}{i}\PYG{p}{,}\PYG{n}{j}\PYG{p}{]}\PYG{o}{.}\PYG{n}{set\PYGZus{}ylim}\PYG{p}{(}\PYG{p}{(}\PYG{o}{\PYGZhy{}}\PYG{l+m+mf}{.2}\PYG{p}{,} \PYG{l+m+mi}{10}\PYG{p}{)}\PYG{p}{)}
        \PYG{n}{axarr}\PYG{p}{[}\PYG{n}{i}\PYG{p}{,}\PYG{n}{j}\PYG{p}{]}\PYG{o}{.}\PYG{n}{legend}\PYG{p}{(}\PYG{n}{loc} \PYG{o}{=} \PYG{l+s+s1}{\PYGZsq{}}\PYG{l+s+s1}{upper left}\PYG{l+s+s1}{\PYGZsq{}}\PYG{p}{,} \PYG{n}{frameon}\PYG{o}{=}\PYG{k+kc}{True}\PYG{p}{,} \PYG{n}{framealpha} \PYG{o}{=} \PYG{l+m+mf}{0.9}\PYG{p}{,} \PYG{n}{fontsize}\PYG{o}{=}\PYG{l+m+mi}{14}\PYG{p}{)}
        
\PYG{n}{plt}\PYG{o}{.}\PYG{n}{subplots\PYGZus{}adjust}\PYG{p}{(}\PYG{n}{hspace}\PYG{o}{=}\PYG{l+m+mf}{0.09}\PYG{p}{,} \PYG{n}{wspace}\PYG{o}{=}\PYG{l+m+mf}{0.07}\PYG{p}{)}
\end{sphinxVerbatim}

\end{sphinxuseclass}\end{sphinxVerbatimInput}
\begin{sphinxVerbatimOutput}

\begin{sphinxuseclass}{cell_output}
\noindent\sphinxincludegraphics{{dc639cc5f376c8ff27e1b948d78c3c185360759f261b69ec78fbc7351bded44c}.png}

\end{sphinxuseclass}\end{sphinxVerbatimOutput}

\end{sphinxuseclass}

\subsection{Kinematika závodu na 100 m}
\label{\detokenize{Prednasky/1_2_Kinematika_v_1D:kinematika-zavodu-na-100-m}}
\sphinxAtStartPar
Příkladem, kde lze analýzu některých aspektů pohybu lidského těla zredukovat na analýzu bodu, je studium biomechaniky běhu na 100 metrů.

\sphinxAtStartPar
Technickou zprávu s kinematickými daty pro světový rekord na 100 m od Usaina Bolta si můžete stáhnout z \sphinxhref{http://www.iaaf.org/development/research}{website for Research Projects} od Mezinárodní asociace atletických federací.
\sphinxhref{http://www.iaaf.org/download/download?filename=76ade5f9-75a0-4fda-b9bf-1b30be6f60d2.pdf\&amp;urlSlug=1-biomechanics-report-wc-berlin-2009-sprint}{Tady je přímý odkaz}. Konkrétně následující tabulka ukazuje údaje pro tři medailisty v tomto závodě:



\sphinxAtStartPar
Sloupec \sphinxstylestrong{RT} v tabulce výše se týká reakční doby každého sportovce. IAAF má velmi přísné pravidlo o reakční době: každý sportovec s reakční dobou kratší než 100 ms je ze soutěže diskvalifikován! Diskuzi o tomto pravidle naleznete na webu \sphinxhref{http://condellpark.com/kd/reactiontime.htm}{Reaction Times and Sprint False Starts}.

\sphinxAtStartPar
Svou vlastní reakční dobu si můžete změřit jednoduchým způsobem na této webové stránce: \sphinxurl{http://www.humanbenchmark.com/tests/reactiontime}.

\sphinxAtStartPar
Článek \sphinxhref{http://www.ncbi.nlm.nih.gov/pmc/articles/PMC3661886/}{A Kinematics Analysis Of Three Best 100 M Performances Ever} od Krzysztofa a Mera představuje podrobnou kinematickou analýzu závodů na 100 m.


\section{Vztah mezi dráhou, rychlostí a zrychlením}
\label{\detokenize{Prednasky/1_2_Kinematika_v_1D:vztah-mezi-drahou-rychlosti-a-zrychlenim}}
\sphinxAtStartPar
\sphinxstylestrong{Graf dráhy}: Ukazuje, jak se mění poloha tělesa v čase.

\sphinxAtStartPar
\sphinxstylestrong{Graf rychlosti}: Ukazuje, jak se mění rychlost tělesa v čase.

\sphinxAtStartPar
\sphinxstylestrong{Graf zrychlení}: Ukazuje, jak se mění zrychlení tělesa v čase.

\sphinxAtStartPar
\sphinxstylestrong{Při rovnoměrném přímočarém pohybu}: Rychlost je konstantní a dráha je přímo úměrná času, zrychlení je nulové.

\sphinxAtStartPar
\sphinxstylestrong{Při rovnoměrně zrychleném přímočarém pohybu:} Rychlost se lineárně zvětšuje s časem a dráha se mění kvadraticky s časem.

\begin{sphinxadmonition}{note}{Note:}
\sphinxAtStartPar
\sphinxstylestrong{Obecně:}
\begin{itemize}
\item {} 
\sphinxAtStartPar
Sklon grafu dráhy určuje velikost rychlosti

\item {} 
\sphinxAtStartPar
Sklon grafu rychlosti určuje velikost zrychlení

\item {} 
\sphinxAtStartPar
Plocha pod grafem zrychlení je rovná změně rychlosti

\item {} 
\sphinxAtStartPar
Plocha po grafem dráhy je rovna změně dráhy

\end{itemize}
\end{sphinxadmonition}

\sphinxAtStartPar
\sphinxincludegraphics{{motslope}.png}

\begin{sphinxadmonition}{tip}{Tip:}
\sphinxAtStartPar
\sphinxhref{https://www.khanacademy.org/science/mechanics-essentials/xafb2c8d81b6e70e3:how-to-analyze-car-crashes-using-skid-mark-analysis/xafb2c8d81b6e70e3:how-fast-are-you-going-right-now/a/position-vs-time-graphs}{Khan academy \sphinxhyphen{} vysvětlení vztahu mezi sklonem grafu polohu a rychlostí}
\end{sphinxadmonition}

\begin{sphinxuseclass}{cell}\begin{sphinxVerbatimInput}

\begin{sphinxuseclass}{cell_input}
\begin{sphinxVerbatim}[commandchars=\\\{\}]
\PYG{k+kn}{from} \PYG{n+nn}{IPython}\PYG{n+nn}{.}\PYG{n+nn}{display} \PYG{k+kn}{import} \PYG{n}{IFrame}
\PYG{n}{IFrame}\PYG{p}{(}\PYG{l+s+s1}{\PYGZsq{}}\PYG{l+s+s1}{https://www.geogebra.org/classic/pdNj3DgD}\PYG{l+s+s1}{\PYGZsq{}}\PYG{p}{,} \PYG{n}{width}\PYG{o}{=}\PYG{l+m+mi}{800}\PYG{p}{,} \PYG{n}{height}\PYG{o}{=}\PYG{l+m+mi}{600}\PYG{p}{,} \PYG{n}{style}\PYG{o}{=}\PYG{l+s+s2}{\PYGZdq{}}\PYG{l+s+s2}{border: 1px solid black}\PYG{l+s+s2}{\PYGZdq{}}\PYG{p}{)}
\end{sphinxVerbatim}

\end{sphinxuseclass}\end{sphinxVerbatimInput}
\begin{sphinxVerbatimOutput}

\begin{sphinxuseclass}{cell_output}
\begin{sphinxVerbatim}[commandchars=\\\{\}]
\PYGZlt{}IPython.lib.display.IFrame at 0x7c2ac05dcf70\PYGZgt{}
\end{sphinxVerbatim}

\end{sphinxuseclass}\end{sphinxVerbatimOutput}

\end{sphinxuseclass}
\sphinxstepscope


\section{Dynamika}
\label{\detokenize{Prednasky/1_3_Dynamika_pohybu_v_1D:dynamika}}\label{\detokenize{Prednasky/1_3_Dynamika_pohybu_v_1D::doc}}\begin{quote}

\sphinxAtStartPar
Dynamika je odvětví mechaniky, které se zabývá studiem pohybu těles a sil, které tento pohyb způsobují nebo ovlivňují.
\end{quote}

\sphinxAtStartPar
Na rozdíl od kinematiky, která popisuje pohyb bez ohledu na jeho příčiny, dynamika zkoumá vztah mezi silami, hmotností a zrychlením.


\subsection{Síla je vektor}
\label{\detokenize{Prednasky/1_3_Dynamika_pohybu_v_1D:sila-je-vektor}}
\sphinxAtStartPar
Síla je definována jako vektorová veličina z několika klíčových důvodů, které vyplývají z její povahy a způsobu, jakým ovlivňuje pohyb těles. Zde jsou hlavní charakteristiky, které definují sílu jako vektor:
\begin{enumerate}
\sphinxsetlistlabels{\arabic}{enumi}{enumii}{}{.}%
\item {} 
\sphinxAtStartPar
\sphinxstylestrong{Velikost} udává, jak silné je vzájemné působení mezi tělesy. Tato velikost se měří v newtonech (N). Velikost síly nám říká, jak moc síla ovlivňuje pohyb tělesa (například jak rychle ho zrychluje).

\item {} 
\sphinxAtStartPar
\sphinxstylestrong{Směr} udává, ve kterém směru síla působí. Směr síly je klíčový pro určení, jakým způsobem se těleso bude pohybovat nebo deformovat. Například, pokud působíte silou na těleso zleva, bude se pohybovat doprava, a naopak.

\item {} 
\sphinxAtStartPar
\sphinxstylestrong{Působiště} je bod, ve kterém síla působí na těleso. Působiště síly ovlivňuje, jakým způsobem se těleso bude otáčet nebo deformovat. Například, pokud působíte silou na okraj dveří, budou se otáčet, zatímco pokud působíte silou na jejich střed, budou se pouze posouvat.

\item {} 
\sphinxAtStartPar
\sphinxstylestrong{Nositelka} je přímka, na které vektor síly leží. Nositela síly je důležitá pro určení momentu síly a pro skládání sil.

\end{enumerate}

\sphinxAtStartPar
\sphinxincludegraphics{{images}.png}

\sphinxAtStartPar
\sphinxincludegraphics{{087b90ce9f357d4aadb49906595c52121ca9cb7f}.png}

\begin{sphinxadmonition}{note}{Note:}
\sphinxAtStartPar
Síla vždy tlačí nebo táhne.
\end{sphinxadmonition}


\subsubsection{Vektorový zápis síly}
\label{\detokenize{Prednasky/1_3_Dynamika_pohybu_v_1D:vektorovy-zapis-sily}}
\sphinxAtStartPar
Síla se obvykle zapisuje jako vektor \(\vec{\mathbf{F}}\), který má složky v různých směrech (například \(F_x, F_y, F_z\) v kartézských souřadnicích). Tento zápis umožňuje matematicky manipulovat se silami a provádět výpočty týkající se jejich účinků na tělesa.


\subsubsection{Součet sil}
\label{\detokenize{Prednasky/1_3_Dynamika_pohybu_v_1D:soucet-sil}}
\sphinxAtStartPar
Při součtu sil, které působí na těleso, se snažíme najít výslednou sílu, která má stejný účinek jako všechny jednotlivé síly dohromady. Existují numerické a grafické metody, jak tuto výslednou sílu určit.
\begin{itemize}
\item {} 
\sphinxAtStartPar
\sphinxstylestrong{Numerické metody} využívají matematické operace s vektory sil k určení výsledné síly.
\begin{enumerate}
\sphinxsetlistlabels{\arabic}{enumi}{enumii}{}{.}%
\item {} 
\sphinxAtStartPar
Rozklad sil do složek
\begin{itemize}
\item {} 
\sphinxAtStartPar
Každou sílu rozložíme na složky v kartézském souřadnicovém systému (\(F_x, F_y, F_z\)).

\item {} 
\sphinxAtStartPar
Sečteme složky sil ve stejném směru:
\begin{itemize}
\item {} 
\sphinxAtStartPar
\(\Sigma F_x = F_{1x} + F_{2x} + ... + F_{nx}\)

\item {} 
\sphinxAtStartPar
\(\Sigma F_y = F_{1y} + F_{2y} + ... + F_{ny}\)

\item {} 
\sphinxAtStartPar
\(\Sigma F_z = F_{1z} + F_{2z} + ... + F_{nz}\)

\end{itemize}

\item {} 
\sphinxAtStartPar
Výsledná síla (\(\mathbf{F}\)) má složky (\(\Sigma F_x\), \(\Sigma F_y\), \(\Sigma F_z\)).

\item {} 
\sphinxAtStartPar
Velikost výsledné síly se vypočítá pomocí normy:
\begin{itemize}
\item {} 
\sphinxAtStartPar
\(|\vec{\mathbf{F}}| = \sqrt{(\Sigma F_x)^2 + (\Sigma F_y)^2 + (\Sigma F_z)^2}\)

\end{itemize}

\item {} 
\sphinxAtStartPar
Směr výsledné síly se určí pomocí trigonometrických funkcí.

\end{itemize}

\item {} 
\sphinxAtStartPar
Vektorový součet
\begin{itemize}
\item {} 
\sphinxAtStartPar
Síly se sčítají vektorově, což znamená, že se jejich účinky sčítají podle pravidel vektorového součtu.

\end{itemize}

\end{enumerate}

\item {} 
\sphinxAtStartPar
\sphinxstylestrong{Grafické metody} využívají geometrické konstrukce k určení výsledné síly.
\begin{enumerate}
\sphinxsetlistlabels{\arabic}{enumi}{enumii}{}{.}%
\item {} 
\sphinxAtStartPar
Pravidlo rovnoběžníku
\begin{itemize}
\item {} 
\sphinxAtStartPar
Používá se pro součet dvou sil.

\item {} 
\sphinxAtStartPar
Síly se znázorní jako vektory se společným počátkem.

\item {} 
\sphinxAtStartPar
Vektory se doplní na rovnoběžník.

\item {} 
\sphinxAtStartPar
Výsledná síla je úhlopříčka rovnoběžníku.

\end{itemize}

\sphinxAtStartPar
\sphinxincludegraphics{{300px-ParallelogramForces}.png}

\item {} 
\sphinxAtStartPar
Pravidlo trojúhelníku
\begin{itemize}
\item {} 
\sphinxAtStartPar
Používá se pro součet dvou sil.

\item {} 
\sphinxAtStartPar
Síly se znázorní jako vektory tak, že konec prvního vektoru je počátek druhého vektoru.

\item {} 
\sphinxAtStartPar
Výsledná síla je vektor, který spojuje počátek prvního vektoru s koncem druhého vektoru.

\end{itemize}

\end{enumerate}

\sphinxAtStartPar
\sphinxincludegraphics{{300px-TriangleForces}.png}
\begin{enumerate}
\sphinxsetlistlabels{\arabic}{enumi}{enumii}{}{.}%
\setcounter{enumi}{3}
\item {} 
\sphinxAtStartPar
Polygon sil
\begin{itemize}
\item {} 
\sphinxAtStartPar
Používá se pro součet více sil.

\item {} 
\sphinxAtStartPar
Síly se znázorní jako vektory tak, že konec jednoho vektoru je počátek dalšího vektoru.

\item {} 
\sphinxAtStartPar
Výsledná síla je vektor, který spojuje počátek prvního vektoru s koncem posledního vektoru.

\end{itemize}

\end{enumerate}

\end{itemize}

\sphinxAtStartPar
\sphinxincludegraphics{{300px-PolygonForces}.png}


\subsection{Newtonovy zákony}
\label{\detokenize{Prednasky/1_3_Dynamika_pohybu_v_1D:newtonovy-zakony}}\begin{enumerate}
\sphinxsetlistlabels{\arabic}{enumi}{enumii}{}{.}%
\item {} 
\sphinxAtStartPar
\sphinxstylestrong{První Newtonův zákon} \sphinxhyphen{} zákon setrvačnosti.
\begin{quote}

\sphinxAtStartPar
\sphinxstyleemphasis{Corpus omne perseverare in statu suo quiescendi vel movendi uniformiter in directum, nisi quatenus illud a viribus impressis cogitur statum suum mutare.}
\end{quote}
\begin{quote}

\sphinxAtStartPar
Jestliže na těleso (hmotu) nepůsobí žádné vnější síly, nebo výslednice sil je 0, pak těleso setrvává v klidu nebo v rovnoměrném přímočarém pohybu.
\end{quote}

\sphinxAtStartPar
Ekvivalentní (srozumitelná a doslovná) formulace zní: Těleso zůstává v klidu nebo rovnoměrném přímočarém pohybu, není\sphinxhyphen{}li nuceno vnějšími silami tento stav změnit.

\item {} 
\sphinxAtStartPar
\sphinxstylestrong{Druhý Newtonův zákon} \sphinxhyphen{} zákon síly
\begin{quote}

\sphinxAtStartPar
\sphinxstyleemphasis{Mutationem motus proportionalem esse vi motrici impressae et fieri secundam lineam rectam qua vis illa imprimitur.}
\end{quote}
\begin{quote}

\sphinxAtStartPar
Jestliže na těleso (hmotu) působí síla, pak se těleso pohybuje zrychlením, které je přímo úměrné působící síle a nepřímo úměrné hmotnosti tělesa.
\end{quote}
\begin{equation*}
\begin{split}\vec{\mathbf{F}} = m  \vec{\mathbf{a}} \end{split}
\end{equation*}\begin{equation*}
\begin{split}\vec{\mathbf{F}} = m  \ddot{\vec{\mathbf{x}}} \end{split}
\end{equation*}
\item {} 
\sphinxAtStartPar
\sphinxstylestrong{Třetí Newtonův zákon} \sphinxhyphen{}  zákon akce a reakce
\begin{quote}

\sphinxAtStartPar
\sphinxstyleemphasis{Actioni contrariam semper et aequalem esse reactionem; sive: corporum duorum actiones in se mutuo semper esse aequales et in partes contrarias dirigi.}
\end{quote}
\begin{quote}

\sphinxAtStartPar
Proti každé akci vždy působí stejně velká ale opačná reakce ; jinak: vzájemná působení dvou těles jsou vždy stejně velká a míří na opačné strany.
\end{quote}

\end{enumerate}

\sphinxAtStartPar
\sphinxincludegraphics{{Newtons-Law-of-Motion}.png}


\section{Kdy je dvojice sil v rovnováze?}
\label{\detokenize{Prednasky/1_3_Dynamika_pohybu_v_1D:kdy-je-dvojice-sil-v-rovnovaze}}
\sphinxAtStartPar
Dvojice sil je v rovnováze tehdy, když splňuje následující tři podmínky:
\begin{enumerate}
\sphinxsetlistlabels{\arabic}{enumi}{enumii}{}{.}%
\item {} 
\sphinxAtStartPar
\sphinxstylestrong{Stejná velikost:} Obě síly musí mít stejnou velikost (velikost vektoru síly).

\item {} 
\sphinxAtStartPar
\sphinxstylestrong{Opačný směr:} Síly musí působit v opačných směrech. To znamená, že vektory sil musí směřovat proti sobě.

\item {} 
\sphinxAtStartPar
\sphinxstylestrong{Společná nositelka:} Síly musí působit na těleso v jedné přímce (jejich vektory musí ležet na jedné přímce).

\end{enumerate}


\subsection{Důsledky rovnováhy dvojice sil}
\label{\detokenize{Prednasky/1_3_Dynamika_pohybu_v_1D:dusledky-rovnovahy-dvojice-sil}}\begin{itemize}
\item {} 
\sphinxAtStartPar
\sphinxstylestrong{Nulová výslednice:} Vektorový součet obou sil je nulový. To znamená, že dvojice sil nemá žádný posuvný účinek na těleso.

\item {} 
\sphinxAtStartPar
\sphinxstylestrong{Nulový moment:} Moment dvojice sil je také nulový. To znamená, že dvojice sil nemá žádný otáčivý účinek na těleso.

\end{itemize}


\subsection{Příklady}
\label{\detokenize{Prednasky/1_3_Dynamika_pohybu_v_1D:priklady}}\begin{itemize}
\item {} 
\sphinxAtStartPar
\sphinxstylestrong{Kniha ležící na stole:} Na knihu působí tíhová síla směrem dolů a stejně velká síla stolu směrem nahoru. Tyto dvě síly tvoří dvojici sil v rovnováze.

\item {} 
\sphinxAtStartPar
\sphinxstylestrong{Závodník držící činku:} Závodník působí silou vzhůru, aby vyrovnal tíhovou sílu působící na činku směrem dolů. Tyto dvě síly tvoří dvojici sil v rovnováze.

\end{itemize}

\sphinxAtStartPar
\sphinxincludegraphics{{object-forces-rest}.jpg}


\subsection{Pasivní síly v 1D}
\label{\detokenize{Prednasky/1_3_Dynamika_pohybu_v_1D:pasivni-sily-v-1d}}\begin{itemize}
\item {} 
\sphinxAtStartPar
\sphinxstylestrong{Tření (\(\vec{\mathbf{F}}_t\))}: Tření je síla, která vzniká mezi dvěma povrchy, které se dotýkají a pohybují proti sobě nebo se snaží o pohyb. Tření vždy působí proti směru pohybu nebo proti směru snahy o pohyb.
\begin{itemize}
\item {} 
\sphinxAtStartPar
\sphinxstylestrong{Statické tření (\(\vec{\mathbf{F}}_{ts}\))}: Působí mezi tělesy, která jsou v klidu a brání jim v pohybu.
\begin{equation*}
\begin{split} F_{ts} \leq \mu_s N\end{split}
\end{equation*}
\sphinxAtStartPar
kde \(\mu_s\) je koeficient statického tření a \(N\) je normálová síla.

\item {} 
\sphinxAtStartPar
\sphinxstylestrong{Kinetické tření (\(\vec{\mathbf{F}}_{td}\))}: Působí mezi tělesy, která se pohybují proti sobě.
\begin{equation*}
\begin{split} F_{tk} \leq \mu_k N\end{split}
\end{equation*}
\sphinxAtStartPar
kde \(\mu_k\) je koeficient kinetického tření a \(N\) je normálová síla.

\end{itemize}

\end{itemize}

\sphinxAtStartPar
\sphinxincludegraphics{{fsta}.png}
\begin{itemize}
\item {} 
\sphinxAtStartPar
\sphinxstylestrong{Odpor prostředí (\(\vec{\mathbf{F}}_o\))}: Odpor prostředí (např. odpor vzduchu nebo odpor kapaliny) je síla, která působí proti pohybu tělesa v daném prostředí. Velikost odporu prostředí závisí na rychlosti tělesa a na vlastnostech prostředí.
\begin{itemize}
\item {} 
\sphinxAtStartPar
\sphinxstylestrong{Odpor vzduchu (pro nízké rychlosti)}:
\begin{equation*}
\begin{split}F_o = -b  v\end{split}
\end{equation*}
\sphinxAtStartPar
kde \(b\) je konstanta odporu prostředí a \(v\) je rychlost tělesa pro malé hodnoty Reynoldsova čísla.

\item {} 
\sphinxAtStartPar
\sphinxstylestrong{Odpor vzduchu (pro vysoké rychlosti)}:
\begin{equation*}
\begin{split}F_o = -1/2  \rho  C_d  A  v^2\end{split}
\end{equation*}
\sphinxAtStartPar
kde \(\rho\) je hustota prostředí, \(C_d\) je součinitel odporu, \(A\) je plocha tělesa a \(v\) je rychlost tělesa.

\end{itemize}

\end{itemize}

\begin{sphinxadmonition}{tip}{Tip:}
\sphinxAtStartPar
\sphinxhref{https://en.wikipedia.org/wiki/File:194144main\_022\_drag.ogv}{Vysvětlení odporu podle NASA}
\end{sphinxadmonition}


\subsection{Vlastnosti pasivních sil}
\label{\detokenize{Prednasky/1_3_Dynamika_pohybu_v_1D:vlastnosti-pasivnich-sil}}\begin{itemize}
\item {} 
\sphinxAtStartPar
\sphinxstylestrong{Působí proti pohybu}: Pasivní síly vždy působí proti směru pohybu nebo proti směru snahy o pohyb.

\item {} 
\sphinxAtStartPar
\sphinxstylestrong{Závisí na podmínkách}: Velikost pasivních sil často závisí na podmínkách, jako je například materiál povrchů (u tření), rychlost tělesa (u odporu prostředí) nebo teplota.

\item {} 
\sphinxAtStartPar
\sphinxstylestrong{Ztráta energie}: Pasivní síly způsobují ztrátu mechanické energie, která se přeměňuje na teplo nebo jiné formy energie.

\end{itemize}

\sphinxstepscope


\section{Mechanická práce}
\label{\detokenize{Prednasky/1_4_Energie:mechanicka-prace}}\label{\detokenize{Prednasky/1_4_Energie::doc}}\begin{quote}

\sphinxAtStartPar
Pohybuje\sphinxhyphen{}li se těleso působením síly, koná se mechanická práce.
\end{quote}

\sphinxAtStartPar
\sphinxstylestrong{Značení:}
\begin{itemize}
\item {} 
\sphinxAtStartPar
Mechanická práce se obvykle značí písmenem \sphinxstylestrong{W} (z anglického “work”).

\end{itemize}

\sphinxAtStartPar
\sphinxstylestrong{Jednotka:}
\begin{itemize}
\item {} 
\sphinxAtStartPar
V soustavě SI (Mezinárodní soustava jednotek) je jednotkou mechanické práce \sphinxstylestrong{joule} (čti džaul).

\item {} 
\sphinxAtStartPar
Značka joulu je \sphinxstylestrong{J}.

\end{itemize}

\sphinxAtStartPar
\sphinxstylestrong{Definice joulu:}
\begin{itemize}
\item {} 
\sphinxAtStartPar
1 joule (1 J) je práce vykonaná silou 1 newtonu (1 N) při posunutí tělesa o 1 metr (1 m) ve směru působící síly.

\item {} 
\sphinxAtStartPar
Matematicky: 1 J = 1 N ⋅ m

\end{itemize}

\sphinxAtStartPar
\sphinxstylestrong{Další jednotky (méně časté):}
\begin{itemize}
\item {} 
\sphinxAtStartPar
\sphinxstylestrong{erg}: Používá se v soustavě CGS (centimetr\sphinxhyphen{}gram\sphinxhyphen{}sekunda). 1 erg = 10⁻⁷ J.

\item {} 
\sphinxAtStartPar
\sphinxstylestrong{kilowatthodina (kWh)}: Používá se pro měření spotřeby elektrické energie. 1 kWh = 3,6 × 10⁶ J.

\end{itemize}
\begin{enumerate}
\sphinxsetlistlabels{\arabic}{enumi}{enumii}{}{.}%
\item {} 
\sphinxAtStartPar
\sphinxstylestrong{Síla působí ve stejném směru jako pohyb tělesa}
\begin{equation*}
\begin{split}W = F s \end{split}
\end{equation*}
\sphinxAtStartPar
kde \(F\) je velikost síly a \(s\) délka dráhy

\item {} 
\sphinxAtStartPar
\sphinxstylestrong{Síla působí v jiném směru než pohyb tělesa}
\begin{equation*}
\begin{split}W = \vec{\mathbf{F}} \cdot \vec{\mathbf{s}} \end{split}
\end{equation*}
\sphinxAtStartPar
kde \(\vec{\mathbf{F}}\) je vektor síly a \(\vec{\mathbf{s}}\) je vektor posunutí. Jinými slovy práci koná složka síly rovnoběžná s trajektorií tělesa. V případě, že je síla konstatní a trajektorie je úsečka můžeme vyjádřit.
\begin{equation*}
\begin{split}W = F s \cos \alpha\end{split}
\end{equation*}
\sphinxAtStartPar
kde \(\alpha\) je je úhel mezi silou a trajektorií pohybu.

\item {} 
\sphinxAtStartPar
\sphinxstylestrong{Síla se mění nebo dráha je zakřivena}

\end{enumerate}
\begin{equation*}
\begin{split}W = \int\limits_0^s \vec{\mathbf{F}} \cdot \mathrm{d}\vec{\mathbf{s}}  \end{split}
\end{equation*}

\section{Energie}
\label{\detokenize{Prednasky/1_4_Energie:energie}}

\subsection{Kinetická energie}
\label{\detokenize{Prednasky/1_4_Energie:kineticka-energie}}
\sphinxAtStartPar
Kinetická energie je energie spojená s pohybem objektu. Těleso v pohybu má kinetickou energii. Kinetická energie částice o hmotnosti m pohybující se rychlostí v je dána vztahem:
\begin{equation*}
\begin{split}E_k = \frac{1}{2} m \vec{\mathbf{v}} \cdot \vec{\mathbf{v}}= \frac{1}{2} m v^2\end{split}
\end{equation*}
\sphinxAtStartPar
kde \(v=|∣\vec{\mathbf{v}}∣|\) je velikost rychlosti. Pro systém částic je celková kinetická energie součtem kinetických energií všech jednotlivých částic:
\begin{equation*}
\begin{split}E_k = \sum_i \frac{1}{2} m_i \dot{\vec{\mathbf{r}}}_i \cdot \dot{\vec{\mathbf{r}}}_i = \sum_i \frac{1}{2} m_i v_i^2\end{split}
\end{equation*}
\sphinxAtStartPar
kde \(m_i\) je hmotnost \(i\)\sphinxhyphen{}té částice a \(\dot{\vec{\mathbf{r}}_i}\) je její rychlost


\subsection{Potenciální energie}
\label{\detokenize{Prednasky/1_4_Energie:potencialni-energie}}
\sphinxAtStartPar
Potenciální energie je energie spojená s konfigurací systému objektů. Představuje schopnost konat práci v důsledku polohy objektu v silovém poli. Potenciální energie je často spojena s konzervativními silami. Síla \(F\) je konzervativní, pokud práce vykonaná touto silou na částici pohybující se mezi dvěma body nezávisí na dráze.

\sphinxAtStartPar
Potenciální energie \(U\) spojená s konzervativní silou \(F\) je definována tak, že:
\begin{equation*}
\begin{split}\mathbf{\vec{F}} = -\nabla U\end{split}
\end{equation*}
\sphinxAtStartPar
kde \(\nabla\) je operátor gradientu. To znamená, že síla je záporný gradient potenciální energie. Změna potenciální energie mezi dvěma body se rovná záporné práci vykonané konzervativní silou při pohybu mezi těmito body.

\sphinxAtStartPar
Mezi běžné příklady potenciální energie patří:
\begin{itemize}
\item {} 
\sphinxAtStartPar
\sphinxstylestrong{Gravitační potenciální energie:} Pro objekt o hmotnosti m blízko povrchu Země je gravitační potenciální energie:
\begin{equation*}
\begin{split}E_p = mgh\end{split}
\end{equation*}
\end{itemize}

\sphinxAtStartPar
kde \(g\) je zrychlení v důsledku gravitace a \(h\) je výška nad referenční úrovní.
\begin{itemize}
\item {} 
\sphinxAtStartPar
\sphinxstylestrong{Elastická potenciální energie:} Pro pružinu s konstantou tuhosti \(k\) protaženou nebo stlačenou o vzdálenost \(x\) z její rovnovážné polohy je elastická potenciální energie:
\begin{equation*}
\begin{split}U_e = \frac{1}{2} k x^2\end{split}
\end{equation*}
\end{itemize}


\subsection{Práce a energie}
\label{\detokenize{Prednasky/1_4_Energie:prace-a-energie}}
\sphinxAtStartPar
Práce \(W\) vykonaná výslednou silou na částici, když se pohybuje z bodu A do bodu B, se rovná změně její kinetické energie:
\begin{equation*}
\begin{split} W = \Delta E_k = E_{kB} - E_{kA} \end{split}
\end{equation*}
\sphinxAtStartPar
Pokud práci konají pouze konzervativní síly, pak můžeme práci spojit i se změnou potenciální energie:
\begin{equation*}
\begin{split} W = -\Delta U = -(U_B - U_A) = U_A - U_B\end{split}
\end{equation*}
\sphinxAtStartPar
Kombinací těchto vztahů získáme zákon zachování mechanické energie:
\begin{equation*}
\begin{split}\Delta E_k + \Delta U = 0  \quad \text{nebo} \quad E_{kA} + U_A =E_{kB} + U_B\end{split}
\end{equation*}
\sphinxAtStartPar
To znamená, že celková mechanická energie (\(E=T+U\)) systému zůstává konstantní, pokud práci konají pouze konzervativní síly.


\section{Výkon}
\label{\detokenize{Prednasky/1_4_Energie:vykon}}
\sphinxAtStartPar
Výkon (\(P\)) je definován jako rychlost, s jakou je konána práce nebo se přenáší energie:
\begin{equation*}
\begin{split}P = \frac{dW}{dt} = \vec{\mathbf{F}} \cdot \vec{\mathbf{v}}\end{split}
\end{equation*}
\sphinxAtStartPar
Výkon se měří ve wattech (W), kde 1 W = 1 J/s.

\sphinxstepscope


\section{Princip minimální akce a Lagrange\sphinxhyphen{}Eulerovy rovnice}
\label{\detokenize{Prednasky/1_5_Lagrange_Eulerovy_rovnice:princip-minimalni-akce-a-lagrange-eulerovy-rovnice}}\label{\detokenize{Prednasky/1_5_Lagrange_Eulerovy_rovnice::doc}}
\begin{sphinxuseclass}{cell}\begin{sphinxVerbatimInput}

\begin{sphinxuseclass}{cell_input}
\begin{sphinxVerbatim}[commandchars=\\\{\}]
\PYG{k+kn}{import} \PYG{n+nn}{numpy} \PYG{k}{as} \PYG{n+nn}{np}
\PYG{k+kn}{import} \PYG{n+nn}{matplotlib}\PYG{n+nn}{.}\PYG{n+nn}{pyplot} \PYG{k}{as} \PYG{n+nn}{plt}
\PYG{k+kn}{import} \PYG{n+nn}{os}
\PYG{k+kn}{import} \PYG{n+nn}{torch}

\PYG{k+kn}{from} \PYG{n+nn}{celluloid} \PYG{k+kn}{import} \PYG{n}{Camera}
\PYG{k+kn}{from} \PYG{n+nn}{IPython}\PYG{n+nn}{.}\PYG{n+nn}{display} \PYG{k+kn}{import} \PYG{n}{HTML}
\PYG{k+kn}{from} \PYG{n+nn}{base64} \PYG{k+kn}{import} \PYG{n}{b64encode}
\end{sphinxVerbatim}

\end{sphinxuseclass}\end{sphinxVerbatimInput}

\end{sphinxuseclass}

\subsection{Newtonovy zákony \sphinxhyphen{} opakování}
\label{\detokenize{Prednasky/1_5_Lagrange_Eulerovy_rovnice:newtonovy-zakony-opakovani}}
\sphinxAtStartPar
\sphinxhref{https://en.wikipedia.org/wiki/Newton's\_laws\_of\_motion}{Newtonovy pohybové zákony} jsou považovány za základy klasické mechaniky. Popisují vztah mezi pohybem tělesa a sílami, které na něj působí. Kůli jednoduchosti budeme uvažovat pohyb častice v 1D prostoru. Odvzozené vztahy jsou obecné je možné je jednoduše vyjádřit také ve 3D nebo pomocí zoběcněných souřadnic.

\sphinxAtStartPar
\textbackslash{}begin\{equation\}
\textbackslash{}vec\{r\}(t) = \textbackslash{}vec\{x\}(t) = x \textbackslash{}vec\{i\}
\textbackslash{}end\{equation\}

\sphinxAtStartPar
A vzhledem k její poloze jsou rychlost a zrychlení částice:

\sphinxAtStartPar
\textbackslash{}begin\{equation\} \textbackslash{}begin\{array\}\{l\}
\textbackslash{}vec\{v\}(t) = \textbackslash{}dfrac\{\textbackslash{}mathrm d \textbackslash{}vec\{x\}(t)\}\{\textbackslash{}mathrm d t\} = \textbackslash{}dfrac\{d x(t)\}\{\textbackslash{}mathrm d t\}\textbackslash{}vec\{i\}= \textbackslash{}dot\{\textbackslash{}vec\{x\}\}\textbackslash{}{[}2em{]}
\textbackslash{}vec\{a\}(t) = \textbackslash{}dfrac\{\textbackslash{}mathrm d \textbackslash{}vec\{x\}(t)\}\{\textbackslash{}mathrm d t\} = \textbackslash{}dfrac\{\textbackslash{}mathrm d\textasciicircum{}2 \textbackslash{}vec\{x\}(t)\}\{\textbackslash{}mathrm d t\textasciicircum{}2\} = \textbackslash{}ddot\{\textbackslash{}vec\{x\}\}
\textbackslash{}end\{array\}
\textbackslash{}end\{equation\}

\sphinxAtStartPar
Hybnost částice je definována jako:

\sphinxAtStartPar
\textbackslash{}begin\{equation\}
\textbackslash{}vec\{p\}(t) = m\textbackslash{}vec\{v\}(t) = m \textbackslash{}dot\{\textbackslash{}vec\{x\}\}
\textbackslash{}end\{equation\}

\sphinxAtStartPar
kde \(m\) a \(\dot{\vec{x}}\) jsou hmotnost a rychlost tělesa.

\sphinxAtStartPar
Druhý Newtonův zákon dává do souvislosti výslednou sílu působící na částici s rychlostí změny její hybnosti, a pokud je hmotnost konstantní:

\sphinxAtStartPar
\textbackslash{}begin\{equation\} \textbackslash{}begin\{array\}\{l\}
\textbackslash{}vec\{F\}(t) = \textbackslash{}dfrac\{\textbackslash{}mathrm d \textbackslash{}vec\{p\}(t)\}\{\textbackslash{}mathrm d t\} = \textbackslash{}dfrac\{\textbackslash{}mathrm d (m \textbackslash{}vec\{v\}(t))\}\{\textbackslash{}mathrm d t\} =  \textbackslash{}dfrac\{\textbackslash{}mathrm d (m \textbackslash{}dot\{\textbackslash{}vec\{x\}\})\}\{\textbackslash{}mathrm d t\}\textbackslash{}{[}2em{]}
\textbackslash{}vec\{F\}(t) = m\textbackslash{}vec\{a\}(t) = m \textbackslash{}ddot\{\textbackslash{}vec\{x\}\}
\textbackslash{}end\{array\} \textbackslash{}end\{equation\}


\sphinxAtStartPar
Z druhého Newtonova zákona, pokud je známa poloha částice v libovolném okamžiku, lze určit výslednou sílu, která na ni působí. Pokud poloha není známa, ale výsledná síla ano, polohu částice lze určit řešením následující obyčejné diferenciální rovnice druhého řádu:

\sphinxAtStartPar
\textbackslash{}begin\{equation\}
\textbackslash{}frac\{\textbackslash{}mathrm d\textasciicircum{}2 \textbackslash{}vec\{x\}(t)\}\{\textbackslash{}mathrm d t\textasciicircum{}2\} = \textbackslash{}frac\{\textbackslash{}vec\{F\}(t)\}\{m\}
\textbackslash{}end\{equation\}

\sphinxAtStartPar
Výše uvedená diferenciální rovnice je označována jako pohybová rovnice částice. Například systém \(N\) částic bude vyžadovat \(N\) rovnic k popisu v 1D a \(3N\) rovnic ve 3D. Pohybová rovnice má obecné řešení

\sphinxAtStartPar
\textbackslash{}begin\{equation\}
\textbackslash{}vec\{x\}(t) = \textbackslash{}int \textbackslash{}left(\textbackslash{}int\textbackslash{}frac\{\textbackslash{}vec\{F\}(t)\}\{m\} \textbackslash{}mathrm\{d\}t\textbackslash{}right) \textbackslash{}mathrm\{d\}t
\textbackslash{}end\{equation\}

\sphinxAtStartPar
což vyžaduje určení dvou konstant, počáteční polohy a rychlosti. Kůli jednoduchosti se na chvíli vzdáme vektorového zápisu a budeme popisovat polohu části pomocí parametru \(x\), polohy na číselné ose. známenko určuje směr vektoru.


\subsection{Energie, síla, hybnost}
\label{\detokenize{Prednasky/1_5_Lagrange_Eulerovy_rovnice:energie-sila-hybnost}}
\sphinxAtStartPar
Pro popis stavu částic můžeme využít také mechanickou energii, která je součtem kinetické a potenciálních energie. Kinetická energie, \(E_k\) částice je dána vztahem:

\sphinxAtStartPar
\textbackslash{}begin\{equation\}
E\_k = \textbackslash{}frac\{1\}\{2\}m v\textasciicircum{}2
\textbackslash{}end\{equation\}

\sphinxAtStartPar
Což lze vyjádřit pomocí hybnosti jako:

\sphinxAtStartPar
\textbackslash{}begin\{equation\}
E\_k = \textbackslash{}frac\{1\}\{2m\} p\textasciicircum{}2
\textbackslash{}end\{equation\}

\sphinxAtStartPar
A pro danou souřadnici pohybu částice lze její hybnost získat z její kinetické energie:

\sphinxAtStartPar
\textbackslash{}begin\{equation\}
\textbackslash{}vec\{p\} = \textbackslash{}frac\{\textbackslash{}partial E\_k\}\{\textbackslash{}partial \textbackslash{}vec\{v\}\}
\textbackslash{}end\{equation\}

\sphinxAtStartPar
Potenciální energie \(U\) je uložená energie částice a její formulace je závislá na síle působící na částici. Pro konzervativní sílu závislou pouze na poloze částice \(x\) v 1D, například v důsledku gravitačního pole v blízkosti zemského povrchu nebo v důsledku lineární pružiny, lze sílu vyjádřit pomocí gradientu potenciální energie:

\sphinxAtStartPar
\textbackslash{}begin\{equation\}
\textbackslash{}vec\{F\} = \sphinxhyphen{}\textbackslash{}nabla U(\textbackslash{}vec\{x\}) = \sphinxhyphen{}\textbackslash{}frac\{\textbackslash{}partial U\}\{\textbackslash{}partial x\}\textbackslash{}vec\{i\}
\textbackslash{}label\{eq12\}
\textbackslash{}end\{equation\}


\subsection{Lagrangeova rovnice v kartézských souřadnicích}
\label{\detokenize{Prednasky/1_5_Lagrange_Eulerovy_rovnice:lagrangeova-rovnice-v-kartezskych-souradnicich}}
\sphinxAtStartPar
Pro jednoduchost si nejprve odvodme Lagrangeovu rovnici pro částici v kartézských souřadnicích a z druhého Newtonova zákona.

\sphinxAtStartPar
Protože chceme odvodit zákony pohybu založené na mechanické energii částice, můžeme vidět, že časová derivace výrazu pro hybnost jako funkce kinetické energie, se rovná síle působící na částici a sílu ve druhém Newtonově zákoně můžeme nahradit tímto výrazem:

\sphinxAtStartPar
\textbackslash{}begin\{equation\}
\textbackslash{}frac\{\textbackslash{}mathrm d \}\{\textbackslash{}mathrm d t\}\textbackslash{}left(\textbackslash{}frac\{\textbackslash{}partial E\_k\}\{\textbackslash{}partial \textbackslash{}dot x\}\textbackslash{}right) = m\textbackslash{}ddot x
\textbackslash{}label\{eq13\}
\textbackslash{}end\{equation\}

\sphinxAtStartPar
Viděli jsme, že konzervativní sílu lze vyjádřit také pomocí potenciální energie částice; nahradíme\sphinxhyphen{}li pravou stranu výše uvedené rovnice tímto výrazem, máme:

\sphinxAtStartPar
\textbackslash{}begin\{equation\}
\textbackslash{}frac\{\textbackslash{}mathrm d \}\{\textbackslash{}mathrm d t\}\textbackslash{}left(\textbackslash{}frac\{\textbackslash{}partial T\}\{\textbackslash{}partial \textbackslash{}dot x\}\textbackslash{}right) = \sphinxhyphen{}\textbackslash{}frac\{\textbackslash{}partial U\}\{\textbackslash{}partial x\}
\textbackslash{}label\{eq14\}
\textbackslash{}end\{equation\}

\sphinxAtStartPar
S využitím skutečnosti, že:

\sphinxAtStartPar
\textbackslash{}begin\{equation\}
\textbackslash{}frac\{\textbackslash{}partial E\_k\}\{\textbackslash{}partial x\} = 0 \textbackslash{}quad a \textbackslash{}quad \textbackslash{}frac\{\textbackslash{}partial U\}\{\textbackslash{}partial \textbackslash{}dot x\} = 0
\textbackslash{}label\{eq15\}
\textbackslash{}end\{equation\}

\sphinxAtStartPar
Můžeme napsat:

\sphinxAtStartPar
\textbackslash{}begin\{equation\}
\textbackslash{}frac\{\textbackslash{}mathrm d \}\{\textbackslash{}mathrm d t\}\textbackslash{}left(\textbackslash{}frac\{\textbackslash{}partial (E\_k\sphinxhyphen{}U)\}\{\textbackslash{}partial \textbackslash{}dot x\}\textbackslash{}right) \sphinxhyphen{} \textbackslash{}frac\{\textbackslash{}partial (E\_k\sphinxhyphen{}U)\}\{\textbackslash{}partial x\} = 0
\textbackslash{}label\{eq16\}
\textbackslash{}end\{equation\}

\sphinxAtStartPar
Definujeme Lagrangeovu funkce nebo Lagranžián \(\mathcal{L}\) jako rozdílu mezi kinetickou a potenciální energií v systému:

\sphinxAtStartPar
\textbackslash{}begin\{equation\}
\textbackslash{}mathcal\{L\} = T \sphinxhyphen{} V
\textbackslash{}label\{eq17\}
\textbackslash{}end\{equation\}

\sphinxAtStartPar
Pro konzervativní sílu působící na částici máme Lagrangeovu rovnici v kartézských souřadnicích:

\sphinxAtStartPar
\textbackslash{}begin\{equation\}
\textbackslash{}frac\{\textbackslash{}mathrm d \}\{\textbackslash{}mathrm d t\}\textbackslash{}left(\textbackslash{}frac\{\textbackslash{}partial \textbackslash{}mathcal\{L\}\}\{\textbackslash{}partial \textbackslash{}dot x\}\textbackslash{}right) \sphinxhyphen{} \textbackslash{}frac\{\textbackslash{}partial \textbackslash{}mathcal\{L\}\}\{\textbackslash{}partial x\} = 0
\textbackslash{}label\{eq18\}
\textbackslash{}end\{equation\}


\subsection{Pohyb jako optimalizace}
\label{\detokenize{Prednasky/1_5_Lagrange_Eulerovy_rovnice:pohyb-jako-optimalizace}}
\sphinxAtStartPar
Tento přístup začíná veličinou nazývanou akce. Pokud minimalizujete akci, můžete získat cestu nejmenší akce, která představuje cestu, kterou fyzický systém projde prostorem a časem. Obecně řečeno, fyzici k této minimalizaci používají analytické nástroje. Princip minimální akce je variační princip umožňující pro libovolný fyzikální systém předpovědět, jak budou vypadat jeho pohybové rovnice.
Laicky řečeno, je to způsob, jak ze všech myslitelných trajektorií systému vybrat tu, podél které je určitá charakteristika systému (například energie) minimální. Tato se skutečně realizuje.


\subsubsection{Standardní přístup}
\label{\detokenize{Prednasky/1_5_Lagrange_Eulerovy_rovnice:standardni-pristup}}
\sphinxAtStartPar
Pro vrh vzhůru s počáteční rychlostí \(v_0\) a počáteční výškou \(y_0\) platí
\begin{equation*}
\begin{split}y(t)=-\frac{1}{2}gt^2+v_0t+y_0\end{split}
\end{equation*}
\begin{sphinxuseclass}{cell}\begin{sphinxVerbatimInput}

\begin{sphinxuseclass}{cell_input}
\begin{sphinxVerbatim}[commandchars=\\\{\}]
\PYG{k}{def} \PYG{n+nf}{falling\PYGZus{}object\PYGZus{}analytic}\PYG{p}{(}\PYG{n}{x0}\PYG{p}{,} \PYG{n}{x1}\PYG{p}{,} \PYG{n}{dt}\PYG{p}{,} \PYG{n}{g}\PYG{o}{=}\PYG{l+m+mi}{1}\PYG{p}{,} \PYG{n}{steps}\PYG{o}{=}\PYG{l+m+mi}{100}\PYG{p}{)}\PYG{p}{:}
    \PYG{n}{v0} \PYG{o}{=} \PYG{p}{(}\PYG{n}{x1} \PYG{o}{\PYGZhy{}} \PYG{n}{x0}\PYG{p}{)} \PYG{o}{/} \PYG{n}{dt}
    \PYG{n}{t} \PYG{o}{=} \PYG{n}{np}\PYG{o}{.}\PYG{n}{linspace}\PYG{p}{(}\PYG{l+m+mi}{0}\PYG{p}{,} \PYG{n}{steps}\PYG{p}{,} \PYG{n}{steps}\PYG{o}{+}\PYG{l+m+mi}{1}\PYG{p}{)} \PYG{o}{*} \PYG{n}{dt}
    \PYG{n}{x} \PYG{o}{=} \PYG{o}{\PYGZhy{}}\PYG{l+m+mf}{.5}\PYG{o}{*}\PYG{n}{g}\PYG{o}{*}\PYG{n}{t}\PYG{o}{*}\PYG{o}{*}\PYG{l+m+mi}{2} \PYG{o}{+} \PYG{n}{v0}\PYG{o}{*}\PYG{n}{t} \PYG{o}{+} \PYG{n}{x0}  \PYG{c+c1}{\PYGZsh{} the equation of motion}
    \PYG{k}{return} \PYG{n}{t}\PYG{p}{,} \PYG{n}{x}

\PYG{n}{x0}\PYG{p}{,} \PYG{n}{x1} \PYG{o}{=} \PYG{p}{[}\PYG{l+m+mi}{0}\PYG{p}{,} \PYG{l+m+mi}{2}\PYG{p}{]}
\PYG{n}{dt} \PYG{o}{=} \PYG{l+m+mf}{0.19}
\PYG{n}{t\PYGZus{}ana}\PYG{p}{,} \PYG{n}{x\PYGZus{}ana} \PYG{o}{=} \PYG{n}{falling\PYGZus{}object\PYGZus{}analytic}\PYG{p}{(}\PYG{n}{x0}\PYG{p}{,} \PYG{n}{x1}\PYG{p}{,} \PYG{n}{dt}\PYG{p}{)}

\PYG{n}{plt}\PYG{o}{.}\PYG{n}{figure}\PYG{p}{(}\PYG{p}{)}
\PYG{n}{plt}\PYG{o}{.}\PYG{n}{plot}\PYG{p}{(}\PYG{n}{t\PYGZus{}ana}\PYG{p}{,} \PYG{n}{x\PYGZus{}ana}\PYG{p}{,} \PYG{l+s+s1}{\PYGZsq{}}\PYG{l+s+s1}{k.\PYGZhy{}}\PYG{l+s+s1}{\PYGZsq{}}\PYG{p}{,} \PYG{n}{label}\PYG{o}{=}\PYG{l+s+s1}{\PYGZsq{}}\PYG{l+s+s1}{Analytické řešení}\PYG{l+s+s1}{\PYGZsq{}}\PYG{p}{)}
\PYG{n}{plt}\PYG{o}{.}\PYG{n}{xlabel}\PYG{p}{(}\PYG{l+s+s1}{\PYGZsq{}}\PYG{l+s+s1}{Čas (s)}\PYG{l+s+s1}{\PYGZsq{}}\PYG{p}{)} \PYG{p}{;} \PYG{n}{plt}\PYG{o}{.}\PYG{n}{ylabel}\PYG{p}{(}\PYG{l+s+s1}{\PYGZsq{}}\PYG{l+s+s1}{Výška (m)}\PYG{l+s+s1}{\PYGZsq{}}\PYG{p}{)} \PYG{p}{;} \PYG{n}{plt}\PYG{o}{.}\PYG{n}{legend}\PYG{p}{(}\PYG{n}{fontsize}\PYG{o}{=}\PYG{l+m+mi}{12}\PYG{p}{)}
\PYG{n}{plt}\PYG{o}{.}\PYG{n}{tight\PYGZus{}layout}\PYG{p}{(}\PYG{p}{)} \PYG{p}{;} \PYG{n}{plt}\PYG{o}{.}\PYG{n}{show}\PYG{p}{(}\PYG{p}{)}
\end{sphinxVerbatim}

\end{sphinxuseclass}\end{sphinxVerbatimInput}
\begin{sphinxVerbatimOutput}

\begin{sphinxuseclass}{cell_output}
\noindent\sphinxincludegraphics{{4e2f3fb115be776f2a555f6a7a95c5c4a16ebd29a2bfc3233236af596e26cbb0}.png}

\end{sphinxuseclass}\end{sphinxVerbatimOutput}

\end{sphinxuseclass}

\subsubsection{Numerický přístup}
\label{\detokenize{Prednasky/1_5_Lagrange_Eulerovy_rovnice:numericky-pristup}}
\sphinxAtStartPar
Ne všechny fyzikální problémy mají analytické řešení. Některé, jako dvojité kyvadlo nebo problém tří těles, jsou deterministické, ale chaotické. Jinými slovy, jejich dynamika je předvídatelná, ale nemůžeme znát jejich stav někdy v budoucnu, aniž bychom simulovali všechny zasahující stavy. Ty můžeme řešit numerickou integrací. U těla v gravitačním poli by numerický přístup vypadal takto:

\begin{sphinxuseclass}{cell}\begin{sphinxVerbatimInput}

\begin{sphinxuseclass}{cell_input}
\begin{sphinxVerbatim}[commandchars=\\\{\}]
\PYG{k}{def} \PYG{n+nf}{falling\PYGZus{}object\PYGZus{}numerical}\PYG{p}{(}\PYG{n}{x0}\PYG{p}{,} \PYG{n}{x1}\PYG{p}{,} \PYG{n}{dt}\PYG{p}{,} \PYG{n}{g}\PYG{o}{=}\PYG{l+m+mi}{1}\PYG{p}{,} \PYG{n}{steps}\PYG{o}{=}\PYG{l+m+mi}{100}\PYG{p}{)}\PYG{p}{:}
    \PYG{n}{xs} \PYG{o}{=} \PYG{p}{[}\PYG{n}{x0}\PYG{p}{,} \PYG{n}{x1}\PYG{p}{]}
    \PYG{n}{ts} \PYG{o}{=} \PYG{p}{[}\PYG{l+m+mi}{0}\PYG{p}{,} \PYG{n}{dt}\PYG{p}{]}
    \PYG{n}{v} \PYG{o}{=} \PYG{p}{(}\PYG{n}{x1} \PYG{o}{\PYGZhy{}} \PYG{n}{x0}\PYG{p}{)} \PYG{o}{/} \PYG{n}{dt}
    \PYG{n}{x} \PYG{o}{=} \PYG{n}{xs}\PYG{p}{[}\PYG{o}{\PYGZhy{}}\PYG{l+m+mi}{1}\PYG{p}{]}
    \PYG{k}{for} \PYG{n}{i} \PYG{o+ow}{in} \PYG{n+nb}{range}\PYG{p}{(}\PYG{n}{steps}\PYG{o}{\PYGZhy{}}\PYG{l+m+mi}{1}\PYG{p}{)}\PYG{p}{:}
        \PYG{n}{v} \PYG{o}{+}\PYG{o}{=} \PYG{o}{\PYGZhy{}}\PYG{n}{g}\PYG{o}{*}\PYG{n}{dt}
        \PYG{n}{x} \PYG{o}{+}\PYG{o}{=} \PYG{n}{v}\PYG{o}{*}\PYG{n}{dt}
        \PYG{n}{xs}\PYG{o}{.}\PYG{n}{append}\PYG{p}{(}\PYG{n}{x}\PYG{p}{)}
        \PYG{n}{ts}\PYG{o}{.}\PYG{n}{append}\PYG{p}{(}\PYG{n}{ts}\PYG{p}{[}\PYG{o}{\PYGZhy{}}\PYG{l+m+mi}{1}\PYG{p}{]}\PYG{o}{+}\PYG{n}{dt}\PYG{p}{)}
    \PYG{k}{return} \PYG{n}{np}\PYG{o}{.}\PYG{n}{asarray}\PYG{p}{(}\PYG{n}{ts}\PYG{p}{)}\PYG{p}{,} \PYG{n}{np}\PYG{o}{.}\PYG{n}{asarray}\PYG{p}{(}\PYG{n}{xs}\PYG{p}{)}

\PYG{n}{t\PYGZus{}num}\PYG{p}{,} \PYG{n}{x\PYGZus{}num} \PYG{o}{=} \PYG{n}{falling\PYGZus{}object\PYGZus{}numerical}\PYG{p}{(}\PYG{n}{x0}\PYG{p}{,} \PYG{n}{x1}\PYG{p}{,} \PYG{n}{dt}\PYG{p}{)}

\PYG{n}{plt}\PYG{o}{.}\PYG{n}{figure}\PYG{p}{(}\PYG{p}{)}
\PYG{n}{plt}\PYG{o}{.}\PYG{n}{plot}\PYG{p}{(}\PYG{n}{t\PYGZus{}ana}\PYG{p}{,} \PYG{n}{x\PYGZus{}ana}\PYG{p}{,} \PYG{l+s+s1}{\PYGZsq{}}\PYG{l+s+s1}{k.\PYGZhy{}}\PYG{l+s+s1}{\PYGZsq{}}\PYG{p}{,} \PYG{n}{label}\PYG{o}{=}\PYG{l+s+s1}{\PYGZsq{}}\PYG{l+s+s1}{Analytické řešení}\PYG{l+s+s1}{\PYGZsq{}}\PYG{p}{)}
\PYG{n}{plt}\PYG{o}{.}\PYG{n}{plot}\PYG{p}{(}\PYG{n}{t\PYGZus{}num}\PYG{p}{,} \PYG{n}{x\PYGZus{}num}\PYG{p}{,} \PYG{l+s+s1}{\PYGZsq{}}\PYG{l+s+s1}{r.\PYGZhy{}}\PYG{l+s+s1}{\PYGZsq{}}\PYG{p}{,} \PYG{n}{label}\PYG{o}{=}\PYG{l+s+s1}{\PYGZsq{}}\PYG{l+s+s1}{Numerické řešení}\PYG{l+s+s1}{\PYGZsq{}}\PYG{p}{)}
\PYG{n}{plt}\PYG{o}{.}\PYG{n}{xlabel}\PYG{p}{(}\PYG{l+s+s1}{\PYGZsq{}}\PYG{l+s+s1}{Čas (s)}\PYG{l+s+s1}{\PYGZsq{}}\PYG{p}{)} \PYG{p}{;} \PYG{n}{plt}\PYG{o}{.}\PYG{n}{ylabel}\PYG{p}{(}\PYG{l+s+s1}{\PYGZsq{}}\PYG{l+s+s1}{Výška (m)}\PYG{l+s+s1}{\PYGZsq{}}\PYG{p}{)} \PYG{p}{;} \PYG{n}{plt}\PYG{o}{.}\PYG{n}{legend}\PYG{p}{(}\PYG{n}{fontsize}\PYG{o}{=}\PYG{l+m+mi}{12}\PYG{p}{)}
\PYG{n}{plt}\PYG{o}{.}\PYG{n}{tight\PYGZus{}layout}\PYG{p}{(}\PYG{p}{)} \PYG{p}{;} \PYG{n}{plt}\PYG{o}{.}\PYG{n}{show}\PYG{p}{(}\PYG{p}{)}
\end{sphinxVerbatim}

\end{sphinxuseclass}\end{sphinxVerbatimInput}
\begin{sphinxVerbatimOutput}

\begin{sphinxuseclass}{cell_output}
\noindent\sphinxincludegraphics{{e0fe521ef2b9bce34794643b4cb4ff58f1c9fe22bf46ad9740ed99232efacdb8}.png}

\end{sphinxuseclass}\end{sphinxVerbatimOutput}

\end{sphinxuseclass}

\subsubsection{Minimalizace akce}
\label{\detokenize{Prednasky/1_5_Lagrange_Eulerovy_rovnice:minimalizace-akce}}
\sphinxAtStartPar
\sphinxstylestrong{Lagrangeova metoda.} Přístupy, které jsme právě probrali, dávají intuitivní smysl. Proto je učíme v úvodních hodinách fyziky. Existuje však zcela jiný způsob pohledu na dynamiku, který se nazývá Lagrangeova metoda. Lagrangeova metoda lépe popisuje realitu, protože dokáže vytvářet pohybové rovnice pro jakýkoli fyzikální systém. Lagrangiány figurují ve všech čtyřech odvětvích fyziky: klasické mechanice, elektřině a magnetismu, termodynamice a kvantové mechanice. Bez Lagrangianovy metody by fyzici tyto nesourodé obory jen těžko sjednocovali. Ale se \sphinxhref{https://www.symmetrymagazine.org/article/the-deconstructed-standard-model-equation}{standardním model Lagrangiánu} dokážou přesně to.

\sphinxAtStartPar
\sphinxstylestrong{Jak to funguje.} Lagrangiánská metoda začíná zvážením všech cest, kterými by se fyzický systém mohl vydat z počátečního stavu \(\bf x(t_0)\) do konečného stavu \(\bf x(t_1)\). Pak poskytuje jednoduché pravidlo pro výběr cesty \(\hat{\bf x}\), kterou příroda skutečně provede: akce \(S\), definovaná v rovnici níže, musí mít nad touto cestou stacionární hodnotu (maximum nebo minimum). Zde \(T\) a \(V\) jsou funkce kinetické a potenciální energie pro systém v jakémkoli daném čase \(t\) v \([t_0,t_1]\).
\$\(
\begin{aligned}
S &:= \int_{t_0}^{t_1} \mathcal L({\bf x}, ~ \dot{\bf x}, ~ t) ~ dt
\quad \textrm{where}\quad \mathcal L = T - V \\
\quad \hat{\bf x} &~~ \textrm{má vlastnost} \quad \frac{d}{dt} \left( \frac{\partial \mathcal L}{\partial \dot{\hat{x}}(t)} \right) = \frac{\partial \mathcal L}{\partial \hat{x}(t)} \quad \textrm{for} \quad t \in [t_0,t_1]
\end{aligned}
\)\(
**Nalezení \)\textbackslash{}hat\{\textbackslash{}bf x\}\( pomocí Euler-Lagrange (co lidé obvykle dělají).** Když je \)S\( stacionární, můžeme ukázat, že Euler-Lagrangeova rovnice (druhý řádek výše uvedených rovnic) platí v intervalu \){[}t\_0,t\_1{]}\( (Morin, 2008). Toto pozorování je cenné, protože nám umožňuje řešit pro \)\textbackslash{}hat\{\textbackslash{}bf x\}\(: nejprve aplikujeme Euler-Lagrangeovu rovnici na Lagrangian \)L\( a odvodíme systém parciálních diferenciálních rovnic. Potom tyto rovnice integrujeme a získáme \)\textbackslash{}hat\{\textbackslash{}bf x\}\$. Důležité je, že tento přístup funguje pro všechny problémy zahrnující klasickou mechaniku, elektrodynamiku, termodynamiku a teorii relativity. Poskytuje ucelený teoretický rámec pro studium klasické fyziky jako celku.

\sphinxAtStartPar
\sphinxstylestrong{Nalezení \(\hat{\bf x}\) s minimalizací akce (co budeme dělat).} Přímější přístup k nalezení \(\hat{\bf x}\) začíná zjištěním, že cesty stacionární akce jsou téměř vždy také cestami nejménší akce (Morin 2008). Bez velké ztráty obecnosti tedy můžeme vyměnit Euler\sphinxhyphen{}Lagrangeovu rovnici za jednoduchý cíl minimalizace uvedený ve třetí části rovnice níže. Mezitím, jak je ukázáno v první části rovnice níže, můžeme předefinovat \(S\) jako diskrétní součet přes \(N\) rovnoměrně rozložených časových řezů:
\$\(
S := \sum_{i=0}^{N} L({\bf x}, ~ \dot{{\bf x}}, ~ t_i) \Delta t \quad \textrm{where} \quad \dot{{\bf x}}(t_i) := \frac{ {\bf x}(t_{i+1}) - {\bf x}(t_{i})}{\Delta t} \quad \textrm{and} \quad \hat{\bf x} := \underset{\bf x}{\textrm{argmin}} ~ S(\bf x)
\)\(
Zůstává jeden problém: po diskretizaci \)\textbackslash{}hat\{ \textbackslash{}bf x\}\( již nemůžeme vzít jeho derivaci k získání přesné hodnoty pro \)\textbackslash{}dot\{ \textbackslash{}bf x\}(t\_i)\(. Místo toho musíme použít aproximaci konečných rozdílů uvedenou v druhé části rovnice výše. Tato aproximace samozřejmě nebude možná pro úplně poslední \)\textbackslash{}dot\{ \textbackslash{}bf x\}\( v součtu, protože \)\textbackslash{}dot\{ \textbackslash{}bf x\}\_\{N+1\}\( neexistuje. Pro tuto hodnotu budeme předpokládat, že pro velké \)N\( je změna rychlosti v intervalu \)\textbackslash{}Delta t\( malá a nechť \)\textbackslash{}dot\{ \textbackslash{}bf x\}\sphinxstyleemphasis{N = \textbackslash{}dot\{ \textbackslash{}bf x\}}\{N\sphinxhyphen{}1\}\(. Po provedení této poslední aproximace nyní můžeme vypočítat gradient \)\textbackslash{}frac\{\textbackslash{}partial S\}\{\textbackslash{}partial \{\textbackslash{}bf x\}\}\( numericky a použít jej k minimalizaci \)S\$. To lze provést pomocí PyTorch (Paszke et al, 2019).

\begin{sphinxuseclass}{cell}\begin{sphinxVerbatimInput}

\begin{sphinxuseclass}{cell_input}
\begin{sphinxVerbatim}[commandchars=\\\{\}]
\PYG{k}{def} \PYG{n+nf}{lagrangian\PYGZus{}freebody}\PYG{p}{(}\PYG{n}{x}\PYG{p}{,} \PYG{n}{xdot}\PYG{p}{,} \PYG{n}{m}\PYG{o}{=}\PYG{l+m+mi}{1}\PYG{p}{,} \PYG{n}{g}\PYG{o}{=}\PYG{l+m+mi}{1}\PYG{p}{)}\PYG{p}{:}
    \PYG{n}{T} \PYG{o}{=} \PYG{l+m+mf}{.5}\PYG{o}{*}\PYG{n}{m}\PYG{o}{*}\PYG{n}{xdot}\PYG{o}{*}\PYG{o}{*}\PYG{l+m+mi}{2}
    \PYG{n}{V} \PYG{o}{=} \PYG{n}{m}\PYG{o}{*}\PYG{n}{g}\PYG{o}{*}\PYG{n}{x}
    \PYG{k}{return} \PYG{n}{T}\PYG{p}{,} \PYG{n}{V}
  
\PYG{k}{def} \PYG{n+nf}{action}\PYG{p}{(}\PYG{n}{x}\PYG{p}{,} \PYG{n}{dt}\PYG{p}{)}\PYG{p}{:}
    \PYG{n}{xdot} \PYG{o}{=} \PYG{p}{(}\PYG{n}{x}\PYG{p}{[}\PYG{l+m+mi}{1}\PYG{p}{:}\PYG{p}{]} \PYG{o}{\PYGZhy{}} \PYG{n}{x}\PYG{p}{[}\PYG{p}{:}\PYG{o}{\PYGZhy{}}\PYG{l+m+mi}{1}\PYG{p}{]}\PYG{p}{)} \PYG{o}{/} \PYG{n}{dt}
    \PYG{n}{xdot} \PYG{o}{=} \PYG{n}{torch}\PYG{o}{.}\PYG{n}{cat}\PYG{p}{(}\PYG{p}{[}\PYG{n}{xdot}\PYG{p}{,} \PYG{n}{xdot}\PYG{p}{[}\PYG{o}{\PYGZhy{}}\PYG{l+m+mi}{1}\PYG{p}{:}\PYG{p}{]}\PYG{p}{]}\PYG{p}{,} \PYG{n}{axis}\PYG{o}{=}\PYG{l+m+mi}{0}\PYG{p}{)}
    \PYG{n}{T}\PYG{p}{,} \PYG{n}{V} \PYG{o}{=} \PYG{n}{lagrangian\PYGZus{}freebody}\PYG{p}{(}\PYG{n}{x}\PYG{p}{,} \PYG{n}{xdot}\PYG{p}{)}
    \PYG{k}{return} \PYG{n}{T}\PYG{o}{.}\PYG{n}{sum}\PYG{p}{(}\PYG{p}{)}\PYG{o}{\PYGZhy{}}\PYG{n}{V}\PYG{o}{.}\PYG{n}{sum}\PYG{p}{(}\PYG{p}{)}
\end{sphinxVerbatim}

\end{sphinxuseclass}\end{sphinxVerbatimInput}

\end{sphinxuseclass}
\sphinxAtStartPar
Nyní hledejme bod stacionární bod. Technicky vzato to může být minimum NEBO inflexní bod. Zde budeme hledat pouze minimum:

\begin{sphinxuseclass}{cell}\begin{sphinxVerbatimInput}

\begin{sphinxuseclass}{cell_input}
\begin{sphinxVerbatim}[commandchars=\\\{\}]
\PYG{k+kn}{import} \PYG{n+nn}{torch}
\PYG{k}{def} \PYG{n+nf}{get\PYGZus{}path\PYGZus{}between}\PYG{p}{(}\PYG{n}{x}\PYG{p}{,} \PYG{n}{steps}\PYG{o}{=}\PYG{l+m+mi}{1000}\PYG{p}{,} \PYG{n}{step\PYGZus{}size}\PYG{o}{=}\PYG{l+m+mf}{1e\PYGZhy{}1}\PYG{p}{,} \PYG{n}{dt}\PYG{o}{=}\PYG{l+m+mi}{1}\PYG{p}{,} \PYG{n}{num\PYGZus{}prints}\PYG{o}{=}\PYG{l+m+mi}{8}\PYG{p}{,} \PYG{n}{num\PYGZus{}stashes}\PYG{o}{=}\PYG{l+m+mi}{80}\PYG{p}{)}\PYG{p}{:}
    \PYG{n}{t} \PYG{o}{=} \PYG{n}{np}\PYG{o}{.}\PYG{n}{linspace}\PYG{p}{(}\PYG{l+m+mi}{0}\PYG{p}{,} \PYG{n+nb}{len}\PYG{p}{(}\PYG{n}{x}\PYG{p}{)}\PYG{o}{\PYGZhy{}}\PYG{l+m+mi}{1}\PYG{p}{,} \PYG{n+nb}{len}\PYG{p}{(}\PYG{n}{x}\PYG{p}{)}\PYG{p}{)} \PYG{o}{*} \PYG{n}{dt}
    \PYG{n}{print\PYGZus{}on} \PYG{o}{=} \PYG{n}{np}\PYG{o}{.}\PYG{n}{linspace}\PYG{p}{(}\PYG{l+m+mi}{0}\PYG{p}{,}\PYG{n+nb}{int}\PYG{p}{(}\PYG{n}{np}\PYG{o}{.}\PYG{n}{sqrt}\PYG{p}{(}\PYG{n}{steps}\PYG{p}{)}\PYG{p}{)}\PYG{p}{,}\PYG{n}{num\PYGZus{}prints}\PYG{p}{)}\PYG{o}{.}\PYG{n}{astype}\PYG{p}{(}\PYG{n}{np}\PYG{o}{.}\PYG{n}{int32}\PYG{p}{)}\PYG{o}{*}\PYG{o}{*}\PYG{l+m+mi}{2} \PYG{c+c1}{\PYGZsh{} print more often early in loop}
    \PYG{n}{stash\PYGZus{}on} \PYG{o}{=} \PYG{n}{np}\PYG{o}{.}\PYG{n}{linspace}\PYG{p}{(}\PYG{l+m+mi}{0}\PYG{p}{,}\PYG{n+nb}{int}\PYG{p}{(}\PYG{n}{np}\PYG{o}{.}\PYG{n}{sqrt}\PYG{p}{(}\PYG{n}{steps}\PYG{p}{)}\PYG{p}{)}\PYG{p}{,}\PYG{n}{num\PYGZus{}stashes}\PYG{p}{)}\PYG{o}{.}\PYG{n}{astype}\PYG{p}{(}\PYG{n}{np}\PYG{o}{.}\PYG{n}{int32}\PYG{p}{)}\PYG{o}{*}\PYG{o}{*}\PYG{l+m+mi}{2}
    \PYG{n}{xs} \PYG{o}{=} \PYG{p}{[}\PYG{p}{]}
    \PYG{k}{for} \PYG{n}{i} \PYG{o+ow}{in} \PYG{n+nb}{range}\PYG{p}{(}\PYG{n}{steps}\PYG{p}{)}\PYG{p}{:}
        \PYG{n}{grad\PYGZus{}x} \PYG{o}{=} \PYG{n}{torch}\PYG{o}{.}\PYG{n}{autograd}\PYG{o}{.}\PYG{n}{grad}\PYG{p}{(}\PYG{n}{action}\PYG{p}{(}\PYG{n}{x}\PYG{p}{,} \PYG{n}{dt}\PYG{p}{)}\PYG{p}{,} \PYG{n}{x}\PYG{p}{)}\PYG{p}{[}\PYG{l+m+mi}{0}\PYG{p}{]}
        \PYG{n}{grad\PYGZus{}x}\PYG{p}{[}\PYG{p}{[}\PYG{l+m+mi}{0}\PYG{p}{,}\PYG{o}{\PYGZhy{}}\PYG{l+m+mi}{1}\PYG{p}{]}\PYG{p}{]} \PYG{o}{*}\PYG{o}{=} \PYG{l+m+mi}{0}  \PYG{c+c1}{\PYGZsh{} fix first and last coordinates by zeroing their grads}
        \PYG{n}{x}\PYG{o}{.}\PYG{n}{data} \PYG{o}{\PYGZhy{}}\PYG{o}{=} \PYG{n}{grad\PYGZus{}x} \PYG{o}{*} \PYG{n}{step\PYGZus{}size}

        \PYG{k}{if} \PYG{n}{i} \PYG{o+ow}{in} \PYG{n}{print\PYGZus{}on}\PYG{p}{:}
            \PYG{n+nb}{print}\PYG{p}{(}\PYG{l+s+s1}{\PYGZsq{}}\PYG{l+s+s1}{step=}\PYG{l+s+si}{\PYGZob{}:04d\PYGZcb{}}\PYG{l+s+s1}{, S=}\PYG{l+s+si}{\PYGZob{}:.4e\PYGZcb{}}\PYG{l+s+s1}{ J*s}\PYG{l+s+s1}{\PYGZsq{}}\PYG{o}{.}\PYG{n}{format}\PYG{p}{(}\PYG{n}{i}\PYG{p}{,} \PYG{n}{action}\PYG{p}{(}\PYG{n}{x}\PYG{p}{,} \PYG{n}{dt}\PYG{p}{)}\PYG{o}{.}\PYG{n}{item}\PYG{p}{(}\PYG{p}{)}\PYG{p}{)}\PYG{p}{)}
        \PYG{k}{if} \PYG{n}{i} \PYG{o+ow}{in} \PYG{n}{stash\PYGZus{}on}\PYG{p}{:}
            \PYG{n}{xs}\PYG{o}{.}\PYG{n}{append}\PYG{p}{(}\PYG{n}{x}\PYG{o}{.}\PYG{n}{clone}\PYG{p}{(}\PYG{p}{)}\PYG{o}{.}\PYG{n}{data}\PYG{o}{.}\PYG{n}{numpy}\PYG{p}{(}\PYG{p}{)}\PYG{p}{)}
    \PYG{k}{return} \PYG{n}{t}\PYG{p}{,} \PYG{n}{x}\PYG{p}{,} \PYG{n}{np}\PYG{o}{.}\PYG{n}{stack}\PYG{p}{(}\PYG{n}{xs}\PYG{p}{)}
\end{sphinxVerbatim}

\end{sphinxuseclass}\end{sphinxVerbatimInput}

\end{sphinxuseclass}
\sphinxAtStartPar
Nyní si to všechno spojíme dohromady. Můžeme inicializovat dráhu naší padající částice jako libovolnou náhodnou dráhu prostorem. V níže uvedeném kódu jsme zvolili dráhu, na které částice náhodně poskakuje kolem x=0 až do času t=19 sekund, kdy vyskočí do konečného stavu x = \(x_num[-1]\) = 21,3 metru. Tato dráha má velkou akční sílu S = 5425 J\sphinxhyphen{}s. Při optimalizaci se tato hodnota plynule snižuje, až se dostaneme na parabolický oblouk s akčním zásahem S = \sphinxhyphen{}2500 Js.

\begin{sphinxuseclass}{cell}\begin{sphinxVerbatimInput}

\begin{sphinxuseclass}{cell_input}
\begin{sphinxVerbatim}[commandchars=\\\{\}]
\PYG{n}{dt} \PYG{o}{=} \PYG{l+m+mf}{0.19}
\PYG{n}{x0} \PYG{o}{=} \PYG{l+m+mf}{1.5}\PYG{o}{*}\PYG{n}{torch}\PYG{o}{.}\PYG{n}{randn}\PYG{p}{(}\PYG{n+nb}{len}\PYG{p}{(}\PYG{n}{x\PYGZus{}num}\PYG{p}{)}\PYG{p}{,} \PYG{n}{requires\PYGZus{}grad}\PYG{o}{=}\PYG{k+kc}{True}\PYG{p}{)}  \PYG{c+c1}{\PYGZsh{} a random path through space}
\PYG{n}{x0}\PYG{p}{[}\PYG{l+m+mi}{0}\PYG{p}{]}\PYG{o}{.}\PYG{n}{data} \PYG{o}{*}\PYG{o}{=} \PYG{l+m+mf}{0.0} \PYG{p}{;} \PYG{n}{x0}\PYG{p}{[}\PYG{o}{\PYGZhy{}}\PYG{l+m+mi}{1}\PYG{p}{]}\PYG{o}{.}\PYG{n}{data} \PYG{o}{*}\PYG{o}{=} \PYG{l+m+mf}{0.0}  \PYG{c+c1}{\PYGZsh{} set first and last points to zero}
\PYG{n}{x0}\PYG{p}{[}\PYG{o}{\PYGZhy{}}\PYG{l+m+mi}{1}\PYG{p}{]}\PYG{o}{.}\PYG{n}{data} \PYG{o}{+}\PYG{o}{=} \PYG{n}{x\PYGZus{}num}\PYG{p}{[}\PYG{o}{\PYGZhy{}}\PYG{l+m+mi}{1}\PYG{p}{]}  \PYG{c+c1}{\PYGZsh{} set last point to be the end height of the numerical solution}

\PYG{n}{t}\PYG{p}{,} \PYG{n}{x}\PYG{p}{,} \PYG{n}{xs\PYGZus{}ncf} \PYG{o}{=} \PYG{n}{get\PYGZus{}path\PYGZus{}between}\PYG{p}{(}\PYG{n}{x0}\PYG{o}{.}\PYG{n}{clone}\PYG{p}{(}\PYG{p}{)}\PYG{p}{,} \PYG{n}{steps}\PYG{o}{=}\PYG{l+m+mi}{20000}\PYG{p}{,} \PYG{n}{step\PYGZus{}size}\PYG{o}{=}\PYG{l+m+mf}{1e\PYGZhy{}2}\PYG{p}{,} \PYG{n}{dt}\PYG{o}{=}\PYG{n}{dt}\PYG{p}{)}

\PYG{n}{plt}\PYG{o}{.}\PYG{n}{figure}\PYG{p}{(}\PYG{n}{figsize}\PYG{o}{=}\PYG{p}{[}\PYG{l+m+mi}{4}\PYG{p}{,}\PYG{l+m+mi}{3}\PYG{p}{]}\PYG{p}{)}
\PYG{c+c1}{\PYGZsh{} plt.title(\PYGZsq{}Minimizing the Action\PYGZsq{})}
\PYG{n}{plt}\PYG{o}{.}\PYG{n}{plot}\PYG{p}{(}\PYG{n}{t\PYGZus{}ana}\PYG{p}{,} \PYG{n}{x\PYGZus{}ana}\PYG{p}{,} \PYG{l+s+s1}{\PYGZsq{}}\PYG{l+s+s1}{.\PYGZhy{}}\PYG{l+s+s1}{\PYGZsq{}}\PYG{p}{,} \PYG{n}{color}\PYG{o}{=}\PYG{l+s+s1}{\PYGZsq{}}\PYG{l+s+s1}{gray}\PYG{l+s+s1}{\PYGZsq{}}\PYG{p}{,} \PYG{n}{label}\PYG{o}{=}\PYG{l+s+s1}{\PYGZsq{}}\PYG{l+s+s1}{Analytic solution}\PYG{l+s+s1}{\PYGZsq{}}\PYG{p}{)}
\PYG{n}{plt}\PYG{o}{.}\PYG{n}{plot}\PYG{p}{(}\PYG{n}{t\PYGZus{}num}\PYG{p}{,} \PYG{n}{x\PYGZus{}num}\PYG{p}{,} \PYG{l+s+s1}{\PYGZsq{}}\PYG{l+s+s1}{.\PYGZhy{}}\PYG{l+s+s1}{\PYGZsq{}}\PYG{p}{,} \PYG{n}{color}\PYG{o}{=}\PYG{l+s+s1}{\PYGZsq{}}\PYG{l+s+s1}{purple}\PYG{l+s+s1}{\PYGZsq{}}\PYG{p}{,} \PYG{n}{label}\PYG{o}{=}\PYG{l+s+s1}{\PYGZsq{}}\PYG{l+s+s1}{Numerical solution}\PYG{l+s+s1}{\PYGZsq{}}\PYG{p}{)}

\PYG{n}{plt}\PYG{o}{.}\PYG{n}{plot}\PYG{p}{(}\PYG{n}{t}\PYG{p}{,} \PYG{n}{x0}\PYG{o}{.}\PYG{n}{detach}\PYG{p}{(}\PYG{p}{)}\PYG{o}{.}\PYG{n}{numpy}\PYG{p}{(}\PYG{p}{)}\PYG{p}{,} \PYG{l+s+s1}{\PYGZsq{}}\PYG{l+s+s1}{y.\PYGZhy{}}\PYG{l+s+s1}{\PYGZsq{}}\PYG{p}{,} \PYG{n}{label}\PYG{o}{=}\PYG{l+s+s1}{\PYGZsq{}}\PYG{l+s+s1}{Initial (random) path}\PYG{l+s+s1}{\PYGZsq{}}\PYG{p}{)}
\PYG{k}{for} \PYG{n}{i}\PYG{p}{,} \PYG{n}{xi} \PYG{o+ow}{in} \PYG{n+nb}{enumerate}\PYG{p}{(}\PYG{n}{xs\PYGZus{}ncf}\PYG{p}{)}\PYG{p}{:}
    \PYG{n}{label} \PYG{o}{=} \PYG{l+s+s1}{\PYGZsq{}}\PYG{l+s+s1}{During optimization}\PYG{l+s+s1}{\PYGZsq{}} \PYG{k}{if} \PYG{n}{i}\PYG{o}{==}\PYG{l+m+mi}{15} \PYG{k}{else} \PYG{k+kc}{None}
    \PYG{n}{plt}\PYG{o}{.}\PYG{n}{plot}\PYG{p}{(}\PYG{n}{t}\PYG{p}{,} \PYG{n}{xi}\PYG{p}{,} \PYG{n}{alpha}\PYG{o}{=}\PYG{l+m+mf}{0.15}\PYG{p}{,} \PYG{n}{color}\PYG{o}{=}\PYG{n}{plt}\PYG{o}{.}\PYG{n}{cm}\PYG{o}{.}\PYG{n}{viridis}\PYG{p}{(} \PYG{l+m+mi}{1}\PYG{o}{\PYGZhy{}}\PYG{n}{i}\PYG{o}{/}\PYG{p}{(}\PYG{n+nb}{len}\PYG{p}{(}\PYG{n}{xs\PYGZus{}ncf}\PYG{p}{)}\PYG{o}{\PYGZhy{}}\PYG{l+m+mi}{1}\PYG{p}{)} \PYG{p}{)}\PYG{p}{,} \PYG{n}{label}\PYG{o}{=}\PYG{n}{label}\PYG{p}{)}
\PYG{n}{plt}\PYG{o}{.}\PYG{n}{plot}\PYG{p}{(}\PYG{n}{t}\PYG{p}{,} \PYG{n}{x}\PYG{o}{.}\PYG{n}{detach}\PYG{p}{(}\PYG{p}{)}\PYG{o}{.}\PYG{n}{numpy}\PYG{p}{(}\PYG{p}{)}\PYG{p}{,} \PYG{l+s+s1}{\PYGZsq{}}\PYG{l+s+s1}{g.\PYGZhy{}}\PYG{l+s+s1}{\PYGZsq{}}\PYG{p}{,} \PYG{n}{label}\PYG{o}{=}\PYG{l+s+s1}{\PYGZsq{}}\PYG{l+s+s1}{Final (optimized) path}\PYG{l+s+s1}{\PYGZsq{}}\PYG{p}{)}
\PYG{n}{plt}\PYG{o}{.}\PYG{n}{plot}\PYG{p}{(}\PYG{n}{t}\PYG{p}{[}\PYG{p}{[}\PYG{l+m+mi}{0}\PYG{p}{,}\PYG{o}{\PYGZhy{}}\PYG{l+m+mi}{1}\PYG{p}{]}\PYG{p}{]}\PYG{p}{,} \PYG{n}{x0}\PYG{o}{.}\PYG{n}{data}\PYG{p}{[}\PYG{p}{[}\PYG{l+m+mi}{0}\PYG{p}{,}\PYG{o}{\PYGZhy{}}\PYG{l+m+mi}{1}\PYG{p}{]}\PYG{p}{]}\PYG{p}{,} \PYG{l+s+s1}{\PYGZsq{}}\PYG{l+s+s1}{g+}\PYG{l+s+s1}{\PYGZsq{}}\PYG{p}{,} \PYG{n}{markersize}\PYG{o}{=}\PYG{l+m+mi}{12}\PYG{p}{,} \PYG{n}{label}\PYG{o}{=}\PYG{l+s+s1}{\PYGZsq{}}\PYG{l+s+s1}{Points held constant}\PYG{l+s+s1}{\PYGZsq{}}\PYG{p}{)}

\PYG{n}{plt}\PYG{o}{.}\PYG{n}{ylim}\PYG{p}{(}\PYG{o}{\PYGZhy{}}\PYG{l+m+mi}{5}\PYG{p}{,} \PYG{l+m+mi}{75}\PYG{p}{)}
\PYG{n}{plt}\PYG{o}{.}\PYG{n}{xlabel}\PYG{p}{(}\PYG{l+s+s1}{\PYGZsq{}}\PYG{l+s+s1}{Time (s)}\PYG{l+s+s1}{\PYGZsq{}}\PYG{p}{)} \PYG{p}{;} \PYG{n}{plt}\PYG{o}{.}\PYG{n}{ylabel}\PYG{p}{(}\PYG{l+s+s1}{\PYGZsq{}}\PYG{l+s+s1}{Height (m)}\PYG{l+s+s1}{\PYGZsq{}}\PYG{p}{)} \PYG{p}{;} \PYG{n}{plt}\PYG{o}{.}\PYG{n}{legend}\PYG{p}{(}\PYG{n}{fontsize}\PYG{o}{=}\PYG{l+m+mi}{5}\PYG{p}{,} \PYG{n}{ncol}\PYG{o}{=}\PYG{l+m+mi}{3}\PYG{p}{)}
\PYG{n}{plt}\PYG{o}{.}\PYG{n}{tight\PYGZus{}layout}\PYG{p}{(}\PYG{p}{)}  \PYG{p}{;} \PYG{n}{plt}\PYG{o}{.}\PYG{n}{show}\PYG{p}{(}\PYG{p}{)}
\end{sphinxVerbatim}

\end{sphinxuseclass}\end{sphinxVerbatimInput}
\begin{sphinxVerbatimOutput}

\begin{sphinxuseclass}{cell_output}
\begin{sphinxVerbatim}[commandchars=\\\{\}]
step=0000, S=4.5384e+03 J*s
step=0400, S=\PYGZhy{}4.5372e+02 J*s
\end{sphinxVerbatim}

\begin{sphinxVerbatim}[commandchars=\\\{\}]
step=1600, S=\PYGZhy{}1.4883e+03 J*s
\end{sphinxVerbatim}

\begin{sphinxVerbatim}[commandchars=\\\{\}]
step=3600, S=\PYGZhy{}2.1659e+03 J*s
\end{sphinxVerbatim}

\begin{sphinxVerbatim}[commandchars=\\\{\}]
step=6400, S=\PYGZhy{}2.4287e+03 J*s
\end{sphinxVerbatim}

\begin{sphinxVerbatim}[commandchars=\\\{\}]
step=10000, S=\PYGZhy{}2.4901e+03 J*s
\end{sphinxVerbatim}

\begin{sphinxVerbatim}[commandchars=\\\{\}]
step=14400, S=\PYGZhy{}2.4990e+03 J*s
\end{sphinxVerbatim}

\begin{sphinxVerbatim}[commandchars=\\\{\}]
step=19881, S=\PYGZhy{}2.4998e+03 J*s
\end{sphinxVerbatim}

\noindent\sphinxincludegraphics{{8a2534c799c1d7b2a8528046cd00058545f97afcaa3cc124d098d7c32644de79}.png}

\end{sphinxuseclass}\end{sphinxVerbatimOutput}

\end{sphinxuseclass}
\begin{sphinxuseclass}{cell}\begin{sphinxVerbatimInput}

\begin{sphinxuseclass}{cell_input}
\begin{sphinxVerbatim}[commandchars=\\\{\}]
\PYG{k}{def} \PYG{n+nf}{make\PYGZus{}video}\PYG{p}{(}\PYG{n}{t}\PYG{p}{,} \PYG{n}{xs}\PYG{p}{,} \PYG{n}{path}\PYG{p}{,} \PYG{n}{interval}\PYG{o}{=}\PYG{l+m+mi}{60}\PYG{p}{,} \PYG{n}{color}\PYG{o}{=}\PYG{l+s+s1}{\PYGZsq{}}\PYG{l+s+s1}{black}\PYG{l+s+s1}{\PYGZsq{}}\PYG{p}{,} \PYG{n}{mode}\PYG{o}{=}\PYG{l+s+s1}{\PYGZsq{}}\PYG{l+s+s1}{ncf}\PYG{l+s+s1}{\PYGZsq{}}\PYG{p}{,} \PYG{o}{*}\PYG{o}{*}\PYG{n}{kwargs}\PYG{p}{)}\PYG{p}{:} \PYG{c+c1}{\PYGZsh{} xs: [time, N, 2]}
    \PYG{n}{fig} \PYG{o}{=} \PYG{n}{plt}\PYG{o}{.}\PYG{n}{gcf}\PYG{p}{(}\PYG{p}{)} \PYG{p}{;} \PYG{n}{fig}\PYG{o}{.}\PYG{n}{set\PYGZus{}dpi}\PYG{p}{(}\PYG{l+m+mi}{200}\PYG{p}{)} \PYG{p}{;} \PYG{n}{fig}\PYG{o}{.}\PYG{n}{set\PYGZus{}size\PYGZus{}inches}\PYG{p}{(}\PYG{l+m+mi}{3}\PYG{p}{,} \PYG{l+m+mi}{3}\PYG{p}{)}
    \PYG{n}{camera} \PYG{o}{=} \PYG{n}{Camera}\PYG{p}{(}\PYG{n}{fig}\PYG{p}{)}
    \PYG{k}{for} \PYG{n}{i} \PYG{o+ow}{in} \PYG{n+nb}{range}\PYG{p}{(}\PYG{n+nb}{len}\PYG{p}{(}\PYG{n}{xs}\PYG{p}{)} \PYG{k}{if} \PYG{n+nb}{type}\PYG{p}{(}\PYG{n}{xs}\PYG{p}{)} \PYG{o+ow}{is} \PYG{n+nb}{list} \PYG{k}{else} \PYG{n}{xs}\PYG{o}{.}\PYG{n}{shape}\PYG{p}{[}\PYG{l+m+mi}{0}\PYG{p}{]}\PYG{p}{)}\PYG{p}{:}
        \PYG{k}{if} \PYG{n}{mode} \PYG{o}{==} \PYG{l+s+s1}{\PYGZsq{}}\PYG{l+s+s1}{ncf}\PYG{l+s+s1}{\PYGZsq{}}\PYG{p}{:}
            \PYG{k}{for} \PYG{n}{j}\PYG{p}{,} \PYG{n}{xi} \PYG{o+ow}{in} \PYG{n+nb}{enumerate}\PYG{p}{(}\PYG{n}{xs}\PYG{p}{[}\PYG{p}{:}\PYG{n}{i}\PYG{p}{]}\PYG{p}{)}\PYG{p}{:}
                \PYG{n}{plt}\PYG{o}{.}\PYG{n}{plot}\PYG{p}{(}\PYG{n}{t}\PYG{p}{,} \PYG{n}{xi}\PYG{p}{,} \PYG{n}{alpha}\PYG{o}{=}\PYG{l+m+mf}{0.15}\PYG{p}{,} \PYG{n}{color}\PYG{o}{=}\PYG{n}{plt}\PYG{o}{.}\PYG{n}{cm}\PYG{o}{.}\PYG{n}{viridis}\PYG{p}{(} \PYG{l+m+mi}{1}\PYG{o}{\PYGZhy{}}\PYG{n}{j}\PYG{o}{/}\PYG{p}{(}\PYG{n+nb}{len}\PYG{p}{(}\PYG{n}{xs}\PYG{p}{)}\PYG{o}{\PYGZhy{}}\PYG{l+m+mi}{1}\PYG{p}{)} \PYG{p}{)}\PYG{p}{,} \PYG{n}{label}\PYG{o}{=}\PYG{n}{label}\PYG{p}{)}
            \PYG{n}{plt}\PYG{o}{.}\PYG{n}{plot}\PYG{p}{(}\PYG{n}{t}\PYG{p}{,} \PYG{n}{xs}\PYG{p}{[}\PYG{l+m+mi}{0}\PYG{p}{]}\PYG{p}{,} \PYG{l+s+s1}{\PYGZsq{}}\PYG{l+s+s1}{.\PYGZhy{}}\PYG{l+s+s1}{\PYGZsq{}}\PYG{p}{,} \PYG{n}{color}\PYG{o}{=}\PYG{n}{plt}\PYG{o}{.}\PYG{n}{cm}\PYG{o}{.}\PYG{n}{viridis}\PYG{p}{(}\PYG{l+m+mf}{0.9}\PYG{p}{)}\PYG{p}{)}
        \PYG{n}{plt}\PYG{o}{.}\PYG{n}{plot}\PYG{p}{(}\PYG{n}{t}\PYG{p}{[}\PYG{p}{[}\PYG{l+m+mi}{0}\PYG{p}{,}\PYG{o}{\PYGZhy{}}\PYG{l+m+mi}{1}\PYG{p}{]}\PYG{p}{]}\PYG{p}{,} \PYG{n}{xs}\PYG{p}{[}\PYG{o}{\PYGZhy{}}\PYG{l+m+mi}{1}\PYG{p}{]}\PYG{p}{[}\PYG{p}{[}\PYG{l+m+mi}{0}\PYG{p}{,}\PYG{o}{\PYGZhy{}}\PYG{l+m+mi}{1}\PYG{p}{]}\PYG{p}{]}\PYG{p}{,} \PYG{l+s+s1}{\PYGZsq{}}\PYG{l+s+s1}{+}\PYG{l+s+s1}{\PYGZsq{}}\PYG{p}{,} \PYG{n}{color}\PYG{o}{=}\PYG{n}{color}\PYG{p}{,} \PYG{n}{markersize}\PYG{o}{=}\PYG{l+m+mi}{16}\PYG{p}{)}
        \PYG{n}{plt}\PYG{o}{.}\PYG{n}{plot}\PYG{p}{(}\PYG{n}{t}\PYG{p}{[}\PYG{p}{:}\PYG{n+nb}{len}\PYG{p}{(}\PYG{n}{xs}\PYG{p}{[}\PYG{n}{i}\PYG{p}{]}\PYG{p}{)}\PYG{p}{]}\PYG{p}{,} \PYG{n}{xs}\PYG{p}{[}\PYG{n}{i}\PYG{p}{]}\PYG{p}{,} \PYG{l+s+s1}{\PYGZsq{}}\PYG{l+s+s1}{.\PYGZhy{}}\PYG{l+s+s1}{\PYGZsq{}}\PYG{p}{,} \PYG{n}{color}\PYG{o}{=}\PYG{n}{color}\PYG{p}{)}
        \PYG{n}{plt}\PYG{o}{.}\PYG{n}{xlim}\PYG{p}{(}\PYG{n}{np}\PYG{o}{.}\PYG{n}{min}\PYG{p}{(}\PYG{n}{t}\PYG{p}{)}\PYG{o}{\PYGZhy{}}\PYG{l+m+mi}{1}\PYG{p}{,} \PYG{n}{np}\PYG{o}{.}\PYG{n}{max}\PYG{p}{(}\PYG{n}{t}\PYG{p}{)}\PYG{o}{+}\PYG{l+m+mi}{1}\PYG{p}{)} \PYG{p}{;} \PYG{n}{plt}\PYG{o}{.}\PYG{n}{ylim}\PYG{p}{(}\PYG{o}{\PYGZhy{}}\PYG{l+m+mi}{5}\PYG{p}{,} \PYG{l+m+mi}{75}\PYG{p}{)}
        \PYG{n}{plt}\PYG{o}{.}\PYG{n}{xticks}\PYG{p}{(}\PYG{p}{[}\PYG{p}{]}\PYG{p}{,} \PYG{p}{[}\PYG{p}{]}\PYG{p}{)}\PYG{p}{;} \PYG{n}{plt}\PYG{o}{.}\PYG{n}{yticks}\PYG{p}{(}\PYG{p}{[}\PYG{p}{]}\PYG{p}{,} \PYG{p}{[}\PYG{p}{]}\PYG{p}{)} \PYG{p}{;} \PYG{n}{plt}\PYG{o}{.}\PYG{n}{xlabel}\PYG{p}{(}\PYG{l+s+s1}{\PYGZsq{}}\PYG{l+s+s1}{Time (s)}\PYG{l+s+s1}{\PYGZsq{}}\PYG{p}{)} \PYG{p}{;} \PYG{n}{plt}\PYG{o}{.}\PYG{n}{ylabel}\PYG{p}{(}\PYG{l+s+s1}{\PYGZsq{}}\PYG{l+s+s1}{Height (m)}\PYG{l+s+s1}{\PYGZsq{}}\PYG{p}{)}
        \PYG{n}{camera}\PYG{o}{.}\PYG{n}{snap}\PYG{p}{(}\PYG{p}{)}
    \PYG{n}{anim} \PYG{o}{=} \PYG{n}{camera}\PYG{o}{.}\PYG{n}{animate}\PYG{p}{(}\PYG{n}{blit}\PYG{o}{=}\PYG{k+kc}{True}\PYG{p}{,} \PYG{n}{interval}\PYG{o}{=}\PYG{n}{interval}\PYG{p}{,} \PYG{o}{*}\PYG{o}{*}\PYG{n}{kwargs}\PYG{p}{)}
    \PYG{n}{anim}\PYG{o}{.}\PYG{n}{save}\PYG{p}{(}\PYG{n}{path}\PYG{p}{)} \PYG{p}{;} \PYG{n}{plt}\PYG{o}{.}\PYG{n}{close}\PYG{p}{(}\PYG{p}{)}
\end{sphinxVerbatim}

\end{sphinxuseclass}\end{sphinxVerbatimInput}

\end{sphinxuseclass}
\begin{sphinxuseclass}{cell}\begin{sphinxVerbatimInput}

\begin{sphinxuseclass}{cell_input}
\begin{sphinxVerbatim}[commandchars=\\\{\}]
\PYG{n}{xs\PYGZus{}ncf\PYGZus{}} \PYG{o}{=} \PYG{n}{np}\PYG{o}{.}\PYG{n}{concatenate}\PYG{p}{(}\PYG{p}{[}\PYG{n}{xs\PYGZus{}ncf}\PYG{p}{[}\PYG{p}{:}\PYG{l+m+mi}{1}\PYG{p}{]}\PYG{p}{]}\PYG{o}{*}\PYG{l+m+mi}{20}\PYG{o}{+}\PYG{p}{[}\PYG{n}{xs\PYGZus{}ncf}\PYG{p}{]}\PYG{o}{+}\PYG{p}{[}\PYG{n}{xs\PYGZus{}ncf}\PYG{p}{[}\PYG{o}{\PYGZhy{}}\PYG{l+m+mi}{1}\PYG{p}{:}\PYG{p}{]}\PYG{p}{]}\PYG{o}{*}\PYG{l+m+mi}{20} \PYG{p}{)}
\PYG{n}{make\PYGZus{}video}\PYG{p}{(}\PYG{n}{t}\PYG{p}{,} \PYG{n}{xs\PYGZus{}ncf\PYGZus{}}\PYG{p}{,} \PYG{n}{path}\PYG{o}{=}\PYG{l+s+s1}{\PYGZsq{}}\PYG{l+s+s1}{static/tutorial\PYGZus{}ncf.mp4}\PYG{l+s+s1}{\PYGZsq{}}\PYG{p}{,} \PYG{n}{color}\PYG{o}{=}\PYG{l+s+s1}{\PYGZsq{}}\PYG{l+s+s1}{green}\PYG{l+s+s1}{\PYGZsq{}}\PYG{p}{,} \PYG{n}{interval}\PYG{o}{=}\PYG{l+m+mi}{80}\PYG{p}{)}

\PYG{n}{mp4} \PYG{o}{=} \PYG{n+nb}{open}\PYG{p}{(}\PYG{l+s+s1}{\PYGZsq{}}\PYG{l+s+s1}{static/tutorial\PYGZus{}ncf.mp4}\PYG{l+s+s1}{\PYGZsq{}}\PYG{p}{,}\PYG{l+s+s1}{\PYGZsq{}}\PYG{l+s+s1}{rb}\PYG{l+s+s1}{\PYGZsq{}}\PYG{p}{)}\PYG{o}{.}\PYG{n}{read}\PYG{p}{(}\PYG{p}{)}
\PYG{n}{data\PYGZus{}url} \PYG{o}{=} \PYG{l+s+s2}{\PYGZdq{}}\PYG{l+s+s2}{data:video/mp4;base64,}\PYG{l+s+s2}{\PYGZdq{}} \PYG{o}{+} \PYG{n}{b64encode}\PYG{p}{(}\PYG{n}{mp4}\PYG{p}{)}\PYG{o}{.}\PYG{n}{decode}\PYG{p}{(}\PYG{p}{)}
\PYG{n}{HTML}\PYG{p}{(}\PYG{l+s+s1}{\PYGZsq{}}\PYG{l+s+s1}{\PYGZlt{}video width=300 controls\PYGZgt{}\PYGZlt{}source src=}\PYG{l+s+s1}{\PYGZdq{}}\PYG{l+s+si}{\PYGZob{}\PYGZcb{}}\PYG{l+s+s1}{\PYGZdq{}}\PYG{l+s+s1}{ type=}\PYG{l+s+s1}{\PYGZdq{}}\PYG{l+s+s1}{video/mp4}\PYG{l+s+s1}{\PYGZdq{}}\PYG{l+s+s1}{\PYGZgt{}\PYGZlt{}/video\PYGZgt{}}\PYG{l+s+s1}{\PYGZsq{}}\PYG{o}{.}\PYG{n}{format}\PYG{p}{(}\PYG{n}{data\PYGZus{}url}\PYG{p}{)}\PYG{p}{)}
\end{sphinxVerbatim}

\end{sphinxuseclass}\end{sphinxVerbatimInput}
\begin{sphinxVerbatimOutput}

\begin{sphinxuseclass}{cell_output}
\begin{sphinxVerbatim}[commandchars=\\\{\}]
\PYGZlt{}IPython.core.display.HTML object\PYGZgt{}
\end{sphinxVerbatim}

\end{sphinxuseclass}\end{sphinxVerbatimOutput}

\end{sphinxuseclass}
\sphinxstepscope


\chapter{Mechanické vlastnosti materiálu}
\label{\detokenize{Prednasky/2_1_Mechanick_xe9_vlastnosti_materi_xe1lu:mechanicke-vlastnosti-materialu}}\label{\detokenize{Prednasky/2_1_Mechanick_xe9_vlastnosti_materi_xe1lu::doc}}

\section{Přehled mechanických vlastností materiálu}
\label{\detokenize{Prednasky/2_1_Mechanick_xe9_vlastnosti_materi_xe1lu:prehled-mechanickych-vlastnosti-materialu}}
\sphinxAtStartPar
Mechanické vlastnosti charakterizují chování materiálů při působení vnějších sil. Mají rozhodující význam pro výpočet strojních součástí. Patří sem pevnost, houževnatost, tvrdost a pružnost.
\begin{enumerate}
\sphinxsetlistlabels{\arabic}{enumi}{enumii}{}{.}%
\item {} 
\sphinxAtStartPar
\sphinxstylestrong{Pevnost} je schopnost materiálu odolávat mechanickým silám, rozlišujeme pevnost v tahu, tlaku, krutu, smyku/střihu a ohybu. Je definována jako největší napětí, které je třeba k porušení materiálu. Za porušení materiálu se považuje nejen rozdělení materiálu na dvě části, ale i vznik trhlin. Pevnost zkoušíme pomocí tzv. statické tahové zkoušky, při které zatěžujeme vzorek pozvolna se zvětšující silou a zkoumáme, při jaké velikosti síly se zkušební vzorek poruší.

\item {} 
\sphinxAtStartPar
\sphinxstylestrong{Pružnost} je schopnost materiálu deformovat se působením vnějších sil a po odstranění těchto sil se vrátit do původního stavu. Pružným materiálem je např. guma (pryž).

\item {} 
\sphinxAtStartPar
\sphinxstylestrong{Pružnost} je schopnost materiálu odolávat rázům. Je definována jako velikost práce, která je nutná k rozdělení materiálu na dvě části. Opakem houževnatosti je křehkost. Křehký materiál je např. sklo, houževnatý materiál je např. ocel. Houževnatost se zkoumá tím, že vzorek vystavíme rázu. Při přeražení zkušební tyče se spotřebuje určité množství práce, velikost této práce dokážeme spočítat.

\item {} 
\sphinxAtStartPar
\sphinxstylestrong{Tvrdost} je odpor materiálu proti vnikání cizího tělesa. Zjišťuje se vtlačováním tvrdého tělíska do zkoušeného materiálu a měřením hloubky popř. šířky vtisku. Tvrdost zvyšujeme u kovů kalením – to je ohřátím na určitou teplotu a prudkým ochlazením. Takto se např. zpracovávají kalené kuličky do kuličkových ložisek.

\end{enumerate}


\section{Přehled technologických vlastností materiálu}
\label{\detokenize{Prednasky/2_1_Mechanick_xe9_vlastnosti_materi_xe1lu:prehled-technologickych-vlastnosti-materialu}}\begin{enumerate}
\sphinxsetlistlabels{\arabic}{enumi}{enumii}{}{.}%
\item {} 
\sphinxAtStartPar
\sphinxstylestrong{Tvárnost}  je vhodnost materiálu ke zpracování pomocí tváření. V tuhém stavu materiál změní působením vnějších sil svůj tvar, ale přitom se neporuší soudržnost \sphinxhyphen{} tj. v materiálu nevzniknou trhlinky. Tvar vzniklý působením vnějších sil zůstaně zachovaný i po odstranění působících sil. Tato vlastnost je typická pro kovy a je důležitá pro kování, válcování, lisování.

\item {} 
\sphinxAtStartPar
\sphinxstylestrong{Obrobitelnost} je způsobilost materiálu pro třískové obrábění (soustružení, frézování apod.). Posuzuje se podle mechanických vlastností, podle řezného odporu a podle způsobu tvorby třísky.

\item {} 
\sphinxAtStartPar
\sphinxstylestrong{Slévatelnost} je soubor vlastností nutných pro vytvoření kvalitního odlitku. Ovlivňují ji teplota tání a tuhnutí, tepelná vodivost, viskozita, tepelná roztažnost a smrštivost.

\item {} 
\sphinxAtStartPar
\sphinxstylestrong{Svařitelnost} je schopnost materiálu vytvořit pomocí svařování ze dvou částí jeden nerozebíratelný celek.

\item {} 
\sphinxAtStartPar
\sphinxstylestrong{Odolnost proti opotřebení} proti opotřebení je odolnost proti působení vnějších sil, které způsobují postupné ubývání materiálu.

\end{enumerate}


\section{Určování mechanických a technologických vlastností}
\label{\detokenize{Prednasky/2_1_Mechanick_xe9_vlastnosti_materi_xe1lu:urcovani-mechanickych-a-technologickych-vlastnosti}}
\sphinxAtStartPar
\sphinxstylestrong{Normy} pro testování materiálů, včetně norem ISO, poskytují pokyny a specifikace pro hodnocení mechanických vlastností materiálů. Tyto normy jsou klíčové pro zajištění konzistence a spolehlivosti při používání materiálů a testovacích metodik napříč různými průmyslovými odvětvími. Testy a postupy vhodné pro měření různých vlastností materiálů, jako je pevnost, odolnost a další charakteristiky, jsou součástí těchto norem. Tyto testy se provádějí s využitím univerzálních testovacích strojů.

\noindent\sphinxincludegraphics{{barrus_astm_iso-768x331}.jpg}

\sphinxAtStartPar
Tyto normy jsou vyvíjeny a publikovány známými organizacemi, jako jsou \sphinxstylestrong{ASTM} (American Society for Testing and Materials), \sphinxstylestrong{ISO} (International Organization for Standardization) a dalšími subjekty. Existují také regionální a národní doporučení, jako například:
\begin{itemize}
\item {} 
\sphinxAtStartPar
EN – Euronorm – evropské normy

\item {} 
\sphinxAtStartPar
DIN – Deutsches Institut für Normung, německá národní normalizační organizace

\item {} 
\sphinxAtStartPar
ČSN \sphinxhyphen{} české technické normy

\end{itemize}

\sphinxAtStartPar
Vzhledem k tomu, že jsou tato doporučení harmonizována, jsou dnes obecně totožná. Tyto organizace zajišťují univerzální použitelnost a přijetí těchto norem. Normy pro testování materiálů definují testovací postupy, zařízení, přípravu vzorků, rozměry vzorků, vlastnosti a metody vyhodnocování výsledků. Obecně přijaté testovací metody usnadňují srovnatelnost výsledků.

\sphinxAtStartPar
Srovnatelnost je klíčová v těměř každém průmyslovém odvětví, včetně letectví, automobilového průmyslu, stavebnictví a zdravotnických zařízení, kde selhání materiálu může mít závažné následky. Během vývoje nových materiálů nebo produktů poskytují normy pro testování materiálů referenční hodnoty pro hodnocení nových materiálů oproti stávajícím.

\sphinxAtStartPar
Zkoušky materálu můžeme rozdělit do dvou skupin: destruktivní a nedestruktivní. Při použití destruktivní metody zkoušení se zkušební vzorek poruší. Při použití nedestruktivní metody zkoušení zkušbní vzorek zůstává v původním tvaru.


\begin{savenotes}\sphinxattablestart
\sphinxthistablewithglobalstyle
\centering
\begin{tabulary}{\linewidth}[t]{TTTT}
\sphinxtoprule
\sphinxstyletheadfamily 
\sphinxAtStartPar
Norma ISO
&\sphinxstyletheadfamily 
\sphinxAtStartPar
Norma ASTM
&\sphinxstyletheadfamily 
\sphinxAtStartPar
Název normy
&\sphinxstyletheadfamily 
\sphinxAtStartPar
Popis zkoušek
\\
\sphinxmidrule
\sphinxtableatstartofbodyhook
\sphinxAtStartPar
ISO 6892\sphinxhyphen{}1
&
\sphinxAtStartPar
ASTM E8 / E8M
&
\sphinxAtStartPar
Tahové testování při pokojové teplotě
&
\sphinxAtStartPar
Stanovuje metody testování pevnosti v tahu, meze kluzu, deformace a zmenšení průřezu kovů. Definuje postup tahových testů na standardizovaných vzorcích kovů při pokojové teplotě a umožňuje stanovení důležitých mechanických vlastností.
\\
\sphinxhline
\sphinxAtStartPar
ISO 6892\sphinxhyphen{}2
&
\sphinxAtStartPar
ASTM E21
&
\sphinxAtStartPar
Tahové testování při vysokých teplotách
&
\sphinxAtStartPar
Definuje tahové testy na kovech při vysokých teplotách.
\\
\sphinxhline
\sphinxAtStartPar
ISO 148\sphinxhyphen{}1
&
\sphinxAtStartPar
ASTM E23
&
\sphinxAtStartPar
Rázová zkouška Charpy
&
\sphinxAtStartPar
Popisuje metodu Charpyho rázové zkoušky kovů. Test měří pevnost a odolnost materiálu při nárazu.
\\
\sphinxhline
\sphinxAtStartPar
ISO 12135
&
\sphinxAtStartPar
ASTM E399
&
\sphinxAtStartPar
Test odolnosti proti iniciaci a růstu trhlin
&
\sphinxAtStartPar
Stanovuje metodu určování vlastností iniciace a růstu trhlin v kovech.
\\
\sphinxhline
\sphinxAtStartPar
ISO 6508
&
\sphinxAtStartPar
ASTM E18
&
\sphinxAtStartPar
Zkouška tvrdosti Rockwell
&
\sphinxAtStartPar
Popisuje zkoušku tvrdosti kovových materiálů metodou Rockwell.
\\
\sphinxhline
\sphinxAtStartPar
ISO 6507
&
\sphinxAtStartPar
ASTM E384
&
\sphinxAtStartPar
Zkouška tvrdosti Vickers
&
\sphinxAtStartPar
Popisuje zkoušku tvrdosti kovových materiálů metodou Vickers.
\\
\sphinxhline
\sphinxAtStartPar
\sphinxhyphen{}
&
\sphinxAtStartPar
ASTM E92
&
\sphinxAtStartPar
Zkouška tvrdosti Vickers \& Knoop
&
\sphinxAtStartPar
Popisuje zkoušku tvrdosti kovových materiálů metodami Vickers a Knoop.
\\
\sphinxhline
\sphinxAtStartPar
ISO 6506
&
\sphinxAtStartPar
ASTM E10
&
\sphinxAtStartPar
Zkouška tvrdosti Brinell
&
\sphinxAtStartPar
Popisuje zkoušku tvrdosti kovových materiálů metodou Brinell.
\\
\sphinxbottomrule
\end{tabulary}
\sphinxtableafterendhook\par
\sphinxattableend\end{savenotes}

\sphinxstepscope


\section{Hookeův zákon}
\label{\detokenize{Prednasky/2_2_Hooke_u016fv_z_xe1kon:hookeuv-zakon}}\label{\detokenize{Prednasky/2_2_Hooke_u016fv_z_xe1kon::doc}}
\begin{sphinxuseclass}{cell}\begin{sphinxVerbatimInput}

\begin{sphinxuseclass}{cell_input}
\begin{sphinxVerbatim}[commandchars=\\\{\}]
\PYG{k+kn}{from} \PYG{n+nn}{IPython}\PYG{n+nn}{.}\PYG{n+nn}{display} \PYG{k+kn}{import} \PYG{n}{IFrame}
\PYG{k+kn}{from} \PYG{n+nn}{IPython}\PYG{n+nn}{.}\PYG{n+nn}{display} \PYG{k+kn}{import} \PYG{n}{YouTubeVideo}
\end{sphinxVerbatim}

\end{sphinxuseclass}\end{sphinxVerbatimInput}

\end{sphinxuseclass}

\subsection{Přechod od pohybových sil ke změnám tvaru}
\label{\detokenize{Prednasky/2_2_Hooke_u016fv_z_xe1kon:prechod-od-pohybovych-sil-ke-zmenam-tvaru}}
\sphinxAtStartPar
Nyní se přesuneme od úvah o silách, které ovlivňují pohyb objektu (například tření a odpor vzduchu), k těm, které ovlivňují tvar objektu. V případě působení velkých sil dojde ke změně tvaru objektu. Například když silně zatlačíme na hřebík, nepohne se z místa ale působící síla změní jeho tvar. Změna tvaru v důsledku působení síly se nazývá \sphinxstylestrong{deformace}.

\noindent\sphinxincludegraphics{{bracedbending2-scaled}.jpg}

\sphinxAtStartPar
Zároveň k deformacím dochází také při působení velmi malých sil. Tyto deformace jsou mnohem menší a platí pro ně dvě důležité charakteristiky:
\begin{enumerate}
\sphinxsetlistlabels{\arabic}{enumi}{enumii}{}{.}%
\item {} 
\sphinxAtStartPar
\sphinxstylestrong{Elasticita} – Objekt se po odstranění síly vrátí do svého původního tvaru, tedy deformace je elastická.

\item {} 
\sphinxAtStartPar
\sphinxstylestrong{Proporcionalita} – Velikost deformace je úměrná působící síle, což znamená, že pro malé deformace platí \sphinxstylestrong{Hookeův zákon}.

\end{enumerate}

\begin{sphinxadmonition}{note}{Note:}
\sphinxAtStartPar
Hookeův zákon
Hookeův zákon je pojmenován po britském fyzikovi Robertu Hookovi ze 17. století a poprvé byl formulován v roce 1676 jako latinská anagramová hádanka \sphinxstyleemphasis{“ceiiinosssttuv”}. Její řešení Hooke zveřejnil v roce 1678 ve formě věty \sphinxstyleemphasis{Ut tensio, sic vis}, což znamená “Jaké prodloužení, taková síla.”
\end{sphinxadmonition}

\sphinxAtStartPar
Matematicky lze Hookeův zákon vyjádřit jako:
\begin{equation*}
\begin{split}
F = k \Delta l
\end{split}
\end{equation*}
\sphinxAtStartPar
kde:
\begin{itemize}
\item {} 
\sphinxAtStartPar
\(\Delta l\) je velikost deformace (například změna délky),

\item {} 
\sphinxAtStartPar
\(F\) je působící síla,

\item {} 
\sphinxAtStartPar
\(k\) je konstanta úměrnosti závislá na tvaru a materiálu objektu. Tuto konstantu úměrnosti označujeme jako \sphinxstylestrong{tuhost} (\sphinxstyleemphasis{stiffness}). Tuhost  je fyzikální veličina charakterizující odpor tělesa nebo materiálu vůči deformaci při působení síly. Vyjadřuje, jak velká síla je potřebná k vyvolání určité deformace. Jednotkou tuhosti je N/m.

\end{itemize}

\begin{sphinxadmonition}{caution}{Caution:}
\sphinxAtStartPar
Elastická síla není konstatní
Je důležité si uvědomit, že tato síla působí proti působící vnější síle podobně jako síla tření a závisí na deformaci \(\Delta l\) – není konstantní jako síla tření.
\end{sphinxadmonition}

\sphinxAtStartPar
Přepíšeme\sphinxhyphen{}li rovnici do tvaru:
\begin{equation*}
\begin{split}
\Delta l = \frac{F}{k}
\end{split}
\end{equation*}
\sphinxAtStartPar
je jasné, že deformace je přímo úměrná aplikované síle (\sphinxstyleemphasis{Ut tensio, sic vis}). Převrácená hodnota tuhosti se označuje jako \sphinxstylestrong{poddajnost} (\sphinxstyleemphasis{compliance}). Poddajnost je fyzikální veličina vyjadřující schopnost tělesa nebo materiálu podléhat deformaci při působení síly.
\begin{equation*}
\begin{split}
C = \frac{1}{k}
\end{split}
\end{equation*}
\sphinxAtStartPar
kde:
\begin{itemize}
\item {} 
\sphinxAtStartPar
\(C\) je poddajnost {[}m/N{]},

\item {} 
\sphinxAtStartPar
\(k\) je tuhost {[}N/m{]}).

\end{itemize}

\sphinxAtStartPar
Vztah mezi deformací \(\Delta l\) a aplikovanou silou  \(F\) lze vyjádřit jako:
\begin{equation*}
\begin{split}
\Delta l = C F
\end{split}
\end{equation*}
\sphinxAtStartPar
Tento vztah ale neplatí pro libovolné síly.  Na obrázku je graf deformace \(\Delta l\) versus aplikovaná síla \(F\). Přímý úsek představuje lineární oblast, kde platí Hookeův zákon. Sklon tohoto přímého úseku je \(1/k\). Při větších silách se graf zakřivuje, ale deformace je stále elastická—po odstranění síly se \(\Delta l\) vrátí na nulu. Při ještě větších silách dochází k trvalé deformaci objektu, která nakonec vede k jeho zlomu. Všimněte si, že v tomto grafu se těsně před zlomením sklon zvětšuje, což znamená, že malý nárůst síly \(F\) způsobuje výrazné prodloužení \(\Delta l\) v blízkosti zlomu.

\noindent\sphinxincludegraphics{{d8d280d00eaed9994fabb303bfcc57e319e05c1b}.jpg}

\begin{sphinxuseclass}{cell}\begin{sphinxVerbatimInput}

\begin{sphinxuseclass}{cell_input}
\begin{sphinxVerbatim}[commandchars=\\\{\}]
\PYG{n}{IFrame}\PYG{p}{(}\PYG{l+s+s2}{\PYGZdq{}}\PYG{l+s+s2}{https://www.geogebra.org/classic/tb98aHHz?embed}\PYG{l+s+s2}{\PYGZdq{}}\PYG{p}{,} \PYG{n}{width}\PYG{o}{=}\PYG{l+s+s2}{\PYGZdq{}}\PYG{l+s+s2}{800}\PYG{l+s+s2}{\PYGZdq{}}\PYG{p}{,} \PYG{n}{height}\PYG{o}{=}\PYG{l+s+s2}{\PYGZdq{}}\PYG{l+s+s2}{600}\PYG{l+s+s2}{\PYGZdq{}}\PYG{p}{)}
\end{sphinxVerbatim}

\end{sphinxuseclass}\end{sphinxVerbatimInput}
\begin{sphinxVerbatimOutput}

\begin{sphinxuseclass}{cell_output}
\begin{sphinxVerbatim}[commandchars=\\\{\}]
\PYGZlt{}IPython.lib.display.IFrame at 0x7bd041110c40\PYGZgt{}
\end{sphinxVerbatim}

\end{sphinxuseclass}\end{sphinxVerbatimOutput}

\end{sphinxuseclass}

\subsection{Na čem záleží tuhost?}
\label{\detokenize{Prednasky/2_2_Hooke_u016fv_z_xe1kon:na-cem-zalezi-tuhost}}
\begin{sphinxadmonition}{caution}{Caution:}
\sphinxAtStartPar
Tuhost je vlastnost tělesa
Tuhost závisí nejen na \sphinxstylestrong{materiálových vlastnostech}, ale také na \sphinxstylestrong{geometrii objektu} (např. průřezu nebo délce). Proto mají dva objekty ze stejného materiálu různou tuhost, pokud mají odlišné rozměry nebo tvar.
\end{sphinxadmonition}

\sphinxAtStartPar
Z Hookeova zákona plyne, že prodloužení \(\Delta l\) je funkcí působící síly. Na čem ale závisí samotná tuhost objektu. Přčdepokládejme, že na gitarové struny působíme stejnou sílou. Stejná síla, v tomto případě tíha \(W\), vede k různé výslední deformaci. Výsledné deformace jsou na obrázku níže znázorněny jako šedé segmenty.

\noindent\sphinxincludegraphics{{Figure_06_03_02a-1}.jpg}
\begin{itemize}
\item {} 
\sphinxAtStartPar
Struna vlevo je \sphinxstylestrong{tenká nylonová} \sphinxhyphen{} pozorujeme velkou deformaci.

\item {} 
\sphinxAtStartPar
Struna uprostřed je \sphinxstylestrong{tlustší nylonová} \sphinxhyphen{} pozorujeme menší deformaci.

\item {} 
\sphinxAtStartPar
Struna vpravo je \sphinxstylestrong{ocelová} \sphinxhyphen{} deformaci téměř nepozorujeme.

\end{itemize}

\sphinxAtStartPar
Každá struna se deformuje jinak, což ukazuje, že tuhost závisí na:
\begin{itemize}
\item {} 
\sphinxAtStartPar
\sphinxstylestrong{Materiálu} – Ocelová struna má vyšší modul pružnosti než nylonová, a proto se méně deformuje.

\item {} 
\sphinxAtStartPar
\sphinxstylestrong{Průměru (ploše průřezu)} – Tlustší nylonová struna je tužší než tenčí, protože větší průřez zvyšuje odpor vůči deformaci.

\item {} 
\sphinxAtStartPar
\sphinxstylestrong{Délce} – I když v tomto případě jsou všechny struny stejně dlouhé, obecně platí, že delší objekty mají menší tuhost (při stejné síle se více deformují).

\end{itemize}

\sphinxAtStartPar
V případě, že by jsme zkoušeli struny různé délky, zjistili by jsme, že celkové prodloužení závisí na velikosti původní délky:
\begin{equation*}
\begin{split}\Delta l \propto l\end{split}
\end{equation*}
\sphinxAtStartPar
Dále by jsme mohli zjistit, že prodloužení je menší u materiálů s větším průřezem
\begin{equation*}
\begin{split}\Delta l \propto \frac{1}{A}\end{split}
\end{equation*}
\sphinxAtStartPar
Když tyto závislosti vložíme do Hookeova zákona zjistíme
\begin{equation*}
\begin{split}\Delta l = E \frac{l}{A} \end{split}
\end{equation*}
\sphinxAtStartPar
nebo můžeme přepsat do tvaru
\begin{equation*}
\begin{split}\sigma = E \varepsilon \end{split}
\end{equation*}
\sphinxAtStartPar
kde:
\begin{itemize}
\item {} 
\sphinxAtStartPar
\(\sigma\) je napětí, {[} N/m\(^2\){]}

\item {} 
\sphinxAtStartPar
\(\varepsilon\) je poměrné prodloužení, {[}1{]}

\item {} 
\sphinxAtStartPar
\(E\) je Youngův modul pružnosti, {[}Pa{]}

\end{itemize}

\sphinxAtStartPar
Hookův zákon v tomto tvaru bývá také označován jako \sphinxstylestrong{elementární Hookův zákon} a můžeme ho formulovat:
\begin{quote}

\sphinxAtStartPar
Normálové napětí je přímo úměrné relativnímu prodloužení.
\begin{equation*}
\begin{split} \sigma = E \varepsilon \end{split}
\end{equation*}\end{quote}


\subsection{Napětí (\sphinxstyleemphasis{stress})}
\label{\detokenize{Prednasky/2_2_Hooke_u016fv_z_xe1kon:napeti-stress}}\begin{quote}

\sphinxAtStartPar
\sphinxstylestrong{Napětí} vyjadřuje vnitřní sílu působící v materiálu v důsledku vnějšího zatížení. Udává, jak velká síla působí na jednotkovou plochu průřezu.
\end{quote}

\sphinxAtStartPar
Matematicky je definováno vztahem:
\begin{equation*}
\begin{split} \sigma = \frac{F}{A}\end{split}
\end{equation*}
\sphinxAtStartPar
kde:
\begin{itemize}
\item {} 
\sphinxAtStartPar
\(\sigma\) je normálové napětí {[}Pa = N/m\textasciicircum{}2{]},

\item {} 
\sphinxAtStartPar
\(F\) je působící síla {[}N{]},

\item {} 
\sphinxAtStartPar
\(A\) je plocha průřezu {[}m\(^2\){]}.

\end{itemize}

\sphinxAtStartPar
Napětí může být:
\begin{itemize}
\item {} 
\sphinxAtStartPar
\sphinxstylestrong{Tahové} (\(\sigma > 0\)) – materiál se prodlužuje.

\item {} 
\sphinxAtStartPar
\sphinxstylestrong{Tlakové} (\(\sigma < 0\)) – materiál se zkracuje.

\end{itemize}

\noindent\sphinxincludegraphics[width=500\sphinxpxdimen]{{normalstress}.jpg}

\sphinxAtStartPar
V rámci tohoto kurzu budeme předpokládat, že všechny materiály jsou \sphinxstylestrong{homogenní, izotropní a elastické}. Také budeme uvažovat, že objekt je \sphinxstylestrong{prizmatický} – to znamená, že průřez je po celé délce stejný (například banán je přibližně prizmatický, zatímco jablko není). Tyto předpoklady nám umožňují říci, že objekt se bude \sphinxstylestrong{deformovat rovnoměrně} v každém bodě svého průřezu a napětí je konstatní.

\sphinxAtStartPar
V případě, že napětí v průřezu není konstatní přenáší malá plocha průřezu \(\Delta A\) malou sílu \(\Delta F\) a můžeme definovat napětí v bodě, když dané plochy a síly budeme zmenšovat:
\begin{equation*}
\begin{split} 
p = \lim\limits_{\Delta A \rightarrow 0} \frac{\Delta F}{\Delta A} = \frac{\mathrm{d}F}{\mathrm{d}A}
\end{split}
\end{equation*}
\sphinxAtStartPar
a součet všech těchto sil musí být roven \sphinxstylestrong{výsledné vnitřní síle} \(F\).
\begin{equation*}
\begin{split} 
F = \int\limits_A p \mathrm{d}A
\end{split}
\end{equation*}
\sphinxAtStartPar
Tento vztah vyjadřuje \sphinxstylestrong{průměrné normálové napětí}, protože jsme vnitřní síly zprůměrovali po celém průřezu.

\sphinxAtStartPar
Napětí je často obtížné pochopit, protože není přímo viditelné. Existuje však metoda, která umožňuje \sphinxstylestrong{vizualizaci napětí} – pozorováním průhledného z určitých materiálů objektu v \sphinxstylestrong{polarizovaném světle}. Tento jev je známý jako \sphinxstylestrong{fotoelasticita}.

\begin{sphinxuseclass}{cell}\begin{sphinxVerbatimInput}

\begin{sphinxuseclass}{cell_input}
\begin{sphinxVerbatim}[commandchars=\\\{\}]
\PYG{n}{YouTubeVideo}\PYG{p}{(}\PYG{l+s+s1}{\PYGZsq{}}\PYG{l+s+s1}{MhcsqX9u1pU}\PYG{l+s+s1}{\PYGZsq{}}\PYG{p}{)}
\end{sphinxVerbatim}

\end{sphinxuseclass}\end{sphinxVerbatimInput}
\begin{sphinxVerbatimOutput}

\begin{sphinxuseclass}{cell_output}
\noindent\sphinxincludegraphics{{6864866c61fd07f066590ed0c01e86cd6e23bd467af955509347ff7142092726}.jpg}

\end{sphinxuseclass}\end{sphinxVerbatimOutput}

\end{sphinxuseclass}
\sphinxAtStartPar
Napětí může v materiálu existovat i bez působícího zatížení. Tomu se říká \sphinxstylestrong{zbytkové napětí}, které může být výhodné pro zvýšení pevnosti materiálů – například při výrobě japonských katan. Naopak, nežádoucí zbytková napětí mohou podporovat růst trhlin a vést k \sphinxstylestrong{porušení materiálu}. Příkladem je zhroucení mostu Silver Bridge v Západní Virginii v roce 1967.

\sphinxAtStartPar
Jedním z nejzajímavějších příkladů zbytkového napětí je fenomén rychlého ochlazování roztaveného skla, známý jako Prince Rupert’s Drop.

\begin{sphinxuseclass}{cell}\begin{sphinxVerbatimInput}

\begin{sphinxuseclass}{cell_input}
\begin{sphinxVerbatim}[commandchars=\\\{\}]
\PYG{n}{YouTubeVideo}\PYG{p}{(}\PYG{l+s+s1}{\PYGZsq{}}\PYG{l+s+s1}{xe\PYGZhy{}f4gokRBs}\PYG{l+s+s1}{\PYGZsq{}}\PYG{p}{)}
\end{sphinxVerbatim}

\end{sphinxuseclass}\end{sphinxVerbatimInput}
\begin{sphinxVerbatimOutput}

\begin{sphinxuseclass}{cell_output}
\noindent\sphinxincludegraphics{{6b9d74b195dd3abcf9a65afa95ffd97f65cd5fc956236c718c0cdba3bb1e8a99}.jpg}

\end{sphinxuseclass}\end{sphinxVerbatimOutput}

\end{sphinxuseclass}

\subsection{Poměrné prodloužení (\sphinxstyleemphasis{strain})}
\label{\detokenize{Prednasky/2_2_Hooke_u016fv_z_xe1kon:pomerne-prodlouzeni-strain}}\begin{quote}

\sphinxAtStartPar
\sphinxstylestrong{Poměrné prodloužení} je bezrozměrná veličina, která vyjadřuje relativní změnu délky materiálu při deformaci.
\end{quote}

\noindent\sphinxincludegraphics{{ec4d46c90ac9707d24050f8c0586e1296f196b49}.png}

\sphinxAtStartPar
Matematicky se definuje jako:
\begin{equation*}
\begin{split}
\varepsilon = \frac{\Delta L}{L_0}
\end{split}
\end{equation*}
\sphinxAtStartPar
kde:
\begin{itemize}
\item {} 
\sphinxAtStartPar
\(\varepsilon \) je poměrné prodloužení {[}\sphinxhyphen{}{]},

\item {} 
\sphinxAtStartPar
\(\Delta l \) je změna délky tělesa {[}m{]},

\item {} 
\sphinxAtStartPar
\(l_0 \) je původní délka tělesa {[}m{]}.

\end{itemize}

\sphinxAtStartPar
Poměrné prodloužení je \sphinxstylestrong{kladné} při \sphinxstylestrong{tahu} a \sphinxstylestrong{záporné} při \sphinxstylestrong{tlaku}.

\sphinxAtStartPar
Na rozdíl od napětí, je možné deformace přímo vidět a měřit. V elastické oblasti jsou ale běžné deformace malé pro inženýrské materiály, typicky v rozsahu mikrometrů. Pro měření proto využíváme senzor, který se označuje pojmem \sphinxstylestrong{tenzometr}.

\sphinxAtStartPar
Příkladem může být \sphinxstylestrong{odporový tenzometr}, u kterého při deformaci dochází ke změně elektrického odporu, což lze přesně změřit.  Tenzometr pracuje na základě \sphinxstylestrong{závislosti elektrického odporu vodiče na jeho délce a průřezu}. Elektrický odpor vodiče je dán vztahem:
\begin{equation*}
\begin{split}
R = \rho \frac{L}{A}
\end{split}
\end{equation*}
\sphinxAtStartPar
kde:
\begin{itemize}
\item {} 
\sphinxAtStartPar
\(R\) je elektrický odpor {[}\(\Omega\){]},

\item {} 
\sphinxAtStartPar
\(\rho\) je měrný elektrický odpor materiálu {[}\(\Omega\) m{]},

\item {} 
\sphinxAtStartPar
\(L\) je délka vodiče {[}m{]},

\item {} 
\sphinxAtStartPar
\(A\) je plocha průřezu vodiče {[}m\(^2\){]}.

\end{itemize}

\sphinxAtStartPar
Při \sphinxstylestrong{natažení} tenzometru se délka vodiče zvětší a průřez zmenší, což vede k \sphinxstylestrong{zvýšení odporu}. Naopak při \sphinxstylestrong{stlačení} se odpor sníží. Změna odporu \(\Delta R\) je úměrná poměrnému prodloužení podle vztahu:

\noindent\sphinxincludegraphics{{6173f43169a48a5aecb07d5906d96f924a8ee2c5}.jpg}
\begin{equation*}
\begin{split}
\frac{\Delta R}{R} = K \cdot \varepsilon
\end{split}
\end{equation*}
\sphinxAtStartPar
kde:
\begin{itemize}
\item {} 
\sphinxAtStartPar
\(K\) je \sphinxstylestrong{tenzometrický faktor} (typicky kolem 2 pro kovové tenzometry),

\item {} 
\sphinxAtStartPar
\(\varepsilon\) je poměrné prodloužení {[}1{]}

\end{itemize}


\subsection{Youngův modul pružnosti (modul pružnosti v tahu)}
\label{\detokenize{Prednasky/2_2_Hooke_u016fv_z_xe1kon:younguv-modul-pruznosti-modul-pruznosti-v-tahu}}\begin{quote}

\sphinxAtStartPar
Youngův modul pružnosti ( E ) charakterizuje \sphinxstylestrong{schopnost materiálu odolávat deformaci při působení napětí}.
\end{quote}

\sphinxAtStartPar
Youngův modul pružnosti definován jako poměr \sphinxstylestrong{normálového napětí} k odpovídající \sphinxstylestrong{poměrné deformaci} v oblasti lineární pružnosti materiálu:
\begin{equation*}
\begin{split}
E = \frac{\sigma}{\varepsilon}
\end{split}
\end{equation*}
\sphinxAtStartPar
kde:
\begin{itemize}
\item {} 
\sphinxAtStartPar
\(E\) je Youngův modul pružnosti (Pa, N/m²)

\item {} 
\sphinxAtStartPar
\(\sigma\) je normálové napětí (Pa, N/m²)

\item {} 
\sphinxAtStartPar
\(\varepsilon\) je poměrná deformace (bezrozměrná veličina)

\end{itemize}

\sphinxAtStartPar
V \sphinxstylestrong{diagramu napětí–deformace} odpovídá Youngův modul pružnosti \sphinxstylestrong{směru (sklonu) přímé části křivky}, tedy oblasti, kde platí Hookeův zákon. Čím \sphinxstylestrong{strmější} je tato část křivky, tím je materiál \sphinxstylestrong{tuhší} a odolnější vůči deformaci.

\noindent\sphinxincludegraphics{{507px-ModulPruznosti2.svg}.png}

\sphinxAtStartPar
Při stejném napětí v tahu prokazuje materiál „B“ podstatně větší deformaci než materiál „A“. Materiál „A“ má tedy větší modul pružnosti v tahu než materiál „B“. Vyšší hodnota modulu pružnosti znamená, že materiál je \sphinxstylestrong{tuhší} a méně se deformuje při stejném zatížení.

\begin{sphinxadmonition}{note}{Note:}
\sphinxAtStartPar
Thomas Young (1773–1829)
\sphinxhref{https://en.wikipedia.org/wiki/Thomas\_Young\_(scientist)}{Thomas Young} byl \sphinxstylestrong{britský fyzik, lékař a polyhistor}, kterému říkali \sphinxhref{https://en.wikipedia.org/wiki/The\_Last\_Man\_Who\_Knew\_Everything}{The Last Man Who Knew Everything}. Popsal \sphinxstylestrong{akomodaci oka} a roli čočky při ostření obrazu a je pokládám za zakladatele fyziologie oka. Proslavil se také svým \sphinxstylestrong{dvojštěrbinovým experimentem}, který poskytl důkaz o vlnové povaze světla. Podílel se na rozluštění \sphinxstylestrong{egyptských hieroglyfů}.

\sphinxAtStartPar
Young popsal charakterizaci pružnosti, která se stala známou jako Youngův modul, označovaný jako \(E\), v roce 1807 a dále ji rozvinul ve svém díle \sphinxhref{https://archive.org/details/bub\_gb\_fGMSAAAAIAAJ}{Course of Lectures on Natural Philosophy and the Mechanical Arts}. Nicméně první experimentální použití konceptu Youngova modulu provedl Giordano Riccati již v roce 1782, tedy o 25 let dříve než Young. Navíc lze tuto myšlenku vystopovat až k článku Leonarda Eulera, publikovanému v roce 1727, tedy přibližně 80 let před Youngovým článkem z roku 1807.
\end{sphinxadmonition}

\sphinxstepscope


\section{Tahová zkouška}
\label{\detokenize{Prednasky/2_3_Tahov_xe1_zkou_u0161ka:tahova-zkouska}}\label{\detokenize{Prednasky/2_3_Tahov_xe1_zkou_u0161ka::doc}}
\begin{sphinxuseclass}{cell}\begin{sphinxVerbatimInput}

\begin{sphinxuseclass}{cell_input}
\begin{sphinxVerbatim}[commandchars=\\\{\}]
\PYG{k+kn}{from} \PYG{n+nn}{IPython}\PYG{n+nn}{.}\PYG{n+nn}{display} \PYG{k+kn}{import} \PYG{n}{IFrame}
\PYG{k+kn}{from} \PYG{n+nn}{IPython}\PYG{n+nn}{.}\PYG{n+nn}{display} \PYG{k+kn}{import} \PYG{n}{YouTubeVideo}
\end{sphinxVerbatim}

\end{sphinxuseclass}\end{sphinxVerbatimInput}

\end{sphinxuseclass}

\section{Tahová zkouška}
\label{\detokenize{Prednasky/2_3_Tahov_xe1_zkou_u0161ka:id1}}
\noindent\sphinxincludegraphics{{stress_strain_2}.png}

\sphinxAtStartPar
Tahová zkouška je destruktivní mechanická zkouška materiálu, která slouží k určení jeho mechanických vlastností při tahovém zatížení. Při zkoušce se zkušební vzorek upne do trhacího stroje a postupně se zatěžuje tahovou silou. Během zkoušky se měří prodloužení vzorku a síla, která na něj působí. Z naměřených dat se pak sestavuje tahový diagram, ze kterého se vyhodnocují mechanické vlastnosti materiálu.

\sphinxAtStartPar
Všeobecně se rozlišují zkoušky tahem pod
\begin{itemize}
\item {} 
\sphinxAtStartPar
\sphinxstylestrong{statickým zatížením} \sphinxhyphen{} stálé konstantní zatížení

\item {} 
\sphinxAtStartPar
\sphinxstylestrong{kvazistatickým zatížením} \sphinxhyphen{} zatížení plynule narůstá a působí plynule (kvazistaticky). Horní mez pro kvazistatické zkušební metody se pohybuje na rychlosti deformace přibližně 10\sphinxhyphen{}1/s, proto maximální nárůst (např. deformace) nesmí být větší než 0,1 \% / s

\item {} 
\sphinxAtStartPar
\sphinxstylestrong{cyklickým zatížením} \sphinxhyphen{} zatížení se periodicky mění

\item {} 
\sphinxAtStartPar
\sphinxstylestrong{rázovým zatížením} \sphinxhyphen{} prudká změna zatížení

\end{itemize}


\section{Postup zkoušky}
\label{\detokenize{Prednasky/2_3_Tahov_xe1_zkou_u0161ka:postup-zkousky}}\begin{enumerate}
\sphinxsetlistlabels{\arabic}{enumi}{enumii}{}{.}%
\item {} 
\sphinxAtStartPar
\sphinxstylestrong{Příprava vzorku}: Z materiálu se vyrobí zkušební vzorek normalizovaných rozměrů a tvaru.

\item {} 
\sphinxAtStartPar
\sphinxstylestrong{Upnutí vzorku}: Vzorek se upne do čelistí trhacíhbo stroje.

\item {} 
\sphinxAtStartPar
\sphinxstylestrong{Zatěžování vzorku}: Vzorek se postupně zatěžuje tahovou silou.

\item {} 
\sphinxAtStartPar
\sphinxstylestrong{Měření dat}: Během zkoušky se měří prodloužení vzorku a působící síla.

\item {} 
\sphinxAtStartPar
\sphinxstylestrong{Sestavení diagramu}: Z naměřených dat se sestaví tahový diagram, který zobrazuje závislost napětí na deformaci.

\item {} 
\sphinxAtStartPar
\sphinxstylestrong{Vyhodnocení parametrů}: Z tahového diagramu se vyhodnotí mechanické vlastnosti materiálu.

\end{enumerate}


\subsection{Tahový diagram}
\label{\detokenize{Prednasky/2_3_Tahov_xe1_zkou_u0161ka:tahovy-diagram}}
\sphinxAtStartPar
Průběh zkoušky je graficky zaznamenán pomocí tahového diagramu, který je závislostí síly (napětí) na prodloužení (poměrném prodloužení) zkušebního tělesa. Rozlišujeme tři typy tahového diagramu:
\begin{enumerate}
\sphinxsetlistlabels{\arabic}{enumi}{enumii}{}{.}%
\item {} 
\sphinxAtStartPar
\sphinxstylestrong{diagram pracovní} \sphinxhyphen{} závislost síly \(F\) na prodloužení \(\Delta l\))

\item {} 
\sphinxAtStartPar
\sphinxstylestrong{diagram smluvní} \sphinxhyphen{} závislost smluvného napětí \(\sigma_s\) na poměrném prodloužení \(\varepsilon\), vztažená k původním rozměrům vzorku) a

\item {} 
\sphinxAtStartPar
\sphinxstylestrong{diagram skutečný} \sphinxhyphen{} závislost skutečného napětí \(\sigma\) na skutečném prodloužení \(\epsilon\), vztažená ke skutečným aktuálním rozměrům vzorku v průběhu zkoušky.

\end{enumerate}

\sphinxAtStartPar
Nejčastěji se využívá smluvní tahový diagram. Smluvní napětí \(\sigma_s\) je dáno vztahem
\begin{equation*}
\begin{split}\sigma_s = \frac{F}{A_0}\end{split}
\end{equation*}
\sphinxAtStartPar
kde \(F\) je zatěžující síla a \(A_0\) je původní průřez zatěžovaného vzorku. Poměrné podélné prodloužení \(\varepsilon\) definované vztahem
\begin{equation*}
\begin{split}\varepsilon = \frac{\Delta l}{l_0} = \frac{l - l_0}{l_0}\end{split}
\end{equation*}
\sphinxAtStartPar
kde \(l\) je délka vzorku po zatížení silou \(F\) a \(l_0\) je původní délka vzorku.

\noindent\sphinxincludegraphics{{609983f621ac4e2251091105de44c157d17e5238}.png}
\begin{enumerate}
\sphinxsetlistlabels{\arabic}{enumi}{enumii}{}{.}%
\item {} 
\sphinxAtStartPar
\sphinxstylestrong{Elastická oblast}
\begin{itemize}
\item {} 
\sphinxAtStartPar
Vztah mezi napětím a deformací je \sphinxstylestrong{lineární}.

\item {} 
\sphinxAtStartPar
Po odstranění zatížení se materiál \sphinxstylestrong{vrátí do původního tvaru}.

\item {} 
\sphinxAtStartPar
Materiál se řídí \sphinxstylestrong{Hookeovým zákonem} a platí přímá úměrnost mezi napětím a deformací.

\item {} 
\sphinxAtStartPar
\sphinxstylestrong{Směrnice přímky} v této oblasti definuje důležitou materiálovou vlastnost – \sphinxstylestrong{Youngův modul pružnosti} (modul pružnosti v tahu).

\end{itemize}

\item {} 
\sphinxAtStartPar
\sphinxstylestrong{Přechod do plastické oblasti}
\begin{itemize}
\item {} 
\sphinxAtStartPar
Při dosažení určitého napětí \sphinxstylestrong{materiál přechází z elastické do plastické oblasti}.

\item {} 
\sphinxAtStartPar
\sphinxstylestrong{Mez kluzu} označuje napětí, při kterém začíná trvalá (plastická) deformace.

\item {} 
\sphinxAtStartPar
\sphinxstylestrong{Elastické deformace jsou vratné, plastické deformace nejsou vratné}.

\end{itemize}

\item {} 
\sphinxAtStartPar
\sphinxstylestrong{Krčkování a lom materiálu}
\begin{itemize}
\item {} 
\sphinxAtStartPar
Při \sphinxstylestrong{krčkování (necking)} dochází k nerovnoměrné plastické deformaci, při které se část vzorku \sphinxstylestrong{výrazně ztenčuje}.

\item {} 
\sphinxAtStartPar
V této oblasti vznikají vysoké koncentrace napětí, což vede ke konečnému \sphinxstylestrong{lomu materiálu}.

\item {} 
\sphinxAtStartPar
Po lomu lze vypočítat \sphinxstylestrong{procentuální prodloužení} a \sphinxstylestrong{celkové zmenšení průřezu vzorku}.

\end{itemize}

\end{enumerate}

\begin{sphinxadmonition}{caution}{Caution:}
\sphinxAtStartPar
Skutečné napětí není rovno smluvnému napětí
V průběhu zkoušky dochází ke změně průřezu. Proto se skutečné napětí liší od smluvného napětí, při kterém se počítá s původní plochou průřezu.
\end{sphinxadmonition}

\begin{sphinxuseclass}{cell}\begin{sphinxVerbatimInput}

\begin{sphinxuseclass}{cell_input}
\begin{sphinxVerbatim}[commandchars=\\\{\}]
\PYG{n}{YouTubeVideo}\PYG{p}{(}\PYG{l+s+s1}{\PYGZsq{}}\PYG{l+s+s1}{67fSwIjYJ\PYGZhy{}E}\PYG{l+s+s1}{\PYGZsq{}}\PYG{p}{,} \PYG{n}{width}\PYG{o}{=}\PYG{l+m+mi}{800}\PYG{p}{)}
\end{sphinxVerbatim}

\end{sphinxuseclass}\end{sphinxVerbatimInput}
\begin{sphinxVerbatimOutput}

\begin{sphinxuseclass}{cell_output}
\noindent\sphinxincludegraphics{{1d40e462b9bec10f67528a47282bfbf61b5d47c4469c08ae56d3ddcf3e49d78f}.jpg}

\end{sphinxuseclass}\end{sphinxVerbatimOutput}

\end{sphinxuseclass}

\subsection{Parametry vyhodnocované z tahové zkoušky}
\label{\detokenize{Prednasky/2_3_Tahov_xe1_zkou_u0161ka:parametry-vyhodnocovane-z-tahove-zkousky}}
\noindent\sphinxincludegraphics{{diagram_prelozeny}.png}


\begin{savenotes}\sphinxattablestart
\sphinxthistablewithglobalstyle
\centering
\begin{tabulary}{\linewidth}[t]{TTTTT}
\sphinxtoprule
\sphinxstyletheadfamily 
\sphinxAtStartPar
\sphinxstylestrong{Parametr}
&\sphinxstyletheadfamily 
\sphinxAtStartPar
\sphinxstylestrong{Meze / Hodnoty}
&\sphinxstyletheadfamily 
\sphinxAtStartPar
\sphinxstylestrong{Označení (ASTM)}
&\sphinxstyletheadfamily 
\sphinxAtStartPar
\sphinxstylestrong{Označení (ISO)}
&\sphinxstyletheadfamily 
\sphinxAtStartPar
\sphinxstylestrong{Definice}
\\
\sphinxmidrule
\sphinxtableatstartofbodyhook
\sphinxAtStartPar
\sphinxstylestrong{Mez úměrnosti} \sphinxstyleemphasis{(Proportional Limit)}
&
\sphinxAtStartPar
\(\sigma_u\) (MPa)
&
\sphinxAtStartPar
PL
&
\sphinxAtStartPar
\sphinxhyphen{}
&
\sphinxAtStartPar
Nejvyšší napětí, při kterém je napětí úměrné deformaci podle Hookeova zákona.
\\
\sphinxhline
\sphinxAtStartPar
\sphinxstylestrong{Mez pružnosti} \sphinxstyleemphasis{(Elasticity Limit)}
&
\sphinxAtStartPar
\(\sigma_e\) (MPa)
&
\sphinxAtStartPar
EL
&
\sphinxAtStartPar
\sphinxhyphen{}
&
\sphinxAtStartPar
Mezní napětí, které po odtížení ( odlehčení ) nevyvolává trvalé deformace
\\
\sphinxhline
\sphinxAtStartPar
\sphinxstylestrong{Modul pružnosti} \sphinxstyleemphasis{(Young’s Modulus)}
&
\sphinxAtStartPar
E (GPa)
&
\sphinxAtStartPar
E
&
\sphinxAtStartPar
E
&
\sphinxAtStartPar
Poměr mezi napětím a deformací v elastické oblasti, určuje tuhost materiálu.
\\
\sphinxhline
\sphinxAtStartPar
\sphinxstylestrong{Mez kluzu} \sphinxstyleemphasis{(Yield Strength)}
&
\sphinxAtStartPar
\(\sigma_k\), \(\sigma_y\) (MPa)
&
\sphinxAtStartPar
YS
&
\sphinxAtStartPar
Re
&
\sphinxAtStartPar
Napětí, při kterém materiál přechází z elastické do plastické oblasti.
\\
\sphinxhline
\sphinxAtStartPar
\sphinxstylestrong{Mez pevnosti v tahu} \sphinxstyleemphasis{(Ultimate Tensile Strength)}
&
\sphinxAtStartPar
\(\sigma_{UTS}, \sigma_P\) (MPa)
&
\sphinxAtStartPar
UTS
&
\sphinxAtStartPar
Rm
&
\sphinxAtStartPar
Nejvyšší dosažené napětí během tahové zkoušky před nástupem krčkování.
\\
\sphinxhline
\sphinxAtStartPar
\sphinxstylestrong{Procentuální prodloužení} \sphinxstyleemphasis{(Elongation at Break)}
&
\sphinxAtStartPar
A (\%)
&
\sphinxAtStartPar
EL
&
\sphinxAtStartPar
A
&
\sphinxAtStartPar
Relativní prodloužení vzorku při lomu vyjádřené v procentech původní délky.
\\
\sphinxhline
\sphinxAtStartPar
\sphinxstylestrong{Kontrakce průřezu} \sphinxstyleemphasis{(Reduction of Area)}
&
\sphinxAtStartPar
Z (\%)
&
\sphinxAtStartPar
RA
&
\sphinxAtStartPar
Z
&
\sphinxAtStartPar
Relativní zmenšení průřezu vzorku při lomu, vyjádřené v procentech původní plochy.
\\
\sphinxhline
\sphinxAtStartPar
\sphinxstylestrong{Skutečné napětí při lomu} \sphinxstyleemphasis{(True Fracture Stress)}
&
\sphinxAtStartPar
σ\_f (MPa)
&
\sphinxAtStartPar
TFS
&
\sphinxAtStartPar
\sphinxhyphen{}
&
\sphinxAtStartPar
Skutečné napětí vypočtené s ohledem na skutečný zmenšený průřez v místě lomu.
\\
\sphinxbottomrule
\end{tabulary}
\sphinxtableafterendhook\par
\sphinxattableend\end{savenotes}
\begin{itemize}
\item {} 
\sphinxAtStartPar
\sphinxstylestrong{Mez kluzu (\(\sigma_Y\))}: Napětí, při kterém dochází k náhlému plastickému toku materiálu (trvalé deformaci) bez nárůstu zatížení.
\begin{equation*}
\begin{split}\sigma_Y = \frac{F_Y}{A}\end{split}
\end{equation*}
\sphinxAtStartPar
kde
\begin{itemize}
\item {} 
\sphinxAtStartPar
\(F_{Y}\) je síla kluzu.

\item {} 
\sphinxAtStartPar
\(A\) je počáteční průřez vzorku.

\end{itemize}

\sphinxAtStartPar
Mez kluzu materiálů často není výrazná, a proto ji nelze při zkoušce tahem jednoznačně určit. V takových případech se vyhodnocuje smluvní mez kluzu. Smluvní mez kluzu se zpravidla určuje při plastickém prodloužení 0,2 \%, proto se tato charakteristická hodnota označuje Rp0,2. V mnoha případech je možné stanovit jak hodnotu horní meze kluzu ReH, tak dolní meze kluzu ReL.

\noindent\sphinxincludegraphics{{Pracovni_diagram_Rp02}.png}

\item {} 
\sphinxAtStartPar
\sphinxstylestrong{Pevnost v tahu (\(\sigma_{UTS}\))}: Maximální napětí, které materiál snese, než dojde k porušení.
\begin{equation*}
\begin{split}
    \sigma_{UTS} = \frac{F_{max}}{A}
    \end{split}
\end{equation*}
\sphinxAtStartPar
kde
\begin{itemize}
\item {} 
\sphinxAtStartPar
\(F_{max}\) je maximální síla.

\item {} 
\sphinxAtStartPar
\(A\) je počáteční průřez vzorku.

\end{itemize}

\sphinxAtStartPar
U kovových materiálů s výraznou mezí kluzu je maximální tahová síla definována jako nejvyšší dosažená síla až za oblastí s mezí kluzu. Maximální tahová síla po překročení meze kluzu může u méně zpevněných materiálů ležet i pod mezí kluzu, v tomto případě je pak mez pevnosti v tahu nižší než hodnota horní meze kluzu.

\end{itemize}


\subsection{Tahový diagram pro různé materiály}
\label{\detokenize{Prednasky/2_3_Tahov_xe1_zkou_u0161ka:tahovy-diagram-pro-ruzne-materialy}}
\sphinxAtStartPar
Tahový diagram je grafické znázornění závislosti napětí na deformaci. Pro různé materiály má diagram typický tvar, který se liší v závislosti na jejich vlastnostech.
\begin{itemize}
\item {} 
\sphinxAtStartPar
\sphinxstylestrong{Křehké materiály} (\sphinxstyleemphasis{brittle}): Mají lineární tahový diagram a po překročení meze pevnosti dochází k náhlému lomu bez výrazných plastických deformací.

\item {} 
\sphinxAtStartPar
\sphinxstylestrong{Houževnaté (tažné) materiály} (\sphinxstyleemphasis{ductile}): Mají nelineární tahový diagram a po překročení meze kluzu dochází k plastickým deformacím, než dojde k lomu.

\end{itemize}

\noindent\sphinxincludegraphics{{ductile%26brittle}.png}

\sphinxAtStartPar
\sphinxstylestrong{Houževnaté materiály}
\begin{itemize}
\item {} 
\sphinxAtStartPar
\sphinxstylestrong{Absorpce energie}: Tažné materiály mohou absorbovat značné množství energie díky své schopnosti procházet plastickou deformací před zlomením.

\item {} 
\sphinxAtStartPar
\sphinxstylestrong{Deformace}: Tyto materiály vydrží značné plastické deformace, než dojde k jejich zlomení.

\item {} 
\sphinxAtStartPar
\sphinxstylestrong{Křivka napětí\sphinxhyphen{}deformace}: Křivka napětí\sphinxhyphen{}deformace tažných materiálů prochází všemi fázemi, jak bylo popsáno výše, včetně elastické deformace, kluzu, zpevnění a neckingu před zlomením.

\item {} 
\sphinxAtStartPar
\sphinxstylestrong{Příklady}: Měkká ocel, hliník, měď, guma a většina plastů.

\end{itemize}

\sphinxAtStartPar
\sphinxstylestrong{Křehké materiály}
\begin{itemize}
\item {} 
\sphinxAtStartPar
\sphinxstylestrong{Chování při zlomení}: Křehké materiály se lámou bez výrazné změny v prodloužení nebo plastické deformaci. Jejich křivka napětí\sphinxhyphen{}deformace je po mezní síle téměř vertikální, což naznačuje velmi malou plastickou deformaci.

\item {} 
\sphinxAtStartPar
\sphinxstylestrong{Konzistence pevnosti}: U křehkých materiálů jsou mez kluzu, mez pevnosti a mez porušení v podstatě stejné.

\item {} 
\sphinxAtStartPar
\sphinxstylestrong{Příklady}: Lité železo, keramika, sklo, beton a kámen.

\end{itemize}

\sphinxstepscope


\section{Přehled mechanických vlastností vybraných materiálů}
\label{\detokenize{Prednasky/2_4_P_u0159ehled_mechanick_xfdch_vlastnost_xed:prehled-mechanickych-vlastnosti-vybranych-materialu}}\label{\detokenize{Prednasky/2_4_P_u0159ehled_mechanick_xfdch_vlastnost_xed::doc}}
\begin{sphinxuseclass}{cell}\begin{sphinxVerbatimInput}

\begin{sphinxuseclass}{cell_input}
\begin{sphinxVerbatim}[commandchars=\\\{\}]
\PYG{k+kn}{import} \PYG{n+nn}{pandas} \PYG{k}{as} \PYG{n+nn}{pd}
\PYG{k+kn}{from} \PYG{n+nn}{IPython}\PYG{n+nn}{.}\PYG{n+nn}{display} \PYG{k+kn}{import} \PYG{n}{display}\PYG{p}{,} \PYG{n}{Math}\PYG{p}{,} \PYG{n}{Latex}
\PYG{n}{table\PYGZus{}MIT} \PYG{o}{=} \PYG{n}{pd}\PYG{o}{.}\PYG{n}{read\PYGZus{}csv}\PYG{p}{(}\PYG{l+s+s1}{\PYGZsq{}}\PYG{l+s+s1}{https://web.mit.edu/course/3/3.11/www/modules/props.csv}\PYG{l+s+s1}{\PYGZsq{}}\PYG{p}{,} \PYG{n}{sep}\PYG{o}{=}\PYG{l+s+s1}{\PYGZsq{}}\PYG{l+s+s1}{,}\PYG{l+s+s1}{\PYGZsq{}}\PYG{p}{)}
\PYG{n}{display}\PYG{p}{(}\PYG{n}{table\PYGZus{}MIT}\PYG{p}{)}
\end{sphinxVerbatim}

\end{sphinxuseclass}\end{sphinxVerbatimInput}
\begin{sphinxVerbatimOutput}

\begin{sphinxuseclass}{cell_output}
\begin{sphinxVerbatim}[commandchars=\\\{\}]
                           MATERIAL       Type   Cost (\PYGZdl{}/kg)   \PYGZbs{}
0                   Alumina (Al2O3)    ceramic           1.90   
1          Aluminum alloy (7075\PYGZhy{}T6)      metal           1.80   
2                   Beryllium alloy      metal         315.00   
3                    Bone (compact)    natural           1.90   
4        Brass (70Cu30Zn, annealed)      metal           2.20   
5                   Cermets (Co/WC)  composite          78.60   
6          CFRP Laminate (graphite)  composite         110.00   
7                          Concrete    ceramic           0.05   
8                     Copper alloys      metal           2.25   
9                              Cork    natural           9.95   
10                  Epoxy thermoset    polymer           5.50   
11            GFRP Laminate (glass)  composite           3.90   
12                     Glass (soda)    ceramic           1.35   
13                          Granite    ceramic           3.15   
14                        Ice (H2O)    ceramic           0.23   
15                      Lead alloys      metal           1.20   
16                    Nickel alloys      metal           6.10   
17                Polyamide (nylon)    polymer           4.30   
18          Polybutadiene elastomer    polymer           1.20   
19                    Polycarbonate    polymer           4.90   
20              Polyester thermoset    polymer           3.00   
21              Polyethylene (HDPE)    polymer           1.00   
22                    Polypropylene    polymer           1.10   
23           Polyurethane elastomer    polymer           4.00   
24   Polyvinyl chloride (rigid PVC)    polymer           1.50   
25                          Silicon    ceramic           2.35   
26            Silicon Carbide (SiC)    ceramic          36.00   
27       Spruce (parallel to grain)    natural           1.00   
28        Steel, high strength 4340      metal           0.25   
29                 Steel, mild 1020      metal           0.50   
30  Steel, stainless austenitic 304      metal           2.70   
31           Titanium alloy (6Al4V)      metal          16.25   
32            Tungsten Carbide (WC)    ceramic          50.00   
33                              NaN        NaN            NaN   

    Density (r,Mg/m3)  Young\PYGZsq{}s Modulus (E, GPa)  Shear Modulus (G, GPa)  \PYGZbs{}
0                3.90                  390.0000                125.0000   
1                2.70                   70.0000                 28.0000   
2                2.90                  245.0000                110.0000   
3                2.00                   14.0000                  3.5000   
4                8.40                  130.0000                 39.0000   
5               11.50                  470.0000                200.0000   
6                1.50                    1.5000                 53.0000   
7                2.50                   48.0000                 20.0000   
8                8.30                  135.0000                 50.0000   
9                0.18                    0.0320                  0.0050   
10               1.20                    3.5000                  1.4000   
11               1.80                   26.0000                 10.0000   
12               2.50                   65.0000                 26.0000   
13               2.60                   66.0000                 26.0000   
14               0.92                    9.1000                  3.6000   
15              11.10                   16.0000                  5.5000   
16               8.50                  180.0000                 70.0000   
17               1.10                    3.0000                  0.7600   
18               0.91                    0.0016                  0.0005   
19               1.20                    2.7000                  0.9700   
20               1.30                    3.5000                  1.4000   
21               0.95                    0.7000                  0.3100   
22               0.89                    0.9000                  0.4200   
23               1.20                    0.0250                  0.0086   
24               1.40                    1.5000                  0.6000   
25               2.30                  110.0000                 44.0000   
26               2.80                  450.0000                190.0000   
27               0.60                    9.0000                  0.8000   
28               7.80                  210.0000                 76.0000   
29               7.80                  210.0000                 76.0000   
30               7.80                  210.0000                 76.0000   
31               4.50                  100.0000                 39.0000   
32              15.50                  550.0000                270.0000   
33                NaN                       NaN                     NaN   

    Poisson\PYGZsq{}s Ratio (n)  Yield Stress (sY, MPa)  UTS (sf,MPa)  \PYGZbs{}
0                  0.26                  4800.0          35.0   
1                  0.34                   500.0         570.0   
2                  0.12                   360.0         500.0   
3                  0.43                   100.0         100.0   
4                  0.33                    75.0         325.0   
5                  0.30                   650.0        1200.0   
6                  0.28                   200.0         550.0   
7                  0.20                    25.0           3.0   
8                  0.35                   510.0         720.0   
9                  0.25                     1.4           1.5   
10                 0.25                    45.0          45.0   
11                 0.28                   125.0         530.0   
12                 0.23                  3500.0          35.0   
13                 0.25                  2500.0          60.0   
14                 0.28                    85.0           6.5   
15                 0.45                    33.0          42.0   
16                 0.31                   900.0        1200.0   
17                 0.42                    40.0          55.0   
18                 0.50                     2.1           2.1   
19                 0.42                    70.0          77.0   
20                 0.25                    50.0           0.7   
21                 0.42                    25.0          33.0   
22                 0.42                    35.0          45.0   
23                 0.50                    30.0          30.0   
24                 0.42                    53.0          60.0   
25                 0.24                  3200.0          35.0   
26                 0.15                  9800.0          35.0   
27                 0.30                    48.0          50.0   
28                 0.29                  1240.0        1550.0   
29                 0.29                   200.0         380.0   
30                 0.28                   240.0         590.0   
31                 0.36                   910.0         950.0   
32                 0.21                  6800.0          35.0   
33                  NaN                     NaN           NaN   

    Breaking strain (ef,\PYGZpc{})  Fracture Toughness (Kc,MN m\PYGZhy{}3/2)  \PYGZbs{}
0                      0.0                             4.400   
1                     12.0                            28.000   
2                      6.0                             5.000   
3                      9.0                             5.000   
4                     70.0                            80.000   
5                      2.5                            13.000   
6                      2.0                            38.000   
7                      0.0                             0.750   
8                      0.3                            94.000   
9                     80.0                             0.074   
10                     4.0                             0.500   
11                     2.0                            40.000   
12                     0.0                             0.710   
13                     0.1                             1.500   
14                     0.0                             0.110   
15                    60.0                            40.000   
16                    30.0                            93.000   
17                     5.0                             3.000   
18                   500.0                             0.087   
19                    60.0                             2.600   
20                     2.0                             0.700   
21                    90.0                             3.500   
22                    90.0                             3.000   
23                   500.0                             0.300   
24                    50.0                             0.540   
25                     0.0                             1.500   
26                     0.0                             4.200   
27                    10.0                             2.500   
28                     2.5                           100.000   
29                    25.0                           140.000   
30                    60.0                            50.000   
31                    15.0                            85.000   
32                     0.0                             3.700   
33                     NaN                               NaN   

    Thermal Expansion (a,10\PYGZhy{}6/C)  Unnamed: 12  
0                            8.1          NaN  
1                           33.0          NaN  
2                           14.0          NaN  
3                           20.0          NaN  
4                           20.0          NaN  
5                            5.8          NaN  
6                           12.0          NaN  
7                           11.0          NaN  
8                           18.0          NaN  
9                          180.0          NaN  
10                          60.0          NaN  
11                          19.0          NaN  
12                           8.8          NaN  
13                           6.5          NaN  
14                          55.0          NaN  
15                          29.0          NaN  
16                          13.0          NaN  
17                         103.0          NaN  
18                         140.0          NaN  
19                          70.0          NaN  
20                         150.0          NaN  
21                         225.0          NaN  
22                          85.0          NaN  
23                         125.0          NaN  
24                          75.0          NaN  
25                           6.0          NaN  
26                           4.2          NaN  
27                           4.0          NaN  
28                          14.0          NaN  
29                          14.0          NaN  
30                          17.0          NaN  
31                           9.4          NaN  
32                           5.8          NaN  
33                           NaN          NaN  
\end{sphinxVerbatim}

\end{sphinxuseclass}\end{sphinxVerbatimOutput}

\end{sphinxuseclass}
\sphinxstepscope


\section{Analýza dat z tahového diagramu}
\label{\detokenize{Prednasky/2_7_P_u0159_xedklad _tahov_xe9_zkou_u0161ky:analyza-dat-z-tahoveho-diagramu}}\label{\detokenize{Prednasky/2_7_P_u0159_xedklad _tahov_xe9_zkou_u0161ky::doc}}
\begin{sphinxuseclass}{cell}\begin{sphinxVerbatimInput}

\begin{sphinxuseclass}{cell_input}
\begin{sphinxVerbatim}[commandchars=\\\{\}]
\PYG{k+kn}{import} \PYG{n+nn}{numpy} \PYG{k}{as} \PYG{n+nn}{np}
\PYG{k+kn}{import} \PYG{n+nn}{pandas} \PYG{k}{as} \PYG{n+nn}{pd}
\PYG{k+kn}{import} \PYG{n+nn}{matplotlib}\PYG{n+nn}{.}\PYG{n+nn}{pyplot} \PYG{k}{as} \PYG{n+nn}{plt}
\PYG{k+kn}{from} \PYG{n+nn}{lmfit} \PYG{k+kn}{import} \PYG{n}{Model}
\PYG{o}{\PYGZpc{}}\PYG{k}{matplotlib} inline
\end{sphinxVerbatim}

\end{sphinxuseclass}\end{sphinxVerbatimInput}

\end{sphinxuseclass}
\sphinxAtStartPar
Dva datové soubory .xls obsahující data z tahové zkoušky jsme získali z věřejně přístupných \sphinxhref{https://professorkazarinoff.github.io/Engineering-Materials-Programming/07-Mechanical-Properties/plotting-stress-strain-curves.html}{zdrojů}. Popisují naměřená data při zkoušce oceli 1045 slitiny hliníku 6061.


\subsection{Načtení dat z Excelu}
\label{\detokenize{Prednasky/2_7_P_u0159_xedklad _tahov_xe9_zkou_u0161ky:nacteni-dat-z-excelu}}
\sphinxAtStartPar
Nejdříve si pomocí funkce \sphinxcode{\sphinxupquote{\%ls}} ověříme, že nás adresář obsahuje data z měření. Tato funkce nám vypíše obsah adresáře.

\sphinxAtStartPar
Pomocí funkce \sphinxcode{\sphinxupquote{pd.read\_excel()}} z balíčku pandas je možné přímo načíst data. Data ze dvou excelových souborů budou uložena ve dvou datových objektech nazvaných \sphinxcode{\sphinxupquote{steel\_df}} a \sphinxcode{\sphinxupquote{al\_df}}. Datový objekt pandas je tabulkový datový typ.

\sphinxAtStartPar
K zobrazení prvních pěti řádků každého datového rámce můžeme použít metodu \sphinxcode{\sphinxupquote{.head()}}.

\begin{sphinxuseclass}{cell}\begin{sphinxVerbatimInput}

\begin{sphinxuseclass}{cell_input}
\begin{sphinxVerbatim}[commandchars=\\\{\}]
\PYG{o}{\PYGZpc{}}\PYG{k}{ls} ../data/ 
\PYG{n}{df\PYGZus{}al} \PYG{o}{=} \PYG{n}{pd}\PYG{o}{.}\PYG{n}{read\PYGZus{}excel}\PYG{p}{(}\PYG{l+s+s1}{\PYGZsq{}}\PYG{l+s+s1}{../data/Al60601\PYGZus{}raw\PYGZus{}data.xls}\PYG{l+s+s1}{\PYGZsq{}}\PYG{p}{)}
\PYG{n}{df\PYGZus{}al}\PYG{o}{.}\PYG{n}{head}\PYG{p}{(}\PYG{p}{)}
\end{sphinxVerbatim}

\end{sphinxuseclass}\end{sphinxVerbatimInput}
\begin{sphinxVerbatimOutput}

\begin{sphinxuseclass}{cell_output}
\begin{sphinxVerbatim}[commandchars=\\\{\}]
Al60601\PYGZus{}raw\PYGZus{}data.xls  Steel1018\PYGZus{}raw\PYGZus{}data.xls
\end{sphinxVerbatim}

\begin{sphinxVerbatim}[commandchars=\\\{\}]
WARNING *** OLE2 inconsistency: SSCS size is 0 but SSAT size is non\PYGZhy{}zero
\end{sphinxVerbatim}

\begin{sphinxVerbatim}[commandchars=\\\{\}]
   TESTNUM  POINTNUM    TIME    POSIT       FORCE       EXT       CH5  CH6  \PYGZbs{}
0      542         1   8.470  0.02256  201.146011 \PYGZhy{}0.001444  0.007552  NaN   
1      542         2   8.632  0.02330  206.599442  0.000302  0.007552  NaN   
2      542         3  10.027  0.02846  287.512573  0.003044  0.018898  NaN   
3      542         4  11.031  0.03232  365.380981  0.009881  0.022061  NaN   
4      542         5  11.928  0.03616  447.813965  0.014085  0.033652  NaN   

   CH7  CH8  
0  NaN  NaN  
1  NaN  NaN  
2  NaN  NaN  
3  NaN  NaN  
4  NaN  NaN  
\end{sphinxVerbatim}

\end{sphinxuseclass}\end{sphinxVerbatimOutput}

\end{sphinxuseclass}
\begin{sphinxuseclass}{cell}\begin{sphinxVerbatimInput}

\begin{sphinxuseclass}{cell_input}
\begin{sphinxVerbatim}[commandchars=\\\{\}]
\PYG{n}{df\PYGZus{}steel} \PYG{o}{=} \PYG{n}{pd}\PYG{o}{.}\PYG{n}{read\PYGZus{}excel}\PYG{p}{(}\PYG{l+s+s1}{\PYGZsq{}}\PYG{l+s+s1}{../data/Steel1018\PYGZus{}raw\PYGZus{}data.xls}\PYG{l+s+s1}{\PYGZsq{}}\PYG{p}{)}
\PYG{n}{df\PYGZus{}steel}\PYG{o}{.}\PYG{n}{head}\PYG{p}{(}\PYG{p}{)}
\end{sphinxVerbatim}

\end{sphinxuseclass}\end{sphinxVerbatimInput}
\begin{sphinxVerbatimOutput}

\begin{sphinxuseclass}{cell_output}
\begin{sphinxVerbatim}[commandchars=\\\{\}]
WARNING *** OLE2 inconsistency: SSCS size is 0 but SSAT size is non\PYGZhy{}zero
\end{sphinxVerbatim}

\begin{sphinxVerbatim}[commandchars=\\\{\}]
   TESTNUM  POINTNUM    TIME    POSIT       FORCE       EXT       CH5  CH6  \PYGZbs{}
0      523         1   6.189  0.07302  202.924728  0.000402 \PYGZhy{}0.028272  NaN   
1      523         2   6.549  0.07396  205.714890 \PYGZhy{}0.000238 \PYGZhy{}0.034549  NaN   
2      523         3   7.148  0.07624  217.763336 \PYGZhy{}0.000713 \PYGZhy{}0.030140  NaN   
3      523         4   9.146  0.08438  316.306122  0.002377 \PYGZhy{}0.025968  NaN   
4      523         5  10.041  0.08822  417.003357  0.003089 \PYGZhy{}0.024100  NaN   

   CH7  CH8  
0  NaN  NaN  
1  NaN  NaN  
2  NaN  NaN  
3  NaN  NaN  
4  NaN  NaN  
\end{sphinxVerbatim}

\end{sphinxuseclass}\end{sphinxVerbatimOutput}

\end{sphinxuseclass}
\sphinxAtStartPar
Vidíme, že naměřená data obashují různé hodnoty. Pro nás je nejdůležitejší sloupec síla (\sphinxstylestrong{FORCE}). Kromě těchto hodnot máme k dispozici ještě hodnotu \sphinxstylestrong{EXT} hodnoty prodloužení z mechanického extenzometeru v \% a \sphinxstylestrong{CH5}, který obsahuje hodnoty prodloužení z laserového extensometru v \%.


\subsection{Úprava dat}
\label{\detokenize{Prednasky/2_7_P_u0159_xedklad _tahov_xe9_zkou_u0161ky:uprava-dat}}
\sphinxAtStartPar
Protože tato data jsme získali z americké laboratoře, máme sílu určenou v librách (pounds) a průmer zkušební tyče je 0.506 palců (inch). Nejdříve musíme tyto hodnoty převést na jednotky SI.
\begin{itemize}
\item {} 
\sphinxAtStartPar
Pro převod síly nebo hmotnosti z liber (lb) na newtony (N) použijeme následující vztah: 1 lb \(\approx\)  4,4822162 N.

\item {} 
\sphinxAtStartPar
Pro převod délky z palců (inch) na milimetry (mm) použijeme následující vztah:
1 inch = 25.4 mm.

\end{itemize}

\begin{sphinxadmonition}{note}{Note:}
\sphinxAtStartPar
Síla v \sphinxstylestrong{N}, rozměry v \sphinxstylestrong{mm}, napětí v \sphinxstylestrong{MPa}
\$\(1 \frac{\mathrm{N}}{\mathrm{mm}^2} =   1\,10^6 \frac{\mathrm{N}}{\mathrm{m}^2} = 1 \mathrm{MPa}\)\$
\end{sphinxadmonition}

\begin{sphinxuseclass}{cell}\begin{sphinxVerbatimInput}

\begin{sphinxuseclass}{cell_input}
\begin{sphinxVerbatim}[commandchars=\\\{\}]
\PYG{n}{d} \PYG{o}{=} \PYG{l+m+mf}{0.506} \PYG{o}{*} \PYG{l+m+mf}{25.4} \PYG{c+c1}{\PYGZsh{} mm}
\PYG{n}{force\PYGZus{}al} \PYG{o}{=} \PYG{n}{df\PYGZus{}al}\PYG{p}{[}\PYG{l+s+s1}{\PYGZsq{}}\PYG{l+s+s1}{FORCE}\PYG{l+s+s1}{\PYGZsq{}}\PYG{p}{]}\PYG{o}{.}\PYG{n}{to\PYGZus{}numpy}\PYG{p}{(}\PYG{p}{)} \PYG{o}{*}  \PYG{l+m+mf}{4.4822162}\PYG{c+c1}{\PYGZsh{}N}
\PYG{n}{force\PYGZus{}steel} \PYG{o}{=} \PYG{n}{df\PYGZus{}steel}\PYG{p}{[}\PYG{l+s+s1}{\PYGZsq{}}\PYG{l+s+s1}{FORCE}\PYG{l+s+s1}{\PYGZsq{}}\PYG{p}{]}\PYG{o}{.}\PYG{n}{to\PYGZus{}numpy}\PYG{p}{(}\PYG{p}{)} \PYG{o}{*} \PYG{l+m+mf}{4.4822162} \PYG{c+c1}{\PYGZsh{}N}
\end{sphinxVerbatim}

\end{sphinxuseclass}\end{sphinxVerbatimInput}

\end{sphinxuseclass}
\sphinxAtStartPar
Následně si určíme smluvné napětí vydělením síly počátečným průřezem a převedem prodloužení z procent. Protože laserový exntezometr je přesnější, budeme počítat s těmito daty.

\begin{sphinxuseclass}{cell}\begin{sphinxVerbatimInput}

\begin{sphinxuseclass}{cell_input}
\begin{sphinxVerbatim}[commandchars=\\\{\}]
\PYG{n}{A} \PYG{o}{=} \PYG{n}{np}\PYG{o}{.}\PYG{n}{pi}\PYG{o}{*}\PYG{n}{d}\PYG{o}{*}\PYG{o}{*}\PYG{l+m+mi}{2}\PYG{o}{/}\PYG{l+m+mi}{4}
\PYG{n}{stress\PYGZus{}al} \PYG{o}{=} \PYG{n}{force\PYGZus{}al} \PYG{o}{/} \PYG{n}{A}
\PYG{n}{stress\PYGZus{}steel} \PYG{o}{=} \PYG{n}{force\PYGZus{}steel}\PYG{o}{/}\PYG{n}{A}

\PYG{n}{strain\PYGZus{}al} \PYG{o}{=} \PYG{n}{df\PYGZus{}al}\PYG{p}{[}\PYG{l+s+s1}{\PYGZsq{}}\PYG{l+s+s1}{CH5}\PYG{l+s+s1}{\PYGZsq{}}\PYG{p}{]}\PYG{o}{.}\PYG{n}{to\PYGZus{}numpy}\PYG{p}{(}\PYG{p}{)}\PYG{o}{*}\PYG{l+m+mf}{0.01}
\PYG{n}{strain\PYGZus{}steel} \PYG{o}{=} \PYG{n}{df\PYGZus{}steel}\PYG{p}{[}\PYG{l+s+s1}{\PYGZsq{}}\PYG{l+s+s1}{CH5}\PYG{l+s+s1}{\PYGZsq{}}\PYG{p}{]}\PYG{o}{.}\PYG{n}{to\PYGZus{}numpy}\PYG{p}{(}\PYG{p}{)}\PYG{o}{*}\PYG{l+m+mf}{0.01}
\end{sphinxVerbatim}

\end{sphinxuseclass}\end{sphinxVerbatimInput}

\end{sphinxuseclass}

\subsection{Vykreslení závislosti}
\label{\detokenize{Prednasky/2_7_P_u0159_xedklad _tahov_xe9_zkou_u0161ky:vykresleni-zavislosti}}
\sphinxAtStartPar
Nyní, když máme data z tahového testu připravena, můžeme vytvořit rychlý graf pomocí metody \sphinxcode{\sphinxupquote{ax.plot()}} knihovny Matplotlib. První pár hodnot (x, y), který předáme do \sphinxcode{\sphinxupquote{ax.plot()}}, je ( strain\_al\}, stress\_al), a druhý pár (x, y), který předáme, je (strain\_steel, stress\_steel). Příkaz \sphinxcode{\sphinxupquote{plt.show()}} zobrazí graf.

\begin{sphinxuseclass}{cell}\begin{sphinxVerbatimInput}

\begin{sphinxuseclass}{cell_input}
\begin{sphinxVerbatim}[commandchars=\\\{\}]
\PYG{n}{fig}\PYG{p}{,}\PYG{n}{ax} \PYG{o}{=} \PYG{n}{plt}\PYG{o}{.}\PYG{n}{subplots}\PYG{p}{(}\PYG{p}{)}

\PYG{n}{ax}\PYG{o}{.}\PYG{n}{plot}\PYG{p}{(}\PYG{n}{strain\PYGZus{}al}\PYG{p}{,} \PYG{n}{stress\PYGZus{}al}\PYG{p}{,} \PYG{n}{strain\PYGZus{}steel}\PYG{p}{,} \PYG{n}{stress\PYGZus{}steel}\PYG{p}{)}

\PYG{n}{plt}\PYG{o}{.}\PYG{n}{show}\PYG{p}{(}\PYG{p}{)}
\end{sphinxVerbatim}

\end{sphinxuseclass}\end{sphinxVerbatimInput}
\begin{sphinxVerbatimOutput}

\begin{sphinxuseclass}{cell_output}
\noindent\sphinxincludegraphics{{0f15dea8873f7ac009714efd9393a4a82d346d6f373785d39d8c16879cec4c28}.png}

\end{sphinxuseclass}\end{sphinxVerbatimOutput}

\end{sphinxuseclass}
\sphinxAtStartPar
Náš graf vylepšíme přidáním popisek jednotlivých křivek a os.


\begin{savenotes}\sphinxattablestart
\sphinxthistablewithglobalstyle
\centering
\begin{tabulary}{\linewidth}[t]{TTT}
\sphinxtoprule
\sphinxstyletheadfamily 
\sphinxAtStartPar
Matplotlib metoda
&\sphinxstyletheadfamily 
\sphinxAtStartPar
Popis
&\sphinxstyletheadfamily 
\sphinxAtStartPar
Příklad
\\
\sphinxmidrule
\sphinxtableatstartofbodyhook
\sphinxAtStartPar
ax.set\_xlabel()
&
\sphinxAtStartPar
popisek osy x
&
\sphinxAtStartPar
plt.xlabel(‘strain (in/in)’)
\\
\sphinxhline
\sphinxAtStartPar
ax.set\_ylabel()
&
\sphinxAtStartPar
popisek osy y
&
\sphinxAtStartPar
plt.ylabel(‘stress (ksi)’)
\\
\sphinxhline
\sphinxAtStartPar
ax.set\_title()
&
\sphinxAtStartPar
poisek grafu
&
\sphinxAtStartPar
plt.title(‘Stress Strain Curve’)
\\
\sphinxhline
\sphinxAtStartPar
ax.legend()
&
\sphinxAtStartPar
legenda
&
\sphinxAtStartPar
plt.plt(x,y, label=’steel’)
\\
\sphinxbottomrule
\end{tabulary}
\sphinxtableafterendhook\par
\sphinxattableend\end{savenotes}

\begin{sphinxuseclass}{cell}\begin{sphinxVerbatimInput}

\begin{sphinxuseclass}{cell_input}
\begin{sphinxVerbatim}[commandchars=\\\{\}]
\PYG{n}{fig}\PYG{p}{,}\PYG{n}{ax} \PYG{o}{=} \PYG{n}{plt}\PYG{o}{.}\PYG{n}{subplots}\PYG{p}{(}\PYG{p}{)}

\PYG{n}{ax}\PYG{o}{.}\PYG{n}{plot}\PYG{p}{(}\PYG{n}{strain\PYGZus{}al}\PYG{p}{,} \PYG{n}{stress\PYGZus{}al}\PYG{p}{)}
\PYG{n}{ax}\PYG{o}{.}\PYG{n}{plot}\PYG{p}{(}\PYG{n}{strain\PYGZus{}steel}\PYG{p}{,} \PYG{n}{stress\PYGZus{}steel}\PYG{p}{)}

\PYG{n}{ax}\PYG{o}{.}\PYG{n}{set\PYGZus{}xlabel}\PYG{p}{(}\PYG{l+s+s1}{\PYGZsq{}}\PYG{l+s+s1}{Poměrná deformace (m/m)}\PYG{l+s+s1}{\PYGZsq{}}\PYG{p}{)}
\PYG{n}{ax}\PYG{o}{.}\PYG{n}{set\PYGZus{}ylabel}\PYG{p}{(}\PYG{l+s+s1}{\PYGZsq{}}\PYG{l+s+s1}{Napětí (MPa)}\PYG{l+s+s1}{\PYGZsq{}}\PYG{p}{)}
\PYG{n}{ax}\PYG{o}{.}\PYG{n}{set\PYGZus{}title}\PYG{p}{(}\PYG{l+s+s1}{\PYGZsq{}}\PYG{l+s+s1}{Smluvní tahový diagram  slitiny hliníku 6061 a oceli 1018}\PYG{l+s+s1}{\PYGZsq{}}\PYG{p}{)}
\PYG{n}{ax}\PYG{o}{.}\PYG{n}{legend}\PYG{p}{(}\PYG{p}{[}\PYG{l+s+s1}{\PYGZsq{}}\PYG{l+s+s1}{Al6061}\PYG{l+s+s1}{\PYGZsq{}}\PYG{p}{,}\PYG{l+s+s1}{\PYGZsq{}}\PYG{l+s+s1}{Steel1018}\PYG{l+s+s1}{\PYGZsq{}}\PYG{p}{]}\PYG{p}{)}

\PYG{n}{plt}\PYG{o}{.}\PYG{n}{show}\PYG{p}{(}\PYG{p}{)}
\end{sphinxVerbatim}

\end{sphinxuseclass}\end{sphinxVerbatimInput}
\begin{sphinxVerbatimOutput}

\begin{sphinxuseclass}{cell_output}
\noindent\sphinxincludegraphics{{8380842d82a3796dd8cc00b045f37a1ba881e955758c6861305b3028c7736d0f}.png}

\end{sphinxuseclass}\end{sphinxVerbatimOutput}

\end{sphinxuseclass}
\sphinxAtStartPar
Nyní můžeme uložit graf jako obrázek ve formátu .png pomocí metody \sphinxcode{\sphinxupquote{plt.savefig()}} knihovny Matplotlib. Kód níže vytvoří graf a uloží obrázek s názvem \sphinxcode{\sphinxupquote{tahovy\_diagram.png}}. Argument \sphinxcode{\sphinxupquote{dpi=300}} uvnitř metody \sphinxcode{\sphinxupquote{plt.savefig()}} specifikuje rozlišení uloženého obrázku. Obrázek bude uložen ve složce \sphinxcode{\sphinxupquote{static}} stejného adresáře jako náš kód.

\begin{sphinxuseclass}{cell}\begin{sphinxVerbatimInput}

\begin{sphinxuseclass}{cell_input}
\begin{sphinxVerbatim}[commandchars=\\\{\}]
\PYG{n}{fig}\PYG{p}{,}\PYG{n}{ax} \PYG{o}{=} \PYG{n}{plt}\PYG{o}{.}\PYG{n}{subplots}\PYG{p}{(}\PYG{p}{)}

\PYG{n}{ax}\PYG{o}{.}\PYG{n}{plot}\PYG{p}{(}\PYG{n}{strain\PYGZus{}al}\PYG{p}{,} \PYG{n}{stress\PYGZus{}al}\PYG{p}{,} \PYG{n}{label} \PYG{o}{=} \PYG{l+s+s1}{\PYGZsq{}}\PYG{l+s+s1}{Al6061}\PYG{l+s+s1}{\PYGZsq{}}\PYG{p}{)}
\PYG{n}{ax}\PYG{o}{.}\PYG{n}{plot}\PYG{p}{(}\PYG{n}{strain\PYGZus{}steel}\PYG{p}{,} \PYG{n}{stress\PYGZus{}steel}\PYG{p}{,} \PYG{n}{label} \PYG{o}{=} \PYG{l+s+s1}{\PYGZsq{}}\PYG{l+s+s1}{Steel1018}\PYG{l+s+s1}{\PYGZsq{}}\PYG{p}{)}

\PYG{n}{ax}\PYG{o}{.}\PYG{n}{set\PYGZus{}xlabel}\PYG{p}{(}\PYG{l+s+s1}{\PYGZsq{}}\PYG{l+s+s1}{Poměrná deformace (m/m)}\PYG{l+s+s1}{\PYGZsq{}}\PYG{p}{)}
\PYG{n}{ax}\PYG{o}{.}\PYG{n}{set\PYGZus{}ylabel}\PYG{p}{(}\PYG{l+s+s1}{\PYGZsq{}}\PYG{l+s+s1}{Napětí (MPa)}\PYG{l+s+s1}{\PYGZsq{}}\PYG{p}{)}
\PYG{n}{ax}\PYG{o}{.}\PYG{n}{set\PYGZus{}title}\PYG{p}{(}\PYG{l+s+s1}{\PYGZsq{}}\PYG{l+s+s1}{Smluvní tahový diagram  slitiny hliníku 6061 a oceli 1018}\PYG{l+s+s1}{\PYGZsq{}}\PYG{p}{)}
\PYG{n}{ax}\PYG{o}{.}\PYG{n}{legend}\PYG{p}{(}\PYG{p}{)}

\PYG{n}{plt}\PYG{o}{.}\PYG{n}{savefig}\PYG{p}{(}\PYG{l+s+s1}{\PYGZsq{}}\PYG{l+s+s1}{static/tahovy\PYGZus{}diagram.png}\PYG{l+s+s1}{\PYGZsq{}}\PYG{p}{,} \PYG{n}{dpi}\PYG{o}{=}\PYG{l+m+mi}{300}\PYG{p}{)}
\PYG{n}{plt}\PYG{o}{.}\PYG{n}{show}\PYG{p}{(}\PYG{p}{)}
\end{sphinxVerbatim}

\end{sphinxuseclass}\end{sphinxVerbatimInput}
\begin{sphinxVerbatimOutput}

\begin{sphinxuseclass}{cell_output}
\noindent\sphinxincludegraphics{{8380842d82a3796dd8cc00b045f37a1ba881e955758c6861305b3028c7736d0f}.png}

\end{sphinxuseclass}\end{sphinxVerbatimOutput}

\end{sphinxuseclass}

\section{Nalezení meze pevnosti}
\label{\detokenize{Prednasky/2_7_P_u0159_xedklad _tahov_xe9_zkou_u0161ky:nalezeni-meze-pevnosti}}
\sphinxAtStartPar
Mez pevnosti je bod na grafu s největší hodnotou napětí.

\begin{sphinxuseclass}{cell}\begin{sphinxVerbatimInput}

\begin{sphinxuseclass}{cell_input}
\begin{sphinxVerbatim}[commandchars=\\\{\}]
\PYG{n}{uts\PYGZus{}steel} \PYG{o}{=} \PYG{n}{np}\PYG{o}{.}\PYG{n}{max}\PYG{p}{(}\PYG{n}{stress\PYGZus{}steel}\PYG{p}{)}
\PYG{n}{uts\PYGZus{}al} \PYG{o}{=} \PYG{n}{np}\PYG{o}{.}\PYG{n}{max}\PYG{p}{(}\PYG{n}{stress\PYGZus{}al}\PYG{p}{)}
\end{sphinxVerbatim}

\end{sphinxuseclass}\end{sphinxVerbatimInput}

\end{sphinxuseclass}
\sphinxAtStartPar
Pro vykreslení dodáme ještě deformace, ve kterých nastává mez pevnosti

\begin{sphinxuseclass}{cell}\begin{sphinxVerbatimInput}

\begin{sphinxuseclass}{cell_input}
\begin{sphinxVerbatim}[commandchars=\\\{\}]
\PYG{n}{uts\PYGZus{}strain\PYGZus{}steel} \PYG{o}{=} \PYG{n}{strain\PYGZus{}steel}\PYG{p}{[}\PYG{n}{stress\PYGZus{}steel}\PYG{o}{==}\PYG{n}{uts\PYGZus{}steel}\PYG{p}{]}\PYG{p}{[}\PYG{l+m+mi}{0}\PYG{p}{]}
\PYG{n}{uts\PYGZus{}strain\PYGZus{}al} \PYG{o}{=} \PYG{n}{strain\PYGZus{}al}\PYG{p}{[}\PYG{n}{stress\PYGZus{}al}\PYG{o}{==}\PYG{n}{uts\PYGZus{}al}\PYG{p}{]}\PYG{p}{[}\PYG{l+m+mi}{0}\PYG{p}{]}



\PYG{n+nb}{print}\PYG{p}{(}\PYG{l+s+sa}{f}\PYG{l+s+s1}{\PYGZsq{}}\PYG{l+s+s1}{Mez pevnosti pro hliník 6061 je }\PYG{l+s+si}{\PYGZob{}}\PYG{n}{uts\PYGZus{}al}\PYG{l+s+si}{\PYGZcb{}}\PYG{l+s+s1}{ MPa a nastává při poměrném prodloužení }\PYG{l+s+si}{\PYGZob{}}\PYG{n}{uts\PYGZus{}strain\PYGZus{}al}\PYG{l+s+si}{\PYGZcb{}}\PYG{l+s+s1}{.}\PYG{l+s+s1}{\PYGZsq{}}\PYG{p}{)}
\PYG{n+nb}{print}\PYG{p}{(}\PYG{l+s+sa}{f}\PYG{l+s+s1}{\PYGZsq{}}\PYG{l+s+s1}{Mez pevnosti pro ocel 1018 je }\PYG{l+s+si}{\PYGZob{}}\PYG{n}{uts\PYGZus{}steel}\PYG{l+s+si}{\PYGZcb{}}\PYG{l+s+s1}{ MPa a nastává při poměrném prodloužení }\PYG{l+s+si}{\PYGZob{}}\PYG{n}{uts\PYGZus{}strain\PYGZus{}steel}\PYG{l+s+si}{\PYGZcb{}}\PYG{l+s+s1}{.}\PYG{l+s+s1}{\PYGZsq{}}\PYG{p}{)}
\end{sphinxVerbatim}

\end{sphinxuseclass}\end{sphinxVerbatimInput}
\begin{sphinxVerbatimOutput}

\begin{sphinxuseclass}{cell_output}
\begin{sphinxVerbatim}[commandchars=\\\{\}]
Mez pevnosti pro hliník 6061 je 328.87169171951257 MPa a nastává při poměrném prodloužení 0.10797487258911133.
Mez pevnosti pro ocel 1018 je 663.4758721964481 MPa a nastává při poměrném prodloužení 0.12940356254577637.
\end{sphinxVerbatim}

\end{sphinxuseclass}\end{sphinxVerbatimOutput}

\end{sphinxuseclass}
\sphinxAtStartPar
Body si můžeme přidat do grafu.

\begin{sphinxuseclass}{cell}\begin{sphinxVerbatimInput}

\begin{sphinxuseclass}{cell_input}
\begin{sphinxVerbatim}[commandchars=\\\{\}]
\PYG{n}{fig}\PYG{p}{,}\PYG{n}{ax} \PYG{o}{=} \PYG{n}{plt}\PYG{o}{.}\PYG{n}{subplots}\PYG{p}{(}\PYG{p}{)}

\PYG{n}{ax}\PYG{o}{.}\PYG{n}{plot}\PYG{p}{(}\PYG{n}{strain\PYGZus{}al}\PYG{p}{,} \PYG{n}{stress\PYGZus{}al}\PYG{p}{,} \PYG{n}{label} \PYG{o}{=} \PYG{l+s+s1}{\PYGZsq{}}\PYG{l+s+s1}{Al6061}\PYG{l+s+s1}{\PYGZsq{}}\PYG{p}{)}
\PYG{n}{ax}\PYG{o}{.}\PYG{n}{plot}\PYG{p}{(}\PYG{n}{strain\PYGZus{}steel}\PYG{p}{,} \PYG{n}{stress\PYGZus{}steel}\PYG{p}{,} \PYG{n}{label} \PYG{o}{=} \PYG{l+s+s1}{\PYGZsq{}}\PYG{l+s+s1}{Steel1018}\PYG{l+s+s1}{\PYGZsq{}}\PYG{p}{)}

\PYG{n}{ax}\PYG{o}{.}\PYG{n}{plot}\PYG{p}{(}\PYG{n}{uts\PYGZus{}strain\PYGZus{}al}\PYG{p}{,} \PYG{n}{uts\PYGZus{}al}\PYG{p}{,} \PYG{l+s+s2}{\PYGZdq{}}\PYG{l+s+s2}{o}\PYG{l+s+s2}{\PYGZdq{}}\PYG{p}{,} \PYG{n}{label} \PYG{o}{=} \PYG{l+s+s1}{\PYGZsq{}}\PYG{l+s+s1}{\PYGZdl{}}\PYG{l+s+s1}{\PYGZbs{}}\PYG{l+s+s1}{sigma\PYGZus{}P\PYGZdl{} Al6061}\PYG{l+s+s1}{\PYGZsq{}}\PYG{p}{)}
\PYG{n}{ax}\PYG{o}{.}\PYG{n}{plot}\PYG{p}{(}\PYG{n}{uts\PYGZus{}strain\PYGZus{}steel}\PYG{p}{,} \PYG{n}{uts\PYGZus{}steel}\PYG{p}{,} \PYG{l+s+s2}{\PYGZdq{}}\PYG{l+s+s2}{o}\PYG{l+s+s2}{\PYGZdq{}}\PYG{p}{,} \PYG{n}{label} \PYG{o}{=} \PYG{l+s+s1}{\PYGZsq{}}\PYG{l+s+s1}{\PYGZdl{}}\PYG{l+s+s1}{\PYGZbs{}}\PYG{l+s+s1}{sigma\PYGZus{}P\PYGZdl{} ocel 1018}\PYG{l+s+s1}{\PYGZsq{}}\PYG{p}{)}


\PYG{n}{ax}\PYG{o}{.}\PYG{n}{set\PYGZus{}xlabel}\PYG{p}{(}\PYG{l+s+s1}{\PYGZsq{}}\PYG{l+s+s1}{Poměrná deformace (m/m)}\PYG{l+s+s1}{\PYGZsq{}}\PYG{p}{)}
\PYG{n}{ax}\PYG{o}{.}\PYG{n}{set\PYGZus{}ylabel}\PYG{p}{(}\PYG{l+s+s1}{\PYGZsq{}}\PYG{l+s+s1}{Napětí (MPa)}\PYG{l+s+s1}{\PYGZsq{}}\PYG{p}{)}
\PYG{n}{ax}\PYG{o}{.}\PYG{n}{set\PYGZus{}title}\PYG{p}{(}\PYG{l+s+s1}{\PYGZsq{}}\PYG{l+s+s1}{Smluvní tahový diagram  slitiny hliníku 6061 a oceli 1018}\PYG{l+s+s1}{\PYGZsq{}}\PYG{p}{)}
\PYG{n}{ax}\PYG{o}{.}\PYG{n}{legend}\PYG{p}{(}\PYG{p}{)}

\PYG{n}{plt}\PYG{o}{.}\PYG{n}{savefig}\PYG{p}{(}\PYG{l+s+s1}{\PYGZsq{}}\PYG{l+s+s1}{static/tahovy\PYGZus{}diagram.png}\PYG{l+s+s1}{\PYGZsq{}}\PYG{p}{,} \PYG{n}{dpi}\PYG{o}{=}\PYG{l+m+mi}{300}\PYG{p}{)}
\PYG{n}{plt}\PYG{o}{.}\PYG{n}{show}\PYG{p}{(}\PYG{p}{)}
\end{sphinxVerbatim}

\end{sphinxuseclass}\end{sphinxVerbatimInput}
\begin{sphinxVerbatimOutput}

\begin{sphinxuseclass}{cell_output}
\noindent\sphinxincludegraphics{{3f88991d1c9ffd3fef3595356c66f0b6809d64eeec348d3fa99bccd3367a6cbe}.png}

\end{sphinxuseclass}\end{sphinxVerbatimOutput}

\end{sphinxuseclass}

\section{Určení Youngova modulu pružnosti}
\label{\detokenize{Prednasky/2_7_P_u0159_xedklad _tahov_xe9_zkou_u0161ky:urceni-youngova-modulu-pruznosti}}
\sphinxAtStartPar
Zvolíme si část křivky, pro kterou je průběh lineárrní. Pro slitinu hliníku to bude přibližně oblast mezi 0 a 250 MPa a pro ocel mezi 0 a 350 MPa.

\begin{sphinxuseclass}{cell}\begin{sphinxVerbatimInput}

\begin{sphinxuseclass}{cell_input}
\begin{sphinxVerbatim}[commandchars=\\\{\}]
\PYG{n}{lin\PYGZus{}stress\PYGZus{}steel} \PYG{o}{=} \PYG{n}{stress\PYGZus{}steel}\PYG{p}{[}\PYG{n}{stress\PYGZus{}steel}\PYG{o}{\PYGZlt{}}\PYG{l+m+mi}{350}\PYG{p}{]}
\PYG{n}{lin\PYGZus{}strain\PYGZus{}steel} \PYG{o}{=} \PYG{n}{strain\PYGZus{}steel}\PYG{p}{[}\PYG{n}{stress\PYGZus{}steel}\PYG{o}{\PYGZlt{}}\PYG{l+m+mi}{350}\PYG{p}{]}

\PYG{n}{lin\PYGZus{}stress\PYGZus{}al} \PYG{o}{=} \PYG{n}{stress\PYGZus{}al}\PYG{p}{[}\PYG{n}{stress\PYGZus{}al}\PYG{o}{\PYGZlt{}}\PYG{l+m+mi}{250}\PYG{p}{]}
\PYG{n}{lin\PYGZus{}strain\PYGZus{}al} \PYG{o}{=} \PYG{n}{strain\PYGZus{}al}\PYG{p}{[}\PYG{n}{stress\PYGZus{}al}\PYG{o}{\PYGZlt{}}\PYG{l+m+mi}{250}\PYG{p}{]}

\PYG{n}{fig}\PYG{p}{,}\PYG{n}{ax} \PYG{o}{=} \PYG{n}{plt}\PYG{o}{.}\PYG{n}{subplots}\PYG{p}{(}\PYG{p}{)}

\PYG{n}{ax}\PYG{o}{.}\PYG{n}{plot}\PYG{p}{(}\PYG{n}{lin\PYGZus{}strain\PYGZus{}al}\PYG{p}{,} \PYG{n}{lin\PYGZus{}stress\PYGZus{}al}\PYG{p}{,} \PYG{n}{label} \PYG{o}{=} \PYG{l+s+s1}{\PYGZsq{}}\PYG{l+s+s1}{Al6061}\PYG{l+s+s1}{\PYGZsq{}}\PYG{p}{)}
\PYG{n}{ax}\PYG{o}{.}\PYG{n}{plot}\PYG{p}{(}\PYG{n}{lin\PYGZus{}strain\PYGZus{}steel}\PYG{p}{,} \PYG{n}{lin\PYGZus{}stress\PYGZus{}steel}\PYG{p}{,} \PYG{n}{label} \PYG{o}{=} \PYG{l+s+s1}{\PYGZsq{}}\PYG{l+s+s1}{Steel1018}\PYG{l+s+s1}{\PYGZsq{}}\PYG{p}{)}

\PYG{n}{ax}\PYG{o}{.}\PYG{n}{set\PYGZus{}xlabel}\PYG{p}{(}\PYG{l+s+s1}{\PYGZsq{}}\PYG{l+s+s1}{Poměrná deformace (m/m)}\PYG{l+s+s1}{\PYGZsq{}}\PYG{p}{)}
\PYG{n}{ax}\PYG{o}{.}\PYG{n}{set\PYGZus{}ylabel}\PYG{p}{(}\PYG{l+s+s1}{\PYGZsq{}}\PYG{l+s+s1}{Napětí (MPa)}\PYG{l+s+s1}{\PYGZsq{}}\PYG{p}{)}
\PYG{n}{ax}\PYG{o}{.}\PYG{n}{set\PYGZus{}title}\PYG{p}{(}\PYG{l+s+s1}{\PYGZsq{}}\PYG{l+s+s1}{Smluvní tahový diagram  slitiny hliníku 6061 a oceli 1018}\PYG{l+s+s1}{\PYGZsq{}}\PYG{p}{)}
\PYG{n}{ax}\PYG{o}{.}\PYG{n}{legend}\PYG{p}{(}\PYG{p}{)}
\end{sphinxVerbatim}

\end{sphinxuseclass}\end{sphinxVerbatimInput}
\begin{sphinxVerbatimOutput}

\begin{sphinxuseclass}{cell_output}
\begin{sphinxVerbatim}[commandchars=\\\{\}]
\PYGZlt{}matplotlib.legend.Legend at 0x74bff472cc10\PYGZgt{}
\end{sphinxVerbatim}

\noindent\sphinxincludegraphics{{393a5ddde4781d65e5f8a9c74e43e5a552246befcced642cf46c6253f21727e0}.png}

\end{sphinxuseclass}\end{sphinxVerbatimOutput}

\end{sphinxuseclass}
\sphinxAtStartPar
V jednoduchém přístupu použijeme funkci aproximace bodů polynomem prvního stupně.

\begin{sphinxuseclass}{cell}\begin{sphinxVerbatimInput}

\begin{sphinxuseclass}{cell_input}
\begin{sphinxVerbatim}[commandchars=\\\{\}]
\PYG{n}{param\PYGZus{}steel} \PYG{o}{=} \PYG{n}{np}\PYG{o}{.}\PYG{n}{polyfit}\PYG{p}{(}\PYG{n}{lin\PYGZus{}strain\PYGZus{}steel}\PYG{p}{,} \PYG{n}{lin\PYGZus{}stress\PYGZus{}steel}\PYG{p}{,} \PYG{l+m+mi}{1}\PYG{p}{)}
\PYG{n}{param\PYGZus{}al} \PYG{o}{=} \PYG{n}{np}\PYG{o}{.}\PYG{n}{polyfit}\PYG{p}{(}\PYG{n}{lin\PYGZus{}strain\PYGZus{}al}\PYG{p}{,} \PYG{n}{lin\PYGZus{}stress\PYGZus{}al}\PYG{p}{,} \PYG{l+m+mi}{1}\PYG{p}{)}

\PYG{n+nb}{print}\PYG{p}{(}\PYG{l+s+sa}{f}\PYG{l+s+s1}{\PYGZsq{}}\PYG{l+s+s1}{Ocel 1018: Napětí = }\PYG{l+s+si}{\PYGZob{}}\PYG{n}{param\PYGZus{}steel}\PYG{p}{[}\PYG{l+m+mi}{1}\PYG{p}{]}\PYG{l+s+si}{\PYGZcb{}}\PYG{l+s+s1}{ + }\PYG{l+s+si}{\PYGZob{}}\PYG{n}{param\PYGZus{}steel}\PYG{p}{[}\PYG{l+m+mi}{0}\PYG{p}{]}\PYG{l+s+si}{\PYGZcb{}}\PYG{l+s+s1}{ * deformace}\PYG{l+s+s1}{\PYGZsq{}}\PYG{p}{)}
\PYG{n+nb}{print}\PYG{p}{(}\PYG{l+s+sa}{f}\PYG{l+s+s1}{\PYGZsq{}}\PYG{l+s+s1}{Al6061: Napětí = }\PYG{l+s+si}{\PYGZob{}}\PYG{n}{param\PYGZus{}al}\PYG{p}{[}\PYG{l+m+mi}{1}\PYG{p}{]}\PYG{l+s+si}{\PYGZcb{}}\PYG{l+s+s1}{ + }\PYG{l+s+si}{\PYGZob{}}\PYG{n}{param\PYGZus{}al}\PYG{p}{[}\PYG{l+m+mi}{0}\PYG{p}{]}\PYG{l+s+si}{\PYGZcb{}}\PYG{l+s+s1}{ * deformace}\PYG{l+s+s1}{\PYGZsq{}}\PYG{p}{)}
\end{sphinxVerbatim}

\end{sphinxuseclass}\end{sphinxVerbatimInput}
\begin{sphinxVerbatimOutput}

\begin{sphinxuseclass}{cell_output}
\begin{sphinxVerbatim}[commandchars=\\\{\}]
Ocel 1018: Napětí = 30.353246542441244 + 267660.33488346846 * deformace
Al6061: Napětí = \PYGZhy{}13.32484752659266 + 68493.0090973414 * deformace
\end{sphinxVerbatim}

\end{sphinxuseclass}\end{sphinxVerbatimOutput}

\end{sphinxuseclass}
\sphinxAtStartPar
A zpětné získáme fitovaná data.

\begin{sphinxuseclass}{cell}\begin{sphinxVerbatimInput}

\begin{sphinxuseclass}{cell_input}
\begin{sphinxVerbatim}[commandchars=\\\{\}]
\PYG{n}{fit\PYGZus{}stress\PYGZus{}steel} \PYG{o}{=} \PYG{n}{np}\PYG{o}{.}\PYG{n}{polyval}\PYG{p}{(}\PYG{n}{param\PYGZus{}steel}\PYG{p}{,} \PYG{n}{lin\PYGZus{}strain\PYGZus{}steel}\PYG{p}{)}
\PYG{n}{fit\PYGZus{}stress\PYGZus{}al} \PYG{o}{=} \PYG{n}{np}\PYG{o}{.}\PYG{n}{polyval}\PYG{p}{(}\PYG{n}{param\PYGZus{}al}\PYG{p}{,} \PYG{n}{lin\PYGZus{}strain\PYGZus{}al}\PYG{p}{)}

\PYG{n}{fig}\PYG{p}{,}\PYG{n}{ax} \PYG{o}{=} \PYG{n}{plt}\PYG{o}{.}\PYG{n}{subplots}\PYG{p}{(}\PYG{p}{)}

\PYG{n}{ax}\PYG{o}{.}\PYG{n}{plot}\PYG{p}{(}\PYG{n}{lin\PYGZus{}strain\PYGZus{}al}\PYG{p}{,} \PYG{n}{lin\PYGZus{}stress\PYGZus{}al}\PYG{p}{,} \PYG{n}{label} \PYG{o}{=} \PYG{l+s+s1}{\PYGZsq{}}\PYG{l+s+s1}{Al6061}\PYG{l+s+s1}{\PYGZsq{}}\PYG{p}{)}
\PYG{n}{ax}\PYG{o}{.}\PYG{n}{plot}\PYG{p}{(}\PYG{n}{lin\PYGZus{}strain\PYGZus{}steel}\PYG{p}{,} \PYG{n}{lin\PYGZus{}stress\PYGZus{}steel}\PYG{p}{,} \PYG{n}{label} \PYG{o}{=} \PYG{l+s+s1}{\PYGZsq{}}\PYG{l+s+s1}{Steel1018}\PYG{l+s+s1}{\PYGZsq{}}\PYG{p}{)}
\PYG{n}{ax}\PYG{o}{.}\PYG{n}{plot}\PYG{p}{(}\PYG{n}{lin\PYGZus{}strain\PYGZus{}al}\PYG{p}{,} \PYG{n}{fit\PYGZus{}stress\PYGZus{}al}\PYG{p}{,}\PYG{l+s+s2}{\PYGZdq{}}\PYG{l+s+s2}{\PYGZhy{}\PYGZhy{}}\PYG{l+s+s2}{\PYGZdq{}} \PYG{p}{,}\PYG{n}{label} \PYG{o}{=} \PYG{l+s+s1}{\PYGZsq{}}\PYG{l+s+s1}{fit Al6061}\PYG{l+s+s1}{\PYGZsq{}}\PYG{p}{)}
\PYG{n}{ax}\PYG{o}{.}\PYG{n}{plot}\PYG{p}{(}\PYG{n}{lin\PYGZus{}strain\PYGZus{}steel}\PYG{p}{,} \PYG{n}{fit\PYGZus{}stress\PYGZus{}steel}\PYG{p}{,}\PYG{l+s+s2}{\PYGZdq{}}\PYG{l+s+s2}{\PYGZhy{}\PYGZhy{}}\PYG{l+s+s2}{\PYGZdq{}}\PYG{p}{,} \PYG{n}{label} \PYG{o}{=} \PYG{l+s+s1}{\PYGZsq{}}\PYG{l+s+s1}{fit Steel1018}\PYG{l+s+s1}{\PYGZsq{}}\PYG{p}{)}

\PYG{n}{ax}\PYG{o}{.}\PYG{n}{set\PYGZus{}xlabel}\PYG{p}{(}\PYG{l+s+s1}{\PYGZsq{}}\PYG{l+s+s1}{Poměrná deformace (m/m)}\PYG{l+s+s1}{\PYGZsq{}}\PYG{p}{)}
\PYG{n}{ax}\PYG{o}{.}\PYG{n}{set\PYGZus{}ylabel}\PYG{p}{(}\PYG{l+s+s1}{\PYGZsq{}}\PYG{l+s+s1}{Napětí (MPa)}\PYG{l+s+s1}{\PYGZsq{}}\PYG{p}{)}
\PYG{n}{ax}\PYG{o}{.}\PYG{n}{set\PYGZus{}title}\PYG{p}{(}\PYG{l+s+s1}{\PYGZsq{}}\PYG{l+s+s1}{Smluvní tahový diagram  slitiny hliníku 6061 a oceli 1018}\PYG{l+s+s1}{\PYGZsq{}}\PYG{p}{)}
\PYG{n}{ax}\PYG{o}{.}\PYG{n}{legend}\PYG{p}{(}\PYG{p}{)}
\end{sphinxVerbatim}

\end{sphinxuseclass}\end{sphinxVerbatimInput}
\begin{sphinxVerbatimOutput}

\begin{sphinxuseclass}{cell_output}
\begin{sphinxVerbatim}[commandchars=\\\{\}]
\PYGZlt{}matplotlib.legend.Legend at 0x74bff468e020\PYGZgt{}
\end{sphinxVerbatim}

\noindent\sphinxincludegraphics{{2692cf927593bb1eff8a6bbea050c5b23a5e734138f8f721865fc8158c138661}.png}

\end{sphinxuseclass}\end{sphinxVerbatimOutput}

\end{sphinxuseclass}
\sphinxAtStartPar
Sklon přímky nám udáva hodnotu Youngova modulu pružnosti.

\begin{sphinxuseclass}{cell}\begin{sphinxVerbatimInput}

\begin{sphinxuseclass}{cell_input}
\begin{sphinxVerbatim}[commandchars=\\\{\}]
\PYG{n+nb}{print}\PYG{p}{(}\PYG{l+s+sa}{f}\PYG{l+s+s1}{\PYGZsq{}}\PYG{l+s+s1}{E(ocel 1018) = }\PYG{l+s+si}{\PYGZob{}}\PYG{n}{param\PYGZus{}steel}\PYG{p}{[}\PYG{l+m+mi}{0}\PYG{p}{]}\PYG{o}{/}\PYG{l+m+mi}{1000}\PYG{l+s+si}{\PYGZcb{}}\PYG{l+s+s1}{ GPa}\PYG{l+s+s1}{\PYGZsq{}}\PYG{p}{)}
\PYG{n+nb}{print}\PYG{p}{(}\PYG{l+s+sa}{f}\PYG{l+s+s1}{\PYGZsq{}}\PYG{l+s+s1}{E(Al6061) = }\PYG{l+s+si}{\PYGZob{}}\PYG{n}{param\PYGZus{}al}\PYG{p}{[}\PYG{l+m+mi}{0}\PYG{p}{]}\PYG{o}{/}\PYG{l+m+mi}{1000}\PYG{l+s+si}{\PYGZcb{}}\PYG{l+s+s1}{ GPa}\PYG{l+s+s1}{\PYGZsq{}}\PYG{p}{)}
\end{sphinxVerbatim}

\end{sphinxuseclass}\end{sphinxVerbatimInput}
\begin{sphinxVerbatimOutput}

\begin{sphinxuseclass}{cell_output}
\begin{sphinxVerbatim}[commandchars=\\\{\}]
E(ocel 1018) = 267.66033488346847 GPa
E(Al6061) = 68.4930090973414 GPa
\end{sphinxVerbatim}

\end{sphinxuseclass}\end{sphinxVerbatimOutput}

\end{sphinxuseclass}
\sphinxAtStartPar
Výsledky můžeme srovnat s výsledky v tabulkách, např. \sphinxhref{https://www.matweb.com/search/datasheet\_print.aspx?matguid=3a9cc570fbb24d119f08db22a53e2421}{matweb.com} nebo \sphinxhref{https://asm.matweb.com/search/specificmaterial.asp?bassnum=ma6061t6}{asm.matweb.com}. Na základě našich měření a hodnot v tabulkách se můžeme pokusit odhadnout o jaký typ zpracování oceli 1018 se jedná.

\sphinxAtStartPar
Alternativou je využití složitějších modelů k fitování křivek, například knihovna \sphinxhref{https://lmfit.github.io/lmfit-py/index.html}{lmfit}.

\begin{sphinxuseclass}{cell}\begin{sphinxVerbatimInput}

\begin{sphinxuseclass}{cell_input}
\begin{sphinxVerbatim}[commandchars=\\\{\}]
\PYG{k}{def} \PYG{n+nf}{model\PYGZus{}hooke}\PYG{p}{(}\PYG{n}{epsilon}\PYG{p}{,} \PYG{n}{E}\PYG{p}{)}\PYG{p}{:}
    \PYG{k}{return} \PYG{n}{epsilon} \PYG{o}{*} \PYG{n}{E}

\PYG{n}{emodel} \PYG{o}{=} \PYG{n}{Model}\PYG{p}{(}\PYG{n}{model\PYGZus{}hooke}\PYG{p}{)}

\PYG{n}{result} \PYG{o}{=} \PYG{n}{emodel}\PYG{o}{.}\PYG{n}{fit}\PYG{p}{(}\PYG{n}{lin\PYGZus{}stress\PYGZus{}steel}\PYG{p}{,} \PYG{n}{epsilon}\PYG{o}{=}\PYG{n}{lin\PYGZus{}strain\PYGZus{}steel}\PYG{p}{,} \PYG{n}{E}\PYG{o}{=}\PYG{l+m+mi}{0}\PYG{o}{*}\PYG{l+m+mf}{1e3}\PYG{p}{)}

\PYG{n+nb}{print}\PYG{p}{(}\PYG{n}{result}\PYG{o}{.}\PYG{n}{fit\PYGZus{}report}\PYG{p}{(}\PYG{p}{)}\PYG{p}{)}

\PYG{n}{plt}\PYG{o}{.}\PYG{n}{plot}\PYG{p}{(}\PYG{n}{lin\PYGZus{}strain\PYGZus{}steel}\PYG{p}{,} \PYG{n}{lin\PYGZus{}stress\PYGZus{}steel}\PYG{p}{,} \PYG{l+s+s1}{\PYGZsq{}}\PYG{l+s+s1}{o}\PYG{l+s+s1}{\PYGZsq{}}\PYG{p}{)}
\PYG{n}{plt}\PYG{o}{.}\PYG{n}{plot}\PYG{p}{(}\PYG{n}{lin\PYGZus{}strain\PYGZus{}steel}\PYG{p}{,} \PYG{n}{result}\PYG{o}{.}\PYG{n}{init\PYGZus{}fit}\PYG{p}{,} \PYG{l+s+s1}{\PYGZsq{}}\PYG{l+s+s1}{\PYGZhy{}\PYGZhy{}}\PYG{l+s+s1}{\PYGZsq{}}\PYG{p}{,} \PYG{n}{label}\PYG{o}{=}\PYG{l+s+s1}{\PYGZsq{}}\PYG{l+s+s1}{initial fit}\PYG{l+s+s1}{\PYGZsq{}}\PYG{p}{)}
\PYG{n}{plt}\PYG{o}{.}\PYG{n}{plot}\PYG{p}{(}\PYG{n}{lin\PYGZus{}strain\PYGZus{}steel}\PYG{p}{,} \PYG{n}{result}\PYG{o}{.}\PYG{n}{best\PYGZus{}fit}\PYG{p}{,} \PYG{l+s+s1}{\PYGZsq{}}\PYG{l+s+s1}{\PYGZhy{}}\PYG{l+s+s1}{\PYGZsq{}}\PYG{p}{,} \PYG{n}{label}\PYG{o}{=}\PYG{l+s+s1}{\PYGZsq{}}\PYG{l+s+s1}{best fit}\PYG{l+s+s1}{\PYGZsq{}}\PYG{p}{)}
\PYG{n}{plt}\PYG{o}{.}\PYG{n}{legend}\PYG{p}{(}\PYG{p}{)}
\PYG{n}{plt}\PYG{o}{.}\PYG{n}{show}\PYG{p}{(}\PYG{p}{)}

\PYG{n+nb}{print}\PYG{p}{(}\PYG{l+s+sa}{f}\PYG{l+s+s1}{\PYGZsq{}}\PYG{l+s+s1}{E(ocel 1018) = }\PYG{l+s+si}{\PYGZob{}}\PYG{n}{result}\PYG{o}{.}\PYG{n}{params}\PYG{p}{[}\PYG{l+s+s2}{\PYGZdq{}}\PYG{l+s+s2}{E}\PYG{l+s+s2}{\PYGZdq{}}\PYG{p}{]}\PYG{o}{/}\PYG{l+m+mi}{1000}\PYG{l+s+si}{\PYGZcb{}}\PYG{l+s+s1}{ GPa}\PYG{l+s+s1}{\PYGZsq{}}\PYG{p}{)}
\end{sphinxVerbatim}

\end{sphinxuseclass}\end{sphinxVerbatimInput}
\begin{sphinxVerbatimOutput}

\begin{sphinxuseclass}{cell_output}
\begin{sphinxVerbatim}[commandchars=\\\{\}]
[[Model]]
    Model(model\PYGZus{}hooke)
[[Fit Statistics]]
    \PYGZsh{} fitting method   = leastsq
    \PYGZsh{} function evals   = 13
    \PYGZsh{} data points      = 91
    \PYGZsh{} variables        = 1
    chi\PYGZhy{}square         = 139374.949
    reduced chi\PYGZhy{}square = 1548.61055
    Akaike info crit   = 669.399783
    Bayesian info crit = 671.910643
    R\PYGZhy{}squared          = 0.84966458
[[Variables]]
    E:  307574.304 +/\PYGZhy{} 6556.48153 (2.13\PYGZpc{}) (init = 0)
\end{sphinxVerbatim}

\noindent\sphinxincludegraphics{{5b5a6ee2804b97e35782aed1ff0a1b9b69a987cda22d7f52b6e1d677a31eb1c8}.png}

\begin{sphinxVerbatim}[commandchars=\\\{\}]
E(ocel 1018) = 307.57430351056655 GPa
\end{sphinxVerbatim}

\end{sphinxuseclass}\end{sphinxVerbatimOutput}

\end{sphinxuseclass}
\sphinxstepscope


\section{Určení vnitřních sil a napětí \sphinxhyphen{} metoda řezun}
\label{\detokenize{Prednasky/2_5_Metoda__u0159ezu:urceni-vnitrnich-sil-a-napeti-metoda-rezun}}\label{\detokenize{Prednasky/2_5_Metoda__u0159ezu::doc}}
\begin{sphinxuseclass}{cell}\begin{sphinxVerbatimInput}

\begin{sphinxuseclass}{cell_input}
\begin{sphinxVerbatim}[commandchars=\\\{\}]
\PYG{k+kn}{from} \PYG{n+nn}{IPython}\PYG{n+nn}{.}\PYG{n+nn}{display} \PYG{k+kn}{import} \PYG{n}{YouTubeVideo}
\end{sphinxVerbatim}

\end{sphinxuseclass}\end{sphinxVerbatimInput}

\end{sphinxuseclass}

\subsection{Normálové síly (\protect\(N\protect\)) a tahové nebo tlakové napětí (\protect\(\sigma\protect\))}
\label{\detokenize{Prednasky/2_5_Metoda__u0159ezu:normalove-sily-n-a-tahove-nebo-tlakove-napeti-sigma}}
\sphinxAtStartPar
Součást námánou jenom sílami v 1D, v ose součásti označujeme jako \sphinxstylestrong{prut} nebo tyč. Průřez prutu může být kruhový (u tyče), nebo mít jiný tvar. Při působení vnějších sil vnějších axiálních sil (\(F\)) vznikají vv prutu nitřní síly \(N\) jako reakční síly.

\sphinxAtStartPar
Normálové (axiální) napětí \(\sigma\) lze obecně definovat poměrem:
\begin{equation*}
\begin{split}\sigma = \frac{N}{A}\end{split}
\end{equation*}
\sphinxAtStartPar
kde \(A\) je plochu příčného řezu {[}mm\(^2\){]}.

\sphinxAtStartPar
Protože síly a průřezy se v průběhu prutu mohou měnit, mění se také velikost vnitřních sil \(N\) a napětí \(\sigma\). Můžeme tedy napsat
\begin{equation*}
\begin{split}N=N(x), \sigma = \sigma(x) \end{split}
\end{equation*}
\sphinxAtStartPar
kde \(x\) je polojová souřadnice místa působení sil a napětí.


\subsection{Metoda řezu}
\label{\detokenize{Prednasky/2_5_Metoda__u0159ezu:metoda-rezu}}

\section{Metoda řezu pro určení vnitřních sil u prutů namáhaných na tah}
\label{\detokenize{Prednasky/2_5_Metoda__u0159ezu:metoda-rezu-pro-urceni-vnitrnich-sil-u-prutu-namahanych-na-tah}}
\sphinxAtStartPar
Metoda řezu je základní analytický nástroj v mechanice kontinua, který umožňuje určit vnitřní síly v prutu nebo nosníku. Pro prut namáhaný na tah se tato metoda aplikuje následujícím způsobem:
\begin{enumerate}
\sphinxsetlistlabels{\arabic}{enumi}{enumii}{}{.}%
\item {} 
\sphinxAtStartPar
\sphinxstylestrong{Identifikace zatížení}: nejprve se stanoví všechny vnější síly působící na prut, včetně aplikovaných sil a reakcí v podporách.

\item {} 
\sphinxAtStartPar
\sphinxstylestrong{Volba řezu}: prut se pomyslně rozřízne v místě, kde chceme určit vnitřní síly. Nejčastěji se volí řez v místě, kde dochází ke změně zatížení nebo geometrie průřezu. Polohu řezu určíme souřadnicí \(x\).

\item {} 
\sphinxAtStartPar
\sphinxstylestrong{Rozdělení na dvě části}: po provedení mysleného řezu vzniknou dvě oddělené části prutu. Každá část musí být v rovnováze, což znamená, že na řezné ploše se objeví vnitřní síla, která udržuje rovnováhu s vnějšími silami.

\item {} 
\sphinxAtStartPar
\sphinxstylestrong{Aplikace rovnic rovnováhy}: síla působící na jednu část prutu musí být vyrovnána silou působící na druhou část. Pro tahové a tlakové síly v axiálním směru platí:

\end{enumerate}
\begin{equation*}
\begin{split}
\vec{N}(x) + \sum\limits_i\vec{F}_i = \vec{0}
\end{split}
\end{equation*}
\sphinxAtStartPar
kde:
\begin{itemize}
\item {} 
\sphinxAtStartPar
\(N\) je normálová síla v řezu,

\item {} 
\sphinxAtStartPar
\(F_i\) jsou normálové síly působící na yvolněnou část řezu.

\end{itemize}

\noindent\sphinxincludegraphics{{images}.jpg}
\begin{enumerate}
\sphinxsetlistlabels{\arabic}{enumi}{enumii}{}{.}%
\setcounter{enumi}{4}
\item {} 
\sphinxAtStartPar
\sphinxstylestrong{Znaménková konvence}
\begin{itemize}
\item {} 
\sphinxAtStartPar
Tahová normálová síla  nebo napětí jsou považovány za \sphinxstylestrong{kladné}, pokud prut v daném řezu prodlužuje.

\item {} 
\sphinxAtStartPar
Tlaková normálová síla nebo napětí jsou \sphinxstylestrong{záporná}, pokud prut v daném řezu zkracuje.

\end{itemize}

\item {} 
\sphinxAtStartPar
\sphinxstylestrong{Vyhodnocení vnitřních sil}\\
Pokud je v celém prutu stejné zatížení a konstantní průřez, normálová síla \(N\) bude konstantní.\\
V případě proměnného zatížení nebo změny průřezu je nutné provést více řezů a analyzovat jednotlivé úseky.

\item {} 
\sphinxAtStartPar
\sphinxstylestrong{Vyhodnocení vnitřních napětí} určíme podělením vnitřní síly \(N(x)\) plochou průřezu \(A(x)\) v daném místě.
\begin{equation*}
\begin{split}\sigma(x) = \frac{N(x)}{A(x)} \end{split}
\end{equation*}
\end{enumerate}


\subsection{Změna délky}
\label{\detokenize{Prednasky/2_5_Metoda__u0159ezu:zmena-delky}}
\sphinxAtStartPar
Poměrná změna délky v místě řezu je určena působícím napětím podle Hookeova zákona
\begin{equation*}
\begin{split}\varepsilon(x) = \frac{\sigma(x)}{E(x)} \end{split}
\end{equation*}
\sphinxAtStartPar
Platí, že změnu celkové délky je možné získat integrovaním poměrné deformace po délce prutu
\begin{equation*}
\begin{split}\Delta l = \int\limits_0^L \varepsilon(x) \mathrm{d}x = \int\limits_0^L  \frac{\sigma(x)}{E(x)} \mathrm{d}x = \int\limits_0^L  \frac{N(x)}{A(x)E(x)} \mathrm{d}x\end{split}
\end{equation*}
\sphinxAtStartPar
Má\sphinxhyphen{}li homogenní izotropní tyč konstantního průřezu \(A\) a  původní délky \(L\) stejnou vnitřní sílou \(N\) platí
\begin{equation*}
\begin{split}
\Delta L = \frac{\sigma L}{E} = \frac{N L}{E A}
\end{split}
\end{equation*}
\sphinxAtStartPar
V případě tyče se skokovou (nespojitou) změnou průřezu nebo při působení izolovaných sil platí:
\begin{equation*}
\begin{split}
\Delta L = \sum\limits_i \Delta L_i = \sum\limits_i \frac{N_i L_i}{E_i A_i}
\end{split}
\end{equation*}
\sphinxAtStartPar
kde index \(i\) značí úseky, který mají konstatní vnitřní sílu, průřez nebo materiál.


\subsection{Další studijní materiály}
\label{\detokenize{Prednasky/2_5_Metoda__u0159ezu:dalsi-studijni-materialy}}
\begin{sphinxuseclass}{cell}\begin{sphinxVerbatimInput}

\begin{sphinxuseclass}{cell_input}
\begin{sphinxVerbatim}[commandchars=\\\{\}]
\PYG{n}{YouTubeVideo}\PYG{p}{(}\PYG{l+s+s1}{\PYGZsq{}}\PYG{l+s+s1}{KUBMgkmXuIw}\PYG{l+s+s1}{\PYGZsq{}}\PYG{p}{,} \PYG{n}{width}\PYG{o}{=}\PYG{l+m+mi}{800}\PYG{p}{)}
\end{sphinxVerbatim}

\end{sphinxuseclass}\end{sphinxVerbatimInput}
\begin{sphinxVerbatimOutput}

\begin{sphinxuseclass}{cell_output}
\noindent\sphinxincludegraphics{{86c962da254c05394f618415aef25013169f5fc960f119795c16bb417da4e061}.jpg}

\end{sphinxuseclass}\end{sphinxVerbatimOutput}

\end{sphinxuseclass}\begin{itemize}
\item {} 
\sphinxAtStartPar
\sphinxhref{https://mi21.vsb.cz/sites/mi21.vsb.cz/files/unit/pruznost\_pevnost\_obraz.pdf}{Radim Halama, Pružnost a pevnost \sphinxhyphen{} interaktivní studijní materiál Vysoká škola báňská – Technická univerzita Ostrava Západočeská univerzita v Plzni}

\end{itemize}

\sphinxstepscope


\section{Skutečné a smluvní napětí a deformace}
\label{\detokenize{Prednasky/2_6_In_u017een_xfdrsk_xe9_a_skute_u010dn_xe9_nap_u011bt_xed:skutecne-a-smluvni-napeti-a-deformace}}\label{\detokenize{Prednasky/2_6_In_u017een_xfdrsk_xe9_a_skute_u010dn_xe9_nap_u011bt_xed::doc}}
\begin{sphinxuseclass}{cell}\begin{sphinxVerbatimInput}

\begin{sphinxuseclass}{cell_input}
\begin{sphinxVerbatim}[commandchars=\\\{\}]
\PYG{k+kn}{import} \PYG{n+nn}{numpy} \PYG{k}{as} \PYG{n+nn}{np}
\PYG{k+kn}{import} \PYG{n+nn}{matplotlib}\PYG{n+nn}{.}\PYG{n+nn}{pyplot} \PYG{k}{as} \PYG{n+nn}{plt}
\end{sphinxVerbatim}

\end{sphinxuseclass}\end{sphinxVerbatimInput}

\end{sphinxuseclass}
\sphinxAtStartPar
V technické mechanice a pevnostní analýze se \sphinxstylestrong{smluvní napětí} a \sphinxstylestrong{smluvní deformace} používají pro popis materiálového chování. Tyto veličiny se zavádějí zejména v souvislosti s tažnými zkouškami a analýzou materiálových vlastností.

\begin{sphinxuseclass}{{image}https://yasincapar.com/wp-content/uploads/2020/10/Untitled-Project2.gif}
\end{sphinxuseclass}

\subsection{Smluvní napětí (engineering stress)}
\label{\detokenize{Prednasky/2_6_In_u017een_xfdrsk_xe9_a_skute_u010dn_xe9_nap_u011bt_xed:smluvni-napeti-engineering-stress}}\begin{quote}

\sphinxAtStartPar
Smluvní napětí \(\sigma_s\) se definuje jako podíl aktuální síly \(F\) a \sphinxstylestrong{počáteční plochy příčného řezu} \(A_0\) před deformací:
\end{quote}
\begin{equation*}
\begin{split}
\sigma_s = \frac{F}{S_0}
\end{split}
\end{equation*}
\sphinxAtStartPar
kde:
\begin{itemize}
\item {} 
\sphinxAtStartPar
\(F\) je působící síla {[}N\}{]},

\item {} 
\sphinxAtStartPar
\(A_0\) je počáteční plocha průřezu před deformací {[}m{]}.

\end{itemize}

\sphinxAtStartPar
Tento přístup zjednodušuje výpočty, protože při velkých plastických deformacích dochází k výrazné změně průřezu, což by komplikovalo výpočet skutečného napětí.


\subsubsection{Rozdíl mezi smluvním a skutečným napětím}
\label{\detokenize{Prednasky/2_6_In_u017een_xfdrsk_xe9_a_skute_u010dn_xe9_nap_u011bt_xed:rozdil-mezi-smluvnim-a-skutecnym-napetim}}\begin{itemize}
\item {} 
\sphinxAtStartPar
\sphinxstylestrong{Smluvní napětí} – počítá s původním průřezem \(A_0\), což je běžné v inženýrských výpočtech.

\item {} 
\sphinxAtStartPar
\sphinxstylestrong{Skutečné napětí (true stress)} – vychází z aktuální plochy průřezu \(S\) v daném okamžiku deformace, což je přesnější, ale složitější na výpočet:

\end{itemize}
\begin{equation*}
\begin{split}
\sigma_t = \frac{F}{A}
\end{split}
\end{equation*}
\sphinxAtStartPar
kde \(A\) je skutečná plocha průřezu v okamžiku měření.


\subsection{Smluvní deformace  (engineering strain)}
\label{\detokenize{Prednasky/2_6_In_u017een_xfdrsk_xe9_a_skute_u010dn_xe9_nap_u011bt_xed:smluvni-deformace-engineering-strain}}\begin{quote}

\sphinxAtStartPar
Smluvní (technická) deformace \((\varepsilon_s)\) se definuje jako podíl prodloužení délky a původní délky vzorku:
\end{quote}
\begin{equation*}
\begin{split}
\varepsilon_s = \frac{\Delta L}{L_0}
\end{split}
\end{equation*}
\sphinxAtStartPar
kde:
\begin{itemize}
\item {} 
\sphinxAtStartPar
\(\Delta L = L - L_0\) je změna délky {[}m{]},

\item {} 
\sphinxAtStartPar
\(L_0\) je původní délka vzorku {[}m{]}).

\end{itemize}

\begin{sphinxuseclass}{{image}https://yasincapar.com/wp-content/uploads/2020/10/11-2.png}
\end{sphinxuseclass}

\subsubsection{Rozdíl mezi smluvní a skutečnou deformací}
\label{\detokenize{Prednasky/2_6_In_u017een_xfdrsk_xe9_a_skute_u010dn_xe9_nap_u011bt_xed:rozdil-mezi-smluvni-a-skutecnou-deformaci}}\begin{itemize}
\item {} 
\sphinxAtStartPar
\sphinxstylestrong{Smluvní deformace} – používá počáteční délku \(L_0\), což zjednodušuje výpočty.

\item {} 
\sphinxAtStartPar
\sphinxstylestrong{Skutečná (logaritmická) deformace (true strain)} – bere v úvahu okamžitou délku (L), což je přesnější zejména pro velké deformace

\end{itemize}


\subsection{Odvození vztahu logaritmické deformace}
\label{\detokenize{Prednasky/2_6_In_u017een_xfdrsk_xe9_a_skute_u010dn_xe9_nap_u011bt_xed:odvozeni-vztahu-logaritmicke-deformace}}
\sphinxAtStartPar
Vyjdeme z definice infinitesimálního přírůstku podélné deformace:
\begin{equation*}
\begin{split}
d\varepsilon_t = \frac{dL}{L}
\end{split}
\end{equation*}
\sphinxAtStartPar
Integrací této rovnice v mezích od (L\_0) (počáteční délka) po (L) (aktuální délka) získáme:
\begin{equation*}
\begin{split}
\varepsilon_t = \int_{L_0}^{L} \frac{dL}{L}
\end{split}
\end{equation*}
\sphinxAtStartPar
Tento integrál je elementární a \sphinxstylestrong{logaritmická deformace} je dána vztahem:
\begin{equation*}
\begin{split}
\varepsilon_t = \ln{\left( \frac{L}{L_0} \right)}
\end{split}
\end{equation*}

\subsubsection{Vztah mezi logaritmickou a smluvní deformací}
\label{\detokenize{Prednasky/2_6_In_u017een_xfdrsk_xe9_a_skute_u010dn_xe9_nap_u011bt_xed:vztah-mezi-logaritmickou-a-smluvni-deformaci}}
\sphinxAtStartPar
Pro malé deformace (\(\varepsilon_s \ll 1\)) se logaritmická deformace přibližně rovná smluvní deformaci:
\begin{equation*}
\begin{split}
\ln(1 + \varepsilon_s) \approx \varepsilon_s, \quad \text{pro } \varepsilon_s \ll 1
\end{split}
\end{equation*}
\sphinxAtStartPar
Pro velké deformace však platí:
\begin{equation*}
\begin{split}
\varepsilon_t = \ln{\frac{L}{L_0}} = \ln\left({\frac{L0}{L_0} + \frac{L-L0}{L_0}}\right) =  \ln(1 + \varepsilon_s)
\end{split}
\end{equation*}
\begin{sphinxadmonition}{caution}{Caution:}
\sphinxAtStartPar
Logaritmická deformace je vždy o něco menší než smluvní deformace při větších hodnotách \(\varepsilon_s\).
\end{sphinxadmonition}

\begin{sphinxuseclass}{cell}\begin{sphinxVerbatimInput}

\begin{sphinxuseclass}{cell_input}
\begin{sphinxVerbatim}[commandchars=\\\{\}]
\PYG{n}{L0} \PYG{o}{=} \PYG{l+m+mi}{1}
\PYG{n}{delta\PYGZus{}L} \PYG{o}{=} \PYG{l+m+mf}{0.5}
\PYG{n}{L} \PYG{o}{=} \PYG{n}{np}\PYG{o}{.}\PYG{n}{linspace}\PYG{p}{(}\PYG{n}{L0}\PYG{p}{,}\PYG{n}{L0} \PYG{o}{+} \PYG{n}{delta\PYGZus{}L}\PYG{p}{,}\PYG{l+m+mi}{100}\PYG{p}{)}

\PYG{n}{epsilon\PYGZus{}s} \PYG{o}{=} \PYG{p}{(}\PYG{n}{L} \PYG{o}{\PYGZhy{}} \PYG{n}{L0}\PYG{p}{)}\PYG{o}{/}\PYG{n}{L0}
\PYG{n}{epsilon\PYGZus{}t} \PYG{o}{=} \PYG{n}{np}\PYG{o}{.}\PYG{n}{log}\PYG{p}{(}\PYG{n}{L}\PYG{o}{/}\PYG{n}{L0}\PYG{p}{)}
\PYG{c+c1}{\PYGZsh{} pro numpy log označuje přirozený logaritmus}

\PYG{n}{plt}\PYG{o}{.}\PYG{n}{plot}\PYG{p}{(}\PYG{n}{L}\PYG{p}{,} \PYG{n}{epsilon\PYGZus{}s}\PYG{p}{,} \PYG{n}{label} \PYG{o}{=} \PYG{l+s+s2}{\PYGZdq{}}\PYG{l+s+s2}{smluvní deformace}\PYG{l+s+s2}{\PYGZdq{}}\PYG{p}{)}
\PYG{n}{plt}\PYG{o}{.}\PYG{n}{plot}\PYG{p}{(}\PYG{n}{L}\PYG{p}{,} \PYG{n}{epsilon\PYGZus{}t}\PYG{p}{,} \PYG{n}{label} \PYG{o}{=} \PYG{l+s+s2}{\PYGZdq{}}\PYG{l+s+s2}{skutečná deformace}\PYG{l+s+s2}{\PYGZdq{}}\PYG{p}{)}
\PYG{n}{plt}\PYG{o}{.}\PYG{n}{xlabel}\PYG{p}{(}\PYG{l+s+s2}{\PYGZdq{}}\PYG{l+s+s2}{Délka vzorku (m)}\PYG{l+s+s2}{\PYGZdq{}}\PYG{p}{)}
\PYG{n}{plt}\PYG{o}{.}\PYG{n}{ylabel}\PYG{p}{(}\PYG{l+s+s2}{\PYGZdq{}}\PYG{l+s+s2}{Poměrná deformace [1]}\PYG{l+s+s2}{\PYGZdq{}}\PYG{p}{)}
\PYG{n}{plt}\PYG{o}{.}\PYG{n}{legend}\PYG{p}{(}\PYG{p}{)}
\PYG{n}{plt}\PYG{o}{.}\PYG{n}{show}\PYG{p}{(}\PYG{p}{)}
\end{sphinxVerbatim}

\end{sphinxuseclass}\end{sphinxVerbatimInput}
\begin{sphinxVerbatimOutput}

\begin{sphinxuseclass}{cell_output}
\noindent\sphinxincludegraphics{{279a751cf7a97f52622ba40f5ec3ed10bf781e46a3f188265369544183f53324}.png}

\end{sphinxuseclass}\end{sphinxVerbatimOutput}

\end{sphinxuseclass}
\sphinxstepscope


\section{Dimenzování součástí namáhaných v tahu a tlaku}
\label{\detokenize{Prednasky/2_8_Dimenzovani:dimenzovani-soucasti-namahanych-v-tahu-a-tlaku}}\label{\detokenize{Prednasky/2_8_Dimenzovani::doc}}
\sphinxAtStartPar
Dimenzování součástí namáhaných na tah a tlak se provádí tak, aby součást byla schopna bezpečně přenášet dané zatížení bez plastické deformace nebo lomu. Hlavními kritérii jsou \sphinxstylestrong{mechanická pevnost materiálu}, \sphinxstylestrong{koeficient bezpečnosti} a \sphinxstylestrong{mezní stavy součásti}.


\subsection{Principy dimenzování}
\label{\detokenize{Prednasky/2_8_Dimenzovani:principy-dimenzovani}}
\sphinxAtStartPar
Při dimenzování součástí se používají dvě základní metody:
\begin{enumerate}
\sphinxsetlistlabels{\arabic}{enumi}{enumii}{}{.}%
\item {} 
\sphinxAtStartPar
\sphinxstylestrong{Dimenzování na zatížení} – určujeme maximální přípustnou sílu, kterou součást může přenášet.

\item {} 
\sphinxAtStartPar
\sphinxstylestrong{Dimenzování na průřez} – navrhujeme odpovídající průřez součásti tak, aby při dané síle nepřekročilo napětí přípustnou hodnotu.

\end{enumerate}


\subsubsection{\sphinxstylestrong{Dimenzování na zatížení}}
\label{\detokenize{Prednasky/2_8_Dimenzovani:dimenzovani-na-zatizeni}}
\sphinxAtStartPar
Dimenzování na sílu spočívá v určení maximální přípustné síly \(F_{\text{max}}\), kterou součást snese bez porušení.

\sphinxAtStartPar
Ze základního vztahu pro normálové napětí:
\begin{equation*}
\begin{split}
\sigma = \frac{N}{A}
\end{split}
\end{equation*}
\sphinxAtStartPar
kde:
\begin{itemize}
\item {} 
\sphinxAtStartPar
\(\sigma\) je normálové napětí {[}Pa{]},

\item {} 
\sphinxAtStartPar
\(N\) je působící vnitřní síla {[}N{]},

\item {} 
\sphinxAtStartPar
\(A\) je plocha průřezu součásti {[}m²{]}.

\end{itemize}

\sphinxAtStartPar
Maximální přípustná síla je dána vztahem:
\begin{equation*}
\begin{split}
N_{\text{max}} = \sigma_\text{dov} A
\end{split}
\end{equation*}
\sphinxAtStartPar
kde \(\sigma_{\text{dov}}\) je dovolené napětí materiálu, které závisí na pevnosti materiálu a koeficientu bezpečnosti.

\begin{sphinxadmonition}{caution}{Caution:}
\sphinxAtStartPar
V případě proměnného zatížení a průřezu v průběhu prutu musí podmínku pevnosti splnit všechny části.
\end{sphinxadmonition}


\subsubsection{\sphinxstylestrong{Dimenzování na průřez}}
\label{\detokenize{Prednasky/2_8_Dimenzovani:dimenzovani-na-prurez}}
\sphinxAtStartPar
Při dimenzování na průřez určujeme potřebnou plochu průřezu \(A_{\text{min}}\), která zajistí, že při působení dané síly nebude překročeno dovolené napětí:
\begin{equation*}
\begin{split}
A_{\text{min}} = \frac{N}{\sigma_{\text{dov}}}
\end{split}
\end{equation*}
\sphinxAtStartPar
Zvolený průřez musí být dostatečně velký, aby součást byla bezpečně namáhána.


\subsection{Koeficient bezpečnosti}
\label{\detokenize{Prednasky/2_8_Dimenzovani:koeficient-bezpecnosti}}\begin{quote}

\sphinxAtStartPar
Koeficient bezpečnosti \(K\) je poměr mezi mezní pevností materiálu a skutečným nebo dovoleným napětím:
\begin{equation*}
\begin{split}K = \frac{\sigma_{\text{mez}}}{\sigma_{\text{dov}}}\end{split}
\end{equation*}\end{quote}

\sphinxAtStartPar
kde:
\begin{itemize}
\item {} 
\sphinxAtStartPar
\(\sigma_{\text{mez}}\) je pevnostní mez materiálu (mez kluzu nebo mez pevnosti),

\item {} 
\sphinxAtStartPar
\(\sigma_{\text{dov}}\) je dovolené napětí, při kterém by nemělo dojít k porušení součásti.

\end{itemize}

\sphinxAtStartPar
Koeficient bezpečnosti zohledňuje nejistoty v návrhu, například:
\begin{itemize}
\item {} 
\sphinxAtStartPar
výrobní tolerance,

\item {} 
\sphinxAtStartPar
variabilitu materiálových vlastností,

\item {} 
\sphinxAtStartPar
dynamické zatížení a únavu materiálu,

\item {} 
\sphinxAtStartPar
bezpečnostní požadavky v provozu.

\end{itemize}

\sphinxAtStartPar
Dovolené napětí se stanovuje podle vztahu:
\begin{equation*}
\begin{split}
\sigma_{\text{dov}} = \frac{\sigma_{\text{mez}}}{K}
\end{split}
\end{equation*}
\begin{sphinxadmonition}{note}{Note:}
\sphinxAtStartPar
Čím vyšší je koeficient bezpečnosti, tím je návrh konzervativnější, ale zároveň vede k robustnějším a těžším součástem.
\end{sphinxadmonition}

\sphinxstepscope


\part{Cvičení}

\sphinxstepscope


\chapter{1. cvičení}
\label{\detokenize{Cviceni/C1:cviceni}}\label{\detokenize{Cviceni/C1::doc}}

\section{Vektorová algebra}
\label{\detokenize{Cviceni/C1:vektorova-algebra}}\begin{enumerate}
\sphinxsetlistlabels{\arabic}{enumi}{enumii}{}{.}%
\item {} 
\sphinxAtStartPar
Vektor \(\vec{a}\) představuje posun 5,0 m na východ. Pokud vektor \(\vec{b}\) představuje 10,0 m posunutí na sever, najděte součet dvou posunutí (\(\vec{R}\)) a určete jeho velikost.

\item {} 
\sphinxAtStartPar
Určete složky \(x\) a \(y\) posunutí, jehož velikost je 30,0 m pod úhlem 23° od osy \(x\).

\item {} 
\sphinxAtStartPar
Na objekt působí dvě síly, ale v různých směrech. Například vy a váš přítel můžete oba tahat za provázky připevněné k jednomu dřevěnému bloku. Najděte velikost a směr výsledné síly za následujících okolností.
\begin{enumerate}
\sphinxsetlistlabels{\arabic}{enumii}{enumiii}{}{.}%
\item {} 
\sphinxAtStartPar
První síla má velikost 10 N a působí na východ. Druhá síla má velikost 4 N a působí na západ.

\item {} 
\sphinxAtStartPar
První síla má velikost 10 N a působí na východ. Druhá síla má velikost 4 N a působí na sever.

\end{enumerate}

\item {} 
\sphinxAtStartPar
Vektor \(\vec{c}\) představuje posun v metrech vyjádřený v jednotkovém vektorovém zápisu jako
\begin{equation*}
\begin{split}\vec{c} = 2 \vec{i} + 6 \vec{j} + 3 \vec{k}\end{split}
\end{equation*}
\sphinxAtStartPar
Vektor \(\vec{d}\) představuje druhé posunutí.
\begin{equation*}
\begin{split}\vec{d} = 5 \vec{i} - 3 \vec{j} - 2 \vec{k}\end{split}
\end{equation*}\begin{itemize}
\item {} 
\sphinxAtStartPar
Najděte skalární a vektorový součin dvou vektorů a úhel mezi nimi.

\end{itemize}

\item {} 
\sphinxAtStartPar
Dokažte, že pro vektor \(\vec{v}\) existují reálne čísla \(r,s\) a \(t\) tak ,aby \(\vec{v} = r \vec{a} + s \vec{b} + t (\vec{a}\times \vec{b})\) pro které platí
\begin{equation*}
\begin{split}r = \frac{(\vec{v}\cdot\vec{a})(\vec{b}\cdot\vec{b}) - (\vec{v}\cdot\vec{b})(\vec{a}\cdot\vec{b})}{||\vec{a}\times\vec{b}) ||^2}\end{split}
\end{equation*}\begin{equation*}
\begin{split}s = \frac{(\vec{v}\cdot\vec{b})(\vec{a}\cdot\vec{a}) - (\vec{v}\cdot\vec{a})(\vec{a}\cdot\vec{b})}{||\vec{a}\times\vec{b}) ||^2}\end{split}
\end{equation*}\begin{equation*}
\begin{split}t = \frac{\vec{v}\cdot(\vec{a}\times\vec{b})}{||\vec{a}\times\vec{b}) ||^2}\end{split}
\end{equation*}\begin{itemize}
\item {} 
\sphinxAtStartPar
Prokažte, že vektor \(r \vec{a} + s \vec{b} + t (\vec{a}\times \vec{b})- \vec{v}\) je kolmý na vektor \(\vec{a}\), \(\vec{b}\) a \(\vec{a}\times \vec{b}\)

\item {} 
\sphinxAtStartPar
Jaká je podmínka pro vektory \(\vec{a}\), \(\vec{b}\), aby platilo vyjádření \(\vec{v}\) pro všechny vektory v 3D.  (\sphinxhref{https://www.reddit.com/r/askmath/comments/18y599e/probably\_the\_most\_difficult\_vectors\_question\_in/}{\sphinxstylestrong{Řešení}})

\end{itemize}

\item {} 
\sphinxAtStartPar
Běžec vyběhne 200 stejných schodů na vrchol kopce a poté běží po vrcholu kopce 50,0 m, než se zastaví u fontánky na pití. Jeho vektor posunutí z bodu A na spodní části schodů do bodu B u fontány je \(\vec{d}_{AB}= (-90,0\vec{i}+30,0\vec{j})\textrm{m}\)\$. Jaká je výška a šířka každého kroku? Jaká je skutečná vzdálenost, kterou běžec urazí? Pokud udělá okruh a vrátí se do bodu A, jaký je jeho výsledný vektor posunutí?

\end{enumerate}

\sphinxAtStartPar
\sphinxincludegraphics{{CNX_UPhysics_02_03_jogger}.jpg}

\sphinxAtStartPar
\sphinxhref{https://pressbooks.online.ucf.edu/osuniversityphysics/chapter/2-3-algebra-of-vectors/}{\sphinxstylestrong{Řešení}}


\section{Kinematika pohybu}
\label{\detokenize{Cviceni/C1:kinematika-pohybu}}\begin{enumerate}
\sphinxsetlistlabels{\arabic}{enumi}{enumii}{}{.}%
\item {} 
\sphinxAtStartPar
Na obrázku jsou čtyři grafy polohy a času lineárních pohybů autíček A, B, C a D.
\begin{enumerate}
\sphinxsetlistlabels{\arabic}{enumii}{enumiii}{}{.}%
\item {} 
\sphinxAtStartPar
Popište slovně pohyb každého vozu.

\item {} 
\sphinxAtStartPar
Určete průměrnou rychlost každého vozu v intervalu od 0 s do 4 s.

\item {} 
\sphinxAtStartPar
Určete okamžitou rychlost každého vozu v čase t = 4 s.

\item {} 
\sphinxAtStartPar
Na jednom obrázku nakreslete grafy rychlosti a času pro každé auto.

\item {} 
\sphinxAtStartPar
Určete celkové posunutí v t1 = 10 s každého vozu (včetně počátečního nenulového posunutí, pokud existuje), za předpokladu, že pohyb pokračuje se stejnou časovou závislostí, jak je znázorněno na obrázku.

\end{enumerate}

\end{enumerate}

\sphinxAtStartPar


\sphinxAtStartPar
\sphinxhref{https://physicstasks.eu/1746/motion-given-by-a-motion-graph-i}{\sphinxstylestrong{Řešení}}
\begin{enumerate}
\sphinxsetlistlabels{\arabic}{enumi}{enumii}{}{.}%
\setcounter{enumi}{1}
\item {} 
\sphinxAtStartPar
Malý Honza vyrazil na kole po rovné silnici. Závislost zrychlení na čase je dána následující tabulkou:

\end{enumerate}


\begin{savenotes}\sphinxattablestart
\sphinxthistablewithglobalstyle
\centering
\begin{tabulary}{\linewidth}[t]{TTTTTTTTTTTT}
\sphinxtoprule
\sphinxstyletheadfamily 
\sphinxAtStartPar
\(t\) / \(10^{-1}\)s
&\sphinxstyletheadfamily 
\sphinxAtStartPar
0
&\sphinxstyletheadfamily 
\sphinxAtStartPar
1
&\sphinxstyletheadfamily 
\sphinxAtStartPar
2
&\sphinxstyletheadfamily 
\sphinxAtStartPar
3
&\sphinxstyletheadfamily 
\sphinxAtStartPar
4
&\sphinxstyletheadfamily 
\sphinxAtStartPar
5
&\sphinxstyletheadfamily 
\sphinxAtStartPar
6
&\sphinxstyletheadfamily 
\sphinxAtStartPar
7
&\sphinxstyletheadfamily 
\sphinxAtStartPar
8
&\sphinxstyletheadfamily 
\sphinxAtStartPar
9
&\sphinxstyletheadfamily 
\sphinxAtStartPar
10
\\
\sphinxmidrule
\sphinxtableatstartofbodyhook
\sphinxAtStartPar
\(a\) / ms\(^{-2}\)
&
\sphinxAtStartPar
4,0
&
\sphinxAtStartPar
3,7
&
\sphinxAtStartPar
3,3
&
\sphinxAtStartPar
3,0
&
\sphinxAtStartPar
2,6
&
\sphinxAtStartPar
2,3
&
\sphinxAtStartPar
1,9
&
\sphinxAtStartPar
1,6
&
\sphinxAtStartPar
1,2
&
\sphinxAtStartPar
0,8
&
\sphinxAtStartPar
0,4
\\
\sphinxbottomrule
\end{tabulary}
\sphinxtableafterendhook\par
\sphinxattableend\end{savenotes}
\begin{itemize}
\item {} 
\sphinxAtStartPar
Nakreslete graf a určete tvar funkce a(t). Určete z něj průběh Honzíkovy rychlosti a závislosti polohy na čase.

\item {} 
\sphinxAtStartPar
Předpokládejme, že se Honzík se nepohyboval v čase \(t = 0\) s a byl v počátku souřadnicového systému.

\end{itemize}

\sphinxAtStartPar
\sphinxhref{https://physicstasks.eu/2227/little-paulie}{\sphinxstylestrong{Řešení}}


\section{Dynamika pohybu}
\label{\detokenize{Cviceni/C1:dynamika-pohybu}}\begin{enumerate}
\sphinxsetlistlabels{\arabic}{enumi}{enumii}{}{.}%
\item {} 
\sphinxAtStartPar
Vozík o hmotnosti 250 g na vzduchové dráze je tažen a urychlován provázkem přes pevnou kladku. Porovnejte velikost zrychlení v následujících scénářích:
\begin{enumerate}
\sphinxsetlistlabels{\arabic}{enumii}{enumiii}{}{.}%
\item {} 
\sphinxAtStartPar
Provázek táhneme silou 0,1 N.

\item {} 
\sphinxAtStartPar
Na provázek zavěsíme závaží o tíze 0,1 N.

\end{enumerate}

\end{enumerate}

\sphinxAtStartPar
Zanedbávejte odpor vzduchu a hmotnosti struny a kladky.

\sphinxAtStartPar
Poznámka: Vzduchová dráha je zařízení, které umožňuje pohyb vozíku bez tření. Vozík je zavěšen nad dráhou pomocí vzduchových trysek

\sphinxAtStartPar


\sphinxAtStartPar
\sphinxhref{https://physicstasks.eu/1750/cart-on-an-air-track}{\sphinxstylestrong{Řešení}}
\begin{enumerate}
\sphinxsetlistlabels{\arabic}{enumi}{enumii}{}{.}%
\setcounter{enumi}{1}
\item {} 
\sphinxAtStartPar
Na kostku ledu o hmotnosti \(10 kg\) působí horizontální síla závislá na čase. Tato závislost je popsána vzorcem \(F_x = p(q − t)\) , kde \(p = 100\) Ns\(^{−1}\), \(q = 1\) s. V čase \(t = 0\) s byla kostka ledu na začátku naší vztažné soustavy, její rychlost byla 0,2 ms\(^{−1}\) a síla byla v souladu s její rychlostí. Kostka ledu se pohybuje po vodorovné ledové rovině bez tření.
\begin{enumerate}
\sphinxsetlistlabels{\arabic}{enumii}{enumiii}{}{.}%
\item {} 
\sphinxAtStartPar
Kdy a kde se kostka zastaví?

\item {} 
\sphinxAtStartPar
Předpokládejte, působící síla je konstantní \(F_x = p\) ale kostka se rozpouští a její hmotnost ubývá rychlostí 10 g s\(^(-1)\). Kdy a kde se kostka zastaví? Jaká bude celková dráha než se kostka úplně rozpustí?

\end{enumerate}

\end{enumerate}

\sphinxAtStartPar
\sphinxhref{https://physicstasks.eu/1980/ice-cube-and-time-varying-force}{\sphinxstylestrong{Řešení}}
\begin{enumerate}
\sphinxsetlistlabels{\arabic}{enumi}{enumii}{}{.}%
\setcounter{enumi}{2}
\item {} 
\sphinxAtStartPar
Do sklenice s medem vložíme ocelovou kuličku o hmotnosti m = 1g. Odporová síla \(F\), která působí na kuličku, je přímo úměrná její rychlosti.
\begin{enumerate}
\sphinxsetlistlabels{\arabic}{enumii}{enumiii}{}{)}%
\item {} 
\sphinxAtStartPar
Určete maximální rychlost \(v_max\), které může kulička dosáhnout.

\item {} 
\sphinxAtStartPar
Určete průběh velikosti rychlosti \(v(t)\) kuličky.

\end{enumerate}

\end{enumerate}

\sphinxAtStartPar


\sphinxAtStartPar
\sphinxhref{https://physicstasks.eu/652/a-ball-in-a-honey}{\sphinxstylestrong{Řešení}}
\begin{enumerate}
\sphinxsetlistlabels{\arabic}{enumi}{enumii}{}{.}%
\setcounter{enumi}{3}
\item {} 
\sphinxAtStartPar
Jakou dráhu urazí parašutista při seskoku než dosáhne maximální rychlost. Určete jeho maximální rychlost. Potřebné parametry odhadněte. Odoporovou sílu počítejte:
\$\(F_o = -\frac{1}{2}  \rho  C_d  A v^2\)\(
kde \)\textbackslash{}rho\( je hustota prostředí, \)C\_d\( je součinitel odporu, \)A\( je plocha tělesa a \)v\$ je rychlost tělesa.

\end{enumerate}

\sphinxAtStartPar
\sphinxincludegraphics{{440px-14ilf1l.svg}.png}
\begin{enumerate}
\sphinxsetlistlabels{\arabic}{enumi}{enumii}{}{.}%
\setcounter{enumi}{4}
\item {} 
\sphinxAtStartPar
Při úplném sešlápnutí brzdového pedálu vozu pohybujícího se rychlostí 80 kmh\(^{-1}\) po rovné asfaltové silnici může vůz zastavit za 50 m. Jak dlouhá bude jeho brzdná dráha na stejné silnici, která svírá s vodorovnou rovinou úhel 5°? Zvažte jak případ, kdy brzdí z kopce, tak i když brzdí do kopce. Předpokládejme, že v čase t = 0 s je souřadnice x = 0 m. Kola neprokluzují. Odpor vzduchu zanedbejte.

\end{enumerate}

\sphinxAtStartPar


\sphinxAtStartPar
\sphinxhref{https://physicstasks.eu/1984/braking-vehicle}{\sphinxstylestrong{Řešení}}


\section{Zákony zachování}
\label{\detokenize{Cviceni/C1:zakony-zachovani}}\begin{enumerate}
\sphinxsetlistlabels{\arabic}{enumi}{enumii}{}{.}%
\item {} 
\sphinxAtStartPar
Raketa ve vesmíru (bez působení gravitace) o hmotnosti \(m_r\)=2t je naložena palivem o hmotnosti \(m_p\)=12t. Raketa je poháněna raketovým motorem, jehož rychlost vystupujících plynů je 5000 kmh\(^{−1}\). Maximální povolené zrychlení posádky je 7\(G\), kde \(G\) znamená gravitační zrychlení.
\begin{enumerate}
\sphinxsetlistlabels{\arabic}{enumii}{enumiii}{}{.}%
\item {} 
\sphinxAtStartPar
Jaká je maximální možná spotřeba paliva za sekundu, jak dlouho může raketový motor při takové spotřebě pracovat a jaká je konečná rychlost rakety?

\item {} 
\sphinxAtStartPar
Záleží maximální dosažená rychlost na rychlosti spotřeby paliva?

\end{enumerate}

\end{enumerate}

\sphinxAtStartPar
\sphinxhref{https://physicstasks.eu/4318/rocket}{\sphinxstylestrong{Řesení}}
\begin{enumerate}
\sphinxsetlistlabels{\arabic}{enumi}{enumii}{}{.}%
\setcounter{enumi}{1}
\item {} 
\sphinxAtStartPar
Kulka o hmotnosti 10 g byla vypálena proti stacionárnímu dřevěnému bloku o hmotnosti 1 kg. Pronikl blokem do hloubky 10 cm. Kdyby byl dřevěný blok pohyblivý, jak hluboko by se střela dostala? Předpokládejme, že dřevo neustále odolává pohybu střely.

\end{enumerate}

\sphinxAtStartPar
\sphinxhref{https://physicstasks.eu/1994/bullet}{\sphinxstylestrong{Řešení}}

\sphinxstepscope


\chapter{2. cvičení}
\label{\detokenize{Cviceni/C2:cviceni}}\label{\detokenize{Cviceni/C2::doc}}

\section{Tuhost}
\label{\detokenize{Cviceni/C2:tuhost}}\begin{enumerate}
\sphinxsetlistlabels{\arabic}{enumi}{enumii}{}{.}%
\item {} 
\sphinxAtStartPar
Jakou silou je potřeba natáhnout ideální pružinu s konstantou pružiny 120 N/m pro dsažení prodloužení
30 cm?

\item {} 
\sphinxAtStartPar
Pružina s tuhostí 600 N/m se používá pro váhu k vážení ryb. Jaká je
hmotnost ryby, která by natáhla pružinu o 7,5 cm od její normální délky?

\item {} 
\sphinxAtStartPar
Pružina v pogo\sphinxhyphen{}sticku je stlačena o 12 cm, když na ní stojí 40 kg vážící dívka. Jaká je
tuhost pružiny pro pružinu pogo\sphinxhyphen{}stick?

\item {} 
\sphinxAtStartPar
Pružina se při působení síly 13 N natáhne o 8,0 cm. Jak daleko se táhne, když působí síla 26 N?

\item {} 
\sphinxAtStartPar
Elastická šňůra je dlouhá 80 cm, když na ní visí hmota 10 kg. Po přidání dalších 4,0 kg je šňůra dlouhá 82,5 cm. Jak velká je tuhost šňůry.

\item {} 
\sphinxAtStartPar
Pružina s tuhostí 50 N/m visí na stojanu. Druhá pružina s tuhostí 100. N/m visí z první pružiny. Jak daleko se táhnou, pokud je závaží
0,50 kg je zavěšeno na spodní pružině? Jaká je tuhost soustavy pružin? Jak se změní tuhost, když pružiny zavěsíme vedle sebe a jak velkou zátěž přenáší každá z pružin.

\end{enumerate}


\section{Tahový diagram}
\label{\detokenize{Cviceni/C2:tahovy-diagram}}\begin{enumerate}
\sphinxsetlistlabels{\arabic}{enumi}{enumii}{}{.}%
\item {} 
\sphinxAtStartPar
Z naměřených dat v experimentu na obrázku určete tuhost pružiny. Vysvětlete průběh naměřené křivky.

\noindent\sphinxincludegraphics{{316038_1}.png}

\item {} 
\sphinxAtStartPar
Při zkoušce různých slitin hliníku jsme získali následující smluvný tahový diagram. Popište
\begin{itemize}
\item {} 
\sphinxAtStartPar
která slitina vydrží největší napětí
\begin{itemize}
\item {} 
\sphinxAtStartPar
do porušení

\item {} 
\sphinxAtStartPar
do tvárné deformace

\end{itemize}

\item {} 
\sphinxAtStartPar
která slitina bude mít najvětší prodloužení

\item {} 
\sphinxAtStartPar
jak se liší Youngův modul pružnosti mezi slitinami

\end{itemize}

\end{enumerate}
\begin{enumerate}
\sphinxsetlistlabels{\arabic}{enumi}{enumii}{}{.}%
\setcounter{enumi}{2}
\item {} 
\sphinxAtStartPar
Během tahové zkoušky ocelové tyče o průměru 14 mm byly zaznamenány následující údaje. Měřená délka byla 50 mm.

\end{enumerate}


\begin{savenotes}\sphinxattablestart
\sphinxthistablewithglobalstyle
\centering
\begin{tabulary}{\linewidth}[t]{TT}
\sphinxtoprule
\sphinxstyletheadfamily 
\sphinxAtStartPar
Zatížení (N)
&\sphinxstyletheadfamily 
\sphinxAtStartPar
Prodloužení (mm)
\\
\sphinxmidrule
\sphinxtableatstartofbodyhook
\sphinxAtStartPar
0
&
\sphinxAtStartPar
0
\\
\sphinxhline
\sphinxAtStartPar
6310
&
\sphinxAtStartPar
0.010
\\
\sphinxhline
\sphinxAtStartPar
12600
&
\sphinxAtStartPar
0.020
\\
\sphinxhline
\sphinxAtStartPar
18800
&
\sphinxAtStartPar
0.030
\\
\sphinxhline
\sphinxAtStartPar
25100
&
\sphinxAtStartPar
0.040
\\
\sphinxhline
\sphinxAtStartPar
31300
&
\sphinxAtStartPar
0.050
\\
\sphinxhline
\sphinxAtStartPar
37900
&
\sphinxAtStartPar
0.060
\\
\sphinxhline
\sphinxAtStartPar
40100
&
\sphinxAtStartPar
0.163
\\
\sphinxhline
\sphinxAtStartPar
41600
&
\sphinxAtStartPar
0.433
\\
\sphinxhline
\sphinxAtStartPar
46200
&
\sphinxAtStartPar
1.25
\\
\sphinxhline
\sphinxAtStartPar
52400
&
\sphinxAtStartPar
2.50
\\
\sphinxhline
\sphinxAtStartPar
58500
&
\sphinxAtStartPar
4.50
\\
\sphinxhline
\sphinxAtStartPar
68000
&
\sphinxAtStartPar
7.50
\\
\sphinxhline
\sphinxAtStartPar
59000
&
\sphinxAtStartPar
12.5
\\
\sphinxhline
\sphinxAtStartPar
67800
&
\sphinxAtStartPar
15.5
\\
\sphinxhline
\sphinxAtStartPar
65000
&
\sphinxAtStartPar
20.0
\\
\sphinxhline
\sphinxAtStartPar
65500
&
\sphinxAtStartPar
porušení
\\
\sphinxbottomrule
\end{tabulary}
\sphinxtableafterendhook\par
\sphinxattableend\end{savenotes}

\sphinxAtStartPar
Určete:
\begin{itemize}
\item {} 
\sphinxAtStartPar
mez úmeřnosti

\item {} 
\sphinxAtStartPar
Youngův modul pružnosti

\item {} 
\sphinxAtStartPar
mez kluzu

\item {} 
\sphinxAtStartPar
mez pevnosti

\item {} 
\sphinxAtStartPar
hodnotu napětí při porušení

\end{itemize}

\sphinxAtStartPar
\sphinxhref{https://mathalino.com/reviewer/mechanics-and-strength-of-materials/solution-to-problem-203-stress-strain-diagram}{\sphinxstylestrong{Řešení}}


\section{Napětí a dimenzování}
\label{\detokenize{Cviceni/C2:napeti-a-dimenzovani}}\begin{enumerate}
\sphinxsetlistlabels{\arabic}{enumi}{enumii}{}{.}%
\item {} 
\sphinxAtStartPar
Socha o hmotnosti 10 000 N spočívá na vodorovném povrchu na vrcholu 6,0 m vysokého svislého pilíře. Plocha průřezu pilíře je 0,20 m\(^2\)  a je vyrobena ze žuly o hustotě 2700 kg/m\(^3\).
Najděte tlakové napětí v průřezu umístěném 3,0 m pod vrcholem pilíře a hodnotu tlakového přetvoření (relativního napětí) horního 3,0m segmentu pilíře. Výsledek uveďte včetně znaménka.

\end{enumerate}

\noindent\sphinxincludegraphics{{2f27384c79485f663fd3ec750beff00e2a7146f6}.jpg}

\sphinxAtStartPar
\sphinxhref{https://openstax.org/books/university-physics-volume-1/pages/12-3-stress-strain-and-elastic-modulus\#CNX\_UPhysics\_12\_03\_Nelson}{\sphinxstylestrong{Řešení}}
\begin{enumerate}
\sphinxsetlistlabels{\arabic}{enumi}{enumii}{}{.}%
\setcounter{enumi}{1}
\item {} 
\sphinxAtStartPar
Jakou nejtěžší sochu je možné umístnit na pylon pred Fakultou informačních technologií ČVUT v Dejvicích. Potřebné informace odhadněte nebo dohledejte. Informace o pylonu \sphinxhref{https://cs.wikipedia.org/wiki/Sloup\_\%E2\%80\%93\_obelisk}{tady}.

\end{enumerate}

\noindent\sphinxincludegraphics{{nova-budova-hq}.jpeg}
\begin{enumerate}
\sphinxsetlistlabels{\arabic}{enumi}{enumii}{}{.}%
\setcounter{enumi}{2}
\item {} 
\sphinxAtStartPar
Navrhněte minimální průměr ocelového prutu, který bude přenášet tahové zatížení 50 kN. Materiál prutu je nízkouhlíková ocel s mezí kluzu 235 MPa a bezpečnostním součinitelem 1,5.

\end{enumerate}


\section{Hooekův zákon}
\label{\detokenize{Cviceni/C2:hooekuv-zakon}}\begin{enumerate}
\sphinxsetlistlabels{\arabic}{enumi}{enumii}{}{.}%
\item {} 
\sphinxAtStartPar
Je dán jednostranně vetknutý prut sestavený ze 3 částí o různých rozměrech a mechanických vlastnostech (viz. obrázek dole) zatížený třemi silami \(F_1\), \(F_2\) a \(F_3\) a změnou teploty \(\Delta T\), která působí jen na části II a III.

\end{enumerate}


\begin{savenotes}\sphinxattablestart
\sphinxthistablewithglobalstyle
\centering
\begin{tabulary}{\linewidth}[t]{TTT}
\sphinxtoprule
\sphinxstyletheadfamily 
\sphinxAtStartPar
Parametr
&\sphinxstyletheadfamily 
\sphinxAtStartPar
Hodnota
&\sphinxstyletheadfamily 
\sphinxAtStartPar
Jednotka
\\
\sphinxmidrule
\sphinxtableatstartofbodyhook
\sphinxAtStartPar
\(F_1\)
&
\sphinxAtStartPar
25
&
\sphinxAtStartPar
kN
\\
\sphinxhline
\sphinxAtStartPar
\(F_2\)
&
\sphinxAtStartPar
10
&
\sphinxAtStartPar
kN
\\
\sphinxhline
\sphinxAtStartPar
\(F_3\)
&
\sphinxAtStartPar
15
&
\sphinxAtStartPar
kN
\\
\sphinxhline
\sphinxAtStartPar
\(\Delta T\)
&
\sphinxAtStartPar
10
&
\sphinxAtStartPar
K
\\
\sphinxbottomrule
\end{tabulary}
\sphinxtableafterendhook\par
\sphinxattableend\end{savenotes}


\subsection{Část I \sphinxhyphen{} Ocel}
\label{\detokenize{Cviceni/C2:cast-i-ocel}}

\begin{savenotes}\sphinxattablestart
\sphinxthistablewithglobalstyle
\centering
\begin{tabulary}{\linewidth}[t]{TTT}
\sphinxtoprule
\sphinxstyletheadfamily 
\sphinxAtStartPar
Parametr
&\sphinxstyletheadfamily 
\sphinxAtStartPar
Hodnota
&\sphinxstyletheadfamily 
\sphinxAtStartPar
Jednotka
\\
\sphinxmidrule
\sphinxtableatstartofbodyhook
\sphinxAtStartPar
\(a\)
&
\sphinxAtStartPar
1
&
\sphinxAtStartPar
m
\\
\sphinxhline
\sphinxAtStartPar
\(E_I\)
&
\sphinxAtStartPar
210
&
\sphinxAtStartPar
GPa
\\
\sphinxhline
\sphinxAtStartPar
\(\alpha_I\)
&
\sphinxAtStartPar
1.0E\sphinxhyphen{}5
&
\sphinxAtStartPar
K⁻¹
\\
\sphinxhline
\sphinxAtStartPar
\(A_I\)
&
\sphinxAtStartPar
0.01
&
\sphinxAtStartPar
m²
\\
\sphinxbottomrule
\end{tabulary}
\sphinxtableafterendhook\par
\sphinxattableend\end{savenotes}


\subsection{Část II \sphinxhyphen{} Ocel}
\label{\detokenize{Cviceni/C2:cast-ii-ocel}}

\begin{savenotes}\sphinxattablestart
\sphinxthistablewithglobalstyle
\centering
\begin{tabulary}{\linewidth}[t]{TTT}
\sphinxtoprule
\sphinxstyletheadfamily 
\sphinxAtStartPar
Parametr
&\sphinxstyletheadfamily 
\sphinxAtStartPar
Hodnota
&\sphinxstyletheadfamily 
\sphinxAtStartPar
Jednotka
\\
\sphinxmidrule
\sphinxtableatstartofbodyhook
\sphinxAtStartPar
\(b\)
&
\sphinxAtStartPar
0.6
&
\sphinxAtStartPar
m
\\
\sphinxhline
\sphinxAtStartPar
\(E_{II}\)
&
\sphinxAtStartPar
210
&
\sphinxAtStartPar
GPa
\\
\sphinxhline
\sphinxAtStartPar
\(\alpha_{II}\)
&
\sphinxAtStartPar
1.0E\sphinxhyphen{}5
&
\sphinxAtStartPar
K⁻¹
\\
\sphinxhline
\sphinxAtStartPar
\(A_{II}\)
&
\sphinxAtStartPar
0.0025
&
\sphinxAtStartPar
m²
\\
\sphinxbottomrule
\end{tabulary}
\sphinxtableafterendhook\par
\sphinxattableend\end{savenotes}


\subsection{Část III \sphinxhyphen{} Hliník}
\label{\detokenize{Cviceni/C2:cast-iii-hlinik}}

\begin{savenotes}\sphinxattablestart
\sphinxthistablewithglobalstyle
\centering
\begin{tabulary}{\linewidth}[t]{TTT}
\sphinxtoprule
\sphinxstyletheadfamily 
\sphinxAtStartPar
Parametr
&\sphinxstyletheadfamily 
\sphinxAtStartPar
Hodnota
&\sphinxstyletheadfamily 
\sphinxAtStartPar
Jednotka
\\
\sphinxmidrule
\sphinxtableatstartofbodyhook
\sphinxAtStartPar
\(c\)
&
\sphinxAtStartPar
0.8
&
\sphinxAtStartPar
m
\\
\sphinxhline
\sphinxAtStartPar
\(E_{III}\)
&
\sphinxAtStartPar
70
&
\sphinxAtStartPar
GPa
\\
\sphinxhline
\sphinxAtStartPar
\(\alpha_{III}\)
&
\sphinxAtStartPar
2.3E\sphinxhyphen{}5
&
\sphinxAtStartPar
K⁻¹
\\
\sphinxhline
\sphinxAtStartPar
\(A_{III}\)
&
\sphinxAtStartPar
0.0025
&
\sphinxAtStartPar
m²
\\
\sphinxbottomrule
\end{tabulary}
\sphinxtableafterendhook\par
\sphinxattableend\end{savenotes}
\begin{itemize}
\item {} 
\sphinxAtStartPar
Určete reakční sílu N ve vetknutí.

\item {} 
\sphinxAtStartPar
Určete velikosti napětí v řezech, které jsou dány rovinami β, γ, δ a η.

\item {} 
\sphinxAtStartPar
Určete o kolik se prodlouží prut vlivem působení sil a vlivem změny teploty \(\Delta T\).

\end{itemize}

\noindent\sphinxincludegraphics{{13_tt_p01_01}.png}

\sphinxAtStartPar
\sphinxhref{https://euler.fav.zcu.cz/kmet/ppe/13\_tt\_p01.php?action=add}{\sphinxstylestrong{Kontrola výsledků}}
\begin{enumerate}
\sphinxsetlistlabels{\arabic}{enumi}{enumii}{}{.}%
\setcounter{enumi}{1}
\item {} 
\sphinxAtStartPar
Bronzová tyč je upevněna mezi ocelovou tyčí a hliníkovou tyčí, jak je znázorněno na obrázku. V uvedených polohách působí axiální zatížení. Najděte největší hodnotu \(P\), která nepřekročí celkovou deformaci 3,0 mm nebo následující napětí: 140 MPa v oceli, 120 MPa v bronzu a 80 MPa v hliníku. Předpokládejme, že sestava je vhodně vyztužena, aby se zabránilo vybočení. Použijte E\(_\mathrm{ocel}\) = 200 GPa, E\(_\mathrm{hlinik}\) = 70 GPa a E\(_\mathrm{bronz}\) = 83 GPa.

\end{enumerate}

\noindent\sphinxincludegraphics{{211-steel-bronze-aluminum}.jpg}

\sphinxAtStartPar
\sphinxhref{https://mathalino.com/reviewer/mechanics-and-strength-of-materials/solution-to-problem-211-axial-deformation}{\sphinxstylestrong{Řešení}}


\section{Komplexní příklad}
\label{\detokenize{Cviceni/C2:komplexni-priklad}}\begin{enumerate}
\sphinxsetlistlabels{\arabic}{enumi}{enumii}{}{.}%
\item {} 
\sphinxAtStartPar
Důlní šachtový výtah je zavěšen na ocelovém laně o průměru 2,5 cm. Celková hmotnost kabiny a přepravovaných osob je 650 kg. Jak se lano prodlužuje a jaká je hmotnost lana, když
\begin{itemize}
\item {} 
\sphinxAtStartPar
v první aproximace zanedbáme hmotnost lana  a
\begin{itemize}
\item {} 
\sphinxAtStartPar
výtah je na povrchu 12 m pod motorem výtahu?

\item {} 
\sphinxAtStartPar
je na dně šachty hluboké 350 m?

\end{itemize}

\item {} 
\sphinxAtStartPar
hmotnost lana nezanedbáme.

\end{itemize}

\end{enumerate}

\noindent\sphinxincludegraphics{{a648721e30f1ba07de18480e3d2a013f6253ac63}.jpg}

\begin{sphinxadmonition}{note}{Note:}
\sphinxAtStartPar
Materiálové vlastnosti oceli


\begin{savenotes}\sphinxattablestart
\sphinxthistablewithglobalstyle
\centering
\begin{tabulary}{\linewidth}[t]{TT}
\sphinxtoprule
\sphinxstyletheadfamily 
\sphinxAtStartPar
Vlastnost
&\sphinxstyletheadfamily 
\sphinxAtStartPar
Hodnota
\\
\sphinxmidrule
\sphinxtableatstartofbodyhook
\sphinxAtStartPar
Hustota
&
\sphinxAtStartPar
7 850 kg/m³
\\
\sphinxhline
\sphinxAtStartPar
Pevnost v tahu
&
\sphinxAtStartPar
400 \sphinxhyphen{} 550 MPa
\\
\sphinxhline
\sphinxAtStartPar
Mez kluzu
&
\sphinxAtStartPar
250 \sphinxhyphen{} 350 MPa
\\
\sphinxhline
\sphinxAtStartPar
Modul pružnosti
&
\sphinxAtStartPar
200 GPa
\\
\sphinxhline
\sphinxAtStartPar
Tvrdost (Rockwell)
&
\sphinxAtStartPar
50 \sphinxhyphen{} 80 HRB
\\
\sphinxhline
\sphinxAtStartPar
Tvrdost (Brinell)
&
\sphinxAtStartPar
120 \sphinxhyphen{} 180 HB
\\
\sphinxhline
\sphinxAtStartPar
Tvrdost (Vickers)
&
\sphinxAtStartPar
140 \sphinxhyphen{} 190 HV
\\
\sphinxhline
\sphinxAtStartPar
Tepelná vodivost
&
\sphinxAtStartPar
45 \sphinxhyphen{} 60 W/m·K
\\
\sphinxhline
\sphinxAtStartPar
Koeficient tepelné roztažnosti
&
\sphinxAtStartPar
11 \sphinxhyphen{} 13 µm/m·K
\\
\sphinxhline
\sphinxAtStartPar
Elektrická vodivost
&
\sphinxAtStartPar
6 \sphinxhyphen{} 10 MS/m
\\
\sphinxhline
\sphinxAtStartPar
Bod tavení
&
\sphinxAtStartPar
1539 °C
\\
\sphinxbottomrule
\end{tabulary}
\sphinxtableafterendhook\par
\sphinxattableend\end{savenotes}
\end{sphinxadmonition}
\begin{enumerate}
\sphinxsetlistlabels{\arabic}{enumi}{enumii}{}{.}%
\setcounter{enumi}{1}
\item {} 
\sphinxAtStartPar
Jaká je maximální možná hloubka šachty z předešlého příkladu.

\item {} 
\sphinxAtStartPar
Jak by mělo vypadat lano, kdyby jsme chtěli dosáhnout vyvrtat šachtu do středu Země.
\begin{enumerate}
\sphinxsetlistlabels{\arabic}{enumii}{enumiii}{}{.}%
\item {} 
\sphinxAtStartPar
Uvažujte konstatní gravitační zrychlení.

\item {} 
\sphinxAtStartPar
Uvažujte změnu gravitačního zrychlení s hloubkou šachty.

\item {} 
\sphinxAtStartPar
O kolik se prodlouží lano v důsledku vlastní váhy když dosáhne do středu Země.

\item {} 
\sphinxAtStartPar
O kolik se prodlouží lano v důsledku změny teploty? Je reálné vytvořit takéto lano?

\end{enumerate}

\end{enumerate}

\begin{sphinxadmonition}{note}{Note:}
\sphinxAtStartPar
Gravitační zrychlení pod povrchem Země.

\sphinxAtStartPar
Gravitační zrychlení pod povrchem Země je možné odhadnout jako
\begin{equation*}
\begin{split}g = \frac{-G M(r)}{r^2} = \frac{- 4 \pi G \rho r}{3} \end{split}
\end{equation*}
\sphinxAtStartPar
kde \(\rho\) je průměrná hustota Země (\(5500 \) kg/m\(^3\)), \(G\) je gravitační konstanta (6.6743 × 10\(^{-11}\) m\(^3\) kg\(^{-1}\) s\(^{-2}\)) a \(r\) je vzdálenost o středu Země. Pro více informací \sphinxhref{https://en.wikipedia.org/wiki/Shell\_theorem}{Shell theorem}.

\sphinxAtStartPar
Teplota pod povrchem Země narůstá v tzv. geotermálním gradientu přibližně 25–30 °C/km. Pro přesnější popis se podívejte na \sphinxhref{https://en.wikipedia.org/wiki/Geothermal\_gradient}{geootermální gradient na wikipedii}.
\end{sphinxadmonition}
\begin{itemize}
\item {} 
\sphinxAtStartPar
\sphinxhref{https://physicstasks.eu/1289/mine-shaft-elevator?context=5}{\sphinxstylestrong{Částečné řešení}}

\item {} 
\sphinxAtStartPar
\sphinxhref{https://mathalino.com/reviewer/mechanics-and-strength-of-materials/solution-to-problem-205-axial-deformation}{\sphinxstylestrong{Řešení s vlastní váhou}}

\item {} 
\sphinxAtStartPar
\sphinxhref{https://www.kmp.tul.cz/resene-priklady/elearning3ead-2.html}{\sphinxstylestrong{Lano stálé pevnosti}}

\end{itemize}


\section{Staticky neurčitý problém}
\label{\detokenize{Cviceni/C2:staticky-neurcity-problem}}\begin{enumerate}
\sphinxsetlistlabels{\arabic}{enumi}{enumii}{}{.}%
\item {} 
\sphinxAtStartPar
Kruhovou ocelovou tyč o délce 1 m a průměru 10 cm vložíme mezi dva pevné podpěry. Jak se změní napětí v tyči když zvýšíme teplotu o 50 K.

\end{enumerate}

\noindent\sphinxincludegraphics{{Untitled-2-6-768x330}.jpg}

\sphinxAtStartPar
\sphinxhref{https://www.purdue.edu/freeform/me323/animations-and-demonstrations/thermal-strains-and-stresses/}{\sphinxstylestrong{Řešení}}
\begin{enumerate}
\sphinxsetlistlabels{\arabic}{enumi}{enumii}{}{.}%
\setcounter{enumi}{1}
\item {} 
\sphinxAtStartPar
Tyč složená s několika materiálů o různých průřezech je bez pnutí před působením axiálního zatížení \(P_1\) a \(P_2\). Za předpokladu, že stěny jsou tuhé, vypočítejte napětí v každém materiálu, pokud \(P_1\) = 150 kN a \(P_2\) = 90 kN.

\end{enumerate}

\noindent\sphinxincludegraphics{{247-aluminum-steel-bronze}.jpg}

\sphinxAtStartPar
\sphinxhref{https://mathalino.com/reviewer/mechanics-and-strength-of-materials/solution-to-problem-247-statically-indeterminate}{\sphinxstylestrong{Řešení}}







\renewcommand{\indexname}{Index}
\printindex
\end{document}